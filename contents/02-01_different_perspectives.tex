% !TeX encoding = UTF-8
% !TeX TS-program = xelatex
% !TeX spellcheck = en_GB
% !TeX root = ../Herbstrith-H10_over_AI.tex

In 1900, David Hilbert held his famous lecture before the \emph{Second
International Congress of Mathematicians} in Paris. The lecture entitled
\begin{german}\enquote{Mathematische Probleme}\end{german} contained 23
mathematical problems left for the 20th century to solve.
% TODO Write about the history of \textsc{H10}
\todo{Write about the history of \textsc{H10}}

\begin{quotation}
    Given a Diophantine equation with any number of unknown quantities and with rational integral numerical coefficients: To devise a process according to which it can be determined by a finite number of operations whether the equation is solvable in rational integers.
\end{quotation}

A \emph{Diophantine equation} is of the form
%
\[ p(α_1, …, α_n) = 0 \]
%
for some polynomial $p ∈ ℤ[X_1, …, X_n]$, allowing only solutions $α_1,…,α_n ∈ ℤ$.


\begin{defin}
    Let $R$ be a commutative ring with unit. A set $S \subseteq R^n$ is said to
    be \emph{Diophantine} if there exists a polynomial $p ∈ R[X_1,…,X_n,
    Y_1,…,Y_m]$ ($m ≥ 0$\footnote{If $m = 0$, one interprets this as $p ∈
    R[X_1,…,X_n]$}) such that

    \[ (α_1,…,α_n) ∈ S \Leftrightarrow ∃ β_1,…,β_m ∈ R: p(α_1,…,α_n,β_1,…,β_m) = 0 \]
\end{defin}

A polynomial $p ∈ R[X_1, …, X_n]$ as above defines $n$ relations $\rel{p}_i$ by
\[
  \rel{p}(α_1, …, α_i)  \Leftrightarrow
   ∃ β_1, …, β_{n - i} ∈ R : p(α_1, …, α_i, β_1, …, β_{i - n}) = 0 \quad
   1 ≤ i ≤ n.
\]
In this sense a set $S \subseteq R^i$ is Diophantine if there exists a
polynomial $p ∈ R[X_1, …, X_n]$ such that for each $\mathbf α ∈ S$ there exists
a $\mathbf β ∈ R^{n - i}$ that is related to $\mathbf α$ via $\rel p_i$.

Viewing $n$-ary relations as subsets of $R^n$, I will sometimes refer to
Diophantine sets as \emph{Diophantine relations}. A function $R^n → R^m$ is
called Diophantine if it is Diophantine viewed as a relation.

\begin{exam}
  \begin{exlist}
    \item Let $R$ be an integral domain.
    Then every finite subset $S$ of $R$ is Diophantine, because the roots of
    \[
      p(X) := \prod_{s ∈ S} (X - s)
    \]
    are precisely the elements of $S$.

    \item The set of composite numbers is Diophantine over $ℕ$, as $α ∈ ℕ$ is
    composite if and only if
    \[
      ∃ β_1, β_2 ∈ ℕ : x = (β_1 + 2) (β_2 + 2).
    \]
    Here adding $2$ to $β_1$ and $β_2$ ensures, that both factors are greater
    than $1$. Choosing
    \[
      p (X, Y_1, Y_2) := X - (Y_1 + 2)(Y_2 + 2)
    \]
    yields the claim.

    \item The usual order relation $≤$ on $ℕ$ is Diophantine over $ℕ$.
    Indeed $α_1 ≤ α_2$ in $ℕ$ if and only if
    \[
      ∃ β ∈ ℕ : α_1 + β  = α_2.
    \]

    \item Let $R$ be a commutative ring with unit. Then divisibility in $R$ is
    Diophantine. Indeed $α_1 \mid α_2$ in $R$ precisely if
    \[
      ∃ β ∈ R : α_1 β = α_2.
    \]

    \item Let $K$ be a number field and $\algint$ its ring of algebraic integers. Then $\algint \setminus \set{0}$ is
    Diophantine over $\algint$. I extend the hint stated in \cite[Prop. 1]{Denef1978} and claim that
    \[
      α ≠ 0 ⇔ ∃ β, γ ∈ \algint : α β = (2 γ - 1)(3 γ - 1).
    \]

    Firstly, note that the polynomial on the right hand side has the roots $1/2$
    and $1/3$ in $ℚ$. As the intersection $\algint ∩ ℚ $ equals $ℤ$ for all
    number fields $K$, one obtains that the polynomial identity can only be
    satisfied for $α ≠ 0$.

    Let now \(α ≠ 0\). We can decompose the ideal \((α) = \mathfrak x_2
    \mathfrak x_3\) such that
    \[
    (2) + \mathfrak x_2 =
    \algint, \; (3) + \mathfrak x_3 = \algint \, \text{and } \, \mathfrak x_2 +
    \mathfrak x_3 = \algint.
    \]
    This is because \(2\) and \(3\) are rational primes and therefore \((2)\)
    and \((3)\) are relative prime.
    % see Baxa Thm 120
    In other words, we find
    \[
      ∃ x_2 ∈ \mathfrak x_2, ∃ y_2 ∈ \algint : 2 y_2 + x_2 = 1 \quad \text{and} \quad
      ∃ x_3 ∈ \mathfrak x_3, ∃ y_3 ∈ \algint : 3 y_3 + x_3 = 1
    \]
     As a consequence of the Chinese remainder theorem~\cite[see][§~I,
     Thm~3.6]{Neukirch2006} the congruences
    \[
      γ \equiv y_2 \mod \mathfrak x_2 \quad \text{and} \quad
      γ \equiv y_3 \mod \mathfrak x_3
    \]
    are simultaneously solvable. This implies that
    \[
      2 γ \equiv 2 y_2 \equiv 1 \mod \mathfrak x_2 \quad \text{and} \quad
      3 γ \equiv 3 y_3 \equiv 1 \mod \mathfrak x_3.
    \]
    Which can be rewritten as
    \[
      2 γ - 1 ∈ \mathfrak x_2  \quad \text{and} \quad
      3 γ - 1 ∈ \mathfrak x_3.
    \]
    As a consequence, \((2 γ - 1)(3 γ - 1)\) is contained in \(\mathfrak x_2
    \mathfrak x_3 = (α)\), or put differently, there exists a \(β ∈ \algint\)
    such that
    \[
      α β = (2 γ - 1)(3 γ - 1).
    \]

    \item\label{ex:U K is Diophantine}
    Let \(R\) be a commutative ring with unit. The set of units \(U\) in \(R\) is Diophantine over \(R\). This can be seen by the polynomial equation
    \[
      x ∈ U ⇔ ∃ y : xy = 1.
    \]



    \item\label{ex:N is Diophantine over Z}
    Each non-negative integer $α$ is the sum of four squares and as a
    consequence
    \[
      x ∈ ℕ ⇔ ∃β_1,β_2,β_3,β_4∈ℤ: β_1^2 + β_2^2 + β_3^2 + β_4^2 = α.
    \]
    is a Diophantine definition of $ℕ$ over $ℤ$. A full proof of the claim using
    \cref{thm:Minkowski} is given in \cite[Remark 4.20]{Milne2017}.
  \end{exlist}
\end{exam}

Let $Σ_{ring} = \set{+,\cdot; 0, 1}$ be the signature of rings and let
$\mathfrak{Z} = ⟨ℤ; +^{\mathfrak{Z}}, \cdot^{\mathfrak{Z}}; 0^{\mathfrak{Z}},
1^{\mathfrak{Z}}⟩$ denote the integers as $Σ_{ring}$-structure. Then \textsc{H10}
asks whether the existential positive theory $\mathtt{Th}_{∃+}(\mathfrak{Z})$ is
decidable. This is the theory of $\mathfrak{Z}$ containing only fully
existentially quantified atomic formulas or conjunctions---no negations,
disjunctions or universal quantifiers.

This claim can be justified by observing that each positive integer $n$ can
be expressed as
\[
  \underbrace{\mathtt{1 + 1 + … + 1}}_{n\text{-times}}.
\]
So the question whether $X_1 - (X_2 + 2) (X_3 + 2)$ has integral roots is
equivalent to asking whether
\[
  \mathfrak Z \models \mathtt{∃ x_1 ∃ x_2 ∃ x_3 \; x_1 \doteq (x_2 + 1 + 1) \cdot (x_3 + 1 + 1)}.
\]
We will see in \cref{lem:intersections and unions} that for certain integral
domains including $\mathfrak Z$ conjunctions of multiple polynomial equations
can be effectively translated to a single polynomial equation.

The beauty of this approach is that it directly gives rise to generalisations of
Hilbert's tenth problem. For example one can exchange the ring structure
$\mathfrak Z$ by some other ring. Even uncountable rings like $\mathfrak C = ⟨ℂ;
+^{\mathfrak C}, \cdot^{\mathfrak C}; 0^{\mathfrak C}, 1^{\mathfrak C}⟩$ are
possible, as $Σ_{ring}$ is finite and therefore each sentence in
$\mathtt{Th}_{∃+}(\mathfrak{C})$ can be encoded by its Gödelisation or---as it
was done when typesetting this thesis---by the concatenation of the
\textsc{UTF-8} encodings of the symbols.
% QUESTION One can show that the \(Σ_{ring}\)-theory of \(ℂ\) is complete
% (Vaught's test) and therefore decidable. Should I present this?
\todo{One can show that the \(Σ_{ring}\)-theory of \(ℂ\) is complete (Vaught's test) and therefore decidable. Should I present this?} %cf.~\url{https://math.stackexchange.com/a/103187/422444}}
A second approach for generalisation is exchanging $Σ_{ring}$ by some other
computable signature (see \cref{sec:computable structures}). Then one is in the
realm of constraint satisfaction problems. Or one could consider the full first
order theory of $\mathfrak Z$. Then Gödel's first incompleteness theorem gives a
negative answer to the question, whether $\mathtt{Th}(\mathfrak Z)$ is
decidable.\footnote{Actually, Gödel's first incompleteness theorem
implies---under the assumption that Peano arithmetic is consistent---that the
full first order theory of $\mathfrak N = ⟨ℕ; +^{\mathfrak N}, \cdot^{\mathfrak
N}; 0^{\mathfrak N}, 1^{\mathfrak N}⟩$ is undecidable, but by \cref{ex:N is
Diophantine over Z} we can transform any first order sentence $\mathtt{ϕ(x_1, …,
x_n)}$ to a first order sentence where $\mathtt ϕ$ is appended by a conjunction
of formulas
\[
  \mathtt{ ∃ x_{i1} ∃x_{i2} ∃x_{i3} ∃x_{i4} \; x_i \doteq x_{i1} \cdot x_{i1} +
  x_{i2} \cdot x_{i2} + x_{i3} \cdot x_{i3} + x_{i4} \cdot x_{i4}}
\]
for $1 ≤ i ≤ n$, stating that $\mathtt{\seq x}$ are non-negative integers.
}

I will use the second approach and replace $Σ_{ring}=\set{+, \cdot;
0, 1}$ by the signature
\[
  Σ_{ring}^* = \set{+, \cdot; 0, 1, η \mid η ∈ ℕ},
\]
where one adds countably many constants representing the elements unequal
to $0$ and $1$ of a countable ring with unit $R$.

Throughout this thesis I will mostly concern myself with the case where $K$ is
an algebraic number field and $\algint$ is its ring of integers over $ℤ$.
Setting $\modalgint = ⟨\algint; +^{\modalgint},
\cdot^{\modalgint}; η \mid η ∈ \algint⟩$, I understand by Hilbert's tenth
problem (\textsc{H10}) over $\algint$ the following problem.

\begin{quote}
  Is there a Turing machine $\mathbb A$ that decides $\mathtt{Th}_{∃+} (\modalgint)$ i.e.
  \[
    \mathbb A (x) =
      \begin{cases}
        \one & \text{if } ∃ \mathtt{ϕ} ∈  \mathtt{Th}_{∃+} (\modalgint) : \enc{\mathtt{ϕ}} = x \\
        \zer & \text{otherwise}
      \end{cases}?
  \]
\end{quote}

Using the notion of computable rings it will be easy to see that
$\mathtt{Th}_{∃+}(\modalgint)$ is semi-decidable for all number fields $K$.
However, it is not known for all number fields, whether their existential
positive theory is undecidable. In the subsequent sections I will prove the
undecidability of \textsc{H10} over selected number fields $K$.
% QUESTION Should I mention \emph{pure Diophantine} definitions and their
% relation to Diophantine definitions? \cite[cf.][§~11]{Davis1976}
\todo{Should I mention \emph{pure Diophantine} definitions and their relation to Diophantine definitions? \cite[cf.][§~11]{Davis1976}}

Considering the full first order theory of $\modalgint$
\textcite{Robinson1959} proved as early as \citeyear{Robinson1959} that
$\mathtt{Th}(\modalgint)$ is undecidable. In \citeyear{Rumely1986},
\textcite{Rumely1986} published his surprising result that \textsc{H10} is
solvable over $\mathcal O$, the ring of all algebraic integers.
\Textcite{Dries1988} extended this result to the full first order theory of
$\mathcal O$ in \citeyear{Dries1988}.

Probably the most prominent open problem in this area is the case of $ℚ$. A
positive answer to \textsc{H10} over $ℚ$ would imply that there is a universal
algorithm deciding whether a variety over $ℚ$ has a rational point. By giving a
first order definition of $ℤ$ over $ℚ$, \textcite{Robinson1949} could derive the
undecidability of the full first order theory $\mathtt{Th}(ℚ)$ from the
undecidability of the theory of $ℤ$ in \citeyear{Robinson1949}. But her
definition involves universal quantifiers and cannot be used for inferring to
\textsc{H10}. \Textcite{Park2013} generalised this result by providing a
universal first order definition for $\algint$ over an arbitrary number field
$K$ in \citeyear{Park2013}.

The surveys \cite{Koenigsmann2014,Poonen2008} offer a more extensive overview of
problems related to undecidability in number theory.
