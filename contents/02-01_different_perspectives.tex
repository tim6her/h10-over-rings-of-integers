% !TEX encoding = UTF-8
% !TEX TS-program = xelatex
% !TEX spellcheck = en_GB
% !TEX engine = xelatex
% !TEX root = ../Herbstrith-H10_over_AI.tex
%
% ██████  ███████ ██████  ███████ ██████  ███████  ██████ ████████
% ██   ██ ██      ██   ██ ██      ██   ██ ██      ██         ██
% ██████  █████   ██████  ███████ ██████  █████   ██         ██
% ██      ██      ██   ██      ██ ██      ██      ██         ██
% ██      ███████ ██   ██ ███████ ██      ███████  ██████    ██ ██

\subsection{Diophantine equations and sets}
% ██████  ██  ██████  ██████  ██   ██        ███████  ██████
% ██   ██ ██ ██    ██ ██   ██ ██   ██        ██      ██    ██
% ██   ██ ██ ██    ██ ██████  ███████        █████   ██    ██
% ██   ██ ██ ██    ██ ██      ██   ██        ██      ██ ▄▄ ██
% ██████  ██  ██████  ██      ██   ██ ██     ███████  ██████  ██
%                                                        ▀▀

In 1900, David Hilbert held his famous lecture~\cite{Hilbert1900} before the
\emph{International Congress of Mathematicians} in Paris. During the lecture
entitled \foreignquote{german}{Mathematische Probleme%
\footnote{%
  \foreignquote{english}{mathematical problems}
}}
Hilbert posed ten mathematical problems left for the twentieth century to solve.
Hilbert's list of problems was later amended to contain twenty-three problems.
The tenth of these questions and its variants are the subject of this thesis.
The problem states

\begin{foreigndisplayquote}{german}
  \textsc{10. Entscheidung der Lösbarkeit einer Diophantischen Gleichung.}
  Eine Diophantische Gleichung mit irgend welchen Unbekannten und mit
  ganzen rationalen Zahlencoefficienten sei vorgelegt: man soll ein Verfahren
  angeben, nach welchem sich mittelst einer endlichen Anzahl von Operationen
  entscheiden läßt, ob die Gleichung in ganzen rationalen Zahlen lösbar ist.%
  \footnote{
    \begin{english}
      \textsc{10. Determination of the solvability of a diophantine equation.}
      Given a diophantine equation with any number of unknown quantities and
      with rational integral numerical coefficients: To devise a process
      according to which it can be determined by a finite number of operations
      whether the equation is solvable in rational integers.
      \hspace*{\fill}\cite[translation published in][]{Hilbert2000}
    \end{english}
  }
  \hspace*{\fill}\cite{Hilbert1900}
\end{foreigndisplayquote}

A \emph{Diophantine equation}---in the classical sense---is of the form
\[
  p(α_1, …, α_n) = 0,
\]
where \(p ∈ ℤ[X_1, …, X_n]\) is a polynomial and one only allows rational integral
solutions \(α_1,…,α_n ∈ ℤ\). Using the tools developed in \cref{sec:model
theory,sec:number theory}, we will exchange one or both occurrences of the
rational integers by values from other rings. It took until the 1930s to
formalize what Hilbert meant by a \foreignquote{german}{Verfahren \textins{mit}
einer endlichen Anzahl von Operationen%
\footnote{\foreignquote{english}{%
  process \textins{with} a finite number of operations}
}}
to the notion of \emph{computation} that was defined in \cref{sec:computability
theory}. In the same section we have also defined what it means to \emph{decide
a problem}, so we are left with the task of identifying Hilbert's question with
a set of strings. In a first approach one could reformulate the tenth problem as
\begin{quote}
  \textsc{H10.} For a fixed polynomial \(p ∈ ℤ[X_1, X_2, …]\) does there exist a
  Turing machine \(\mathbb{A}_p\) that returns \(\one\) if \(p\) has a root and
  \(\zer\) otherwise?
\end{quote}
This formalization is however trivially solvable, as the Turing machine in
question takes no input and thus must always return either \(\one\) or \(\zer\).
But such machines returning a constant value are easily constructable. We must
therefore exchange the quantifiers and ask
\begin{quote}
  \textsc{H10.} Does there exist a Turing machine \(\mathbb{A}\) and an encoding
  \(\enc{\cdot}\) such that for all polynomials \(p ∈ ℤ[X_1, X_2, …]\) the
  output \(\mathbb{A}(\enc{p})\) is \(\one\), if \(p\) has integral roots, and
  \(\zer\) otherwise.
\end{quote}

We will see that if we restrict ourselves to encodings \(\enc{\cdot}\) that
allow to efficiently obtain the evaluation \(\enc{p(\mathbf{α})}\) from
\(\enc{p}\) and \(\enc{\mathbf{α}}\), then the answer to the question above is
negative. In fact, for all rings of algebraic integers \(\algint\), that we will
consider, we will find a single multivariate polynomial \(p_{\mathcal{K}} ∈
\algint{}[X, \seq{Y}]\) such that for all Turing machines \(\mathbb{A}\) there
exists an algebraic integer \(α ∈ \algint\) with the property that
\(\mathbb{A}\) cannot correctly decide whether the partially evaluated
polynomial
\[
  p_{\mathcal{K}}(α, \seq{Y})
\]
has roots in \(\algint\). The index \(\mathcal{K}\) of the polynomial above is
not chosen at random. Indeed, the polynomial \(p_{\mathcal{K}}\) represents an
encoded version of the halting set \(\mathcal{K}\) in the sense of the following
definition.

\begin{defin}
    Let \(R\) be a commutative ring with unity. A set \(S \subseteq R^n\) is said to
    be \emph{Diophantine} over \(R\) if there exists a polynomial \(p ∈
    R[X_1,…,X_n, Y_1,…,Y_m]\) in \(n + m\) many indeterminates (\(m,n ≥ 0\)) such
    that
    \[
      (α_1,…,α_n) ∈ S \quad ⇔ \quad
      ∃ β_1,…,β_m ∈ R: p(α_1,…,α_n,β_1,…,β_m) = 0
    \]
\end{defin}

A polynomial \(p ∈ R[X_1, …, X_n]\) as above defines an \(n\)-ary relation \(\rel{p}\)
on \(R\) by
\[
  \rel{p}(α_1, …, α_n)  \quad :⇔ \quad p(α_1, …, α_n) = 0
\]
In this sense a set \(S \subseteq R^i\) is Diophantine if there exists a
polynomial \(p ∈ R[X_1, …, X_n]\) such that
\[
  (\seq[i]{α}) ∈ S \quad ⇔ \quad
  ∃ \seq[n - i]{β} : \rel{p}(\seq[i]{α}, \seq[n - i]{β}).
\]
If the ring \(R\) is computable, it is immediate that the relation \(\rel{p}\)
is computable. Thus, we have the following lemma.
\begin{lem}
  Let \(R\) be a computable commutative ring with unity. Then every Diophantine
  subset of \(R\) is semi-decidable.
\end{lem}

Viewing \(n\)-ary relations as subsets of \(R^n\), I will sometimes refer to
Diophantine sets as \emph{Diophantine relations}. A function \(R^n → R^m\) is
called \emph{Diophantine} if it is Diophantine viewed as an \((n + m)\)-ary
relation.

\begin{exam}\label{ex:Diophantine sets}
  \begin{exlist}
    \item Let \(R\) be an integral domain.
    Then every finite subset \(S\) of \(R\) is Diophantine, because the roots of
    \[
      p(X) := \prod_{s ∈ S} (X - s)
    \]
    are precisely the elements of \(S\).

    \item Let \(R\) be an integral domain. Then for every polynomial \(p ∈
    R[X_1, …, X_n]\) the associated polynomial function \(p: R^n → R\) is
    Diophantine. To see this we set
    \[
      q(X_1, …, X_n, X_{n + 1}) := p(\seq{X}) - X_{n + 1},
    \]
    and notice that \(q\) has a root \((\seq{α}, α_{n + 1}) ∈ R\) if and only if
    \(p(\seq{α}) = α_{n + 1}\) as claimed.

    \item Let \(R\) be a commutative ring with unit. Then divisibility in \(R\) is
    Diophantine. Indeed \(α_1 \mid α_2\) in \(R\) precisely if
    \[
      ∃ β ∈ R : α_1 β = α_2.
    \]

    \item Let \(K\) be a number field and \(\algint\) its ring of algebraic integers. Then \(\algint \setminus \set{0}\) is
    Diophantine over \(\algint\). I extend the hint stated in \cite[Prop. 1]{Denef1978} and claim that
    \[
      α ≠ 0 \quad ⇔ \quad
      ∃ β, γ ∈ \algint : α β = (2 γ - 1)(3 γ - 1).
    \]

    Firstly, note that the polynomial on the right hand side has the roots \(1/2\)
    and \(1/3\) in \(ℚ\). As the intersection \(\algint ∩ ℚ \) equals \(ℤ\) for all
    number fields \(K\), one obtains that the polynomial identity can only be
    satisfied for \(α ≠ 0\).

    Let now \(α ≠ 0\). We can decompose the ideal \((α) = \mathfrak x_2
    \mathfrak x_3\) such that
    \[
    (2) + \mathfrak x_2 =
    \algint, \; (3) + \mathfrak x_3 = \algint \, \text{and } \, \mathfrak x_2 +
    \mathfrak x_3 = \algint
    \]
    hold. This is because \(2\) and \(3\) are rational primes and therefore \((2)\)
    and \((3)\) are relative prime.
    % see Baxa Thm 120
    In other words, we find
    \[
      ∃ x_2 ∈ \mathfrak x_2, ∃ y_2 ∈ \algint : 2 y_2 + x_2 = 1 \quad \text{and} \quad
      ∃ x_3 ∈ \mathfrak x_3, ∃ y_3 ∈ \algint : 3 y_3 + x_3 = 1
    \]
     As a consequence of the Chinese remainder theorem (\cref{thm:Chinese
     remainder}) the congruences
    \[
      γ \equiv y_2 \mod \mathfrak x_2 \quad \text{and} \quad
      γ \equiv y_3 \mod \mathfrak x_3
    \]
    are simultaneously solvable. This implies that
    \[
      2 γ \equiv 2 y_2 \equiv 1 \mod \mathfrak x_2 \quad \text{and} \quad
      3 γ \equiv 3 y_3 \equiv 1 \mod \mathfrak x_3.
    \]
    Which can be rewritten as
    \[
      2 γ - 1 ∈ \mathfrak x_2  \quad \text{and} \quad
      3 γ - 1 ∈ \mathfrak x_3.
    \]
    We deduce that \((2 γ - 1)(3 γ - 1)\) is contained in \(\mathfrak x_2
    \mathfrak x_3 = (α)\), or put differently, there exists a \(β ∈ \algint\)
    such that
    \[
      α β = (2 γ - 1)(3 γ - 1).
    \]

    \item\label{ex:U K is Diophantine}
    Let \(R\) be a commutative ring with unit. The set of units \(U\) in \(R\) is Diophantine over \(R\). This can be seen by the polynomial equation
    \[
      x ∈ U \quad ⇔ \quad ∃ y : xy = 1.
    \]
  \end{exlist}
\end{exam}

In the examples above we have seen that many sets and relations are Diophantine.
Before we go on proving some structural results for Diophantine sets, we turn
our attention to the classical case of Diophantine subsets of \(ℤ\) and study
their relations with subsets of \(ℕ\).

\begin{exam}[Diophantine subsets of \(ℕ\)]\label{ex:N is Diophantine over Z}
  If one wants to study sets that are Diophantine over \(ℕ\), one runs into the
  problem that \(ℕ\) is not a ring. An approach that has been carried out
  \cite[cf.~e.g.][]{Davis1973} is considering sets \(S ⊂ ℕ^n\) that are
  Diophantine over \(ℤ\) and allow only for witnesses in \(ℕ\). I will show that
  this construction can be carried out in a Diophantine way.

  First, we note that if \(S_1, S_2 ∈ ℤ^n\) are Diophantine over \(ℤ\), then
  their intersection is Diophantine over \(ℤ\) as well. This is because if
  \(p_1 ∈ ℤ[\seq{X}, \seq[m_1]{Y}]\) represents \(S_1\) via
  \[
    (\seq{α}) ∈ S_1 \quad ⇔ \quad
    ∃ \seq[m_1]{β} : p_1(\seq{α}, \seq[m_1]{β}) = 0
  \]
  and \(p_2 ∈ ℤ[\seq{X}, \seq[m_1]{Y}]\) represents \(S_2\), we set \(m := m_1 +
  m_2\) and consider \(p_1\) and \(p_2\) as polynomials in \(n + m\) many
  indeterminates, where for all \(i ∈ \set{n + 1, …, m}\) indeterminate \(Y_i\)
  either appears in \(p_1\) or \(p_2\) but not in both. Then \((\seq{α},
  \seq[m]{β}) ∈ ℤ^{m + n}\) is a root of
  \[
    q(\seq{X}, \seq[m]{Y}) :=
      p_1(\seq{X}, \seq[m]{Y})^2 + p_2(\seq{X}, \seq[m]{Y})^2
  \]
  if and only if \((\seq{α}, \seq[m]{β})\) is a root of \(p_1\) and \(p_2\).
  Thus, we find that
  \[
    S_1 ∩ S_2 =
    \set{\seq{α} ∈ ℤ \mid ∃ \seq[m]{β} ∈ ℤ : q(\seq{α}, \seq[m]{β}) = 0}.
  \]

  By Lagrange's four square-theorem (\cref{pro:Lagranges four square theorem})
  we know that every non-negative integer \(α\) is the sum of four squares and
  as a consequence
  \[
    x ∈ ℕ \quad ⇔ \quad
    ∃β_1,β_2,β_3,β_4∈ℤ: β_1^2 + β_2^2 + β_3^2 + β_4^2 = α.
  \]
  is a Diophantine definition of \(ℕ\) over \(ℤ\). Therefore, we can check for a
  given polynomial equation whether all variables take only non-negative values
  in a Diophantine way. More formally, we say that  a subset \(S ⊂ ℕ^n\) is
  \emph{Diophantine over \(ℕ\)} if there exists a polynomial \(p ∈ ℤ[X_1, …,
  X_n, Y_1, …, Y_m]\) such that
  \[
    (\seq{α}) ∈ S \quad ⇔ \quad
    ∃ \seq[m]{β} ∈ ℕ : p(\seq{α}, \seq[m]{β}) = 0.
  \]
  If this is the case, we find that \(S\) is Diophantine over \(ℤ\) as well, by
  conjugating the identity with the clause
  \[
    \left(\bigwedge_{i = 1}^n ∃ γ_{i1}, …, γ_{i4} ∈ ℤ :
      α_i = \sum_{j = 1}^4 γ_{ij}^2 \right) ∧
    \left(\bigwedge_{i = 1}^m ∃ δ_{i1}, …, δ_{i4} ∈ ℤ :
      β_i = \sum_{j = 1}^4 δ_{ij}^2 \right).
  \]

  We now list some examples of sets that are Diophantine over \(ℕ\).
  \begin{exlist}
    \item The set of composite numbers is Diophantine over \(ℕ\), as \(α ∈ ℕ\) is
    composite if and only if
    \[
      ∃ β_1, β_2 ∈ ℕ : x = (β_1 + 2) (β_2 + 2).
    \]
    Here adding \(2\) to \(β_1\) and \(β_2\) ensures, that both factors are greater
    than \(1\). Choosing
    \[
      p(X, Y_1, Y_2) := X - (Y_1 + 2)(Y_2 + 2)
    \]
    yields the claim. To transform this into a Diophantine definition over
    \(ℤ\), we must conjugate the clauses stating that \(α, β_1\) and \(β_2\) are
    non-negative. Thus, we obtain
    \begin{align*}
      ∃ β_1, β_2, γ_1, …, γ_4, δ_{11}, …, δ_{14}, δ_{21}, …, δ_{24} ∈ ℤ: (
        & x = (β_1 + 2) (β_2 + 2) ∧\\
        & x = γ_1^2 + γ_2^2 + γ_3^2 + γ_4^2 ∧\\
        & β_1 = δ_{11}^2 + δ_{12}^2 + δ_{13}^2 + δ_{14}^2 ∧\\
        & β_2 = δ_{21}^2 + δ_{22}^2 + δ_{23}^2 + δ_{24}^2),
    \end{align*}
    which can be rewritten as the single Diophantine identity
    \begin{align*}
      ∃ β_1, β_2, & γ_1, …, γ_4, δ_{11}, …, δ_{14}, δ_{21}, …, δ_{24} ∈ ℤ:\\
        & \left(\left(\left(x - (β_1 + 2) (β_2 + 2)\right)^2 +
          \left(x - (γ_1^2 + γ_2^2 + γ_3^2 + γ_4^2)\right)^2\right)^2 +\right.\\
        & \left. +
          \left(
            β_1 - (δ_{11}^2 + δ_{12}^2 + δ_{13}^2 + δ_{14}^2)\right)^2
          \right)^2 +
          \left(β_2 - (δ_{21}^2 + δ_{22}^2 + δ_{23}^2 + δ_{24}^2)\right)^2.
    \end{align*}

    \item The usual order relation \(≤\) on \(ℕ\) is Diophantine over \(ℕ\).
    Indeed \(α_1 ≤ α_2\) in \(ℕ\) if and only if
    \[
      ∃ β ∈ ℕ : α_1 + β  = α_2.
    \]
  \end{exlist}
\end{exam}

We will now see how one can describe Diophantine sets from the view of model
theory.

\begin{lem}
  Let \(R\) be a commutative ring with unity and let \(\mathfrak{R}\) be its
  \(\lang_{ring}\)-structure. Then \(S ⊂ R^n\) is Diophantine if and only if
  there exists an atomic \(\lang_R\)-formula \(ϕ(\mathtt{\seq{x}, \seq[m]{y}})\)
  such that
  \[
    (\seq{α}) ∈ S \quad ⇔ \quad
    \mathfrak{R} \models \mathtt{∃ y_1 : … ∃ y_m: }
        ϕ(\seq{α}, \mathtt{\seq[m]{y}})
  \]
  holds.
\end{lem}
\begin{proof}
  By \cref{thm:Diophantine theory} the formula \(ϕ(\seq{α}, \seq[m]{β})\) is
  true in \(\mathfrak{R}\) if and only if the polynomial associated with \(ϕ\)
  has a root in \((\seq{α}, \seq[m]{β})\).
\end{proof}

Note that even more is true as a partially evaluated polynomial is still a
polynomial. Thus, one can decide membership in all Diophantine sets if and only
if one can decide for all polynomials whether they have roots. As a consequence,
we will identify Hilbert's tenth problem over \(R\) with the set of Gödel
numbers of \(\mathtt{H10}(\mathfrak{R})\) if \(R\) is a countable commutative
ring with unity, and restate Hilbert's problem as
\begin{quote}
  \textsc{H10.} Is the Diophantine theory \(\mathtt{H10}(\mathfrak{R})\)
  decidable?
\end{quote}
In some cases we can modify Hilbert's question even more and allow for
disjunctions and conjunctions to appear in our theory.

\begin{lem}\label{lem:intersections and unions}
    Let \(R\) be an integral domain, whose quotient field \(\Quot R\) is not
    algebraically closed. Then if \(S_1, S_2 \subseteq R\) are Diophantine so
    are
    \[
      S_1 ∩ S_2 \quad \text{and} \quad S_1 ∪ S_2.
    \]
    If \(R\) is computable, then there is an algorithm that derives the defining
    polynomial equations for union and intersection efficiently from the
    equations of \(S_1\) and \(S_2\).
\end{lem}

In other words, conjunctions and disjunctions of existentially quantified atomic
formulae can be replaced by a single existentially quantified atomic formula. Or
again put differently, conjunction \(∧\) and disjunction \(∨\) are
\(\lang_R\)-definable, efficiently computable predicates.

\begin{proof}
  Let \(p(\seq{X}, \seq[m_1]{Y})\) and \(q(\seq{X}, \seq[m_2]{Y})\) give
  Diophantine definitions of \(S_1\) and \(S_2\) resp. Then as in \cref{ex:N is
  Diophantine over Z} we set \(m = m_1 + m_2\) and interpret \(p, q\) as
  polynomials in \(n + m\) many indeterminates such that for all \(i ∈ \set{n +
  1, …, m}\) indeterminate \(Y_i\) appears either in \(p\) or \(q\) but not
  in both.

  Now set
  \[
    h := p q.
  \]
  Then \(h\) vanishes if and only if \(p\) or \(q\) vanishes. As a consequence,
  the \(n\)-tuple \((\seq{α}) ∈ R^n\) is in the union of \(S_1\) and \(S_2\) if
  and only if
  \[
    ∃ \seq[m]{β} ∈ R : h(\seq{α}, \seq[m]{β}) = 0.
  \]

  To make notation easier when proving the claim for intersections, I will
  assume that \(n = 1\) and \(m = 2\). The general cases follows completely
  analogously. Let then
  \[
    h(T) = a_k T^k + … + a_1 T + a_0 ∈ R[T]
  \]
  be a polynomial of degree \(k > 0\) without roots in \(\Quot R\). Then
  \(\overline h(T) = T^k h(T^{-1})\) does not have roots in \(\Quot R\) either.
  As if \(α ∈ \Quot R\) is a root of \(\overline h\) then
  \[
    0 = \overline h(α) = a_k + a_{k-1} α + a_1 α^{-1} + a_0 α^k
  \]
  and \(α = 0\) implies that \(a_k = 0\). Otherwise, \(α^{-1}\) is a root of
  \(α^k h\) and therefore of \(h\).

  Now consider
  \[
    H(α, β_1, β_2) =
    \sum_{i=0}^k a_i p(α, β_1)^i q(α, β_2)^{k - i}.
  \]
  I will prove for all \(α, β_1, β_2 ∈ R\) that \(H(α, β_1, β_2) = 0\) precisely
  if \(p(α, β_1)\) and \(p(α, β_2)\) vanish. Then \(H\) represents the
  intersection via
  \[
    α ∈ S_1 ∩ S_2 \quad ⇔ \quad
    ∃ β_1, β_2 ∈ R : H(α, β_1, β_2) = 0.
  \]

  If \(H(α, β_1, β_2) = 0\) but \(p(α, β_1) ≠ 0\) then
  \[
    0 = \frac{H}{p^k} (α, β_1, β_2) =
    \frac{1}{p(α, β_1)^k} \sum_{i=0}^k a_i p(α, β_1)^i q(α, β_2)^{k - i} =
    \overline h \left(\frac{q}{p}(α, β_1, β_2) \right),
  \]
  which is a contradiction to \(\overline h\) not having roots. If on the
  other hand \(H(α, β_1, β_2) = 0\) but \(q(α, β_2) ≠ 0\)
  one finds
  \[
    0 = \frac H {q^k}(α, β_1, β_2) =
    \frac{1}{q(α, β_2)^k} \sum_{i=0}^k a_i p(α, β_1)^i q(α, β_2)^{k - i} =
    h \left( \frac pq (α, β_1, β_2) \right).
  \]
  The converse direction is clear as the powers of \(p\) and \(q\) sum up
  to \(k\) for each summand in the definition of \(H\).

  To prove the effectiveness of these methods one observes, that the defining
  equations contain only polynomials in \(p\) and \(q\). Thus,
  \cref{ex:polynomials are computable} implies that the polynomial equations for
  union and intersection of Diophantine sets can be computed from the
  polynomials \(p\) and \(q\).
\end{proof}

Note that the algorithm presented above does not depend on the initial equations
\(p\) and \(q\) but it does depend on the integral domain \(R\). We might
need different polynomials \(h\) without roots for each ring \(R\) in the case
of conjunctions.

\begin{rem}
  Using \(h(X) = X^2 + 1\) as the polynomial without roots in \(ℤ\) for the
  construction described in the proof of \cref{lem:intersections and unions},
  one obtains
  \[
    H = p^2 + q^2
  \]
  precisely as in \cref{ex:N is Diophantine over Z}. However, we could have
  equivalently chosen \(h(X) = X^2 - 2X - 2\) as a polynomial without rational
  roots---\(h\) has the irrational roots \(1 ± √3\)---and obtain
  \[
    H = p^2 - 2 pq + 2 q^2.
  \]
\end{rem}

Using induction and the lemma above, one immediately obtains that arbitrary
finite unions and intersections of Diophantine sets are Diophantine. For the
special case that \(R\) is computable, one can thusly deduce that Hilbert's
tenth problem is essentially the same as the primitive positive diagram
\(D_{∃+}(\mathfrak{R})\).

\begin{cor}
  Let \(R\) be a computable integral domain and \(\mathfrak{R}\) its
  \(\lang_{ring}\)-structure. Then \(D_{∃+}(\mathfrak{R})\) is many-one
  reducible to \(\mathtt{H10}(\mathfrak{R})\).
\end{cor}
\begin{proof}
  This follows immediately from the lemma above and the properties of the
  Gödelization.
\end{proof}

Of course one is tempted to consider Hilbert's tenth problem over the complex
pane \(ℂ\). There is however a technicality in our way, as \(ℂ\) is uncountable.
Thus, \(ℂ[X_1, X_2, …]\) is uncountable as well and one cannot encode all
polynomials. For this reason it is sometimes preferable to consider \emph{purely
Diophantine sets}.

\subsection{Purely Diophantine sets}
% ██████  ██    ██ ██████  ███████     ██████  ██  ██████  ██████  ██   ██
% ██   ██ ██    ██ ██   ██ ██          ██   ██ ██ ██    ██ ██   ██ ██   ██
% ██████  ██    ██ ██████  █████       ██   ██ ██ ██    ██ ██████  ███████
% ██      ██    ██ ██   ██ ██          ██   ██ ██ ██    ██ ██      ██   ██
% ██       ██████  ██   ██ ███████     ██████  ██  ██████  ██      ██   ██ ██

\begin{defin}
    Let \(R\) be a commutative ring with unit. A set \(S \subseteq R^n\) is said to
    be \emph{purely Diophantine} over \(R\) if there exists a polynomial \(p ∈
    ℤ[X_1,…,X_n, Y_1,…,Y_m]\) in \(n + m\) many indeterminates (\(m,n ≥ 0\)) such
    that
    \[
      (α_1,…,α_n) ∈ S \quad ⇔ \quad
      ∃ β_1,…,β_m ∈ R: p(α_1,…,α_n,β_1,…,β_m) = 0
    \]
\end{defin}

By demanding that the coefficients are rational integers, we immediately obtain
that there can only be countably many purely Diophantine sets over a fixed ring
with arbitrary cardinality. Whilst the choice of coefficients may seem random to
the algebraist, it is perfectly natural from the perspective of model theory, as
is shown in the following lemma.

\begin{lem}
  Let \(R\) be a commutative ring with unity and let \(\mathfrak{R}\) be its
  \(\lang_{ring}\)-structure. Then \(S ⊂ R^n\) is purely Diophantine if and only
  if there exists an atomic \(\lang_{ring}\)-formula \(ϕ(\mathtt{\seq{x},
  \seq[m]{y}})\) such that
  \[
    (\seq{α}) ∈ S \quad ⇔ \quad
    \mathfrak{R} \models \mathtt{∃ y_1 : … ∃ y_m: }
        ϕ(\seq{α}, \mathtt{\seq[m]{y}})
  \]
  holds.
\end{lem}
\begin{proof}
  Follows from \cref{lem:terms of rings are polynomials} and the analogue of
  part (ii) of \cref{thm:Diophantine theory}.
\end{proof}

At second sight, the construction is even less surprising, as for every ring
\(R\) with \(1\) there exists exists exactly one ring-homomorphism \(φ: ℤ → R\)
mapping \(1 ∈ ℤ\) to \(1 ∈ R\). Looking back at \cref{ex:Diophantine sets}, we
note that the Diophantine sets of (3), (4), and (5) are in fact purely
Diophantine, whilst finite sets (1) are in general not. As for polynomial
functions \(p: R^n → R\), we obtain that they are purely Diophantine if and only
if the coefficients of \(p\) are rational integers. Note however that a
partially evaluated polynomial with rational integral coefficients need not be a
polynomial in \(ℤ[X_1, X_2, …]\). Thus, one needs to be a bit more careful when
dealing with purely Diophantine sets. However, the analogue of
\cref{lem:intersections and unions} holds for purely Diophantine sets.

\begin{lem}
  Let \(S_1, S_2 ⊂ \algint^n\) be purely Diophantine subsets of a ring of
  algebraic integers of some number field \(K\). Then their union and
  intersection are purely Diophantine. The defining equations for union and
  intersection can be obtained effectively.
\end{lem}
\begin{proof}
  I claim that there exists a polynomial \(h ∈ ℤ[X]\) without roots in \(K\).
  Then we can use the same construction as in \cref{lem:intersections and
  unions} to prove the lemma.

  Such a polynomial \(h ∈ ℤ[X]\) must exist in every number field \(K\), as
  otherwise the normal closure of \(K\) contains all algebraic integers and thus
  is the algebraic closure of \(\overline{ℚ}\) by \cref{thm:primitive element}.
  But then the degree \([\overline{ℚ} : K]\) is finite, implying that \([K :
  ℚ]\) is infinite, which is a contradiction.
\end{proof}

We now want to identify the purely Diophantine subsets within the Diophantine
subsets of algebraic integers. For this purpose we reformulate a result of
\textcite{Robinson1951}. But before we state his result let us look at a simple
example: In \(ℚ{[√[4]{2}]}\) the polynomial \(p(X) := X^2 - 2\) does
not give rise to a purely Diophantine representation of \(√2\) because \(-√2\)
is a root of \(p\) as well. We can however represent \(√2\) as follows:
\[
  α = √2 \quad ⇔ \quad
  ∃ β ∈ ℚ{[√[4]{2}]} : (β^4 = 2 ∧ β^2 = α).
\]
This is because \(ℚ{[√[4]{2}]} ⊂ ℝ\) is real and the square of a real number is
non-negative. In general, we have the following proposition.

\begin{pro}\label{pro:Diophantine singletons}
  Let \(K\) be an algebraic number field. If \(x ∈ K\) is fixed by all
  automorphisms of \(K\), then there exist polynomials \(p,q ∈ ℤ[X]\) and a
  constant \(c ∈ ℤ\) such that \(x\) is the only element of \(K\) satisfying
  \[
    ∃ y ∈ \algint : (p(y) = 0 ∧ q(y) = cx).
  \]
  If \(x\) is an algebraic integer, then \(\set{x}\) is purely Diophantine over
  \(\algint\).
\end{pro}
\begin{proof}
  By the primitive element theorem~\ref{thm:primitive element} there exists an
  algebraic integer \(δ ∈ \algint\) such that \(K = ℚ[δ]\). Let \(μ_{ℚ, δ} ∈
  ℤ[X]\) be the minimal polynomial of \(δ\) over the rationals \(ℚ\) and let
  \(δ = \seq[k]{δ} ∈ \algint\) be the roots of \(μ_{ℚ, δ}\) that are contained
  in \(K\). Since every \(z ∈ K\) can be written as \(z = f(δ)\), where \(f(X) ∈
  ℚ[X]\) and the rationals are fixed by all automorphisms \(σ\) of \(K\), we
  find that \(σ(z) = f(σ(δ))\) holds for all automorphisms. Thus, \(\id_K =
  \seq[k]{σ}\), where \(σ_i(δ) = δ_i\), are all automorphisms of \(K\).

  As \(x\) is fixed by all of the \(σ_i\), we find that
  \[
    f(δ) = x = σ_i(x) = σ_i(f(δ)) = f(σ_i(δ)) = f(δ_i)
  \]
  holds for all \(1 ≤ i ≤ k\). Now since \(μ_{ℚ, δ}\) defines
  \(\set{\seq[k]{δ}}\) in a Diophantine way, we obtain that
  \[
    α = x \quad ⇔ \quad ∃ y ∈ \algint : (μ_{ℚ, δ}(y) = 0 ∧ f(y) = x).
  \]
  To finish the prove set \(c\) to be the least common multiple of all
  denominators of coefficients in \(f\) and multiply the right equation with
  \(c\). Since \(\seq[k]{δ} ∈ \algint\) are the only roots of \(μ_{ℚ, δ}\), the
  singleton \(\set{x}\) is in fact purely Diophantine over \(\algint\) as
  claimed.
\end{proof}

Note that the assumption of \(x ∈ K\) being fixed by all automorphisms is
necessary. Indeed, if \(p(X, Y) ∈ ℤ[X, Y]\) is a polynomial with rational
integral coefficients such that there exists a \(y ∈ K\) with \(p(x, y) = 0\),
then for every automorphism \(σ\), we find that
\[
  p(σ(x), σ(y)) = σ(p(x, y)) = 0,
\]
and thus, \(σ(x)\) satisfies the same relation. Building on this result for
singletons, \textcite{Davis1976} gave the following characterization of purely
Diophantine sets within Diophantine sets over rings of algebraic integers.

\begin{thm}\label{thm:purely Diophantine sets}
  Let \(\algint\) be the ring of algebraic integers of a number field \(K\). A
  set \(S ⊂ \algint^n\) is purely Diophantine if and only if \(S\) is
  Diophantine and self-conjugate, i.e.\ for all automorphisms \(σ: K → K\)
  and all \((\seq{α}) ∈ S\) we have that the image \((σ(α_1), …, σ(α_n))\) is
  contained in \(S\).
\end{thm}
\begin{proof}
  We may assume that \(S\) is Diophantine, as being purely Diophantine clearly
  implies the former. Thus, let \(p ∈ \algint{}[\seq{X}, \seq[m]{Y}]\) be a
  polynomial witnessing that \(S\) is Diophantine i.e.\ we have that
  \[
    (\seq{α}) ∈ S \quad ⇔ \quad
    ∃ β_1, …, β_m ∈ \algint : p(\seq{α}, \seq[m]{β}) = 0.
  \]
  To simplify notation I will assume that \(n = m = 1\) and thus that \(p(X,
  Y)\) is bivariat. The general case follows completely analogously.

  To see the first direction, we assume that \(p\) has in fact rational integral
  coefficients and let \(α\) be in \(S\). Then there exists an integer \(β ∈
  \algint\) such that \(p(α, β) = 0\). Let now \(\seq[k]{σ}\) be all
  automorphisms of \(K\). Since each \(σ_i\) preserves \(ℤ\) point-wise (\(1 ≤
  i ≤ k\)), we know that
  \[
    p(σ_i(α), σ_i(β)) = σ_i(p(α, β)) = 0
  \]
  and \(σ_i(α) ∈ S\) as claimed.

  Conversely, let \(S\) be self-conjugate and let \(p_i\) for \(1 ≤ i ≤ k\)
  denote the polynomial obtained from \(p\) by replacing the coefficients of
  \(p\) by their images under \(σ_i\). We define
  \[
    q(X, Y) := \prod_{i = 1}^k p_i(X, Y)
  \]
  and note that the coefficients of \(q\) are preserved by all automorphisms
  \(σ_i\). As a consequence of \cref{pro:Diophantine singletons}
  we can find for all coefficients \(a ∈ \algint\) of \(q\), polynomials \(P_a,
  Q_a ∈ ℤ[Y]\) and a constant \(c_a ∈ ℤ\) such that
  \[
    α = a \quad ⇔ \quad ∃ β ∈ \algint : (P_a(β) = 0 ∧ Q_a(β) = c_aα).
  \]
  Therefore, the relation defined by \(q\) can be rewritten in a purely
  Diophantine form. To see this we assume that \(J ⊂ ℕ^2\) is finite and
  \[
    q(X, Y) = \sum_{(i,j) ∈ J} a_{ij} X^i Y^j.
  \]
  Then, we have the following equivalence for all \(α ∈ \algint\)
  \begin{align*}
    ∃ β ∈ \algint: q(α, β) ⇔ &\\
    ∃ β, (β_{ij})_{(i,j) ∈ J} ∈ \algint :
        & \sum_{(i,j) ∈ J} β_{ij} α^i β^j = 0 ∧\\
        & \bigwedge_{(i,j) ∈ J} ∃ γ_{ij} ∈ \algint :
          (P_{a_{ij}}(γ_{ij}) = 0 ∧ Q_{a_{ij}}(γ_{ij}) = β_{ij}).
  \end{align*}

  All that is left to prove is that \(q\) and \(p\) represent \(S\) i.e.\ that
  \[
    ∃ β ∈ \algint : p(α, β) \quad ⇔ \quad ∃ β ∈ \algint : q(α, β)
  \]
  holds for all algebraic integers \(α ∈ \algint\). To see this first assume
  that \(p(α, β) = 0\) holds. Then since the identity is an automorphism of
  \(K\), we find that one factor of
  \[
    q(α, β) := \prod_{i = 1}^k p_i(α, β)
  \]
  is zero, and thus that \(q(α, β) = 0\). If on the other hand \(q(α, β) = 0\)
  then one of the factors of \(q\) must be zero. Say \(p_i(α, β) = 0\) and let
  \(σ_j\) be the inverse of \(σ_i\). Then we find that
  \[
    0 = σ_j(p_i(α, β)) = p(σ_j(α), σ_j(β))
  \]
  and therefore \(σ_j(α) ∈ S\). Now since \(S\) is self-conjugate by assumption
  we can deduce \(α = σ_i(σ_j(α)) ∈ S\) as claimed.
\end{proof}

The following diagram summarizes all the reducibility relations between the
theories of rings of algebraic integers that we have proven so far. By the
transitivity of many-one reducibility (cf.~\cref{pro:m reducibility and
decidability}) all that is left to prove is many-one reducibility of the
halting set \(\mathcal{K}\) to the purely Diophantine theory
\(\mathtt{H10}^*(\struc{O}_K)\). Then the diagram collapses as depicted at the
end of \cref{sec:theories}.
\begin{center}
  \includegraphics{res/theories_3}
\end{center}

\subsection{Related problems}
% ██████  ███████ ██             ██████  ██████   ██████  ██████  ███████
% ██   ██ ██      ██             ██   ██ ██   ██ ██    ██ ██   ██ ██
% ██████  █████   ██             ██████  ██████  ██    ██ ██████  ███████
% ██   ██ ██      ██             ██      ██   ██ ██    ██ ██   ██      ██
% ██   ██ ███████ ███████ ██     ██      ██   ██  ██████  ██████  ███████ ██

The beauty of the model theoretic approach to Hilbert's tenth problem is that it
directly gives rise to various generalizations. To conclude this section I will
list some results on variants of the problem.

In \citeyear{Rosser1936} \textcite{Rosser1936} proved---extending a result of
\textcite{Goedel1931}---that the full first order theory
\(\mathtt{Th}(\mathfrak{N})\) of the natural numbers is undecidable.%
\footnote{At this point I should mention that I assume throughout this thesis
that the Peano arithmetic is consistent. See e.g.\ Chap.~8 of the
textbook~\cite{Cooper2004} for a more rigorous discussion of Gödel's results.}
As a consequence, the full first order theory of \(ℤ\) is undecidable, as one
can translate a sentence in \(ℕ\) to an equivalent sentence in \(ℤ\) via the
construction described in \cref{ex:N is Diophantine over Z}. Considering the
full first order theory of \(\algint\), \textcite{Robinson1959} proved as early
as \citeyear{Robinson1959} that \(\mathtt{Th}(\modalgint)\) is undecidable. In
\citeyear{Matijasevic1970} \textcite{Matijasevic1970} showed---building on the
work of Davis, Putnam, and J.~Robinson---the undecidability of Hilbert's tenth
problem over \(ℤ\). More specifically, he provided the last piece in the proof
of the Davis-Putnam-Robinson-Matijasevič theorem (\textsc{DPRM}).

\begin{thm}[DPRM-theorem]\label{thm:DPRM}
  A subset of the natural numbers is semi-decidable if and only if it is
  Diophantine over \(ℕ\).
\end{thm}

This result is remarkably similar to the key theorem~(7) of Gödel's
proof~\cite{Goedel1931} of his celebrated first incompleteness theorem. Quite in
his spirit [cf.~Thm.~9 ibid.] we can deduce

\begin{cor}\label{cor:H10 over Z}
  The halting set \(\mathcal{K}\) is many-one reducible to
  \(\mathtt{H10}(\mathfrak{Z})\). Thus, Hilbert's tenth problem over \(ℤ\) is
  undecidable.
\end{cor}
\begin{proof}
  Throughout this proof I will identify \(ℤ\) with the domain of one of its
  computable representations. Then \(ℕ ⊂ ℤ\) is decidable. Indeed, for a given
  integer \(n ∈ ℤ\) we know that
  \begin{align*}
    n ∈ ℕ \quad &⇔ \quad
    ∃ x_1, x_2, x_3, x_4 ∈ ℤ : n = x_1^2 + x_2^2 + x_3^2 + x_4^2\\
  \intertext{and}
    n \not∈ ℕ \quad &⇔ \quad
    ∃ x_1, x_2, x_3, x_4 ∈ ℤ : n =
      - \left(x_1^2 + x_2^2 + x_3^2 + x_4^2 + 1\right)
  \end{align*}
  hold by \cref{pro:Lagranges four square theorem} and \(ℕ\) is decidable by
  \cref{pro:Posts theorem}. As a consequence, we have found a computable
  representation of the \(\lang_{ring}\)-structure of the non-negative integers
  and by \cref{ex:N is computably categorical} there exists a computable
  bijection \(f: ω → ℕ\) with an computable inverse.

  Now consider \(f(\mathcal{K}) ⊂ ℤ\). Then the inverse mapping \(f^{-1}\)
  witnesses that \(f(\mathcal{K})\) is many-one reducible to \(\mathcal{K}\).
  Thus, \(f(\mathcal{K})\) is semi-decidable and by the \textsc{DPRM}-theorem
  \(f(\mathcal{K})\) is Diophantine over \(ℕ\) and therefore Diophantine over
  \(ℤ\) as well. Hence, there exists a polynomial \(p_{\mathcal{K}} ∈ ℤ[X, Y_1,
  …, Y_m]\) with the property that
  \[
    α ∈ f(\mathcal{K}) \quad ⇔ \quad
    ∃ \seq[m]{β} ∈ ℤ : p_{\mathcal{K}}(α, \seq[m]{β}) = 0.
  \]
  Finally, we find for all \(x ∈ ω\) that the \(\lang_ℤ\)-sentence
  \begin{equation}\label{eq:representing halting set}
    \mathtt{∃ y_1 : … ∃ y_m :} p_{\mathcal{K}}(f(x), \mathtt{\seq[m]{y}})
      \doteq 0
  \end{equation}
  is contained in \(\mathtt{H10}(\mathfrak{Z})\) precisely if \(x ∈
  \mathcal{K}\). Thus, the function mapping \(x ∈ ω\) to the Gödelization of
  \eqref{eq:representing halting set} witnesses that \(\mathcal{K}\) is
  many-one reducible to \(\mathtt{H10}(\mathfrak{Z})\).

  Now \cref{pro:m reducibility and decidability} implies that
  \(\mathtt{H10}(\mathfrak{Z})\) is undecidable, as \(\mathcal{K}\) is
  undecidable.
\end{proof}

In \citeyear{Rumely1986}, \textcite{Rumely1986} published his surprising result
that \textsc{H10} is solvable over \(\mathcal O\), the ring of all algebraic
integers. \Textcite{Dries1988} extended this result to the full first order
theory of \(\mathcal O\) in \citeyear{Dries1988}.

Probably the most prominent open problem in this area is the case of \(ℚ\). A
positive answer to \textsc{H10} over \(ℚ\) would imply that there is a universal
algorithm deciding whether a variety over \(ℚ\) has a rational point. By giving
a first order definition of \(ℤ\) over \(ℚ\), \textcite{Robinson1949} could
derive the undecidability of the full first order theory of \(ℚ\) from the
undecidability of the theory of \(ℤ\) in \citeyear{Robinson1949}. But her
definition involves universal quantifiers and cannot be used for inferring to
\textsc{H10}. \Textcite{Park2013} strengthend the results of
\textcite{Robinson1949,Robinson1959} by providing a universal first order
definition of \(\algint\) over an arbitrary number field \(K\) in
\citeyear{Park2013}. Again moving to larger rings proves to be easier.
\textcite{Tarski1931} showed in \citeyear{Tarski1931} using the method of
\emph{quantifier elimination} that the full \(\lang_{ring}\)-theory of the real
numbers is decidable. For complex numbers the tools for proving the analogous
result were already known in the nineteenth century.

The surveys~\cite{Koenigsmann2014,Poonen2008} offer a more extensive overview of
problems related to undecidability in number theory.
