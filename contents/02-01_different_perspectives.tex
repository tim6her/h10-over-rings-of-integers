% !TeX encoding = UTF-8
% !TeX TS-program = xelatex
% !TeX spellcheck = en_GB
% !TeX root = ../Herbstrith-H10_over_AI.tex

In 1900, David Hilbert held his famous lecture before the \emph{Second International Congress of Mathematicians} in Paris.
The lecture entitled \begin{german}\enquote{Mathematische Probleme}\end{german} contained 23 mathematical problems left for the 20th century to solve.

\todo{Write about the history of \textsc{H10}}

\begin{quotation}
    Given a Diophantine equation with any number of unknown quantities and with rational integral numerical coefficients: To devise a process according to which it can be determined by a finite number of operations whether the equation is solvable in rational integers.
\end{quotation}

A \emph{Diophantine equation} is of the form
%
\[ p(x_1, …, x_n) = 0 \]
%   
for some polynomial $p ∈ ℤ[X_1, …, X_n]$, allowing only solutions $x_1,…,x_n ∈ ℤ$.


\begin{defin}
    Let $R$ be a commutative ring with unit.
    A set $S \subseteq R^n$ is said to be \emph{Diophantine} if there exists a polynomial $p ∈ R[X_1,…,X_n, Y_1,…,Y_m]$ ($m ≥ 0$\footnote{If $m = 0$ we interpret this as $p ∈ R[X_1,…,X_n]$}) such that
    
    \[ (x_1,…,x_n) ∈ S \Leftrightarrow ∃y_1,…,y_m ∈ R: p(x_1,…,x_n,y_1,…,y_m) = 0 \]
\end{defin}

A polynomial $p$ as above defines a relation $\rel{p}$ on $R^n \times R^m$ by
%
\[ \mathbf{x} \rel{p} \mathbf{y} \Leftrightarrow
   p(\mathbf{x}, \mathbf{y}) = 0. \]
%
In this sense a set $S \subseteq R^n$ is Diophantine if there exists a polynomial $p$ such that for each $\mathbf x ∈ S$ there exists a $\mathbf y ∈ R^m$ that is related to $\mathbf x$ via $\rel p$.

Viewing $n$-ary relations as subsets of $R^n$, we will sometimes refer to Diophantine sets as \emph{Diophantine relations.}

\begin{exam}
    \begin{exlist}
        \item Let $R$ be an integral domain.
        Then every finite subset $S$ of $R$ is Diophantine, because the roots of 
        
        \[ p(X) := \prod_{s ∈ S} (X - s) \]
        
        are precisely the elements of $S$.
        
        \item The set of composite numbers is Diophantine over $ℕ$, as $x ∈ ℕ$ is composite if and only if
        
        \[ ∃ y_1, y_2 ∈ ℕ : x = (y_1 + 2) (y_2 + 2). \]
        
        Here adding $2$ to $y_1$ and $y_2$ ensures, that both factors are greater than $1$.
        Choosing
        
        \[ p (X, Y_1, Y_2) := X - (Y_1 + 2)(Y_2 + 2) \]
        
        yields the claim.
        
        \item The usual order relation $≤$ on $ℕ$ is Diophantine over $ℕ$.
        Indeed $x_1 ≤ x_2$ in $ℕ$ if and only if
        
        \[ ∃ y ∈ ℕ : x_1 + y  = x_2. \]
        
        \item Let $R$ be a commutative ring with unit.
        Then divisibility in $R$ is Diophantine.
        Indeed $x_1 | x_2$ in $R$ precisely if
       
        \[ ∃ y ∈ R : x_1 y = x_2. \]
        
    \end{exlist}
\end{exam}

\todo{Here should be an introduction and the subsequent table should be continued.}

\begin{table}
    \begin{tabular}{l l}
        \toprule
        Number Theory          & Model Theory \\
        \midrule
        polynomial         & relation          \\
        root of polynomial & diophantine set   \\
        etc.               &                   \\
        \bottomrule
    \end{tabular}
    \caption{Comparison of central concepts in number and model theory}
\end{table}