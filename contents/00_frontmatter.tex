% !TeX encoding = UTF-8
% !TeX TS-program = xelatex
% !TeX spellcheck = en_GB
% !TeX root = ../Herbstrith-H10_over_AI.tex

\begin{titlepage}
%\vspace*{-2cm}  % bei Verwendung von vmargin.sty
\begin{flushright}
    \includegraphics{uni-logo}
\end{flushright}
\vspace{0.5cm}

\begin{center}  % Diplomarbeit ODER Magisterarbeit ODER Dissertation
    \Huge{\textsf{\textbf{\MakeUppercase{
        Masterarbeit
    }}}}
    \vspace{1.5cm}

    \large{\textsf{  % Diplomarbeit ODER Magisterarbeit ODER Dissertation
                     % (Dies ist erst die Ueberschrift!)
        Titel der Masterarbeit
    }}
    \vspace{.1cm}

    \LARGE{\textsf{ On Hilbert's Tenth Problem over\\
                    Rings of Algebraic Integers
    }}
    \vfill

    \large{\textsf{  % Verfasserin ODER Verfasser (Ueberschrift)
        Verfasser
    }}

    \Large{\textsf{  Tim Benedikt Herbstrith
    }}
    \vfill

    \large{\textsf{
        angestrebter akademischer Grad  % (Ueberschrift)
    }}

    \Large{\textsf{  % Magistra ODER Magister ODER Doktorin ODER Doktor
                     % ACHTUNG: Kuerzel "Mag.a" oder "Dr.in" nicht zulaessig
        Master of Science (MSc.)
    }}

\vspace{1.5cm}

\noindent\textsf{Wien, im Monat Mai 2018}  % <<<<< ORT, MONAT UND JAHR EINTRAGEN
\vfill

\noindent\begin{tabular}{@{}ll}
\textsf{Studienkennzahl lt.\ Studienblatt:}
&
\textsf{A 033 821}  % <<<<< STUDIENKENNZAHL EINTRAGEN
\\
    % BEI DISSERTATIONEN:
%\textsf{Dissertationsgebiet lt. Studienblatt:}
    % ANSONSTEN:
\textsf{Studienrichtung lt.\ Studienblatt:}
&
\textsf{Mathematik}  % <<<<< DISSGEBIET/STUDIENRICHTUNG EINTRAGEN
\\
% Betreuerin ODER Betreuer:
\textsf{Betreuer: }
&
\textsf{ao.~Univ.-Prof.~Mag.~Dr.~Ch.~Baxa}  % <<<<< NAME EINTRAGEN
\end{tabular}
\end{center}
\end{titlepage}

\cleardoublepage


\begin{german}
\section*{Vorwort}
\todo{Schreibe das Vorwort}
\end{german}

\vspace{1.5cm}

\section*{Abstract}
\todo{Write the abstract}
\newpage
\thispagestyle{empty}
\tableofcontents
