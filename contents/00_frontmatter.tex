% !TEX encoding = UTF-8
% !TEX TS-program = xelatex
% !TEX spellcheck = en_GB
% !TEX engine = xelatex
% !TEX root = ../Herbstrith-H10_over_AI.tex

% ######## ########   #######  ##    ## ########
% ##       ##     ## ##     ## ###   ##    ##
% ##       ##     ## ##     ## ####  ##    ##
% ######   ########  ##     ## ## ## ##    ##
% ##       ##   ##   ##     ## ##  ####    ##
% ##       ##    ##  ##     ## ##   ###    ##
% ##       ##     ##  #######  ##    ##    ##

\begin{titlepage}
%\vspace*{-2cm}  % bei Verwendung von vmargin.sty
\begin{flushright}
    \includegraphics{res/uni-logo}
\end{flushright}
\vspace{0.5cm}

\begin{center}  % Diplomarbeit ODER Magisterarbeit ODER Dissertation
    \Huge{\textsf{\textbf{\MakeUppercase{
        Masterarbeit
    }}}}
    \vspace{1.5cm}

    \large{\textsf{  % Diplomarbeit ODER Magisterarbeit ODER Dissertation
                     % (Dies ist erst die Ueberschrift!)
        Titel der Masterarbeit
    }}
    \vspace{.1cm}

    \LARGE{\textsf{ On Hilbert's Tenth Problem over\\
                    Rings of Algebraic Integers
    }}
    \vfill

    \large{\textsf{  % Verfasserin ODER Verfasser (Ueberschrift)
        Verfasser
    }}

    \Large{\textsf{  Tim Benedikt Herbstrith
    }}
    \vfill

    \large{\textsf{
        angestrebter akademischer Grad  % (Ueberschrift)
    }}

    \Large{\textsf{  % Magistra ODER Magister ODER Doktorin ODER Doktor
                     % ACHTUNG: Kuerzel "Mag.a" oder "Dr.in" nicht zulaessig
        Master of Science (MSc.)
    }}

\vspace{1.5cm}

\noindent\textsf{Wien, im Monat September 2018}
  % <<<<< ORT, MONAT UND JAHR EINTRAGEN
\vfill

\noindent\begin{tabular}{@{}ll}
\textsf{Studienkennzahl lt.\ Studienblatt:}
&
\textsf{A 033 821}  % <<<<< STUDIENKENNZAHL EINTRAGEN
\\
    % BEI DISSERTATIONEN:
%\textsf{Dissertationsgebiet lt. Studienblatt:}
    % ANSONSTEN:
\textsf{Studienrichtung lt.\ Studienblatt:}
&
\textsf{Mathematik}  % <<<<< DISSGEBIET/STUDIENRICHTUNG EINTRAGEN
\\
% Betreuerin ODER Betreuer:
\textsf{Betreuer: }
&
\textsf{ao.~Univ.-Prof.~Mag.~Dr.~Ch.~Baxa}  % <<<<< NAME EINTRAGEN
\end{tabular}
\end{center}
\end{titlepage}

\newpage%
\thispagestyle{empty}%
\vspace*{\fill}%

\begin{footnotesize}%
\noindent%
© Tim B.\ Herbstrith, 2019: 
If not stated otherwise, \emph{On Hilbert's Tenth Problem over Rings
of Algebraic Integers} by Tim B. Herbstrith is licenced under a \textsc{Creative
Commons Attribution-NonCommercial-ShareAlike 4.0 International License}. All
code snippets are provided `as is' without warranty of any kind, express or
implied, including but not limited to the warranties of merchantability, fitness
for a particular purpose and non-infringement under the terms of the
\textsc{MIT} licence. The source codes and the aforementioned licences are
available at
\begin{center}
 \url{https://github.com/tim6her/h10-over-rings-of-integers}
\end{center}
\hspace{\fill}
\includegraphics[height=4ex]{res/Cc-by-nc-sa_euro_icon}
\includegraphics[height=4ex]{res/License_icon-mit-88x31-2-2}
\end{footnotesize}
\cleardoublepage


\begin{german}
\section*{Abriss}

Hilberts zehntes Problem fragt ob ein Algorithmus existiert, der zu gegebenen
multivarianten Polynom mit ganzzahligen Koeffizienten entscheiden kann, ob
dieses ganzzahlige Nullstellen besitzt. Obwohl das Problem breits im Jahr 1900
von \textcite{Hilbert1900} formuliert wurde, dauerte es bis 1970 bis
\textcite{Matijasevic1970} beweisen konnte, dass es kein solchen Algorithmus
geben kann. Das Problem lässt sich direkt auf andere kommutative Ringe \(R\) mit
\(1\) übertragen, indem Koeffizienten aus \(ℤ\) oder \(R\) und Nullstellen aus
\(R\) gewählt werden. In dieser Masterarbeit werden wir uns vor allem mit dem
Fall von Ringen ganzalgebraischer Zahlen beschäftigen. Wie eng Hilberts zehntes
Problem mit anderen Entscheidungsproblemen verwandt ist, kommt allerdings erst
dann zu Tage, wenn wir Hilberts Problem als die Frage der Entscheidbarkeit einer
Theorie auffassen. Wir werden zum Beispiel erkennen, dass Matijasevic'
\textsc{DPRM}-Theorem~(\ref{thm:DPRM}) sehr ähnlich zu Gödels Haupttheorem in
seinem Beweis \cite{Goedel1931} des ersten Unvollständigkeitssatzes ist.

Um Hilberts Problem in dieser Allgemeinheit verstehen zu können, werden im
ersten Abschnitt Grundlagen der Berechenbarkeitstheorie und der Modelltheorie
wiederholt. Dabei werden wir auf das Halteproblem stoßen, dessen
Unentscheidbarkeit die Schlüsselzutat für alle unsere Beweise der
Unentscheidbarkeit sein wird. Weiters werden wir die für uns relevanten Begriffe
und Resultate der algebraischen Zahlentheorie sowie der Geometrie der Zahlen
einführen und teilweise beweisen.

Nach diesen einführenden Kapiteln werden wir im zweiten Teil der Arbeit Hilberts
zehntes Problem formalisieren und eine ausführlichere Betrachtung verwandter
Probleme und der historischen Entwicklung dieser anstellen. Um das Problem
schließlich negativ für ausgewählte Ringe zu entscheiden, werden wir
diophantische Mengen über kommutativen Ringen \(R\) mit \(1\) einführen. Das
Hauptresultat dieser Arbeit wird sein, dass für einen Ring ganzalgebraischer
Zahlen \(\algint\), über dem die ganzen Zahlen \(ℤ\) eine diophantische Menge
bilden, unabhängig davon, ob Koeffizienten aus \(ℤ\) oder \(\algint\) gewählt
werden, das zehnte hilbertsche Problem unentscheidbar ist.

Im letzten Abschnitt der Arbeit werden die Resultate von \textcite{Denef1980}
sowie \textcite{Pheidas1988} bzw. \textcite{Shapiro1989} präsentiert. Diese
konnten im konkreten Fall von total-reellen algebraischen Zahlkörpern \(K ≠ ℚ\)
und algebraischen Zahlkörpern \(K\) mit mindestens einer reellen und genau einem
Paar komplexer Einbettungen zeigen, dass \(ℤ\) über \(\algint\) eine
diophantische Menge ist.
\end{german}

\vspace{1.5cm}

\section*{Abstract}
\todo{Write the abstract}
\newpage
\thispagestyle{empty}
\tableofcontents
