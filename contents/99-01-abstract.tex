% !TEX encoding = UTF-8
% !TEX TS-program = xelatex
% !TEX spellcheck = en_GB
% !TEX engine = xelatex
% !TEX root = ../Herbstrith-H10_over_AI.tex

% ███████ ██    ██ ███    ███ ███    ███  █████  ██████  ██    ██
% ██      ██    ██ ████  ████ ████  ████ ██   ██ ██   ██  ██  ██
% ███████ ██    ██ ██ ████ ██ ██ ████ ██ ███████ ██████    ████
%      ██ ██    ██ ██  ██  ██ ██  ██  ██ ██   ██ ██   ██    ██
% ███████  ██████  ██      ██ ██      ██ ██   ██ ██   ██    ██

\begin{german}
\section{Zusammenfassung}

Hilberts zehntes Problem fragt, ob ein Algorithmus existiert, der zu gegebenen
multivariaten Polynom mit ganzzahligen Koeffizienten entscheiden kann, ob
dieses ganzzahlige Nullstellen besitzt. Obwohl das Problem breits im Jahr 1900
von \textcite{Hilbert1900} formuliert wurde, dauerte es bis 1970, bis
\textcite{Matijasevic1970} beweisen konnte, dass es keinen solchen Algorithmus
geben kann. Das Problem lässt sich direkt auf andere kommutative Ringe \(R\) mit
\(1\) übertragen, indem Koeffizienten aus \(ℤ\) oder \(R\) und Nullstellen aus
\(R\) gewählt werden. In dieser Masterarbeit werden wir uns vor allem mit dem
Fall von Ringen ganzalgebraischer Zahlen beschäftigen. Wie eng Hilberts zehntes
Problem mit anderen Entscheidungsproblemen verwandt ist, kommt allerdings erst
dann zu Tage, wenn wir Hilberts Problem als die Frage der Entscheidbarkeit einer
Theorie auffassen. Wir werden zum Beispiel erkennen, dass Matijasevic'
\textsc{DPRM}-Theorem~(\ref{thm:DPRM}) sehr ähnlich zu Gödels Haupttheorem in
seinem Beweis \cite{Goedel1931} des ersten Unvollständigkeitssatzes ist.

Um Hilberts Problem in dieser Allgemeinheit verstehen zu können, werden im
ersten Abschnitt Grundlagen der Berechenbarkeitstheorie und der Modelltheorie
vorgestellt. Dabei werden wir auf das Halteproblem stoßen, dessen
Unentscheidbarkeit die Schlüsselzutat für alle unsere Beweise der
Unentscheidbarkeit sein wird. Weiters werden wir die für uns relevanten Begriffe
und Resultate der algebraischen Zahlentheorie sowie der Geometrie der Zahlen
wiederholen und teilweise beweisen.

Nach diesen einführenden Kapiteln werden wir im zweiten Teil der Arbeit Hilberts
zehntes Problem formalisieren und eine ausführlichere Betrachtung verwandter
Probleme und der historischen Entwicklung dieser anstellen. Um das Problem
schließlich negativ für ausgewählte Ringe zu entscheiden, werden wir
diophantische Mengen über kommutativen Ringen mit \(1\) einführen und
einige wichtige strukturelle Eigenschaften diophantischer Mengen beweisen. Das
Hauptresultat dieser Arbeit ist, dass über einem Ring ganzalgebraischer
Zahlen \(\algint\), über dem die ganzen Zahlen \(ℤ\) eine diophantische Menge
bilden, unabhängig davon, ob Koeffizienten aus \(ℤ\) oder \(\algint\) gewählt
werden, das zehnte hilbertsche Problem unentscheidbar ist.

Im letzten Abschnitt der Arbeit werden die Resultate von \textcite{Denef1980},
\textcite{Pheidas1988} und \textcite{Shlapentokh1989} präsentiert. Diese
konnten im Fall von total-reellen algebraischen Zahlkörpern \(K ≠ ℚ\)
und algebraischen Zahlkörpern \(K\) mit mindestens einer reellen und genau einem
Paar komplexer Einbettungen zeigen, dass \(ℤ\) über \(\algint\) eine
diophantische Menge ist. Damit ist Hilberts zehntes Problem über \(\algint\) in
diesen Fällen unentscheidbar. Für allgemeine Zahlkörper steht noch nicht fest,
ob Hilberts Problem entscheidbar ist. Die Vermutung von \textcite{Denef1978},
dass für alle Zahlkörper \(K\) Hilberts zehntes Problem über \(\algint\)
unentscheidbar ist, ist noch unbewiesen.
\end{german}
\clearpage

\section{Summary}

Hilbert's tenth problem asks, whether there exists an algorithm that can decide
for a given multivariate polynomial \(p\) with integral coefficients, if \(p\)
has integral roots. Even though the problem was already posed in 1900 by
\textcite{Hilbert1900}, it took until 1970 til \textcite{Matijasevic1970} could
prove that such an algorithm cannot exist. The problem can be translated
directly to other commutative rings \(R\) with \(1\) by letting the coefficients
range over \(ℤ\) or \(R\) and consider roots in \(R\). In this thesis we put
special interest on the case of \(R\) being a ring of algebraic integers. How
closely Hilbert's tenth problem is related to other decision problems, will
however only become apparent when we consider the problem as a question of
decidability of a theory. For instance, we will see that Matijasevič'
\textsc{DPRM}-theorem~(\ref{thm:DPRM}) is very similar to Gödel's central
theorem in his proof~\cite{Goedel1931} of the first incompleteness theorem.

To understand Hilbert's problem in this general setting, we introduce the basics
of computability theory and model theory in the first part of this thesis.
During these introductory sections we will present the halting problem. The
undecidability of this fundamental problem will be the key ingredient in every
proof of undecidability we will encounter. Furthermore, we will remind the
reader of the relevant results of algebraic number theory and geometry of
numbers.

In a second step we will formalize Hilbert's tenth problem and will extensively
study related problems and their historical developments. In order to eventually
answer the problem to the negative for selected rings, we will define
Diophantine sets over commutative rings with \(1\) and prove some of their
important structural properties. The main result of this thesis is, that
Hilbert's tenth problem is unsolvable over a ring of algebraic integers
\(\algint\) if \(ℤ\) is Diophantine over \(\algint\). This statement remains
true whether we allow the polynomials to have coefficients in \(ℤ\) or
\(\algint\).

In the final section of this thesis we will present the results of
\textcite{Denef1980}, \textcite{Pheidas1988}, and \textcite{Shlapentokh1989}.
They where able to prove in the case of totally real number fields \(K ≠ ℚ\) and
number fields of degree at least \(3\) over \(ℚ\) with exactly one pair of
non-real embeddings, that \(ℤ\) is Diophantine over \(\algint\). As a
consequence, Hilbert's tenth problem is undecidable over \(\algint\). For
general number fields it is not known whether Hilbert's tenth problem is
decidable over their ring of algebraic integers. The conjecture by
\textcite{Denef1978}, that Hilbert's tenth problem is undecidable over
\(\algint\) for all algebraic number fields \(K\), is still unproven.
