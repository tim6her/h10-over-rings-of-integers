% !TEX encoding = UTF-8
% !TEX TS-program = xelatex
% !TEX spellcheck = en_GB
% !TEX engine = xelatex
% !TEX root = ../Herbstrith-H10_over_AI.tex

%  █████  ██████  ███████ ████████ ██████   █████   ██████ ████████
% ██   ██ ██   ██ ██         ██    ██   ██ ██   ██ ██         ██
% ███████ ██████  ███████    ██    ██████  ███████ ██         ██
% ██   ██ ██   ██      ██    ██    ██   ██ ██   ██ ██         ██
% ██   ██ ██████  ███████    ██    ██   ██ ██   ██  ██████    ██

\begin{german}
\section{Zusammenfassung}

Hilberts zehntes Problem fragt, ob ein Algorithmus existiert, der zu gegebenen
multivariaten Polynom mit ganzzahligen Koeffizienten entscheiden kann, ob
dieses ganzzahlige Nullstellen besitzt. Obwohl das Problem breits im Jahr 1900
von \textcite{Hilbert1900} formuliert wurde, dauerte es bis 1970, bis
\textcite{Matijasevic1970} beweisen konnte, dass es keinen solchen Algorithmus
geben kann. Das Problem lässt sich direkt auf andere kommutative Ringe \(R\) mit
\(1\) übertragen, indem Koeffizienten aus \(ℤ\) oder \(R\) und Nullstellen aus
\(R\) gewählt werden. In dieser Masterarbeit werden wir uns vor allem mit dem
Fall von Ringen ganzalgebraischer Zahlen beschäftigen. Wie eng Hilberts zehntes
Problem mit anderen Entscheidungsproblemen verwandt ist, kommt allerdings erst
dann zu Tage, wenn wir Hilberts Problem als die Frage der Entscheidbarkeit einer
Theorie auffassen. Wir werden zum Beispiel erkennen, dass Matijasevic'
\textsc{DPRM}-Theorem~(\ref{thm:DPRM}) sehr ähnlich zu Gödels Haupttheorem in
seinem Beweis \cite{Goedel1931} des ersten Unvollständigkeitssatzes ist.

Um Hilberts Problem in dieser Allgemeinheit verstehen zu können, werden im
ersten Abschnitt Grundlagen der Berechenbarkeitstheorie und der Modelltheorie
vorgestellt. Dabei werden wir auf das Halteproblem stoßen, dessen
Unentscheidbarkeit die Schlüsselzutat für alle unsere Beweise der
Unentscheidbarkeit sein wird. Weiters werden wir die für uns relevanten Begriffe
und Resultate der algebraischen Zahlentheorie sowie der Geometrie der Zahlen
wiederholen und teilweise beweisen.

Nach diesen einführenden Kapiteln werden wir im zweiten Teil der Arbeit Hilberts
zehntes Problem formalisieren und eine ausführlichere Betrachtung verwandter
Probleme und der historischen Entwicklung dieser anstellen. Um das Problem
schließlich negativ für ausgewählte Ringe zu entscheiden, werden wir
diophantische Mengen über kommutativen Ringen mit \(1\) einführen und
einige wichtige strukturelle Eigenschaften diphantischer Mengen beweisen. Das
Hauptresultat dieser Arbeit ist, dass über einem Ring ganzalgebraischer
Zahlen \(\algint\), über dem die ganzen Zahlen \(ℤ\) eine diophantische Menge
bilden, unabhängig davon, ob Koeffizienten aus \(ℤ\) oder \(\algint\) gewählt
werden, das zehnte hilbertsche Problem unentscheidbar ist.

Im letzten Abschnitt der Arbeit werden die Resultate von \textcite{Denef1980}
sowie \textcite{Pheidas1988} bzw. \textcite{Shlapentokh1989} präsentiert. Diese
konnten im konkreten Fall von total-reellen algebraischen Zahlkörpern \(K ≠ ℚ\)
und algebraischen Zahlkörpern \(K\) mit mindestens einer reellen und genau einem
Paar komplexer Einbettungen zeigen, dass \(ℤ\) über \(\algint\) eine
diophantische Menge ist. Damit ist Hilberts zehntes Problem über \(\algint\) in
diesen Fällen unentscheidbar. Für allgemeine Zahlkörper steht noch nicht fest,
ob Hilberts Problem entscheidbar ist. Die Vermutung von \textcite{Denef1978},
dass für alle Zahlkörper \(K\) Hilberts zehntes Problem über \(\algint\)
unentscheidbar ist, ist noch unbewiesen.
\end{german}

\section{Summary}
\todo{Write the abstract}