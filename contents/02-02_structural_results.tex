% !TeX encoding = UTF-8
% !TeX TS-program = xelatex
% !TeX spellcheck = en_GB
% !TeX root = ../Herbstrith-H10_over_AI.tex

\subsection{Computable structures} \label{sec:computable structures}

Up to this point the encoding of problems was treated as some kind of black-box.
This subsection takes a categorical view on computability and ensures us,
that---up to a sensible definition---encodings of the rings we concern ourselves
with do not matter. The interested reader may whish to read the excellent survey by \textcite{Stoltenberg1999} on this subject.

\begin{defin}
  Let $Σ = \set{f_1, f_2, …}$ be an at most countable signature.
  \begin{thmlist}
    \item An algebraic structure $\mathfrak A = ⟨A; f_1^{\mathfrak A},
    f_2^{\mathfrak A}, …⟩$, with $A \subseteq ω$, is called \emph{computable
    structure} if $A$ is decidable and all operations $f_i^{\mathfrak A}$ are
    computable.

    \item An algebraic structure $\mathfrak A = ⟨A; f_1^{\mathfrak A},
    f_2^{\mathfrak A}, …⟩$ is called \emph{efficiently presentable} if there exists a computable $Σ$-structure $⟨Ω_A; φ_1, φ_2, …⟩$ and a $Σ$-isomorphism $\enc{\cdot}: A → Ω_A$.

    \item A computable $Σ$-morphism between computable $Σ$-structures is called \emph{computable morphism}.
  \end{thmlist}
\end{defin}

\begin{rem}
  \begin{exlist}
    \item An efficient presentation of a ring $R$ is a ring-homomorphism $\enc{\cdot}: R → Ω_R$ of $R$, where $Ω_R \subseteq ω$ is decidable and all operations of $Ω_R$ are computable functions.
    \item \Textcite{Stoltenberg1999} use a slightly modified definition of computable algebraic rings. They consider \emph{effective enumberations} $α_R : Ω_R → R$, where $Ω_R \subseteq ω$ is a computable $Σ_{ring}^*$-structure in the sense of the definiton above and $α_R$ is a $Σ_{ring}^*$-epimorphism. Then the ring $R$ is called computable if the equivalence relation
    \[
      x_1 \cong_{α_R} y_1  ⇔ α_R(x_1) = α_R(x_2)
    \]
    on $Ω_R$ is decidable.

    This definition can have slight technical advantages but note that in this case $Ω_R$ need not be a ring in the sense of abstract algebra, a $Σ_{ring}^*$ structure in the sense of universal algebra suffices.
    By setting $\enc{η} = [α_R^{-1}(\set{η})] ∈ Ω_R / \cong_{α_R}$ for each $η ∈ R$ one obtains a ring-isomorphism $R → Ω_R / \cong_{α_R}$ that gives rise to an efficient presentation as defined in this thesis. So ring $R$ is computable in the sense of \textcite{Stoltenberg1999} if and only if it is efficiently presentable.
  \end{exlist}
\end{rem}

\begin{exam}
  \begin{exlist}
    \item In \cref{ex:tally encoding} we have encoded the non-negative integer
    $n$ by a string of $n$ consecutive $\one$-s. We have also already
    presented the algorithm deciding $\enc{ℕ} \subseteq ω$. under this encoding.
    Considering $ℕ$ as $Σ_{ring}$-structure, one finds that the tally
    encoding gives rise to a efficient presentation of $ℕ$.

    The constants $0$ and $1$ are trivially computable, by clearing the tape in
    the first case and writing a single $\one$ in the second case. Using the
    pairing function of \cref{ex:tally pairing} the binary operations $+$ and
    $\cdot$ are also easily seen to be computable. As for $+$ the algorithm
    takes the input
    \[
      \one … \one \zer \one … \one
    \]
    and replaces the $\zer$-symbol by an $\one$ and deletes the rightmost
    $\one$.

    \item In general $ℤ$ and $\algint$ viewed as $Σ_{ring}^*$ structures are efficiently presentable. For the integers this is proven by each programming language allowing arbitrarily large integer arithmetic---like \emph{Haskell} or \emph{Python}.

    To present algebraic integers one uses an integral basis say $ζ_1, …, ζ_n$. Then any integer $η$ can be encoded as an $n$-tuple of integers. Addition and subtraction are defined coordinate-wise. To implement the multiplication one stores the finite multiplication table of the basis elements
    \[
      \begin{array}{r | r r r r}
            & ζ_1   & ζ_2     & … & ζ_n     \\
        \hline
        ζ_1 & ζ_1^2 & ζ_1 ζ_2 & … & ζ_1 ζ_n \\
        ζ_2 &       & ζ_2^2   & … & ζ_2 ζ_n \\
        \vdots &    &   & \ddots  & \vdots  \\
        ζ_n &       &         &   & ζ_n^2
      \end{array}
    \]
    in memory and extends to all of $\algint$ linearly. A full implementation
    can be found in the Appendix.\todo{Write the appendix}

    \item \label{ex:polynomials are computable}
    If $R$ is a computable integral domain, then the polynomial algebras
    $R[X_1, …, X_n]$ in arbitrary many indeterminates ($n ∈ ℕ^*$) and $R[X_1,
    X_2, …]$ in countably many indeterminates are computable $R$-algebras.

    A possible implementation starts by implementing the monoid $⟨M; \cdot; X_i \mid i ∈ ℕ⟩$ and extends it to the $R$-algebra
    $R[X_1, X_2, …]$. Within $R[X_1, X_2, …]$ every subalgebra $R[X_1, …, X_n]$
    is decidable ($n ∈ ℕ^*$) and as a consequence is computable. See
    \cite{Stoltenberg1999} for a more detailed discussion and \cref{app:polynomials} for a
    sample implementation based on this idea.

    \item $⟨ℕ, ≤⟩$ is efficiently presentable using the tally encoding and $n ≤
    m$ if and only if $\max(n - m, 0) = 0$. So deciding $n ≤ m$ boils down to
    applying floor subtraction and checking whether the tape is empty. Both
    operations are clearly computable.
  \end{exlist}
\end{exam}

It is a natural question whether two efficient presentations of the same
structure are computably isomorphic i.e. if there exists a computable
isomorphism between them. We will see that the last example differs from the
others in this regard.

\begin{defin}
  Let $Σ = \set{f_1, f_2, …}$ be an at most countable signature. An
  algebraic structure $\mathfrak A = ⟨A; f_1^{\mathfrak A}, f_2^{\mathfrak A},
  …⟩$ is called \emph{computably categorical} if it is efficiently
  representable and every pair of efficient representations is computably
  isomorphic.
\end{defin}

In the case of rings of algebraic integers the following theorem comes to our
rescue, assuring us that the decidability of \textsc{H10} does in fact not
depend on the encoding chosen. For a more detailed discussion of efficient
representations of rings see \cite{Stoltenberg1999}.

\begin{thm}
  Let $R$ be a finitely generated, efficiently representable ring with unit.
  Then $R$ is computably categorical.
\end{thm}

I will not give a proof here but sketch how one proceeds in proving the
theorem.

Let $ζ_1, …, ζ_n ∈ R$ be a set of generators of $R$ over $R$ and let $φ_1: R →
R_1, φ_2: R → R_2$ be the efficient representations of $R$ together with the
respective ring isomorphisms. Then $φ_1(ζ_1), …, φ_1(ζ_n)$ generate $R_1$ over
$R_1$ and $φ_2(ζ_1), …, φ_2(ζ_n)$ generate $R_2$ over $R_2$. Storing these
finitely many values of the isomorphism $φ_2 \circ φ_1^{-1}$ in memory one can
use the computabilty of $R_1$ and $R_2$ respectively to extend the partial
mapping in the obvious way.

With this technicality out of the way, from now on I will not differentiate
between $\algint$ and its efficient representations. However, there are
structures where the choice of representation matters. In fact, $⟨ℕ, ≤⟩$ is not
computably categorical. A proof using the undecidability of the halting problem
can be found in \cite[Prob. 1.6]{Shore}.

Using that $\algint$ is computably categorical for each algebraic number field
$K$ we observe that \textsc{H10} is semi-decidable over $\algint$. An algorithm
affirming the existence of roots for a given polynomial $p ∈ \algint{[X_1, …,
X_n]}$ starts by trying whether the empty string is the encoding of $n$
algebraic integers $x_1, …, x_n$ and if this is the case evaluates $p(x_1, …,
x_n)$. Both operations are computable as $\algint$ is efficiently presentable. If $p(x_1, …, x_n) = 0$ the algorithm finishes, otherwise it writes the next
string on the tape and starts over.

\subsection{Important techniques}

\todo{Diophantine predicates}

Before tackling \textsc{H10} over selected rings of algebraic integers, we list
some structural results and methods used within the subsequent proofs. For
further structural results see \cite{Shlapentokh2000}.

\begin{lem}\label{lem:intersections and unions}
    Let $R$ be an integral domain, whose quotient field is not
    algebraically closed. Then if $S_1, S_2 \subseteq R$ are Diophantine so are
    \[
      S_1 ∩ S_2 \quad \text{and} \quad S_1 ∪ S_2.
    \]

    If $R$ is computable, then there is an algorithm that finds the defining
    polynomial equations for union and intersection efficiently.
\end{lem}

In other words, conjunctions and disjunctions of existentially quantified atomic
formulas can be replaced by a single existentially quantified atomic formula. Or
again put differently, conjunction $∧$ and disjunction $∨$ are
$Σ_{ring}^*$ definable.

\begin{proof}
Let $f(\mathbf{X}, \mathbf{Y}) ∈ R[X_1, …, X_{n_1}, Y_1, …, Y_{n_2}]$ and
$g(\mathbf{X}, \mathbf{Y}) ∈ R[X_1, …, X_{n_1}, Y_1, …, Y_{n_2}]$ give
Diophantine definitions\footnote{By inserting dummy indeterminates we may whish
that both polynomials have the same number of indeterminates.} of $S_1$ and
$S_2$ resp. then
\[
  h := f g
\]
vanishes if and only if $f$ or $g$ vanishes. As a consequence, the
following identity gives a Diophantine definition of the union.

\[ S_1 ∪ S_2 = \lbrace \mathbf{x} \mid \mathbf{y} \; h(\mathbf{x}, \mathbf{y}) = 0 \rbrace. \]

To prove the claim for intersections of Diophantine sets, let
\[
  h(X) = a_m T^m + … + a_1 T + a_0 ∈ R[T]
\]
be a polynomial of degree $m > 0$ without roots in $R$. Then
$\overline h(X) = X^m h(X^{-1})$ does not have roots in $R$ either. As if
$α ∈ R$ is a root of $\overline h$ then
\[
  0 = \bar h(α) = a_m + a_{m-1} α + a_1 α^{,-1} + a_0 α^m
\]
and $α = 0$ implies $a_m = 0$. Otherwise, $α^{-1}$ is a root of
$α^m h$ and therefore of $h$.

Now consider
\[
  H(\mathbf{X}, \mathbf{Y}) = \sum_{i=0}^m a_i f(\mathbf{X}, \mathbf{Y})^i g(\mathbf{X}, \mathbf{Y})^{m - i}.
\]
We will prove for all $\mathbf{α} ∈ R^{n_1}$ and $\mathbf{β} ∈ R^{n_2}$ the
following equivalence, showing that $H$ witnesses that $S_1 ∩ S_2$ is
Diophantine.

\[ H(\mathbf α, \mathbf β) = 0 \Leftrightarrow f(\mathbf α, \mathbf β) = 0 ∧ g(\mathbf α, \mathbf β) = 0 \]

If $H(\mathbf α, \mathbf β) = 0$ but $f(\mathbf α, \mathbf β) ≠ 0$ then
\[
  0 = \frac H {f^n} (\mathbf α, \mathbf β) = \overline h \left(\frac gf (\mathbf α, \mathbf β) \right),
\]
which is a contradiction to $\overline h$ not having roots. If, on the
other hand, $H(\mathbf α, \mathbf β) = 0$ but $g(\mathbf α, \mathbf β) ≠ 0$
we find
\[
  0 = \frac H {g^n}(\mathbf α, \mathbf β) = \overline h \left( \frac fg (\mathbf α, \mathbf β) \right).
\]

The converse direction is clear as the powers of $f$ and $g$ sum up
to $n$ for each summand in the definition of $H$.

To prove the effectiveness of these methods one observes, that the defining
equations contain only polynomials in $f$ and $g$. Interpreting $f$ and $g$ as
polynomials in countably many indeterminates \cref{ex:polynomials are
computable} implies that the polynomial equations for union and intersection of
Diophantine sets can be computed from the polynomials $f$ and $g$.
\end{proof}

Using induction and the lemma above, one immediately obtains that arbitrary
finite unions and intersections of Diophantine sets are Diophantine.

An important corollary of the lemma above is---as was remarked before---that
\textsc{H10} over $\algint$ is decidable if and only if $\mathtt{Th}_{∃+}
(\mathfrak O)$ is decidable. Indeed, if one has a Turing machine that decides
whether a single polynomial over $\algint$ has roots in $\algint$. Then upon
using the algorithm described in \cref{lem:intersections and unions} the same
Turing machine can decide whether a collection of polynomial equations can be
satisfied simultaneously.

Note that the algorithm presented above does not depend on the initial equations
$f$ and $g$ but it does depend on the integral domain $R$. As we might need
different polynomials $h$ without roots for each ring $R$ in the case of
conjunctions.

\textcite{Shlapentokh2000} notes that the following lemma and its corollary are
`the only tool successfully used to show the undecidability of \textsc{H10}
for various subrings of the number fields' They explain the usefulness
of Diopahtine definitions.

\begin{lem} \label{lem:moving up}
Let $R_1 \subseteq R_2$ be computable rings and integral domains such
that the quotient field of $R_2$ is not algebraically closed. If \textsc{H10}
is undecidable over $R_1$ and $R_1$ has a Diophantine definition over
$R_2$, then \textsc{H10} in undecidable over $R_2$.
\end{lem}
\begin{proof}
Let $f(X, Y_1, …, Y_n)$ be a Diophantine definition of $R_1$ over $R_2$ and
assume that \textsc{H10} is decidable over $R_2$. We prove that \textsc{H10} is
decidable over $R_1$ as well, contradicting our assumption.

Let $p ∈ R_1[T_1, …, T_m]$ be a Diophantine definition of $S$ over
$R_1$. Then
\[
  S = \left\lbrace ∃α_1,…,α_m, β_{1,1}, …, β_{m,n} ∈ R_2 \;\middle|\; p(α_1, …, α_m) = 0 ∧ \bigwedge_{i=1}^m f(α_i, β_{i,1},…,β_{i,n}) = 0 \right\rbrace
\]
is a Diophantine definition of $S$ over $R_2$. If $\mathfrak R_1$ is the
$Σ_{ring}^*$-structure of $R_1$ and $\mathfrak R_2$ the respective
structure of $R_2$ this equation can be restated as
\begin{align*}
  \mathfrak R_1 &\models ∃ α_1 … ∃ y_m \; p(α_1, …, α_m) \doteq 0 ⇔ \\
  \mathfrak R_2 &\models ∃α_1 … ∃ α_m ∃ β_{1,1} … ∃ β_{m,n} \; p(α_1, …, α_m) \doteq 0 ∧ \bigwedge_{i=1}^m f(α_i, β_{i,1},…,β_{i,n}) \doteq 0,
\end{align*}
where the latter sentence can clearly be obtained efficiently from the first one
again by an algorithm depending on $R_1$ and $R_2$ but not on $p$. Therefore,
the decidability of $\mathtt{Th}_{∃+}(\mathfrak R_2)$ implies the decidability
of $\mathtt{Th}_{∃+}(\mathfrak R_1)$ by first applying the transformation of
sentences and then the deciding algorithm in $R_2$.
\end{proof}

The following corollary follows directly from the lemma above and the
undecidability of Hilbert's tenth problem over $ℤ$.

\begin{cor}
  Let $ℤ \subseteq R$ be a computable integral domain, whose quotient field is
  not algebraically closed. If $ℤ$ has a Diophantine definition over $R$ then
  \textsc{H10} is not decidable over $R$.
\end{cor}
