% !TeX encoding = UTF-8
% !TeX TS-program = xelatex
% !TeX spellcheck = en_GB
% !TeX root = ../Herbstrith-H10_over_AI.tex

\subsection{Computable structures} \label{sec:computable structures}

Up to this point the encoding of problems was treated as some kind of black-box.
This subsection takes a categorical view on computability and ensures us,
that---up to a sensible definition---encodings of the rings we concern ourselves
with do not matter.

\begin{defin}
  Let $\mathcal{L} = \set{f_1, f_2, …}$ be an at most countable language.
  \begin{thmlist}
    \item An algebraic structure $\mathfrak A = ⟨A; f_1^{\mathfrak A},
    f_2^{\mathfrak A}, …⟩$, with $A \subseteq ω$, is called \emph{computable
    structure} if $A$ is decidable and all operations $f_i^{\mathfrak A}$ are
    universally computable.

    \item An algebraic structure $\mathfrak A = ⟨A; f_1^{\mathfrak A},
    f_2^{\mathfrak A}, …⟩$ is called \emph{efficiently presentable} if there
    is an $\mathcal L$-isomorphism $A → B$, where $\mathfrak B = ⟨B;
    f_1^{\mathfrak B}, f_2^{\mathfrak B}, …⟩$ is a computable structure.

    \item An computable $\mathcal L$-morphism between computable $\mathcal
    L$-structures is called \emph{computable morphism}.
  \end{thmlist}
\end{defin}

An efficient presentation of say a ring $R$ is an encoding of $R$ that preserves the ring structure of $R$.

\begin{exam}
  \begin{exlist}
    \item In \cref{ex:tally encoding} we have encoded the non-negative integer
    $n$ by a string of $n$ consecutive $\one$-s. We have also already
    presented the algorithm deciding $\enc{ℕ} \subseteq ω$. under this encoding.
    Considering $ℕ$ as $\mathcal L_{ring}$-structure, one finds that the tally
    encoding gives rise to a efficient presentation of $ℕ$.

    The constants $0$ and $1$ are trivially computable, by clearing the tape in
    the first case and writing a single $\one$ in the second case. Using the
    pairing function of \cref{ex:tally pairing} the binary operations $+$ and
    $\cdot$ are also easily seen to be computable. As for $+$ the algorithm
    takes the input
    \[
      \one … \one \zer \one … \one
    \]
    and replaces the $\zer$-symbol by an $\one$ and deletes the rightmost
    $\one$.

    \item In general $ℤ$ and $\algint$ viewed as $\mathcal L_{ring}^*$ structures are efficiently representable. For the integers this is proven by each programming language allowing arbitrarily large integer arithmetic---like \emph{Haskell} or \emph{Python}.

    To present algebraic integers one uses an integral basis say $ζ_1, …, ζ_n$. Then any integer $η$ can be encoded as an $n$-tuple of integers. Addition and subtraction are defined coordinate-wise. To implement the multiplication one stores the finite multiplication table of the basis elements
    \[
      \begin{array}{r | r r r r}
            & ζ_1   & ζ_2     & … & ζ_n     \\
        \hline
        ζ_1 & ζ_1^2 & ζ_1 ζ_2 & … & ζ_1 ζ_n \\
        ζ_2 &       & ζ_2^2   & … & ζ_2 ζ_n \\
        \vdots &    &   & \ddots  & \vdots  \\
        ζ_n &       &         &   & ζ_n^2
      \end{array}
    \]
    in memory and extends to all of $\algint$ linearly. A full implementation can be found in the Appendix.\todo{Write the appendix}

    \item $⟨ℕ, ≤⟩$ is efficiently presentable using the tally encoding and $n ≤
    m$ if and only if $\max(n - m, 0) = 0$. So deciding $n ≤ m$ boils down to
    applying floor subtraction and checking whether the tape is empty. Both
    operations are clearly computable.
  \end{exlist}
\end{exam}

It is a natural question whether two efficient presentations of the same
structure are computably isomorphic i.e. if there exists a computable
isomporphism between them. We will see that the last example differs from the
others in this regard.

\begin{defin}
  Let $\mathcal{L} = \set{f_1, f_2, …}$ be an at most countable language. An
  algebraic structure $\mathfrak A = ⟨A; f_1^{\mathfrak A}, f_2^{\mathfrak A},
  …⟩$ is called \emph{computably categorical} if it is efficiently
  representable and every pair of efficient representations is computably
  isomorphic.
\end{defin}

In the case of rings of algebraic integers the following theorem comes to our
rescue, assuring us that the decidability of \textsc{H10} does in fact not
depend on the encoding chosen.

\begin{thm}
  Let $R$ be a finitely generated, efficiently representable ring with unit.
  Then $R$ is computably categorical.
\end{thm}

I will not give a proof here but sketch how one proceeds in proving the
theorem.

Let $ζ_1, …, ζ_n ∈ R$ be a set of generators of $R$ over $R$ and let $φ_1: R →
R_1, φ_2: R → R_2$ be the efficient representations of $R$ together with the
respective ring isomorphisms. Then $φ_1(ζ_1), …, φ_1(ζ_n)$ generate $R_1$ over
$R_1$ and $φ_2(ζ_1), …, φ_2(ζ_n)$ generate $R_2$ over $R_2$. Storing these
finitely many values of the isomorphism $φ_2 \circ φ_1^{-1}$ in memory one can
use the computabilty of $R_1$ and $R_2$ respectively to extend the partial
mapping in the obvious way.

For a more detailed discussion of computable rings see \cite{Stoltenberg1999}.

With this technicallity out of the way, from now on I will not differentiate
between $\algint$ and its efficient representation. However, there are
structures where the choice of representation matters. In fact, $⟨ℕ, ≤⟩$ is not
computably categorical. A proof using the undecidability of the halting problem
can be found in \cite[Prob. 1.6]{Shore}.

Using that $\algint$ is computably categorical for each algebraic number field
$K$ we observe that \textsc{H10} is semi-decidable over $\algint$. An algorithm
affirming the existence of roots for a given polynomial $p ∈ \algint{[X_1, …,
X_n]}$ starts by trying whether the empty string is the encoding of $n$
algebraic integers $x_1, …, x_n$ and if this is the case evaluates $p(x_1, …,
x_n)$. Both operations are computable as $\algint$ is efficiently presentable. If $p(x_1, …, x_n) = 0$ the algorithm finishes, otherwise it writes the next
string on the tape and starts over.

\subsection{Important techniques}

\todo{Diophantine predicates}

Before tackling \textsc{H10} over selected rings of algebraic integers, we list
some structural results and methods used within the subsequent proofs. For
further structural results see \cite{Shlapentokh2000}.

\begin{lem}\label{lem:intersections and unions}
    Let $R$ be an integral domain, whose quotient field is not
    algebraically closed. Then if $S_1, S_2 \subseteq R$ are Diophantine so are

    \[ S_1 ∩ S_2 \quad \text{and} \quad S_1 ∪ S_2 \]
\end{lem}

\begin{proof}
Let $f(X, Y_1, …, Y_n)$ and $g(X, Y_1, …, Y_n)$ give Diophantine
definitions\footnote{By inserting dummy indeterminates we may whish that
both polynomials have the same number of indeterminates.} of $S_1$
and $S_2$ resp. then

\[ h := fg \]

vanishes if and only if $f$ or $g$ vanishes. As a consequence, the
following set gives a Diophantine definition of the union.

\[ S_1 ∪ S_2 = \lbrace x \mid ∃ y_1, … , ∃ y_n \; h(x, y_1, … , y_n) = 0 \rbrace. \]

To prove the claim for intersections of Diophantine sets, let

\[h(X) = a_m X^m + … + a_1 X + a_0 ∈ R[X, Y_1, …, Y_n]\]

be a polynomial of degree $m > 0$ without roots in $R$. Then
$\bar h(X) = X^m h(X^{-1})$ does not have roots in $R$ either. As if
$α ∈ R$ is a root of $\bar h$ then

\[ 0 = \bar h(α) = a_m + a_{m-1} α + a_1 α^{,-1} + a_0 α^m\]

and $α = 0$ implies $a_m = 0$. Otherwise, $α^{-1}$ is a root of
$α^m h$ and therefore of $h$.

Now consider

\[ H(X, Y_1, …, Y_n) = \sum_{i=0}^m a_i f(X, Y_1, …, Y_n)^i g(X, Y_1, …, Y_n)^{m - i}.\]

We will prove for all $α_0, α_1, …, α_n ∈ R$ the following
equivalence, showing that $H$ witnesses that $S_1 ∩ S_2$ is
Diophantine.

\[ H(α_0, α_1, …, α_n) = 0 \Leftrightarrow f(α_0, α_1, …, α_n) = 0 ∧ g(α_0, α_1, …, α_n) = 0 \]

If there are $α_0, α_1, …, α_n ∈ R$ such that
$H(α_0, α_1, …, α_n) = 0$ but $f(α_0, α_1, …, α_n) ≠ 0$ then

\[ 0 = \frac H {f^n} (α_0, α_1, …, α_n) = \bar h \left(\frac gf (α_0, α_1, …, α_n) \right), \]

which is a contradiction to $\bar h$ not having roots. If, on the
other hand, $H(α_0, α_1, …, α_n) = 0$ but $g(α_0, α_1, …, α_n) ≠ 0$
we find

\[ 0 = \frac H {g^n}(α_0, α_1, …, α_n) = \bar h \left( \frac fg (α_0, α_1, …, α_n) \right). \]

The converse direction is clear as the powers of $f$ and $g$ sum up
to $n$ for each summand in the definition of $H$.
\end{proof}

Using induction and the lemma above, one immediately obtains that
arbitrary finite unions and intersections of Diophantine sets are
Diophantine. The lemma can also easily be extended to Diophantine sets
contained in $R^n$ as the existential quantorisation is nowhere used
in the proof.

\textcite{Shlapentokh2000} notes that the following lemma and its corollary are
`the only tool successfully used to show the undecidability of \textsc{H10}
for various subrings of the number fields' They explain the usefulness
of diopahtine definitions.

\begin{lem} \label{lem:moving up}
Let $R_1 \subseteq R_2$ be recursive rings and integral domains such
that the quotient field of $R_2$ is not algebraically closed. If H10
is undecidable over $R_1$ and $R_1$ has a Diophantine definition over
$R_2$, then H10 in undecidable over $R_2$.
\end{lem}

\begin{proof}
Let $f(X, Y_1, …, Y_n)$ be a Diophantine definition of $R_1$ over
$R_2$ and assume that H10 is decidable over $R_2$. We prove that H10
is decidable over $R_1$ as well, contradicting our assumption.

Let $p ∈ R_1[T_1, …, T_m]$ be Diophantine definition of $S$ over
$R_1$. Then

\[ S = \left\lbrace ∃α_1,…,α_m, β_{1,1}, …, β_{m,n} ∈ R_2 \middle| P(α_1, …, α_m) = 0 ∧ \bigwedge_{i=1}^m f(α_i, β_{i,1},…,β_{i,n}) = 0 \right\rbrace\]

is a Diophantine defition of $S$ over $R_2$. Firstly, note that the
defining relation of $S$ can be rewritten as a single polynomial
relation by \cref{lem:intersections and unions}. Since there is
a Turing machine deciding \textsc{H10} in $R_2$
\end{proof}
