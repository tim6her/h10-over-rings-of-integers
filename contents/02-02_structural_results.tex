% !TeX encoding = UTF-8
% !TeX TS-program = xelatex
% !TeX spellcheck = en_GB
% !TeX root = ../Herbstrith-H10_over_AI.tex

Before tackling \textsc{H10} in selected number fields, we list some
structural results and methods used within the subsequent proofs. For
further structural results see \cite{Shlapentokh2000}.

\begin{lem}\label{lem:intersections and unions}
    Let $R$ be an integral domain, whose quotient field is not
    algebraically closed. Then if $S_1, S_2 \subseteq R$ are Diophantine so are

    \[ S_1 ∩ S_2 \quad \text{and} \quad S_1 ∪ S_2 \]
\end{lem}

\begin{proof}
Let $f(X, Y_1, …, Y_n)$ and $g(X, Y_1, …, Y_n)$ give Diophantine
definitions\footnote{By inserting dummy indeterminates we may whish that
both polynomials have the same number of indeterminates.} of $S_1$
and $S_2$ resp. then

\[ h := fg \]

vanishes if and only if $f$ or $g$ vanishes. As a consequence, the
following set gives a Diophantine definition of the union.

\[ S_1 ∪ S_2 = \lbrace x \mid ∃ y_1, … , ∃ y_n \; h(x, y_1, … , y_n) = 0 \rbrace. \]

To prove the claim for intersections of Diophantine sets, let

\[h(X) = a_m X^m + … + a_1 X + a_0 ∈ R[X, Y_1, …, Y_n]\]

be a polynomial of degree $m > 0$ without roots in $R$. Then
$\bar h(X) = X^m h(X^{-1})$ does not have roots in $R$ either. As if
$α ∈ R$ is a root of $\bar h$ then

\[ 0 = \bar h(α) = a_m + a_{m-1} α + a_1 α^{,-1} + a_0 α^m\]

and $α = 0$ implies $a_m = 0$. Otherwise, $α^{-1}$ is a root of
$α^m h$ and therefore of $h$.

Now consider

\[ H(X, Y_1, …, Y_n) = \sum_{i=0}^m a_i f(X, Y_1, …, Y_n)^i g(X, Y_1, …, Y_n)^{m - i}.\]

We will prove for all $α_0, α_1, …, α_n ∈ R$ the following
equivalence, showing that $H$ witnesses that $S_1 ∩ S_2$ is
Diophantine.

\[ H(α_0, α_1, …, α_n) = 0 \Leftrightarrow f(α_0, α_1, …, α_n) = 0 ∧ g(α_0, α_1, …, α_n) = 0 \]

If there are $α_0, α_1, …, α_n ∈ R$ such that
$H(α_0, α_1, …, α_n) = 0$ but $f(α_0, α_1, …, α_n) ≠ 0$ then

\[ 0 = \frac H {f^n} (α_0, α_1, …, α_n) = \bar h \left(\frac gf (α_0, α_1, …, α_n) \right), \]

which is a contradiction to $\bar h$ not having roots. If, on the
other hand, $H(α_0, α_1, …, α_n) = 0$ but $g(α_0, α_1, …, α_n) ≠ 0$
we find

\[ 0 = \frac H {g^n}(α_0, α_1, …, α_n) = \bar h \left( \frac fg (α_0, α_1, …, α_n) \right). \]

The converse direction is clear as the powers of $f$ and $g$ sum up
to $n$ for each summand in the definition of $H$.
\end{proof}

Using induction and the lemma above, one immediately obtains that
arbitrary finite unions and intersections of Diophantine sets are
Diophantine. The lemma can also easily be extended to Diophantine sets
contained in $R^n$ as the existential quantorisation is nowhere used
in the proof.

\textcite{Shlapentokh2000} notes that the following lemma and its corollary are
`the only tool successfully used to show the undecidability of \textsc{H10}
for various subrings of the number fields' They explain the usefulness
of diopahtine definitions.

\begin{lem} \label{lem:moving up}
Let $R_1 \subseteq R_2$ be recursive rings and integral domains such
that the quotient field of $R_2$ is not algebraically closed. If H10
is undecidable over $R_1$ and $R_1$ has a Diophantine definition over
$R_2$, then H10 in undecidable over $R_2$.
\end{lem}

\begin{proof}
Let $f(X, Y_1, …, Y_n)$ be a Diophantine definition of $R_1$ over
$R_2$ and assume that H10 is decidable over $R_2$. We prove that H10
is decidable over $R_1$ as well, contradicting our assumption.

Let $p ∈ R_1[T_1, …, T_m]$ be Diophantine definition of $S$ over
$R_1$. Then

\[ S = \left\lbrace ∃α_1,…,α_m, β_{1,1}, …, β_{m,n} ∈ R_2 \middle| P(α_1, …, α_m) = 0 ∧ \bigwedge_{i=1}^m f(α_i, β_{i,1},…,β_{i,n}) = 0 \right\rbrace\]

is a Diophantine defition of $S$ over $R_2$. Firstly, note that the
defining relation of $S$ can be rewritten as a single polynomial
relation by \cref{lem:intersections and unions}. Since there is
a Turing machine deciding \textsc{H10} in $R_2$
\end{proof}
