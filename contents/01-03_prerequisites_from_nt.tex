% !TEX encoding = UTF-8
% !TEX TS-program = xelatex
% !TEX spellcheck = en_GB
% !TEX engine = xelatex
% !TEX root = ../Herbstrith-H10_over_AI.tex
%
% ███    ██ ██    ██ ███    ███ ██████  ███████ ██████      ████████ ██   ██
% ████   ██ ██    ██ ████  ████ ██   ██ ██      ██   ██        ██    ██   ██
% ██ ██  ██ ██    ██ ██ ████ ██ ██████  █████   ██████         ██    ███████
% ██  ██ ██ ██    ██ ██  ██  ██ ██   ██ ██      ██   ██        ██    ██   ██
% ██   ████  ██████  ██      ██ ██████  ███████ ██   ██        ██    ██   ██ ██

In this section I will closely follow Chapter 1 of the German
textbook~\cite{Neukirch2006}. However, the content is also present in the
English reference~\cite[Chap.~2]{Milne2017}, and sometimes the presentation of
this reference will be recited.

\subsection{Number fields and rings of algebraic integers}
% ███    ██ ██    ██ ███    ███        ███████ ██      ██████  ███████
% ████   ██ ██    ██ ████  ████        ██      ██      ██   ██ ██
% ██ ██  ██ ██    ██ ██ ████ ██        █████   ██      ██   ██ ███████
% ██  ██ ██ ██    ██ ██  ██  ██        ██      ██      ██   ██      ██
% ██   ████  ██████  ██      ██ ██     ██      ███████ ██████  ███████

We start with a series of definitions and remind the reader on some important
results from algebraic number theory and commutative algebra. But at first let
us fix an important notation.

Let \(R\) and \(S\) be commutative rings with unity. Let \(φ: R → S\) be a
ring-homomorphism mapping \(1_R\) to \(1_S\), then \(S\) is called an
\emph{\(R\)-algebra} and we write \(a α\) as a short form for \(φ(a) \cdot α\)
(\(a ∈ R\) and \(α ∈ S\)). We are especially interested in the case
where \(R \subseteq S\) and \(φ\) is chosen to be the embedding of \(R\) into
\(S\). In this situation we denote  by \(R[\seq{α}]\) the smallest ring inside
\(S\) containing \(R\) and all \(\seq{α} ∈ S\). Then \(R[\seq{α}]\) contains all
polynomial expressions in \(\seq{α}\) with coefficients in \(R\), i.e.\ all
elements of the form
\[
  \sum_{(\seq[m]{i}) ∈ ℕ^m} a_{i_1, …, i_m} α^{i_1} … α^{i_m},
\]
where only finitely many \(a_{i_1, …, i_m} ∈ R\) are non-zero.

\begin{defin}
  A finite field-extension \(K\) of the rationals \(ℚ\) is called
  \emph{algebraic number field} i.e. \(K\) is a field and at the same time a
  \(ℚ\)-algebra, that is finite-dimensional viewed as a \(ℚ\)-vector space. The
  \emph{degree} \([K : ℚ]\) is the dimension of \(K\) viewed as a \(ℚ\)-vector
  space.
\end{defin}

For convenience we will always assume that \(K\) is a subset of the complex
pane \(ℂ\).

\begin{exam}
  Examples of number fields include.
  \begin{exlist}
    \item \(ℚ\) is (up to isomorphism) the only algebraic number field of
    degree \(1\).

    \item \(ℚ[√2] = \set{a + b √2 \mid a, b ∈ ℚ}\) is an algebraic number field
    of degree \(2\). The inverse of \(a + b √2\), where not both \(a\) and \(b\)
    are \(0\), is given by \(\frac{a - b √2}{a^2 - 2 b^2}\).

    \item \(ℚ[√[3]{2}] = \set{a + b √[3]{2} + c √[3]{4} \mid a, b ∈ ℚ}\) is an
    algebraic number field of degree \(3\).

    \item \(ℚ[\e^{\frac{2 i π}{3}}] = \set{a + b \e^{\frac{2 i π}{3}}}\) is an
    algebraic number field of degree \(2\).
  \end{exlist}
\end{exam}

\begin{defin}
  Let \(R \subseteq S\) be commutative rings with unity. Then \(α ∈ S\) is called
  \emph{integral} over \(R\) if it is the root of a monic polynomial with
  coefficients in \(R\) i.e.\ if \(α\) satisfies an equation of the form
  \[
    α^n + a_{n-1}α^{n - 1} + … + a_0 = 0
  \]
  for some \(n ≥ 1\) and some \(\seq[n - 1]{a} ∈ R\). If all elements of \(S\)
  are integral over \(R\), then \(S\) is called \emph{integral} over \(R\).
\end{defin}

Of course, when one wants to use tools from algebra some structure on the
considered sets is needed. Thus, the following theorem, implying that if \(α, β
∈ S\) are algebraic over \(S\), so are their sum and product, is very desirable.

\begin{thm}\label{thm:integral closure}
  Let \(R \subseteq S\) be commutative rings with unity. Then the elements of
  \(S\) that are integral over \(R\) form a subring of \(S\).
\end{thm}

Dedekind gave a proof of this theorem using the following proposition.

\begin{pro}\label{pro:characterization of integral elements}
  Let \(R \subseteq S\) be commutative rings with unity. Then \(α ∈ S\) is
  integral over \(R\) if and only if there exists a finitely generated
  \(R\)-module \(M \subseteq S\) such that \(αM \subseteq M\), in fact \(M =
  R[α]\) can be chosen.
\end{pro}

\begin{proof}[Proof of \cref{thm:integral closure}]
  Let \(α, β ∈ S\) be integral over \(R\) and let \(M\) and \(N\) be finitely
  generated \(R\)-modules contained in \(S\) such that \(αM \subseteq M\) and
  \(βN \subseteq N\) hold. We define the product of the two modules as
  \[
    MN := \set{\sum_{i=1}^k m_i n_i \;\middle\vert\;
     k ∈ ℕ, \seq[k]{m} ∈ M, \seq[k]{n} ∈ N}.
  \]
  Clearly, \(MN\) contains \(0\) as \(M\) (and \(N\)) contains \(0\).
  Furthermore, it is closed under addition, and the inverse of an element in
  \(MN\) can be found by inverting all \(m_i ∈ M\) (or \(n_i ∈ N\)) in the sum.
  Thus, \(MN\) forms a subgroup of \(S\). Note that \(a m_i\) is contained in
  \(M\) for all \(a ∈ R\) and \(m_i ∈ M\) since \(M\) is an \(R\)-module. As a
  consequence,
  \[
    a \sum_{i=1}^k m_i n_i = \sum_{i=1}^k \underbrace{a m_i}_{∈ M} n_i ∈ MN
  \]
  for all \(a ∈ R\) and all \(\seq[k]{m} ∈ M\), \(\seq[k]{n} ∈ N\), and we can
  deduce that \(MN\) is an \(R\)-module.

  Let \(\set{\seq[m]{e}} \subseteq M\) generate \(M\) and \(\set{\seq[n]{f}}
  \subseteq N\) generate \(N\). Then it is easily seen that the finite set
  \[
    \set{e_i f_j \mid 1 ≤ i ≤ m, 1 ≤ j ≤ n} \subseteq MN
  \]
  generates \(MN\).

  We finish the proof by showing that \(αβ\) and \(α ± β\) satisfy \(αβ\, MN
  \subseteq MN\) and \((α ± β) MN\) respectively. Then the proposition implies
  the claim. But this is easy since
  \[
    αβ \sum_{i=1}^k m_i n_i = \sum_{i=1}^k \underbrace{αm_i}_{∈ M} \,
    \underbrace{βn_i}_{∈ N} ∈ MN
  \]
  holds for all \(\seq[k]{m} ∈ M\) and all \(\seq[k]{n} ∈ N\), and
  \[
    (α ± β) \sum_{i=1}^k m_i n_i =
    \sum_{i=1}^k \underbrace{αm_i}_{∈ M}\, n_i ±
      \sum_{i=1}^k m_i \, \underbrace{βn_i}_{∈ N} ∈ MN
  \]
  holds as well.
\end{proof}

Similarly, one can deduce from \cref{pro:characterization of integral elements}
that being integral is a transitive relation on rings. More formally, we have
the following proposition.

\begin{pro}\label{pro:being integral is transitive}
  Let \(R \subseteq S \subseteq T\) be commutative rings with unity. If \(S\) is
  integral over \(R\) and \(α ∈ T\) is integral over \(S\), then \(α\) is
  integral over \(R\).
\end{pro}

The set \(\overline{R}{}^S\) of all elements of \(S\) that are algebraic over
\(R\) is called \emph{algebraic closure} of \(R\) in \(S\). By the theorem above
\(\overline{R}{}^S\) is a subring of \(S\).

We will now return our attention from the general case to our specific situation
and consider the elements of \(ℂ\) that are integral over \(ℤ\). These elements
are called \emph{algebraic integers} and the integral closure of \(ℤ\) in \(ℂ\)
is denoted by \(\algint[]\). Given a number field \(K\), we denote by
\(\algint\) the intersection of \(\algint[]\) with \(K\). In other words,
\(\algint\) is the integral closure of \(ℤ\) in \(K\). To emphasize that we are
considering the ring \(ℤ\) and not any \(\algint\), we call \(ℤ\) the ring of
\emph{rational integers}.

We have that \(\overline{ℤ}{}^ℚ = \algint[] ∩ ℚ = ℤ\). This follows from a more
general result stating that factorial rings \(R\) equal their algebraic closure
\(\overline{R}{}^{\Quot R}\) in their quotient fields \(\Quot R\). By
\cref{pro:being integral is transitive} this property of \(ℤ\) extends to all
rings of algebraic integers, we have \(\overline{\algint}{}^K = \algint\). To
make the analogue complete we prove that \(K\) is the fraction field of
\(\algint\) (see~\cref{thm:K is the quotient field of O K}). However, even more
is true, as one can choose the denominator in the quotient to be a rational
integer. More precisely, the following holds.

\begin{pro}
  Let \(K\) be a number field and \(\algint\) its ring of algebraic integers.
  For all \(x ∈ K\) there exists a non-zero rational integer \(n ∈ ℤ \setminus
  \set{0}\) such that \(nx ∈ \algint\) is an algebraic integer.
\end{pro}

\begin{thm}\label{thm:K is the quotient field of O K}
  The quotient field of \(\algint\) is (isomorphic to) \(K\) for all number
  fields \(K\).
\end{thm}

% TODO Fill out missing facts
\todo{Fill out missing facts}
Missing facts
\begin{enumerate}
  \item Dedekind domain and factorization into prime ideals
  \item GCD and LCM
  \item Principal ideals of different primes are relative prime
  \item\label{thm:Chinese remainder}
   Chinese remainder theorem \cite[see][Chap.~1,
  Thm~3.6]{Neukirch2006}
  \item\label{thm:primitive element} Primitive element theorem
  \item\label{cor:Lagranges four square theorem}
  Lagrange's four-square theorem. A full proof of the claim using
  \cref{thm:Minkowski} is given in \cite[Remark 4.20]{Milne2017}.
  \item\label{ex:O K is computable}
  Integral bases and computability
  \item normal extension/closure
  \item Galois group acts transitively on \(μ_a\).
  \item self-conjugate elements
\end{enumerate}


\begin{defin}[Dedekind domain]
  pass
\end{defin}

\begin{defin}[class number \(c_k\)]
    pass
\end{defin}

\begin{defin}[Abelian extension]
    pass
\end{defin}

\subsection{Geometry of numbers}

\subsubsection{Minkowski's theorem on convex bodies}

In this section I closely follow \cite[Chap.~I, §2]{Schmidt1991}.

Let \(V\) be an \(n\)-dimensional vector space over the reals \(ℝ\). If \(\seq[k]{e} ∈
V\) (\(k ≤ n\)) are linearly independent over \(ℝ\), then the free Abelian group
\[
  Λ = ℤ e_1 + … + ℤ e_k
\]
is called a \emph{lattice}. Note that \(ℤ + \sqrt{2} ℤ\) is not a lattice in this
sense, because \(1\) and \(\sqrt{2}\) are linearly dependent over \(ℝ\).

Let \(μ\) be the measure corresponding to the usual Euclidean volume%
\footnote{I.e. \(μ\) is the Lebesgue measure on \(ℝ^n\) and therefore the Haar
measure with respect to the locally compact, Abelian group \(⟨Λ, +⟩\).}
on \(ℝ^n\). Then the \emph{fundamental parallelepiped}
\[
  D = \set{\sum_{i=1}^n α_i e_i \;\middle\vert\; α_i ∈ [0, 1]}
\]
has the volume
\[
  μ(D) = |\det \left( \seq{e} \right)|
\]
for a fixed lattice \(Λ = ℤ e_1 + … + ℤ e_n\) in \(ℝ^n\). One calls such a lattice whose rank coincides with the dimension of the vector space \(ℝ^n\) a \emph{full} lattice.

\begin{thm}[Minkowski's theorem on convex bodies] \label{thm:Minkowski}
  Let \(Λ = ℤ e_1 + … + ℤ e_n\) be a full lattice in the \(n\)-dimensional
  \(ℝ\)-vector space \(V\) and let \(D\) denote its fundamental parallelepiped. If \(T
  \subseteq V\) is compact, convex and symmetric in the origin, i.e.\ if \(α ∈ T\)
  so is \(-α ∈ T\), and
  \[
    μ(T) ≥ 2^n μ(D).
  \]
  Then \(T\) contains a non-zero lattice point \(α ∈ Λ \setminus \set{0}\).
\end{thm}

For a proof of Minkowski's theorem and further details on lattices see
\cite[4.4, p.~28]{Neukirch2006} or \cite[Thm.~4.19]{Milne2017}.

\subsubsection{Kronecker's theorem}

I will closely follow the German reference \cite[Chap.~2]{Hlawka1986}. However,
the subject can as well be found in \cite[see][Chap.~\textsc{XXIII}]{Hardy1975}.


\begin{thm}[Dirichlet's unit theorem]\label{thm:Dirichlet}
    see \cite[Thm.~5.1]{Milne2017}
\end{thm}

\begin{thm}[Local-Global--Principle / Hasse-Minkowski theorem]
    see \cite[§ 66]{Meara2000}
\end{thm}
