% !TEX encoding = UTF-8
% !TEX TS-program = xelatex
% !TEX spellcheck = en_GB
% !TEX engine = xelatex
% !TEX root = ../Herbstrith-H10_over_AI.tex
%
% ███    ██ ██    ██ ███    ███ ██████  ███████ ██████      ████████ ██   ██
% ████   ██ ██    ██ ████  ████ ██   ██ ██      ██   ██        ██    ██   ██
% ██ ██  ██ ██    ██ ██ ████ ██ ██████  █████   ██████         ██    ███████
% ██  ██ ██ ██    ██ ██  ██  ██ ██   ██ ██      ██   ██        ██    ██   ██
% ██   ████  ██████  ██      ██ ██████  ███████ ██   ██        ██    ██   ██ ██

\subsection{Number fields and rings of algebraic integers}
% ███    ██ ██    ██ ███    ███        ███████ ██      ██████  ███████
% ████   ██ ██    ██ ████  ████        ██      ██      ██   ██ ██
% ██ ██  ██ ██    ██ ██ ████ ██        █████   ██      ██   ██ ███████
% ██  ██ ██ ██    ██ ██  ██  ██        ██      ██      ██   ██      ██
% ██   ████  ██████  ██      ██ ██     ██      ███████ ██████  ███████

In this section I will closely follow Chapter 1 of the German
textbook~\cite{Neukirch2006}. However, the content is also present in the
English reference~\cite[Chap.~2]{Milne2017}, and sometimes the presentation of
this reference will be recited. We start with a series of definitions and remind
the reader of some important results from algebraic number theory and
commutative algebra. But at first let us fix an important notation.

Let \(R\) and \(S\) be commutative rings with unity. Let \(φ: R → S\) be a
ring-homomorphism mapping \(1_R\) to \(1_S\), then \(S\) is called an
\emph{\(R\)-algebra} and we write \(a α\) as a short form for \(φ(a) \cdot α\)
(\(a ∈ R\) and \(α ∈ S\)). We are especially interested in the case
where \(R \subseteq S\) and \(φ\) is chosen to be the embedding of \(R\) into
\(S\). In this situation we denote  by \(R[\seq{α}]\) the smallest ring inside
\(S\) containing \(R\) and all \(\seq{α} ∈ S\). Then \(R[\seq{α}]\) contains all
polynomial expressions in \(\seq{α}\) with coefficients in \(R\), i.e.\ all
elements of the form
\[
  \sum_{(\seq{i}) ∈ ℕ^n} a_{i_1, …, i_n} α^{i_1} … α^{i_n},
\]
where only finitely many \(a_{i_1, …, i_n} ∈ R\) are non-zero.

\begin{defin}
  A finite field-extension \(K\) of the rationals \(ℚ\) is called
  \emph{algebraic number field}. This means that \(K\) is a field and at the
  same time a \(ℚ\)-algebra, that is finite-dimensional viewed as a \(ℚ\)-vector
  space. The \emph{degree} \([K : ℚ]\) is the dimension of \(K\) viewed as a
  \(ℚ\)-vector space.
\end{defin}

For convenience we will always assume that \(K\) is a subset of the complex
pane \(ℂ\).

\begin{exam}
  \begin{exlist}
    \item \(ℚ\) is (up to isomorphism) the only algebraic number field of
    degree \(1\).

    \item \(ℚ[√2] = \set{a + b √2 : a, b ∈ ℚ}\) is an algebraic number field
    of degree \(2\). The inverse of \(a + b √2\), where not both \(a\) and \(b\)
    are \(0\), is given by \(\frac{a - b √2}{a^2 - 2 b^2}\).

    \item \(ℚ[√[3]{2}] = \set{a + b √[3]{2} + c √[3]{4} : a, b, c ∈ ℚ}\) is an
    algebraic number field of degree \(3\).
  \end{exlist}
\end{exam}

Let \(K/M\) be an extension of number fields, then \(K/M\) is in fact an
\emph{algebraic extension}. This means that every \(x ∈ K\) is the root of some
non-zero polynomial with coefficients in \(M\). We denote by \(μ_{M, x} ∈ M[X]\)
the monic polynomial with root \(x\) dividing (in \(M[X]\)) all other
polynomials with root \(x\). The polynomial \(μ_{M, x}\) is called \emph{minimal
polynomial} of \(x\) over \(M\). By the minimality condition (w.r.t.\
divisibility in \(M[X]\)) on \(μ_{M, x}\) this polynomial must be irreducible.

In fact, every element \(x\) in \(K\) is algebraic over the rationals \(ℚ\).
Note that the field of all algebraic elements \(\overline{ℚ}\)---called
\emph{algebraic closure} of \(ℚ\)---is however not a number field, as the
extension \(\overline{ℚ} / ℚ\) is (countably) infinite.

\begin{defin}
  Let \(R ⊂ S\) be commutative rings with unity. Then \(α ∈ S\) is called
  \emph{integral} over \(R\) if it is the root of a monic polynomial with
  coefficients in \(R\), i.e.\ if \(α\) satisfies an equation of the form
  \[
    α^n + a_{n-1}α^{n - 1} + … + a_0 = 0
  \]
  for some \(n ≥ 1\) and some \(a_0,…,a_{n-1} ∈ R\). If all elements of \(S\)
  are integral over \(R\) then \(S\) is called \emph{integral} over \(R\). If
  \(R\) is an integral domain then \(R\) is \emph{integrally closed} if for all
  elements \(x ∈ \Quot(R)\) being integral over \(R\) implies \(x ∈ R\).
\end{defin}

Of course, if one wants to use tools from algebra some structure on the
considered sets is needed. Thus, the following theorem, implying that if \(α, β
∈ S\) are algebraic over \(S\), so are their sum and product, is very desirable.

\begin{thm}\label{thm:integral closure}
  Let \(R \subseteq S\) be commutative rings with unity. Then the elements of
  \(S\) that are integral over \(R\) form a subring of \(S\).
\end{thm}

Richard Dedekind gave a proof of this theorem using the following proposition.

\begin{pro}\label{pro:characterization of integral elements}
  Let \(R \subseteq S\) be commutative rings with unity. Then \(α ∈ S\) is
  integral over \(R\) if and only if there exists a finitely generated non-zero
  \(R\)-module \(M \subseteq S\) such that \(αM \subseteq M\), in fact \(M =
  R[α]\) can be chosen.
\end{pro}

\begin{proof}[Proof of \cref{thm:integral closure}]
  Let \(α, β ∈ S\) be integral over \(R\) and let \(M\) and \(N\) be finitely
  generated \(R\)-modules contained in \(S\) such that \(αM \subseteq M\) and
  \(βN \subseteq N\) hold. We define the product of the two modules as
  \[
    MN := \set{\sum_{i=1}^k m_i n_i :
     k ∈ ℕ, \seq[k]{m} ∈ M, \seq[k]{n} ∈ N}.
  \]
  Clearly, \(MN\) contains \(0\) as \(M\) (and \(N\)) contains \(0\).
  Furthermore, it is closed under addition, and the inverse of an element in
  \(MN\) can be found by inverting all \(m_i ∈ M\) (or \(n_i ∈ N\)) in the sum.
  Thus, \(MN\) forms a subgroup of \(S\). Note that \(a m_i\) is contained in
  \(M\) for all \(a ∈ R\) and \(m_i ∈ M\) since \(M\) is an \(R\)-module. As a
  consequence,
  \[
    a \sum_{i=1}^k m_i n_i = \sum_{i=1}^k \underbrace{a m_i}_{∈ M} n_i
  \]
  is contained in \(MN\) for all \(a ∈ R\) and all \(\seq[k]{m} ∈ M\),
  \(\seq[k]{n} ∈ N\), and we can deduce that \(MN\) is an \(R\)-module.

  Let \(\set{\seq[m]{e}} \subseteq M\) generate \(M\) and \(\set{\seq[n]{f}}
  \subseteq N\) generate \(N\). Then it is easily seen that the finite set
  \[
    \set{e_i f_j : 1 ≤ i ≤ m, 1 ≤ j ≤ n} \subseteq MN
  \]
  generates \(MN\).

  We finish the proof by showing that \(αβ\) and \(α ± β\) satisfy \(αβ\, MN
  \subseteq MN\) and \((α ± β) MN ⊂ MN\) respectively. Then the proposition
  implies the claim. But this is the case since
  \[
    αβ \sum_{i=1}^k m_i n_i = \sum_{i=1}^k \underbrace{αm_i}_{∈ M} \,
    \underbrace{βn_i}_{∈ N} ∈ MN
  \]
  holds for all \(\seq[k]{m} ∈ M\) and all \(\seq[k]{n} ∈ N\), and
  \[
    (α ± β) \sum_{i=1}^k m_i n_i =
    \sum_{i=1}^k \underbrace{αm_i}_{∈ M}\, n_i ±
      \sum_{i=1}^k m_i \, \underbrace{βn_i}_{∈ N} ∈ MN
  \]
  holds as well.
\end{proof}

Similarly, one can deduce from \cref{pro:characterization of integral elements}
that being integral is a transitive relation on rings. More formally, we have
the following proposition.

\begin{pro}\label{pro:being integral is transitive}
  Let \(R \subseteq S \subseteq T\) be commutative rings with unity. If \(S\) is
  integral over \(R\) and \(α ∈ T\) is integral over \(S\), then \(α\) is
  integral over \(R\).
\end{pro}

The set \(\overline{R}{}^S\) of all elements of \(S\) that are integral over
\(R\) is called \emph{integral closure} of \(R\) in \(S\). By the theorem above
\(\overline{R}{}^S\) is a subring of \(S\).

We will now return our attention from the general case to our specific situation
and consider the elements of \(ℂ\) that are integral over \(ℤ\). These elements
are called \emph{algebraic integers} and the integral closure of \(ℤ\) in \(ℂ\)
is denoted by \(\algint[]\). Given a number field \(K\), we denote by
\(\algint\) the intersection of \(\algint[]\) with \(K\). In other words,
\(\algint\) is the integral closure of \(ℤ\) in \(K\). To emphasize that we are
considering the ring \(ℤ\) and not any \(\algint\), we call \(ℤ\) the ring of
\emph{rational integers}.

We have that \(\overline{ℤ}{}^ℚ = \algint[] ∩ ℚ = ℤ\). Thus, \(ℤ\) is integrally
closed. This follows from a more general result stating that factorial rings are
integrally closed. By \cref{pro:being integral is transitive} this property of
\(ℤ\) extends to all rings of algebraic integers, formally we have
\(\overline{\algint}{}^K = \algint\). To make the analogue complete we prove
that \(K\) is the fraction field of \(\algint\) (see~\cref{thm:K is the quotient
field of O K}). However, even more is true, as one can choose the denominator in
the quotient to be a rational integer. More precisely, the following holds.

\begin{pro}
  Let \(K\) be a number field and \(\algint\) its ring of algebraic integers.
  For all \(x ∈ K\) there exists a non-zero rational integer \(n ∈ ℤ \setminus
  \set{0}\) such that their product \(nx\) is an algebraic integer.
\end{pro}

\begin{thm}\label{thm:K is the quotient field of O K}
  The quotient field of \(\algint\) is (isomorphic to) \(K\) for all number
  fields \(K\).
\end{thm}
\begin{proof}
  By the proposition above every \(x ∈ K\) can be written as \(x = αn^{-1}\),
  where \(α ∈ \algint\) is an algebraic and \(n ∈ ℤ \setminus \set{0}\) is a
  rational integer. If \(x = βm^{-1}\) is another representation of this form,
  then \(αm = βn\) must hold in \(\algint\) and thus we can embed \(K\) into the
  quotient field \(\Quot(\algint)\) by mapping \(x = αn^{-1}\) to the
  representative \([α, n] ∈ \Quot(\algint)\).

  If on the other hand \([α, β] ∈ \Quot(\algint)\) with \(β ≠ 0\) is given,
  then \(α β^{-1}\) an element of \(K\). Thus, there exist \(γ ∈ \algint\) and
  \(n ∈ ℤ \setminus \set{0}\) such that \(γ n^{-1} = α β^{-1}\)---or put
  differently, such that \([γ, n] ∈ \Quot(\algint)\) is in the same equivalence
  class as \([α, β]\). As a consequence, the embedding defined above is
  surjective.
\end{proof}

We can now deduce that an element \(x ∈ K\) is an algebraic integer if and only
if its minimal polynomial \(μ_{ℚ, x}\) has rational integral coefficients.
Indeed, if \(x\) is a root of the monic polynomial \(p ∈ ℤ[X]\) then \(μ_{ℚ,
x}\) divides \(p\) and thus every root of \(μ_{ℚ, x}\) is an algebraic integer
as well. Now decompose
\[
  μ_{ℚ, x}(X) = \prod_{i=1}^n (X - α_i),
\]
for some \(\seq{α} ∈ K\), then since \(\algint\) is a ring, the minimal
polynomial must have coefficients in \(\algint ∩ ℚ = ℤ\) and the claim is
proven. In fact, we can always find an algebraic integer \(α ∈ K\) that
completely determines the number field \(K\).

\begin{thm}[primitive element theorem]\label{thm:primitive element}
  Let \(L/K\) be an extension of number fields then there exists a
  \emph{primitive element} \(α ∈ \algint[L]\) such that \(L = K[α]\). Moreover,
  if \(μ_{K, α} ∈ K[X]\) is the minimal polynomial of \(α\) over \(K\) then the
  degree of \(μ_{K, α}\) and the degree of the field extension \(L/K\) coincide.
  A \(K\)-basis of \(L\) is given by \(\set{1, α, …, α^{n - 1}}\), where \(n =
  [L : K]\).
\end{thm}

Important tools for studying number fields and algebraic integers are given
by the norm and trace, which are defined below.

\begin{defin}
  For an extension \(L/K\) of number fields and a fixed element \(x ∈ L\) we
  consider the linear transformation \(λ_x : L → L\) defined by \(λ_x(z) = xz\)
  and define the \emph{trace} of \(x\) as
  \[
    Tr_{L/K} (x) := Tr(λ_x)
  \]
  as well as the \emph{norm} of \(x\) as
  \[
    N_{L/K} (x) := \det(λ_x).
  \]
\end{defin}

 By basic facts from linear algebra, we find that the trace \(Tr_{L/K}: L → K\)
 is in fact a homomorphism between the additive groups of the number fields and
 the norm \(N_{L/K}: L^* → K^*\) is in fact a homomorphism between the groups of
 units.

 From the view of Galois theory one can reinterpret the norm and trace as
 follows.

 \begin{thm}\label{thm:norm and trace}
   Let \(L/K\) be an extension of number fields of degree \(n\). Then there
   exist exactly \(n\) embeddings \(\seq{σ}: L → ℂ\) that fix \(K\) point-wise.
   Furthermore, for all \(x ∈ L\) we have that
   \begin{thmlist}
     \item \(Tr_{L/K}(x) = \sum_{i = 1}^n σ_i(x)\), and
     \item \(N_{L_K}(x) = \prod_{i = 1}^n σ_i(x)\).
   \end{thmlist}
 \end{thm}

One calls an extension \(L / K\) of number fields \emph{normal extension} if the
embeddings \(σ_i: L → ℂ\) that fix \(K\) point-wise are in fact automorphisms of
\(L\). We do however have the following equivalent characterizations as well.

\begin{pro}
  For an extension \(L/K\) of number fields the following properties are
  equivalent.
  \begin{thmlist}
    \item \(L/K\) is a normal extension.

    \item If an irreducible polynomial \(p ∈ K[X]\) has one root in \(L\), then
    \(p\) splits in linear factors over \(L\).

    \item \(L\) is the splitting field of some irreducible polynomial \(p ∈
    K[X]\).
  \end{thmlist}
\end{pro}

From the proposition above we see immediately that every extension \(K / ℚ\) of
degree \(2\) is normal. Moving on to degree \(3\) this changes as for instance
\(ℚ[√[3]{2}] / ℚ\) is not normal. The irreducible polynomial \(X^3 - 2 ∈ ℚ[X]\)
has the root \(√[3]{2}\) in \(ℚ[√[3]{2}]\), but both of the other non-real roots
are not contained in the number field. One can however enlarge \(ℚ[√[3]{2}]\) to
obtain a normal extension. More generally, if \(L/K\) is an extension of
number fields, then there exists (up to isomorphism) a unique number field \(N
\supseteq L\), such that the extension \(N/K\) is normal. We call \(N\) the
\emph{normal closure} of the extension. In fact, if \(L = K[α]\) then \(N\) is
the splitting field of \(μ_{K, α}\). Using the normal closure of \(L/K\) one can
show that norm and trace behave well w.r.t.\ towers of field extensions.

\begin{cor}
  Let \(K ⊂ M ⊂ L\) be a tower of extensions of number fields. Then we have
  \[
    Tr_{L/K} = Tr_{M/K} • Tr_{L/M} \quad \text{and} \quad
    N_{L/K} = N_{M/K} • N_{L/M}.
  \]
\end{cor}

We fix an extension \(L/K\) of number fields and take another look at
\cref{thm:norm and trace}. Then we find for an algebraic integer \(α ∈
\algint[L]\) that its norm \(N_{L/K}(α)\) and trace \(Tr_{L/K}(α)\) are in fact
products and sums of algebraic integers and thus algebraic integers themselves.
Now since norm and trace are mappings from \(L\) to \(K\) we can deduce that
both \(N_{L/K}(α)\) and \(Tr_{L/K}(α)\) are contained in \(\algint\). In
particular, we find for all \(α ∈ \algint[L]\) that \(N_{L/ℚ}(α)\) and
\(Tr_{L/ℚ}(α)\) are rational integers.

Furthermore, one finds that an algebraic integer \(α ∈ \algint[L]\) is a unit if
and only if its norm \(N_{L/K}(α)\) is a unit in \(\algint\). Indeed, if \(β
N_{L/K}(α) = 1\) holds for some algebraic integer \(β ∈ \algint\), we can
rewrite the norm to find that
\[
  1 = α \underbrace{β \prod_{i = 2}^n σ_i(α)}_{∈ \algint[L]}
\]
holds, where \(\id = \seq{σ}\) denote all the complex embeddings of \(L\) that
fix \(K\) point-wise. In the special case that \(K = ℚ\) we find that \(α ∈
\algint[L]\) is a unit if and only if \(N_{L/ℚ}(α)\) is \(±1\).

As a next step we will further investigate the algebraic structure of
\(\algint\).

\begin{thm}
  Let \(K\) be an algebraic number field. Then \(\algint\) is a finitely
  generated free \(ℤ\)-module.
\end{thm}

We call a module-basis of \(\algint[L]\) over \(\algint\) an \emph{integral
basis} of \(L\) over \(K\). In particular, we can deduce that \(\algint[L]\) is
a finitely generated free \(ℤ\)-module by setting \(K = ℚ\) in the theorem
above. Every integral basis \(\set{\seq{ξ}} ⊂ \algint[L]\) is in fact a vector
space basis of \(L\) over \(K\) as well, thus its cardinality must coincide with
the degree of the extension. Note however, that not every \(K\)-basis of \(L\)
containing only algebraic integers is an integral basis of \(L\) over \(K\). In
full generality it is hard to find an integral basis, but once the basis is
known the structure of \(\algint[L]\) behaves very nicely with respect to
computability, which is the content of the following corollary.

\begin{cor}\label{cor:O K is computable}
  Let \(K\) be a number field and \(\algint\) its ring of algebraic integers.
  Then \(\algint\) is an efficiently presentable and computably categorical
  \(\lang_{ring}\)-structure.
\end{cor}
\begin{proof}
  By the theorem above \(\algint\) is a finitely generated free \(ℤ\)-module. In
  fact, it even carries a \(ℤ\)-algebra structure. Thus, it is efficiently
  presentable by \cref{ex:Z is computable}. Since \(\algint\) is a ring with
  unity, it is finitely generated and as a consequence of \cref{thm:computable
  categoricity} \(\algint\) is computably categorical.
\end{proof}

\subsection{Ideals of \(\algint\)}
% ██ ██████  ███████  █████  ██      ███████
% ██ ██   ██ ██      ██   ██ ██      ██
% ██ ██   ██ █████   ███████ ██      ███████
% ██ ██   ██ ██      ██   ██ ██           ██
% ██ ██████  ███████ ██   ██ ███████ ███████

We view algebraic integers as generalizations of rational integers. Given a
fixed algebraic integer \(α\) one can show using induction on the absolute value
of its norm \(N_{K/ℚ}(α)\) that \(α\) decomposes into a product of irreducible
elements. However, unlike in the case of rational integers this decomposition is
not unique. Indeed, in \(ℚ[i√{5}]\) one can decompose
\[
  21 = 3 \cdot 7 = (1 + i 2 √{5}) \cdot (1 - i 2 √{5}),
\]
where \(3, 7, 1 + i 2 √{5}\) and \(1 - i2 √{5}\) are irreducible and pair-wise
non-associated algebraic integers.%
\footnote{For full details see the first example in~\cite[Chap.~1,
§~3]{Neukirch2006}.}

It was the idea of German mathematician Ernst Eduard Kummer to generalize the
prime decomposition to ‘ideal numbers’. In his view, there should be ideal
primes \(\mathfrak{p}_1, \mathfrak{p}_2, \mathfrak{p}_3\) and \(\mathfrak{p}_4\)
such that
\[
  3 = \mathfrak{p}_1 \mathfrak{p}_2, \quad
  7 = \mathfrak{p}_3 \mathfrak{p}_4, \quad
  1 + 2 √{-5} = \mathfrak{p}_1 \mathfrak{p}_3, \; \text{and} \;
  1 - 2 √{-5} = \mathfrak{p}_2 \mathfrak{p}_4
\]
then
\[
  21 = (\mathfrak{p}_1 \mathfrak{p}_2)(\mathfrak{p}_3 \mathfrak{p}_4) =
    (\mathfrak{p}_1 \mathfrak{p}_3) (\mathfrak{p}_2 \mathfrak{p}_4)
\]
and the decomposition is again unique.

Since divisibility by a fixed number \(n ∈ ℤ\) gives rise to a congruence
relation \(m_1 \equiv m_2 \mod n\) defined by \(n \mid m_1 - m_2\), it is quite
natural---and was indeed carried out by Richard Dedekind---to view these ‘ideal
numbers’ as congruence relations on \(\algint\). Then the equivalence class
containing \(0\) is an \emph{ideal} in the sense of modern abstract algebra and
‘divisibility’ of ideals \(\mathfrak{a}\) by \(\mathfrak{b}\) can be replaced by
the inclusion of sets \(\mathfrak{a} ⊂ \mathfrak{b}\). On the other hand, for a
given ideal \(\mathfrak{a}\) we get back to the congruence if we define \(α
\equiv β \mod \mathfrak{a}\) by \(α - β ∈ \mathfrak{a}\).

Compare this to the well known case of rational integers. Here every ideal is a
principal ideal. Thus, there exist \(α, β ∈ ℤ\) such that \(\mathfrak{a} = (α)\)
and \(\mathfrak{b} = (β)\) and \(α\) is divisible by \(β\) if and only if
\(\mathfrak{a} ⊂ \mathfrak{b}\).

As with rational integers, one can define addition and multiplication of ideals
by
\begin{align*}
  \mathfrak{a} + \mathfrak{b} &:=
    \set{α + β : α ∈ \mathfrak{a}, β ∈ \mathfrak{b}}, \text{ and}\\
  \mathfrak{a} \mathfrak{b} &:=
    \set{\sum_{i=0}^n α_i β_i :
      n ∈ ℕ, α_0,\seq{α} ∈ \mathfrak{a}, β_0,\seq{β} ∈ \mathfrak{b}}.
\end{align*}
It is easy to prove that sums and products of ideals are again ideals. In fact,
the set of all ideals of \(\algint\) is a monoid with respect to multiplication,
where the neutral element is given by \(\algint = (1)\). However, unlike in the
case of rational integers we have that
\[
  \mathfrak{a} \mathfrak{b} ⊂ \mathfrak{a, b} ⊂ \mathfrak{a} + \mathfrak{b}
\]
and thus that
\[
  \mathfrak{a, b} \mid \mathfrak{a} \mathfrak{b} \quad \text{and} \quad
  \mathfrak{a} + \mathfrak{b} \mid \mathfrak{a, b}.
\]
Before we can further study divisibility of ideals, we need to investigate the
algebraic properties of rings of algebraic integers \(\algint\).

\begin{defin}
  An integral domain \(D\) is called \emph{Dedekind domain} if \(D\) is
  Noetherian---i.e.\ every ideal of \(D\) is finitely generated---integrally
  closed, and every non-zero prime ideal \(\mathfrak{p} ⊂ D\) is a maximal
  ideal.
\end{defin}

To study the ideals of algebraic integers the following theorem is essential.

\begin{thm}\label{thm:algebraic integers are Dedekind domain}
  Let \(K\) be a number field. Then its ring of algebraic integers \(\algint\)
  is a Dedekind domain.
\end{thm}

Note that for two ideals \((0) \subsetneq \mathfrak{a, b} \subsetneq \algint\),
the sum \(\mathfrak{a} + \mathfrak{b}\) is the smallest (w.r.t.\ set-inclusion)
ideal containing both \(\mathfrak{a}\) and \(\mathfrak{b}\). Indeed, if
\(\mathfrak{c}\) contains \(\mathfrak{a}\) and \(\mathfrak{b}\) then it contains
all sums of elements in \(\mathfrak{a}\) and \(\mathfrak{b}\). As a consequence,
we call \(\mathfrak{a} + \mathfrak{b}\) the \emph{greatest common divisor} of
\(\mathfrak{a}\) and \(\mathfrak{b}\).

Similarly, the intersection \(\mathfrak{a} ∩ \mathfrak{b}\) is the greatest
ideal of \(\algint\) contained in both \(\mathfrak{a}\) and \(\mathfrak{b}\).
Thus, we call \(\mathfrak{a} ∩ \mathfrak{b}\) the \emph{least common multiple}
of the ideals \(\mathfrak{a}\) and \(\mathfrak{b}\). Before we study the role of
prime ideals with respect to this notion of divisibility, an example is in
order.

\begin{exam}
  Consider the ring of rational integers \(ℤ\) and fix two integers \(n_1, n_2
  ∈ ℤ\). We denote by \(d\) their greatest common divisor and by \(m\) their
  least common multiple. As for their principal ideals the following hold
  \begin{align*}
    (n_1) ∩ (n_2) &= n_1 ℤ ∩ n_2 ℤ =
      \set{n ∈ ℤ : n_1 \mid n \text{ and } n_2 \mid n} =
      \set{n ∈ ℤ : m \mid n} = (m),\\
    (n_1)(n_2) &=
      \set{\sum_{i=0}^n α_i β_i :
           n ∈ ℕ, α_0,\seq{α} ∈ n_1 ℤ, β_0,\seq{β} ∈ n_2 ℤ} =\\
      &= \set{\sum_{i=0}^n n_1 k_i n_2 ℓ_i :
              n ∈ ℕ, k_0,\seq{k}, ℓ_0,\seq{ℓ} ∈ ℤ} =
        (n_1 n_2),\\
    \intertext{and using Bézout's identity}
    (n_1) + (n_2) &= \set{α + β : α ∈ (n_1), β ∈ (n_2)} =
       \set{n_1 k + n_2 ℓ : k, ℓ ∈ ℤ} = (d).
  \end{align*}
  Thus, in the case of rational integers greatest common divisor and least
  common multiple have their intended meaning if one replaces integers \(n\)
  with their respective principal ideals \((n)\).
\end{exam}

\begin{thm}\label{thm:prime ideal factorization}
  Let \(\algint\) be the ring of of algebraic integers in some number field
  \(K\) and let \(\mathfrak{a} \subsetneq \algint\) be a non-zero ideal. Then
  there exist up to reordering unique prime ideals \(\seq{\mathfrak{p}} ⊂
  \algint\) such that
  \[
    \mathfrak{a} = \mathfrak{p}_1 … \mathfrak{p}_n.
  \]
\end{thm}

Combining multiple occurrences of the same prime ideal in the decomposition
described in the theorem, one writes
\[
  \mathfrak{a} =
    \prod_{\substack{\mathfrak{p} ⊂ \algint\\\mathfrak{p} \text{ prime ideal}}}
      \mathfrak{p}^{ν_{\mathfrak{p}}},
\]
where all \(ν_{\mathfrak{p}}\) are a non-negative integers and all but finitely
many exponents are zero.%
\footnote{The constructivist reader will be pleased to hear that since all
ideals of \(\algint\) are finitely generated by \cref{thm:algebraic integers are
Dedekind domain}, the ring \(\algint\) contains only countably many prime
ideals. Thus, one can fix a linear order on the set of prime ideals, such that
for all ideals \(\mathfrak{a} ⊂ \algint\) all non-zero exponents
\(ν_{\mathfrak{p}}\) appear in a finite initial segment of the order.}
Using this product notation of the prime
decomposition of ideals, one obtains for
\[
\mathfrak{a} =
  \prod_{\substack{\mathfrak{p} ⊂ \algint\\\mathfrak{p} \text{ prime ideal}}}
    \mathfrak{p}^{ν_{\mathfrak{p}}}
\quad \text{and} \quad
\mathfrak{b} =
  \prod_{\substack{\mathfrak{p} ⊂ \algint\\\mathfrak{p} \text{ prime ideal}}}
    \mathfrak{p}^{μ_{\mathfrak{p}}},
\]
that their greatest common divisor has the factorization
\[
\mathfrak{a + b} =
  \prod_{\substack{\mathfrak{p} ⊂ \algint\\\mathfrak{p} \text{ prime ideal}}}
    \mathfrak{p}^{\min(ν_{\mathfrak{p}}, μ_{\mathfrak{p}})}.
\]
Thus, if \(\mathfrak{a} + \mathfrak{b} = (1)\), we say \(\mathfrak{a}\) and
\(\mathfrak{b}\) are \emph{relative prime}.

If we notice that the product of ideals \(\mathfrak{a = a_1 … a_n}\), where the
\(\mathfrak{a}_i\) are pair-wise relative prime, is equal to the intersection
\[
  \mathfrak{a} = \bigcap_{i = 1}^n \mathfrak{a}_i,
\]
we have all the tools at hand to restate another important property of the
integers.

\begin{thm}[Chinese remainder theorem]\label{thm:Chinese remainder}
  Let \(\seq{\mathfrak{a}} ⊂ \algint\) ideals, which are pair-wise relative
  prime, and let \(\mathfrak{a} = \cap_{i=1}^n \mathfrak{a}_i\). Then the
  following isomorphism of rings holds
  \[
    \algint / \mathfrak{a} \; \cong \;
    \bigoplus_{i = 1}^n \algint / \mathfrak{a}_i.
  \]
\end{thm}
\begin{proof}
  We consider the ring-homomorphism \(φ: \algint → \bigoplus_{i = 1}^n \algint /
  \mathfrak{a}_i\) defined by
  \[
    α ↦ (α + \mathfrak{a}_i)_{i = 1}^n.
  \]
  Its kernel is given by \(\mathfrak{a} = \cap_{i=1}^n \mathfrak{a}_i\). Thus,
  it suffices to prove that \(φ\) is surjective. For this we proceed by
  induction on \(n\). If \(n = 1\), then the claim is trivial. Thus, we consider
  the case \(n = 2\). Then we can find \(β_1 ∈ \mathfrak{a}_1\) and \(β_2 ∈
  \mathfrak{a}_2\) such that \(1 = β_1 + β_2\). In other words, we find \(β_1,
  β_2 ∈ \algint\) such that
  \[
    β_i \equiv 1 \mod \mathfrak{a}_{3 - i} \quad \text{and} \quad
    β_i \equiv 0 \mod \mathfrak{a}_i
  \]
  hold simultaneously for \(i ∈ \set{1, 2}\). If now an arbitrary element
  \((x_1 + \mathfrak{a}_1, x_2 + \mathfrak{a}_2) ∈ \algint / \mathfrak{a}_1 ×
  \algint / \mathfrak{a}_2\) is given then
  \[
    x := x_1 β_2 + x_2 β_1
  \]
  has the property that \(x \equiv x_i \mod \mathfrak{a}_{i}\) holds for \(i ∈
  \set{1, 2}\). Thus, we have found that \(φ\) is surjective for \(n = 2\).

  Let now \(n ≥ 2\) and note that we have the isomorphism of direct sums
  \[
    \bigoplus_{i = 1}^{n} \algint / \mathfrak{a}_i \; \cong \;
    \algint / \mathfrak{a}_n \times
      \bigoplus_{i = 1}^{n - 1} \algint / \mathfrak{a}_i.
  \]
  If we set \(\mathfrak{b} := \mathfrak{a}_1 … \mathfrak{a}_{n - 1}\) then by
  the induction hypothesis the factor rings \(\algint / \mathfrak{b}\) and
  \(\bigoplus_{i = 1}^{n - 1} \algint / \mathfrak{a}_i\) are isomorphic. Thus,
  we can deduce that
  \[
    \bigoplus_{i = 1}^{n} \algint / \mathfrak{a}_i \; \cong \;
    \algint / \mathfrak{a}_n \times \algint / \mathfrak{b}
  \]
  holds. To conclude the proof note that the ideals \(\mathfrak{a}_n\) and
  \(\mathfrak{b} := \mathfrak{a}_1 … \mathfrak{a}_{n - 1}\) are relative prime.
  Now by our observation for the case \(n = 2\), we find that the mapping
  \(\tilde{φ} : \algint → \algint / \mathfrak{a}_n \times \algint /
  \mathfrak{b}\) defined by
  \[
    α ↦ \left(α + \mathfrak{a}_n,
               α + \mathfrak{b}\right)
  \]
  is surjective and has kernel \(\mathfrak{a}_n ∩ \mathfrak{b} = \mathfrak{a}\).
  By the reduction steps observed above the claim holds.
\end{proof}

\begin{rem}
  Let \(\seq{\mathfrak{a}}\) be pair-wise relative prime. Then the Chinese
  remainder theorem tells us, that the collection of congruences
  \[
    x \equiv a_1 \mod \mathfrak{a}_1, \; …, \;
    x \equiv a_n \mod \mathfrak{a}_n
  \]
  can be solved simultaneously. Indeed, in the proof of the theorem we have
  shown that the ring-homomorphism \(φ: \algint → \bigoplus_{i = 1}^n \algint /
  \mathfrak{a}_i\) defined by
  \[
    x ↦ (x + \mathfrak{a}_i)_{i = 1}^n
  \]
  is a surjection and thus the \(n\)-tuple \((a_1 + \mathfrak{a}_1, …, a_n +
  \mathfrak{a}_n)\) is the image of some \(x ∈ \algint\). In other words, there
  exists an \(x ∈ \algint\) such that \(x \equiv a_i \mod \mathfrak{a}_i\) for
  all \(1 ≤ i ≤ n\).
\end{rem}

In the field of rational numbers one can extend the prime decomposition of
integers to a composition of positive rationals by allowing for negative powers
of primes. As for ideals there exists a similar construction.

\begin{defin}
  A \(\algint\)-submodule \(\mathfrak{m}\) of \(K\) is called \emph{fractional
  ideal} of \(K\) if there exists an algebraic integer \(α ∈ \algint \setminus
  \set{0}\) such that \(α \mathfrak{m} ⊂ \algint\).

  Let \(x := α / β ∈ K\), where \(α, β ∈ \algint\) and \(β ≠ 0\), then \(x
  \algint := β^{-1}(α \algint)\) is called the \emph{principal fractional ideal}
  generated by \(x\).
\end{defin}

As with usual ideals every non-zero fractional ideal \(\mathfrak{m}\) can be
written as
\[
  \mathfrak{m} =
  \prod_{\substack{\mathfrak{p} ⊂ \algint\\\mathfrak{p} \text{ prime ideal}}}
    \mathfrak{p}^{ν_{\mathfrak{p}}},
\]
where \(ν_{\mathfrak{p}} ∈ ℤ\) and all but finitely many exponents are zero.

To conclude this subsection we consider the principal ideals \((p)\) generated
by rational primes \(p ∈ ℤ\). But first let us have a look at an example. It
is easy to deduce that the equality
\[
  (21) = (3)(7) = (1 + i 2 √{5}) (1 - i 2 √{5}),
\]
of principal ideals holds in \(K = ℚ[i √5]\) by plugging-in the definition of
products of ideals. Thus, neither \((3)\) nor \((7)\) can be prime ideals in
\(\algint\). We do however know that no other rational prime can divide these
principal ideals. More formally we have
\begin{pro}
  Let \(\mathfrak{p} ≠ (0)\) be a prime ideal in the ring of algebraic integers
  of some number field \(K\). Then there exists a unique rational prime \(p ∈
  ℤ\) such that \(\mathfrak{p}\) divides the principal ideal \((p) = p
  \algint\).
\end{pro}
For a short argrument proving the proposition see the proof of Thm~3.1 in the
textbook~\cite{Neukirch2006}.

\subsection{Geometry of numbers}
%  ██████  ███████  ██████  ███    ███ ███████ ████████ ██████  ██    ██
% ██       ██      ██    ██ ████  ████ ██         ██    ██   ██  ██  ██
% ██   ███ █████   ██    ██ ██ ████ ██ █████      ██    ██████    ████
% ██    ██ ██      ██    ██ ██  ██  ██ ██         ██    ██   ██    ██
%  ██████  ███████  ██████  ██      ██ ███████    ██    ██   ██    ██

In this section we want to study approximations of real numbers by rational
quantities. The first main result will be Minkowski's theorem on convex
bodies~(\ref{thm:Minkowski}), which can be applied to prove Dirichlet's unit
theorem~(\ref{thm:Dirichlet}). The second main result is Kronecker's
approximation theorem~(\ref{thm:Kronecker}), whose proof is presented as in
Chap.~2 of the textbook~\cite{Hlawka1991}.

Let \(\seq{e} ∈ ℝ^n\) be a collection of linearly independent vectors over
\(ℝ\), then the free abelian group
\[
  Λ = ℤ e_1 + … + ℤ e_n
\]
is called a \emph{lattice} and its elements are \emph{lattice points}.
The set of generators \(\set{\seq{e}}\) is called \emph{basis} of \(Λ\). Note
that \(ℤ + \sqrt{2} ℤ\) is not a lattice in this sense, because \(1\) and
\(\sqrt{2}\) are linearly dependent over \(ℝ\).

The basis \(\set{\seq{e}}\) of a lattice \(Λ\) is not unique. For instance, a
second basis is given by the elements \(\set{e_1 + e_2, e_2, …, e_n}\). However,
if \(\set{\seq{f}}\) is another basis then the \(n \times n\)-matrix \(C :=
(c_{ij})_{1 ≤ i, j ≤ n}\) defined by
\[
  f_{i} = \sum_{j = 1}^n c_{ij} e_j
\]
has rational integral coefficients and is invertible. Thus, the determinant of
\(C\) is either \(-1\) or \(1\).

Let \(\Vol\) be the measure corresponding to the usual euclidean volume%
\footnote{More specifically, \(\Vol\) denotes the Lebesgue measure on \(ℝ^n\).
Since the Lebesgue measure is translation invariant, \(\Vol\) is also the Haar
measure with respect to the locally compact, abelian group \(⟨Λ, +⟩\).}
on \(ℝ^n\). Then for a fixed lattice \(Λ = ℤ e_1 + … + ℤ e_n\) in \(ℝ^n\) the
\emph{fundamental parallelepiped}
\[
  D = \set{\sum_{i=1}^n α_i e_i : α_i ∈ [0, 1]}
\]
has the volume
\[
  \Vol(D) = |\det \left( \seq{e} \right)|.
\]
Note that the fundamental parallelepiped \(D\) does depend on the choice of
basis, whereas its volume \(\Vol(D)\) is an invariant of the lattice. This is
because the determinant of the matrix for change of bases has absolute value
one. A lattice \(Λ\) in \(ℝ^2\) and its fundamental parallelepiped are depicted
in \cref{fig:lattice}. All elements of \(Λ\) appear at intersection of the
lines.

\begin{figure}
  \begin{center}
    \includegraphics{res/lattice_1}
    \caption{A lattice in \(ℝ^2\) and its fundamental parallelepiped \(D\)}
    \label{fig:lattice}
  \end{center}
\end{figure}

We have now all tools at hand to state our first main result. The following
proof is presented as in Thm.~4.4 of the textbook~\cite{Neukirch2006}.

\begin{thm}[Minkowski's theorem on convex bodies]\label{thm:Minkowski}
  Let \(Λ = ℤ e_1 + … + ℤ e_n\) be a lattice in the \(n\)-dimensional
  \(ℝ\)-vector space \(V\) and let \(D\) denote its fundamental parallelepiped.
  If \(T \subseteq V\) is convex and symmetric in the origin, i.e.\ \(α ∈ T\)
  implies \(-α ∈ T\), and
  \[
    \Vol(T) > 2^n \Vol(D).
  \]
  Then \(T\) contains a non-zero lattice point \(γ ∈ Λ \setminus \set{0}\).
\end{thm}
\begin{proof}
  We prove that there exist two distinct lattice points \(γ_1, γ_2 ∈ Λ\) such
  that the intersection of sets
  \[
    \left( ½ T + γ_1 \right) ∩ \left( ½ T + γ_2 \right)
  \]
  is non-empty. If this is the case then there exist \(x_1, x_2 ∈ T\) such that
  \[
    ½ x_1 + γ_1 = ½ x_2 + γ_2
  \]
  and thus
  \[
    0 ≠ γ := γ_1 - γ_2 = ½ (x_2 - x_1)
  \]
  lies in \(T ∩ Λ\), since it is the centre of the line segment between \(x_1\)
  and \(x_2 ∈ T\).

  To obtain a contradiction assume that the members of the family of sets
  \(\left(½ T + γ\right)_{γ ∈ Λ}\) are pair-wise disjoint. Then their
  intersections \(D ∩ \left(½ T + γ\right)\) with the fundamental parallelepiped
  \(D\) are pairwise disjoint as well. It follows that
  \[
    \Vol(D) ≥ \sum_{γ ∈ Λ} \Vol\left(D ∩ \left(½ T + γ\right)\right).
  \]

  On the other hand, since the euclidean volume is invariant under translation,
  we find that
  \[
    \Vol\left(D ∩ \left(½ T + γ\right)\right) =
    \Vol\left((D - γ) ∩ \left(½ T\right)\right)
  \]
  Furthermore, the sets \(D - γ\) cover all of \(ℝ^n\) and therefore all of
  \(½ T\) as well. Finally, we conclude that
  \[
    \Vol(D) ≥ \sum_{γ ∈ Λ} \Vol\left((D - γ) ∩ \left(½ T\right)\right) =
      \Vol\left(½ T\right) = \frac{1}{2^n} \Vol(T),
  \]
  which contradicts our assumption on the volume of \(T\).
\end{proof}

Note that the approximation of \(\Vol(T)\) cannot be improved as for instance
the open square \(\set{(x, y) ∈ ℝ^2 : |x|, |y| < 1}\) has volume \(2^2\) but
contains no non-zero lattice point of the two-di\-men\-sional lattice
\((1, 0)ℤ + (0, 1)ℤ\).

We will now use Minkowski's theorem to reprove an old result of Lagrange, that
is of utmost importance to our task of settling Hilbert's tenth problem. The
proof is taken from Remark~4.20 of Milne's lecture notes~\cite{Milne2017}. But
first we need a lemma.

\begin{lem}
  Let \(Λ ⊂ Λ'\) be two lattices in \(ℝ^n\) and let \(D\) and \(D'\) be one of
  their respective fundamental parallelepipeds. Then
  \[
    \Vol(D) = \Vol(D') [Λ' : Λ]
  \]
  holds, where \([Λ' : Λ] = | Λ' / Λ |\) denotes the index of \(Λ\) in \(Λ'\).
\end{lem}

For the proof of the proposition it will be convenient to identify the free
abelian group \(a_1 ℤ \times … \times a_n ℤ\) with the lattice generated by the
basis \(\set{a_1 e_1, …, a_n e_n} ⊂ ℝ^n\), where \(e_i\) denotes the \(i\)-th
vector of the standard basis of \(ℝ^n\).

\begin{pro}[Lagrange's four-square theorem]%
\label{pro:Lagranges four square theorem}
  Every non-negative integer is the sum of four squares of integers.
\end{pro}
\begin{proof}
  The integers \(0, 1\) and \(2\) can be written as
  \[
    0 = 0^2 + 0^2 + 0^2 + 0^2, \quad
    1 = 1^2 + 0^2 + 0^2 + 0^2, \; \text{and} \;
    2 = 1^2 + 1^2 + 0^2 + 0^2.
  \]
  Thus, we may assume that \(n > 2\). Furthermore, the set of integers
  representable as sum of four squares is closed under multiplication as
  \begin{align*}
    (a_1^2 &+ a_2^2 + a_3^2 + a_4^2)(b_1^2 + b_2^2 + b_3^2 + b_4^2) =\\
      & (a_1 b_1 - a_2 b_2 - a_3 b_3 - a_4 b_4)^2 +
        (a_1 b_2 + a_2 b_1 + a_3 b_4 - a_4 b_3)^2 +\\
      & (a_1 b_3 - a_2 b_4 + a_3 b_1 + a_4 b_2)^2 +
        (a_1 b_4 + a_2 b_3 - a_3 b_2 + a_4 b_1)^2
  \end{align*}
  holds. Hence, all that is left is to prove the claim for odd primes.

  For a fixed odd prime \(p\) the squares of an integer \(m\) take exactly \((p
  + 1)/2\) distinct values modulo \(p\) when \(m\) runs through \(0, 1, …, p -
  1\). Indeed, note that
  \[
    m \equiv - (p - m) \mod p \quad \text{and} \quad
    m^2 \equiv (p - m)^2 \mod p
  \]
  hold for all \(m ∈ \set{0, 1, …, p - 1}\). Thus, we obtain \((p - 1)/2\)
  pairs of numbers \((m, p - m)\) with the same square modulo \(p\), plus the
  value \(0 = 0^2\) when \(m\) runs through \(0, 1, …, p - 1\).
  
  By the same argument \(-n^2 - 1\) runs through exactly \((p + 1)/2\)
  distinct values modulo \(p\) for \(0 ≤ n ≤ p - 1\) as well. Hence, by the
  pigeonhole principle there exist integers \(m, n ∈ \set{0, 1, …, p - 1}\)
  solving the congruence
  \[
    m^2 + n^2 + 1 \equiv 0 \mod p.
  \]

  For a fixed solution \((m, n)\) of the above congruence, we consider the set
  \(Λ\) of all integral solutions \((a, b, c, d) ∈ ℤ^4\) of the simultaneous
  congruence
  \[
    c \equiv m a + n b \mod p \quad \text{and} \quad
    d \equiv m b - n a \mod p.
  \]
  It is not hard to see, that \(Λ\) is in fact a (free abelian) subgroup of
  \(ℤ^4\) of rank \(4\) and thus can be considered as an lattice. As \((p, p, p,
  p)\) is a solution of the congruences, we find that \(p ℤ^4 ⊂ Λ\) is a
  subgroup of \(Λ\). Considering the quotient \(Λ / p ℤ^4\) we note that \(a\)
  and \(b\) can be chosen arbitrarily modulo \(p\), but then \(c\) and \(d\) are
  uniquely determined. Thus, the index \([Λ : p ℤ^4]\) equals \(p^2\). We
  conclude that the index \([ℤ^4 : Λ]\) equals \(p^2\) as well and by the
  previous lemma the volume \(\Vol(D)\) of a fundamental parallelepiped \(D\)
  of \(Λ\) is \(1 \cdot p^2\).

  Consider the closed four-dimensional ball \(T\) of radius \(r\) around the
  origin. Its volume is \(π^2 r^4 / 2\) and if we choose \(2 p > r^2 > 4 √2 p /
  π\) then
  \[
    \Vol(T) > 16 p^2 = 2^4 \Vol(D)
  \]
  holds. By Minkowski's theorem there exists a non-zero lattice point \((a, b,
  c, d) ∈ (Λ ∩ T) \setminus \set{0}\). Since \((a, b, c, d)\) is in \(Λ\), we
  know that
  \begin{align*}
    a^2 + b^2 + c^2 + d^2 &\equiv
        a^2 + b^2 + (m a + n b)^2 + (m b - n a)^2 \\
      &\equiv a^2 (m^2 + n^2 + 1) + b^2 (m^2 + n^2 + 1) \equiv 0 \mod p
  \end{align*}
  holds. On the other hand, since \((a, b, c, d)\) is in \(T\), we have that
  \[
    a^2 + b^2 + c^2 + d^2 < 2p
  \]
  and \(p = a^2 + b^2 + c^2 + d^2\) is the desired representation.
\end{proof}

We now want to give a structural description of the group of units \(U_K :=
\algint^*\) of a number field \(K\). It is easy to see that all algebraic
integers \(ζ ∈ \algint\) with finite order, say \(k ∈ ℕ\), are roots of unity.
Indeed, the property \(ζ^k = 1\) shows that \(ζ\) is a \(k\)-th root of unity.
The set of all roots of unity \(ζ ∈ U_K\) is denoted by \(μ(K)\). If one can
show, that \(μ(K)\) is finite then \(μ(K)\) is a cyclic subgroup of \(K^*\).

By the fundamental theorem of finitely generated abelian groups, we know that
every finitely generated abelian group \(G\) is isomorphic to \(G_{tors} \times
ℤ^t\), where \(G_{tors}\) is the finite subgroup of elements with finite order,
called the \emph{torsion part} of \(G\), and \(t ∈ ℕ\) is called the \emph{free
rank} of \(G\), denoted by \(\rk G = t\). Thus, if \(U_K\) is finitely
generated then its torsion part is \(μ(K)\) and all that is left to fully
describe \(U_K\) is finding its free rank. This classification is the content of
the following important theorem.

\begin{thm}[Dirichlet's unit theorem]\label{thm:Dirichlet}
  Let \(K\) be a number field of degree \(n\) over the rationals \(ℚ\). If \(r\)
  is the number of real embeddings \(σ: K → ℝ\) of \(K\) then \(s := (n - r) /
  2\) is the number of pairs of complex-conjugate embeddings \(σ,\overline{σ}:
  K → ℂ\). In this case the group of units \(U_K\) is isomorphic to
  \[
    μ(K) \times ℤ^{r + s - 1}.
  \]
\end{thm}

In other words, Dirichlet's theorem states that there exists a collection of
units \(\seq[r + s - 1]{u} ∈ U_K\), called \emph{fundamental system of
units}, such that every unit \(u ∈ U_K\) can be written as
\[
  u = ζ u_1^{m_1} … u_{r + s - 1}^{m_{r + s - 1}}
\]
where \(ζ ∈ μ(K)\) is a root of unity and \(m_i ∈ ℤ\) is a rational integer for
all \(i ∈ \set{1, …, r + s - 1}\).

A full proof of the theorem exceeds the scope of this thesis, but among others
Chap.~5 of the textbook~\cite{Milne2017} and Chap.~1, §7 of the German
reference~\cite{Neukirch2006} contain proofs based on Minkowski's theorem. The
idea is to consider the mapping \(Σ: K → ℝ^r \times ℂ^s\) defined by
\[
  Σ(x) := (σ_1(x), …, σ_r(x), σ_{r + 1}(x), …, σ_{r + s}(x)),
\]
where \(\seq[r]{σ}\) are all real embeddings of \(K\) and \(σ_{r + 1},
\overline{σ}_{r + 1}, …, σ_{r + s}, \overline{σ}_{r + s}\) are all non-real
embeddings. Then \(Σ\) preserves sums and we obtain a group-homomorphism by
taking logarithms. More formally, we consider \(L : K^* → ℝ^{r + s}\) defined by
\[
  L(x) := (\log |σ_1(x)|, …, \log |σ_r(x)|,
           \log |σ_{r + 1}(x)|, …, \log |σ_{r + s}(x)|).
\]
Now, since the norm \(N_{K / ℚ}(u)\) is \(±1\) for every unit \(u ∈ U_K\), we
know that
\[
  |σ_1(u)| … |σ_r(u)| |σ_{r + 1}(u)|^2 … |σ_{r + s}(u)|^2 = 1
\]
and upon taking the logarithm we have that
\[
  \log |σ_1(u)| + … + \log |σ_r(u)| +
  2\log |σ_{r + 1}(u)| + … + 2 |σ_{r + s}(u)| = 0.
\]
In other words, the image \(L(U_K)\) is contained in the hyperplane \(H\)
defined by
\[
  H: x_1 + … + x_r + 2 x_{r + 1} + … + 2 x_{r + s} = 0,
\]
which is an \(r + s - 1\)-dimensional \(ℝ\)-vector space. The key to proving
Dirichlet's theorem is showing that \(L(U_K)\) can be considered as an \(r + s -
1\)-dimensional lattice in \(H\).

We will now turn our attention to approximations of real numbers by the
rationals and start with a result of Dirichlet. Dirichlet's direct proof
makes use of the pigeonhole principle. In fact, it was he who popularized this
simple combinatorial fact by giving it its German name
\foreignquote{german}{Schubfachprinzip}. However, we base our proof on
Minkowski's theorem.

\begin{thm}[Dirichlet's approximation theorem]
  For each real number \(α ∈ ℝ\) and each integer \(N > 1\) there exist integers
  \(n, p ∈ ℤ\) with \(0 < n ≤ N\) such that
  \[
    |n α - p | < \frac{1}{N}
  \]
  holds.
\end{thm}
\begin{proof}
  Consider the set
  \[
    T := \set{(x, y) ∈ ℝ^2 : -N - ½ ≤ x ≤ N + ½, |αx - y| ≤ \frac{1}{N}}.
  \]
  If we can prove that \(T\) contains a non-zero integral tuple \((n, p) ∈
  ℤ^2\), we are done, as if \(n < 0\) we can replace \(p\) by \(-p\) as well as
  \(n\) by \(-n\) and have found the claimed approximation. Note that \(n\)
  cannot be zero, as otherwise since \(1/N\) is smaller than one, \(p\) must be
  zero as well.

  As was mentioned before, we want to apply Minkowski's theorem and thus need to
  check that \(T\) is convex and symmetric in the origin. Symmetry is satisfied
  as the first condition on \(x\) is symmetric and the second condition is
  invariant under replacing \((x, y)\) by \((-x, -y)\). Geometrically, \(T\) is
  a ribbon of width \(2/N\) around the graph of \(f(x) = αx\) between the
  perpendicular lines at \(±(N + 1/2)\). In \cref{fig:dirichlet} the set \(T\)
  is represented by the shaded area, the thick line represents the function
  \(f\) and the dashed lines mark the area where the condition
  \[
    -N - ½ ≤ x ≤ N + ½
  \]
  is met.\footnote{In the example depicted \(N\) equals \(3\).}
  We need to prove that \(T\) contains a non-zero lattice point \(γ ∈ ℤ^2\). But
  this is now easy as by our geometrical observation \(T\) is convex and its
  area is equal to
  \[
    \Vol(T) = \frac{2}{N} (2 N + 1) = 4 + \frac{2}{N} > 41
  \]
  and Minkowski's theorem implies the existence of the claimed lattice point.
\end{proof}

Using the set
\[
  T := \set{(x, \seq[k]{y}) ∈ ℝ^{1 + k} : -N^k - ½ ≤ x ≤ N^k + ½,
            |α_i x - y_i| ≤ \frac{1}{N}},
\]
one proves completely analogously the multidimensional version
of Dirichlet's approximation theorem.

\begin{thm}[Multidimensional Dirichlet approximation theorem]%
  \label{thm:Dirichlet approximation}%
  Given \(k\) real numbers \(\seq[k]{α} ∈ ℝ\) and a fixed integer \(N ≥ 1\).
  There exist integers \(\seq[k]{p}, n ∈ ℤ\) with \(0 < n ≤ N^k\) such
  that for all \(1 ≤ i ≤ ℓ\) we have that
  \[
    |α_i n - p_i | ≤ \frac{1}{N}
  \]
  holds.
\end{thm}

\begin{figure}
  \begin{center}
    \includegraphics{res/dirichlet}
    \caption{The convex set \(T\) (shaded area) contains the lattice point
             \(γ\)}
    \label{fig:dirichlet}
  \end{center}
\end{figure}

Intuitively, Dirichlet's approximation theorem tells us that \(α n\) for \(α ∈
ℝ\) can be made arbitrarily close to an integer by varying \(n ∈ ℕ\). The
closely related approximation theorem by Leopold Kronecker tells us, that we
cannot only approximate integral values, but if \(α\) is irrational then \(α n -
p\) can be made arbitrarily close to a fixed \(β ∈ ℝ\) if we vary the integers
\(n\) and \(p\).

If we identify two real numbers \(x, y\) whenever there exists an integer \(p\)
with the property that \(x + p = y\) then we have constructed the additive
quotient group \(ℝ / ℤ\). Geometrically, this construction can be seen as
rolling up the half-open unit interval \([0, 1)\) to form a unit circle
(cf.~\cref{fig:unitcircle}). Considering the quotient topology on \(ℝ / ℤ\),
Dirichlet's theorem tells us that \(α n\) can be made arbitrarily close to \(0
+ ℤ\), while Kronecker's theorem states that every point of the unit circle is a
cluster point of the sequence \((α n)_{n ∈ ℕ}\) if \(α\) is irrational.

\begin{figure}
  \begin{center}
    \includegraphics{res/unitcircle}
    \caption{The half-open unit interval is rolled up to form a unit circle.
             Both can be seen as representations of the factor group \(ℝ/ℤ\)}%
    \label{fig:unitcircle}
  \end{center}
\end{figure}

In the following I will recite the presentation of the proof of Kronecker's
theorem from Chap.~2 of the textbook~\cite{Hlawka1991}. Note however that the
material is also presented in Chap.~23 of the reference~\cite{Hardy1975} and the
remark following the theorem stems from this book.

\begin{thm}[Kronecker's approximation theorem]
  For all irrational numbers \(α ∈ ℝ \setminus ℚ\), all real numbers \(β ∈ ℝ\),
  all integers \(N ≥ 1\), and all \(ε > 0\) there exist integers \(p, n ∈ ℤ\)
  with \(|n| ≥ N\) such that
  \[
    |αn - β - p| < ε
  \]
  holds.
\end{thm}
\begin{proof}
  By Dirichlet's approximation theorem there exist integers \(g, q ∈ ℤ\) with
  \(0 < q\) such that
  \[
    0 < |α q - g | < ε
  \]
  holds. Indeed, for the left inequality we notice that \(α\) is irrational and
  the right inequality follows by setting \(N > ε^{-1}\). Now we set \(n := kq\)
  and \(p := kg + c\), where the exact values of \(k\) and \(c\) will be
  determined in the course of the proof.

  We transform the expression of interest
  \begin{equation}\label{eq:Kronecker 1}
    |α n - β - p | = |k (α q - g) - β - c| =
    |αq - g| \left\vert k - \frac{β + c}{α q - g}\right\vert
  \end{equation}
  and set
  \[
    k := \left\lfloor \frac{β + c}{α q - g} \right\rfloor + 1,
  \]
  where \(\lfloor x \rfloor\) denotes the greatest integer smaller than \(x\).
  This ensures that the last factor in \eqref{eq:Kronecker 1} remains \(≤ 1\).
  Choose \(c\) to be any integer with the same sign as \(α q - g\) that
  satisfies
  \[
    |c| ≥ N |αq - g| + |β|
  \]
  and set \(k\) accordingly. Then
  \begin{align*}
    k &= \left\lfloor \frac{β + c}{α q - g} \right\rfloor + 1 ≥
         \frac{c}{α q -g } - \left\vert \frac{β}{α q - g} \right\vert ≥\\
      &≥ \left(N + \left\vert \frac{β}{α q - g} \right\vert \right) -
        \left\vert \frac{β}{α q - g} \right\vert = N > 0
  \end{align*}
  and therefore \(n = kq ≥ k ≥ N\). From \eqref{eq:Kronecker 1} we can now
  deduce that
  \[
    |α n - β - p | =
      \underbrace{|αq - g|}_{< ε}
      \underbrace{\left\vert k - \frac{β + c}{α q - g}\right\vert}_{≤ 1} < ε
  \]
  is fulfilled, which was to be shown.
\end{proof}

\begin{rem}
  Note that the condition on \(α\) being irrational in Kronecker's theorem is
  necessary. Indeed, if we assume otherwise that \(α = a/b ∈ ℚ\) then \(αn -
  \lfloor αn \rfloor\) runs only through the values
  \[
    0, \frac{1}{b}, \frac{2}{b}, …, \frac{b - 1}{b}
  \]
  for all \(n ∈ ℕ\). Thus, if we choose any \(β ∈ [0, 1)\) that is not among
  these values, then
  \[
    \set{\left\vert \frac{r}{b} - β \right\vert : 0 ≤ r < b}
  \]
  has a positive maximum, say \(δ\) and the condition
  \[
    |αn - β - p| < ε
  \]
  cannot be satisfied for \(ε < δ\).
\end{rem}

As with Dirichlet's approximation theorem there is a multidimensional version of
Kronecker's theorem as well. To state this theorem we need a definition.

\begin{defin}
  A set of real numbers \(\seq[n]{α} ∈ ℝ\) is called \emph{linearly
  independent over \(ℤ\)} if for all integers \(\seq[n]{c}\) the fact that
  \[
    \sum_{i = 1}^n c_i α_i
  \]
  is an integer implies that \(c_1 = c_2 = … = c_n = 0\).
\end{defin}

Note that a set of real numbers \(\set{\seq[ℓ]{α}} ⊂ ℝ\) is linearly independent
over \(ℤ\) if and only if \(\set{1, \seq[ℓ]{α}}\) is linearly independent over
\(ℚ\) in the sense of linear algebra. Indeed, if \(\set{\seq[ℓ]{α}}\) is
linearly independent over \(ℤ\) and there are rationals \(x_0, \seq[ℓ]{x} ∈ ℚ\)
such that
\[
  \sum_{i = 1}^ℓ x_i α_i = x_0 \cdot 1
\]
then upon multiplying with the least common multiple of the denominators of the
non-zero \(x_i\)-s we obtain a \(ℤ\)-linear combination of the \(α_i\)-s with an
integral value and thus all the \(x_i\)-s must be zero.

If on the other hand, \(\set{1, \seq[ℓ]{α}}\) is linearly independent over \(ℚ\)
then
\[
  \sum_{i = 1}^ℓ c_i α_i = a ∈ ℤ
\]
for some \(\seq[ℓ]{c} ∈ ℤ\) implies that all of the \(c_i\) (and \(a\)) must be
zero. Thus, \(\set{\seq[ℓ]{α}}\) is linearly independent over \(ℤ\).

\begin{thm}[Multidimensional Kronecker approximation theorem]%
  \label{thm:Kronecker}
  Let \(\seq[ℓ]{α}\) be real numbers that are linearly independent
  over \(ℤ\). Then for all real \(\seq[ℓ]{β} ∈ ℝ\), all \(ε > 0\), and all
  integers \(N ≥ 1\) one can find integers \(n, \seq[ℓ]{p} ∈ ℤ\) with \(|n| ≥
  N\) such that for all \(i ∈ \set{1, …, ℓ}\) the inequality
  \[
    |α_i n - β_i - p_i| < ε
  \]
  is satisfied.
\end{thm}

As did \textcite{Hlawka1991} I will present the inductive proof of
\textcite{Estermann1933} published in 1933.

\begin{proof}
  For \(ℓ = 1\) we have already carried out a proof in the previous theorem, as
  for a single real number \(α\) to be linearly independent over \(ℤ\) is the
  same as being irrational. Thus let us assume that \(ℓ > 1\) and that the claim
  holds true for all collections of less than \(ℓ\) linearly independent real
  numbers.

  We set \(δ := ε/2\) and apply the multidimensional Dirichlet
  approximation theorem~(\ref{thm:Dirichlet approximation}) to obtain integers
  \(q, \seq[ℓ]{g} ∈ ℤ\) with \(q > 0\) such that
  \[
    0 < |α_i q - g_i| < δ
  \]
  holds for all \(i ∈ \set{1, …, ℓ}\). Again, the left inequality holds
  since \(α_i\) is irrational.

  As in the one-dimensional case we set \(n = k q\) and \(p_i = k g_i + c_i\)
  for \(1 ≤ i ≤ ℓ\) and integers \(k, \seq[ℓ]{c}\) whose values will be
  determined later. Considering the expression of interest we can again obtain
  \[
    |α_i n - β_i - p_i| = |k(α_i q - g_i) - β_i - c_i| =
      |α_i q - g_i| \left\vert k - \frac{β_i + c_i}{α_i q - g_i}\right\vert
  \]
  for all \(1 ≤ i ≤ ℓ\). Now if
  \[
    k := \left\lfloor \frac{β_ℓ + c_ℓ}{α_ℓ q - g_ℓ} \right\rfloor + 1 \quad
    \text{and} \quad
    |c_ℓ| ≥ N |α_ℓ q - g_ℓ| + |β_ℓ|
  \]
  are satisfied then one obtains analogously to the one-dimensional case that
  \begin{equation}\label{eq:Kronecker 3}
    |α_ℓ n - β_ℓ - p_ℓ| < δ \quad \text{and} \quad
    |n| ≥ N
  \end{equation}
  hold.

  Let us denote \(ϑ := k - (β_ℓ + c_ℓ)/(α_ℓ q - g_ℓ)\). For \(1 ≤ j < ℓ\) we
  consider
  \begin{equation}\label{eq:Kronecker 2}
    \begin{split}
      α_j n - β_j - p_j &= α_j k q - β_j - k g_j - c_j =\\
        &= α_j q \left( \frac{β_ℓ + c_ℓ}{α_ℓ q - g_ℓ} + ϑ\right) -
           \left( \frac{β_ℓ + c_ℓ}{α_ℓ q - g_ℓ} + ϑ\right) g_j - β_j - c_j =\\
        &= c_ℓ \left(\frac{α_j q - g_j}{α_ℓ q - g_ℓ}\right) -
           \left(β_j - \frac{β_ℓ (α_j q - g_j)}{α_ℓ q - g_ℓ}\right) -
           c_j + ϑ(α_j q - g_j)
    \end{split}
  \end{equation}
  and define for \(1 ≤ j < ℓ\)
  \[
    \tilde{α}_j := \frac{α_j q - g_j}{α_ℓ q - g_ℓ} \quad \text{and} \quad
    \tilde{β}_j := β_j - \frac{β_ℓ (α_j q - g_j)}{α_ℓ q - g_ℓ}.
  \]
  I claim that the real numbers \(\seq[ℓ-1]{\tilde{α}}\) are linearly
  independent over \(ℤ\), so that the induction hypothesis can be applied to the
  \(\tilde{α}_j\) and \(\tilde{β}_j\). Indeed, if we have integers \(\seq[ℓ]{f}
  ∈ ℤ\) such that
  \[
    \sum_{j = 1}^{ℓ - 1} f_j \tilde{α}_j = -f_ℓ
  \]
  holds. We can transform this identity to the expression
  \[
    0 = f_ℓ + \sum_{j = 1}^{ℓ - 1} f_j \tilde{α}_j =
        f_ℓ + \left(\sum_{j = 1}^{ℓ - 1} f_j α_j q -
                    \sum_{j = 1}^{ℓ - 1} f_j g_j \right)
        \frac{1}{α_ℓ q - g_ℓ},
  \]
  which is equivalent to
  \[
  \sum_{j = 1}^{ℓ} f_j α_j q = \sum_{j = 1}^{ℓ} f_j g_j ∈ ℤ.
  \]
  Now since \(q\) is a non-zero integer and \(\seq[ℓ]{α}\) are linearly
  independent over \(ℤ\), we find that \(f_1 = … = f_{ℓ - 1} = f_ℓ = 0\) and
  thus the claim holds true.

  With these definitions for \(\tilde{α}_j\) and \(\tilde{β}_j\) we can deduce
  from \eqref{eq:Kronecker 2} that
  \begin{equation}\label{eq:Kronecker 4}
    |α_j n - β_j - p_j| ≤
         |c_ℓ \tilde{α}_j - \tilde{β}_j - c_j| + |α_j q - g_j| <
         |c_ℓ \tilde{α}_j - \tilde{β}_j - c_j| + δ
  \end{equation}
  holds for all \(1 ≤ j < ℓ\). We apply the inductive assumption to obtain
  an estimate of the left term in the last expression. More formally, there
  exists an integer \(\tilde{n}\) with the property
  \[
    |\tilde{n}| ≥ N |α_ℓ q - g_ℓ| + |β_ℓ|
  \]
  and integers \(\seq[ℓ-1]{\tilde{p}}\), such that for all \(j ∈
  \set{1, …, ℓ-1}\) we have that
  \[
    |\tilde{α}_j \tilde{n} - \tilde{β}_j - \tilde{p}_j| < δ.
  \]
  Since \(c_ℓ\) needs only to satisfy
  \[
    |c_ℓ| ≥ N |α_ℓ q - g_ℓ| + |β_ℓ|
  \]
  we can set \(c_ℓ := \tilde{n}\) and \(c_j := \tilde{p}_j\) (\(1 ≤ j < ℓ\))
  and therefore
  \[
    |c_ℓ \tilde{α}_j - \tilde{β}_j - c_j| < δ.
  \]
  Then we find not only that \(|n| ≥ N\) and
  \[
    |α_ℓ n - β_ℓ - p_ℓ| < δ < ε
  \]
  are satisfied by \eqref{eq:Kronecker 3} but also that
  \[
    |α_j n - β_j - p_j| < 2δ = ε
  \]
  holds true for all \(1 ≤ j < ℓ\) by \eqref{eq:Kronecker 4}. Thus, the proof is
  concluded.
\end{proof}

\subsection{Absolute values and local fields}
% ██       ██████   ██████    ███████ ██      ██████  ███████
% ██      ██    ██ ██         ██      ██      ██   ██ ██
% ██      ██    ██ ██         █████   ██      ██   ██ ███████
% ██      ██    ██ ██         ██      ██      ██   ██      ██
% ███████  ██████   ██████ ██ ██      ███████ ██████  ███████

In this section we introduce some notions required to formulate an important
principle of Helmut Hasse and Hermann Minkowski. We will only briefly discuss
these topics and refer the reader to Chap.~3 of the textbook~\cite{Neukirch2006}
or Chap.~7 of the reference~\cite{Milne2017} for a more rigour discussion.

\begin{defin}
  An \emph{absolute value} on a field \(K\) is a function \(|\cdot| : K → ℝ, x ↦
  |x|\) with the properties
  \begin{thmlist}
    \item \(|x| ≥ 0\) for all \(x ∈ K\) and \(|x| = 0\) if and only if \(x = 0\);
    \item \(|xy| = |x| |y|\) for all \(x, y ∈ K\); and
    \item \(|x + y| ≤ |x| + |y|\) for all \(x, y ∈ K\).
  \end{thmlist}
  If additionally the stronger condition
  \begin{thmlist}[resume]
    \item \(|x + y| ≤ \max(|x|, |y|)\)
  \end{thmlist}
  holds for all \(x, y ∈ K\) then \(|\cdot|\) is called a \emph{non-archimedian
  absolute value}.
\end{defin}

For notational convenience we introduce the function
\(\ord_{\mathfrak{p}}\) for all non-zero prime ideals \(\mathfrak{p} ⊂ \algint\)
mapping a non-zero fractional ideal \(\mathfrak{m} ⊂ K\) to the power of
\(\mathfrak{p}\) in its prime decomposition and \(\ord_{\mathfrak{p}}(0) := ∞\).
If \(K = ℚ\) we write \(\ord_{p}\) instead of \(\ord_{(p)}\) for all primes
\(p ∈ ℤ\).

\begin{exam}
  Let \(K\) be a number field. Then \(K\) has the following absolute values
  \begin{exlist}
    \item a trivial absolute value defined by \(|0|_1 := 0\) and \(|x|_1 :=
    1\) for all non-zero \(x ∈ K \setminus \set{0}\);

    \item one absolute value for each embedding \(σ: K → ℂ\) by setting \(|x| :=
    |σ(a)|_ℂ\), where \(|\cdot|_{ℂ}\) denotes the complex modulus; and

    \item one \emph{\(\mathfrak{p}\)-adic absolute value} for each non-zero
    prime ideal \(\mathfrak{p}\) defined by
    \[
      |x|_{\mathfrak{p}} :=
       \left(\frac{1}{ℕ\mathfrak{p}}\right)^{\ord_{\mathfrak{p}}(x\algint)},
    \]
    where \(ℕ\mathfrak{p} := [\algint : \mathfrak{p}]\).
  \end{exlist}
\end{exam}

\begin{lem}\label{lem:p adic absolute values}
  Let \(K\) be a number field and \(x  ∈ K\). Then \(x\) is an algebraic integer
  if and only if \(|x|_{\mathfrak{p}} ≤ 1\) for all prime ideals \(\mathfrak{p}
  ⊂ \algint\).
\end{lem}
\begin{proof}
  If \(x ∈ \algint\) is an algebraic integer, then \(x \algint ⊂ \algint\) is a
  principal ideal. Thus, in the factorization of \(x \algint\) into a product of
  prime ideals (cf.~\cref{thm:prime ideal factorization}) all exponents are
  non-negative. In other words \(\ord_{\mathfrak{p}}(x \algint) ≥ 0\) holds for
  all prime ideals \(\mathfrak{p}\). Since \(ℕ\mathfrak{p} > 1\) for every prime
  ideal \(\mathfrak{p}\), the \(\mathfrak{p}\)-adic absolute value
  \(|x|_{\mathfrak{p}}\) can at most equal \(1\).
  
  If, on the other hand, \(|x|_{\mathfrak{p}} > 1\) holds for some prime ideal
  \(\mathfrak{p} ⊂ \algint\), then \(\ord_{\mathfrak{p}}(x \algint)\) must be
  negative and thus \(x \algint \supsetneq \algint\) is a proper fractional
  ideal. Hence, \(x\) cannot be an algebraic integer.
\end{proof}

Note that an absolute value \(|\cdot|\) defines a metric on \(K\) by setting
\[
  d(x, y) := |x - y|.
\]
Thus, we can view \(K\) as a topological space and define two absolute values to
be \emph{equivalent} if they induce the same topology on \(K\). An equivalence
class of absolute values is called a \emph{prime} or \emph{place} of \(K\). The
\emph{completion} of a number field with respect to a \emph{prime} \(v\) is the
completion of \(K\) with respect to the topology induced by \(v\). More
formally, we consider first the set \(C_K\) of all Cauchy series in \(K\) with
respect to the prime \(v\) and notice that \(C_v\) forms a ring with respect to
pointwise addition and multiplication. A maximal ideal \(M_v\) is given by the
set of null sequences in \(K\). Thus, we can define the completion of \(K\) with
respect to \(v\) to be the quotient ring
\[
  \hat{K} := C_v / M_v.
\]
It is easy to check that this field is indeed topologically complete. Note that
one can identify \(x ∈ K\) with the equivalence class of the constant sequence
\((x, x, x, …)\). A completion is called \emph{non-archimedian} if it is induced
by a non-archimedian absolute value. As for number fields the archimedian
completions are precisely those induced by the embeddings \(σ\) of \(K\) into
\(ℂ\). If the embedding \(σ\) is real, one obtains \(ℝ\) as an completion of
\(K\), while non-real embeddings yield \(ℂ\) as a completion. The completions of
number fields are examples of so called \emph{local fields}.

We say a multivariate polynomial \(p\) is \emph{homogeneous} if all non-zero
monomials appearing in \(p\) have the same degree. Thus, \(Y_1^5 + 2 Y_1^3
Y_2^2 - 7 Y_1 Y_2^4\) is homogeneous while \(Y_1^4 - Y_1^2 Y_2\) is not. We call
a polynomial \(q ∈ R[Y_1, …, Y_n]\) a \emph{quadratic form} over an integral
domain \(R\) if \(q\) is homogeneous and has degree \(2\). If \(F\) is a field
of characteristic unequal to two, one can alternatively define a quadratic form
as a polynomial \(q(Y_1, …, Y_n)\) that can be written in the form
\[
  q(Y_1, …, Y_n) = (Y_1, …, Y_n) A
  \begin{pmatrix} Y_1 \\ \vdots \\ Y_n\end{pmatrix},
\]
where \(A ∈ M_{n}(F)\) is a symmetric \(n × n\)-matrix over \(F\). If \(A\) is
non-singular, we call \(q\) a \emph{regular} quadratic form. We say \(x ∈ R\) is
\emph{represented} by \(q\) over \(R\) if there exist \(\seq{y} ∈ R\) such that
\(x = q(\seq{y})\). We call a quadratic form \(q ∈ R[\seq{Y}]\) \emph{universal}
if it represents every element of \(R\). As for representabilty in a number
field \(K\) we have the following theorem.

\begin{thm}[Hasse-Minkowski theorem]\label{thm:Hasse Minkowski}
  A number \(x ∈ K\) is represented by a regular quadratic form \(q\) in a
  number field \(K\) if and only if \(x\) is represented by \(q\) in all
  completions of \(K\).
\end{thm}

A proof of this theorem can be found in §66 of the textbook~\cite{Meara2000}.
With regard to universal quadratic forms we have furthermore, that if the
regular quadratic form \(q\) has at least four indeterminates then \(q\) is
universal in all non-archimedian completions of \(K\). A direct application of
the Hasse-Minkowski theorem is the following lemma, taken from
\cite[Lem.~5.1.1]{Shlapentokh2007}.

\begin{lem}\label{lem:quadratic form}
  Let \(K\) be a number field and fix \(x ∈ K\). Furthermore, let \(x =
  \seq{x}\) be all the conjugates of \(x\) over \(ℚ\). Then the quadratic form
  \[
    q(Y_1, Y_2, Y_3, Y_4) := Y_1^2 + Y_2^2 + c Y_3^2 + Y_4^2
  \]
  represents \(x\) over \(K\) if \(c = \seq{c}\) are all the conjugates of \(c ∈
  K \setminus \set{0}\) over \(ℚ\) and \(c_i < 0\) whenever \(x_i < 0\).
\end{lem}
\begin{proof}
  Note that \(q\) can be written as \((Y_1, Y_2, Y_3, Y_4)A(Y_1, Y_2, Y_3,
  Y_4)^t\), where \(A\) is a non-singular diagonal matrix. Hence, \(q\) is
  regular. Since \(q\) has four indeterminates it suffices to check that \(q\)
  represents \(x_i\) over \(ℂ\) and \(ℝ\). Then the Hasse-Minkowski theorem
  implies that \(q\) represents \(x\) over \(K\). But since \(ℂ\) is
  algebraically closed, the polynomial \(X^2 - x_i\) has a root \(y_{i1}\) in
  \(ℂ\). Thus,
  \[
    x_i = y_{i1}^2 + 0^2 + c_i\,0^2 + 0^2
  \]
  is the desired representation. On the other hand, in the case of \(ℝ\) we
  distinguish two cases. If \(x_i ≥ 0\) then we can proceed as in the case of
  \(ℂ\). If \(x_i < 0\) then \(x_i / c_i\) is positive and thus a square in
  \(ℝ\). Now set \(y_{i3} := √{x_i / c_i}\) and notice that
  \[
    x_i = 0^2 + 0^2 + c_i\,y_{i3}^2 + 0^2.
  \]
\end{proof}

Furthermore, the following theorem will be useful.

\begin{thm}[Strong approximation theorem]\label{thm:strong approximation}
  Let \(K\) be a number field, let \(\mathcal{M}_K\) be the set of all the
  absolute values of \(K\), let \(\mathcal{F}_K = \set{|\cdot|_1,… ,|\cdot|_ℓ}
  ⊂ \mathcal{M}_K\) be a non-empty finite subset, and let \(a_1,…,a_{ℓ - 1} ∈
  K\). Then for any \(ε > 0\) there exists an \(x ∈ K\) such that the following
  conditions are satisfied.
  \begin{thmlist}
    \item For \(1 ≤ i ≤ ℓ - 1\) we have that \(|x − a_i|_i <ε\).

    \item For any absolute value \(|\cdot|\) not contained in \(\mathcal{F}_K\)
    we have that \(|x| ≤ 1\).
  \end{thmlist}
\end{thm}
For a proof of this theorem see §21 of the textbook~\cite{Meara2000}.
