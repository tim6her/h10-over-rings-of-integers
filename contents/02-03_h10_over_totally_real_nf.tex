% !TeX encoding = UTF-8
% !TeX TS-program = xelatex
% !TeX spellcheck = en_GB
% !TeX root = ../Herbstrith-H10_over_AI.tex

We closely follow the papers of \textcite{Denef1980} and \textcite{Pheidas1988},
who in term follow the structure of the original paper \citetitle{Davis1973} by
\textcite{Davis1973}. This way, we can prove the undecidability of Hilbert's
tenth problem over rings of algebraic integers in totally real number fields and
number fields with one pair of non-real embeddings in one go.

\subsection{Finitely many easy lemmas}

We start by defining two sequences, that satisfy the Pell equation stated below.

\begin{equation} \label{eq:Pell}
    x^2 - d y^2 = 1
\end{equation}

\begin{defin}
  Let $K$ be an algebraic number field, $\algint$ its ring of algebraic
  integers and fix $a ∈ \algint$. We define $δ(a) = \sqrt{a^2 - 1}$ and $ε(a) =
  a + δ(a)$. If $δ(a) \not\in K$ we define $x_m(a), y_m(a) ∈ \algint$ for $m ∈
  ℕ$ by

  \[
    x_m(a) + δ(a) y_m(a) = (ε(a))^m.
  \]
\end{defin}

This definition includes the case $K = ℚ$ with $\algint = ℤ$ of
\cite{Davis1973}. However, we are using the slightly modified notation of
\cite{Denef1980,Pheidas1988}. Note that the sequences $x_m(a)$ and $y_m(a)$ are
well defined for each $m ∈ ℕ$ as they correspond to the coefficients of
$(ε(a))^m$ in $K[δ(a)]/K$ with respect to the basis $\lbrace 1, δ(a)\rbrace$. If
the reference is clear we will omit the dependency on $a$ writing $δ, ε, x_m$
and $y_m$.

\begin{rem}
  As $K[δ]/K$ has degree two, there is exactly one pair of field
  automorphisms\footnote{A field extension of degree two is necessarily normal.}
  on $K[δ]$ preserving $K$ point-wise, namely $σ_1(δ) = δ$ and $σ_2(δ) = -δ$.
  The latter will be denoted by $\overline{α} = σ_2(α)$ to emphasise the analogy
  of complex conjugation.
\end{rem}

We will now collect some properties of these sequences. The proofs are
generalised versions of the ones given in \cite{Davis1973}.

\begin{lem}
  Let $K$ be an algebraic number field and $a ∈ \algint$ such that $δ(a) \not\in K$. Then
  \begin{thmlist}
    \item \label{lem:epsilon is unit}
    $ε$ is a unit in $\algint[K(δ(a))]$, its inverse is given by $ε^{-1} = a - δ = \overline{ε}$, and
    \item $x_m, y_m$ satisfy the Pell equation~\eqref{eq:Pell} for all $m ∈ ℕ$, using $d = δ(a)^2$ as parameter.
  \end{thmlist}
\end{lem}
\begin{proof}
  \begin{plist}
    \item We have $ε (a - δ) = (a + δ) (a - δ) = a^2 - δ^2 = 1$ as desired.
    \item We use induction on $m$. If $m = 0$, the pair $x_0 = 1$ and $y_0 = 0$
    yields a trivial solution to equation~\eqref{eq:Pell}. Let the claim be
    proven for all pairs $x_n, y_n$ with $n ≤ m$. Then rewriting the definition
    of $x_{m + 1}, y_{m + 1}$ we obtain

    \[
      x_{m + 1} + δ y_{m + 1} = ε^{m + 1} = (x_m + δ y_m)ε.
    \]

    Applying the automorphism $\overline \cdot$ we obtain

    \[
      \overline{x_{m + 1} + δ y_{m + 1}} = x_{m + 1} - δ y_{m + 1} = (x_m - δ y_m) ε^{-1}
    \]

    and multiplication of both equations yields

    \[
      x_{m + 1}^2 - d y_{m + 1}^2 = \overline{x_{m + 1} + δ y_{m + 1}} (x_{m + 1} - δ y_{m + 1}) = 1
    \]

    as claimed.
  \end{plist}
\end{proof}

The defining equation

\[
  x_m + δ y_m = ε^m = (x_1 + δ y_1)^m
\]

Can be seen as an analogue of the trigonometric identity

\[
  \cos m + i \sin m = e^{im} = (\cos 1 + i \sin 1)^m,
\]

where $x_m$ plays the role of $\cos m$, $y_m$ the one of $\sin m$, and $i$ is replaced by $δ$. In this view the Pell equation~\eqref{eq:Pell} is the analogue of the Pythagorean identity

\[
  \cos (m) ^2 + \sin (m) ^2 = 1.
\]

The next lemma proves the identities corresponding to $\cos m = \Re e^m$, $\sin
m = \Im e^m$, and the addition formulas.

\begin{lem}
  Let $K$ be an algebraic number field and $a ∈ \algint$ such that $δ = δ(a) \not\in K$. Then for all $m, k ∈ ℕ$ we have
  \begin{thmlist}
    \item $x_m = (ε^m + ε^{-m}) / 2$ and $y_m = (ε^m - ε^{-m}) / (2 δ)$,
    \item \label{lem:addition formulas}
    $x_{m ± k} = x_m x_k ± δ^2 y_m y_k$, and
    $y_{m ± k} = x_k y_m ± x_m y_k$
  \end{thmlist}
\end{lem}
\begin{proof}
  \begin{plist}
    \item In \cref{lem:epsilon is unit} we have seen that $ε^{-1} =
    \overline{ε}$ and therefore $ε^{-m} = \left(\overline{ε}\right)^m$. Observe that for arbitrary $x, y ∈ \algint$ we have

    \[
      x + δ y + \overline{x + δ y} = 2x \quad \text{and} \quad
      x + δ y - \overline{x - δ y} = 2δ y.
    \]

     Setting $x + δ y = ε^m$ yields the claim.
    \item By the defining equation for $x_m$ and $y_m$ we have

    \begin{align*}
      x_{m + k} + δ y_{m + k} &= ε^{m + k} = (x_m + δ y_m) (x_k + δ y_k) =\\
                            &= (x_m x_k + δ^2 y_m y_k) + δ (x_m y_k + x_k y_m)
    \end{align*}

    and thusly

    \begin{align*}
      x_{m + k} &= x_m x_k + δ^2 y_m y_k, \\
      y_{m + k} &= x_m y_k + x_k y_m.
    \end{align*}

    The identities for $x_{m - k}$ and $y_{m - k}$ follow analogously.
  \end{plist}
\end{proof}

Setting $k = 1$ in the lemma above, we obtain $x_{m ± 1} = a x_m ± δ^2 y_m$ and
$y_{m ± 1} = a y_m ± x_m$. A further immediate consequence of this lemma is the
subsequent one, which brings divisibility into play.

\begin{lem}
  Let $K$ be a number field and $a ∈ \algint$ such that $δ = δ(a) \not\in K$.
  Then for all $m, k ∈ ℕ$, $k ≠ 0$ we have
  \begin{thmlist}
    \item $y_m$ divides $y_{mk}$ in $\algint$ and
    \item $y_{mk} \equiv k x_m^{k - 1} y_m \mod \left(y_m^3\right)$ in
    $\algint$ i.e. $y_{mk} - k x_m^{k - 1} y_m$ is contained in the principal
    ideal generated by $y_m^3$ in $\algint$.
  \end{thmlist}
\end{lem}
\begin{proof}
  \begin{plist}
    \item We argue by induction on $k$. The identity is trivial if $k = 1$ and
    \cref{lem:addition formulas} implies that

    \[
      y_{m(k + 1)} = x_m y_{mk} + x{mk} y_m.
    \]

    If the claim is proven for all factors lower than $k + 1$, we find that
    $y_m | y_{mk}$ and $y_m | y_m$ trivially. As a consequence, $y_m |
    y_{m(k + 1)}.$

    \item Again the defining equation yields
    \begin{align*}
      x_{mk} + δ y_{mk} &= ε^{mk} = (x_m + δ y_m)^k = \\
                        &= \sum_{j = 0}^k \binom k j x_m^{k - j} y_m^j δ^j \\
      \intertext{and}
      y_{mk} &= \sum_{\substack{j=0\\ j \text{ odd}}}^k
                \binom k j x_m^{k - j} y_m^j δ^{j -1}.
    \end{align*}
    In the equation above all terms for $j > 1$ are divisible by $y_m^3$ and
    thus vanish modulo $\left(y_m^3\right)$. The only term remaining is $k
    x_m^{k - 1} y_m$ as claimed.
  \end{plist}
\end{proof}
