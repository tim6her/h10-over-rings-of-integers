% !TeX encoding = UTF-8
% !TeX TS-program = xelatex
% !TeX spellcheck = en_GB
% !TeX root = ../Herbstrith-H10_over_AI.tex

I closely follow the papers of \textcite{Denef1980} and \textcite{Pheidas1988},
whose structure in term heavily depends on the article
\citetitle{Davis1973} by \textcite{Davis1973}. This way, one can prove the
undecidability of Hilbert's tenth problem over rings of algebraic integers in
totally real number fields and number fields with one pair of non-real
embeddings in one go. Compare this approach to Sections 6.3 and 7.2 of the
study~\cite{Shlapentokh2007} by Shlapentokh.

\subsection{Finitely many easy lemmas}

We start by defining two sequences, that satisfy Pell's equation stated below.

\begin{equation} \label{eq:Pell}
    x^2 - d y^2 = 1
\end{equation}

Using modified versions of the techniques presented
by~\textcite{Matijasevic1970}, it will be shown that the index \(m\) can be
obtained in a Diophantine way from \(\py_m(a)\) for certain subsequences of the
sequences defined below.

\begin{defin}
  Let \(K\) be an algebraic number field, \(\algint\) its ring of algebraic
  integers and fix \(a ∈ \algint\). One defines \(δ(a) := \sqrt{a^2 - 1}\) and
  \(ε(a) := a + δ(a)\), where we demand that \(-π/2 < \arg δ(a) ≤ π/2\). If
  \(δ(a) \not\in K\) one defines \(\px_m(a), \py_m(a) ∈ \algint\) for \(m ∈ ℕ\)
  by
  \[
    \px_m(a) + δ(a) \py_m(a) = {(ε(a))}^m.
  \]
\end{defin}

This definition includes the case \(K = ℚ\) with \(\algint = ℤ\) of~\cite{Davis1973}. However, I am using the slightly modified notation of~\cite{Denef1980,Pheidas1988}.

Under the assumptions of the definition \(δ(a)\) is a root of the monic
quadratic polynomial
\[
  X^2 - a^2 + 1 ∈ \algint {[X]}.
\]
Therefore, the extension \(K[δ(a)] / K\) is quadratic and \(δ(a) ∈
\algint[{K[δ(a)]}]\). The sequences \(\px_m(a)\) and \(\py_m(a)\) are well
defined for each \(m ∈ ℕ\) as they correspond to the coefficients of
\({(ε(a))}^m\) in \(K[δ(a)]/K\) with respect to the basis \(\lbrace 1,
δ(a)\rbrace\). If the reference is clear, I will omit the dependency on \(a\)
writing \(δ, ε, \px_m\) and \(\py_m\).
% QUESTION
\todo{Why does every number field contain such an element \(a\)?}
In the following the number field \(K[δ(a)]\) will be denoted by \(L\).

\begin{rem}
  As \(L/K\) has degree two, there is exactly one pair of field
  automorphisms\footnote{A field extension of degree two is necessarily normal.}
  on \(L\) preserving \(K\) point-wise, namely \(σ_1^{K}(α + δβ) = α + δβ\) and
  \(σ_2^K(α + δβ) = α - δβ\) for \(α, β ∈ \algint\). The latter will be denoted
  by \(\overline{η} = σ_2^K(η)\) to emphasize the analogy of complex
  conjugation.
\end{rem}

Let me now collect some properties of these sequences. The proofs are
generalized versions of the ones given in~\cite{Davis1973}.

\begin{lem}
  Let \(K\) be an algebraic number field and \(a ∈ \algint\) such that \(δ(a) \not\in K\). Then
  \begin{thmlist}
    \item\label{lem:epsilon is unit} \(ε\) is a unit in \(\algint[L]\), its
    inverse is given by \(ε^{-1} = a - δ = \overline{ε}\), and
    \item \(\px_m, \py_m\) satisfy Pell's equation~\eqref{eq:Pell} for all \(m
    ∈ ℕ\), using \(d = {δ(a)}^2\) as parameter.
  \end{thmlist}
\end{lem}
\begin{proof}
  \begin{plist}
    \item We have \(ε \; (a - δ) = (a + δ) (a - δ) = a^2 - δ^2 = 1\) as desired.
    \item One uses induction on \(m\). If \(m = 0\), the pair \(\px_0 = 1\) and \(\py_0 =
    0\) yields a trivial solution to equation~\eqref{eq:Pell}. Let the claim be
    proven for all pairs \(\px_n, \py_n\) with \(n ≤ m\). Then rewriting the definition
    of \(\px_{m + 1}, \py_{m + 1}\) one obtains
    \[
      \px_{m + 1} + δ \py_{m + 1} = ε^{m + 1} = (\px_m + δ \py_m)ε.
    \]
    Applying the automorphism \(\overline \cdot\) implies
    \[
      \overline{\px_{m + 1} + δ \py_{m + 1}} = \px_{m + 1} - δ \py_{m + 1} = (\px_m - δ \py_m) ε^{-1}
    \]
    and multiplication of both equations yields
    \[
      \px_{m + 1}^2 - d \py_{m + 1}^2 = \overline{\px_{m + 1} + δ \py_{m + 1}} (\px_{m + 1} - δ \py_{m + 1}) = 1
    \]
    as claimed.
  \end{plist}
\end{proof}

The defining equation
\[
  \px_m + δ \py_m = ε^m = {(\px_1 + δ \py_1)}^m
\]
can be seen as an analogue of the trigonometric identity
\[
  \cos m + i \sin m = e^{im} = {(\cos 1 + i \sin 1)}^m,
\]
where \(\px_m\) plays the role of \(\cos m\), \(\py_m\) the one of \(\sin m\), and \(i\) is replaced by \(δ\). In this view Pell's equation~\eqref{eq:Pell} is the analogue of the Pythagorean identity
\[
  {\cos (m)}^2 + {\sin (m)}^2 = 1.
\]

The next lemma proves the identities corresponding to \(\cos m = \Re e^{im}\),
\(\sin m = \Im e^{im}\), and the addition formulas.

\begin{lem}
  Let \(K\) be an algebraic number field and \(a ∈ \algint\) such that \(δ = δ(a) \not\in K\). Then for all \(m, k ∈ ℕ\) one has
  \begin{thmlist}
    \item\label{lem:real part of epsilon}
    \(\px_m = (ε^m + ε^{-m}) / 2\) and \(\py_m = (ε^m - ε^{-m}) / (2 δ)\), as well as,
    \item\label{lem:addition formulas}
    \(\px_{m ± k} = \px_m \px_k ± δ^2 \py_m \py_k\), and
    \(\py_{m ± k} = \px_k \py_m ± \px_m \py_k\).
  \end{thmlist}
\end{lem}
\begin{proof}
  \begin{plist}
    \item In \cref{lem:epsilon is unit} we have seen that \(ε^{-1} =
    \overline{ε}\) and therefore \(ε^{-m} = {\left(\overline{ε}\right)}^m\). Observe that for arbitrary \(α, β ∈ \algint\) we have
    \[
      α + β y + \overline{α + δ β} = 2α \quad \text{and} \quad
      α + β y - \overline{α - δ β} = 2δ β.
    \]
    Now, setting \(α + δ β = ε^m\) yields the claim.
    \item By the defining equation for \(\px_m\) and \(\py_m\) we have
    \begin{align*}
      \px_{m + k} + δ \py_{m + k} &= ε^{m + k} = (\px_m + δ \py_m) (\px_k + δ \py_k) =\\
                            &= (\px_m \px_k + δ^2 \py_m \py_k) + δ (\px_m \py_k + \px_k \py_m)
    \end{align*}
    and thusly
    \begin{align*}
      \px_{m + k} &= \px_m \px_k + δ^2 \py_m \py_k, \\
      \py_{m + k} &= \px_m \py_k + \px_k \py_m.
    \end{align*}
    The identities for \(\px_{m - k}\) and \(\py_{m - k}\) follow analogously.
  \end{plist}
\end{proof}

Setting \(k = 1\) in the lemma above, one obtains \(\px_{m ± 1} = a \px_m ± δ^2 \py_m\)
and \(\py_{m ± 1} = a \py_m ± \px_m\). A further immediate consequence of this lemma is
the subsequent one, which brings divisibility into play.

\begin{lem}
  Let \(K\) be a number field and \(a ∈ \algint\) such that \(δ = δ(a) \not\in K\).
  Then for all \(m, k ∈ ℕ\), \(k ≠ 0\) we have
  \begin{thmlist}
    \item \(\py_m\) divides \(\py_{mk}\) in \(\algint\) and
    \item \(\py_{mk} \equiv k \px_m^{k - 1} \py_m \mod \left(\py_m^3\right)\) in
    \(\algint\) i.e. \(\py_{mk} - k \px_m^{k - 1} \py_m\) is contained in the principal
    ideal generated by \(\py_m^3\) in \(\algint\).
  \end{thmlist}
\end{lem}
\begin{proof}
  \begin{plist}
    \item I argue by induction on \(k\). The claim is trivial if \(k = 1\) and
    \cref{lem:addition formulas} implies that
    \[
      \py_{m(k + 1)} = \px_m \py_{mk} + x{mk} \py_m.
    \]

    If the claim is proven for all factors lower than \(k + 1\), one finds that
    \(\py_m \mid \py_{mk}\) and \(\py_m \mid \py_m\) trivially. As a consequence, \(\py_m \mid
    \py_{m(k + 1)}\).

    \item Again the defining equation yields
    \begin{align*}
      \px_{mk} + δ \py_{mk} &= ε^{mk} = {(\px_m + δ \py_m)}^k = \\
                        &= \sum_{j = 0}^k \binom k j \px_m^{k - j} \py_m^j δ^j \\
      \intertext{and}
      \py_{mk} &= \sum_{\substack{j=0\\ j \text{ odd}}}^k
                \binom k j \px_m^{k - j} \py_m^j δ^{j -1}.
    \end{align*}
    In the equation above all terms for \(j > 1\) are divisible by \(\py_m^3\) and
    thus vanish modulo \(\left(\py_m^3\right)\). The only term remaining is \(k
    \px_m^{k - 1} \py_m\) as claimed.
  \end{plist}
\end{proof}

The next lemma even though being easy to prove provides a valuable tool in
studying the sequences \(\px_m\) and \(\py_m\). It derives an recursive definition and
lets one prove properties of the sequences, by proving them for \(m ∈ \lbrace
0, 1 \rbrace\) and inferring the properties for \(m + 1\) from \(m\) and \(m - 1\).

\begin{lem}\label{lem:recursion for x_m and y_m}
  Let \(K\) be a number field and \(a ∈ \algint\) such that \(δ = δ(a) \not\in
  K\). For \(m > 1\) the following recursive conditions hold in $\algint$.
  \begin{align*}
    \px_{m + 1} &= 2 a \px_m - \px_{m - 1}, & \px_1 = a, \;& \px_0 = 1 \\
    \py_{m + 1} &= 2 a \py_m - \py_{m - 1}, & \py_1 = 1, \;& \py_0 = 0 \\
  \end{align*}
\end{lem}
\begin{proof}
  The initial conditions follow from \(ε = a + δ\) and \(ε^0 = 1\). To prove the
  the difference equations one uses \cref{lem:addition formulas} and obtains
  \begin{align*}
    \px_{m + 1} &= a \px_m + δ^2 \py_m,  &  \py_{m + 1} &= a \py_m + \px_m, \\
    \px_{m - 1} &= a \px_m - δ^2 \py_m,  &  \py_{m - 1} &= a \py_m - \px_m.
  \end{align*}
  Summation yields \(\px_{m + 1} + \px_{m - 1} = 2 a \px_m\) and \(\py_{m + 1} + \py_{m - 1}
  = 2 a \py_m\).
\end{proof}

One applies the previous lemma to prove some congruence conditions.

\begin{lem}
  Let \(K\) be a number field and \(a, b, c ∈ \algint\) such that \(δ(a), δ(b)
  \not\in K\). Then for all \(m ∈ ℕ\) the following congruences hold in
  $\algint$.
  \begin{plist}
    \item \(\py_m (a) \equiv m \mod (a - 1)\)
    \item If \(a \equiv b \mod (c)\), then \(\px_m (a) \equiv \px_m (b) \mod (c)\) and
    \(\py_m(a) \equiv \py_m(b) \mod (c)\).
  \end{plist}
\end{lem}
\begin{proof}
  Both congruences become equalities if \(m = 0\). As for \(m = 1\),
  the first congruence is again an equality as \(\py_1 (a) = 1\) independently of
  \(a\). The second claim is trivial since \(x_1 (η) = η\) and \(\py_1 (η) = 1\) for \(η
  ∈ \lbrace a, b \rbrace\). At this point one proceeds inductively and assumes
  the claims to be proven for all indices lower than \(m + 1\).

  \begin{plist}
    \item Note that \(a \equiv 1 \mod (a - 1)\) and thusly by
    \cref{lem:recursion for x_m and y_m}
    \[
      \py_{m + 1} = 2 a \py_m - \py_{m - 1} \equiv 2 m - (m - 1) = m + 1 \mod (a - 1)
    \]
    as claimed.

    \item Using \cref{lem:recursion for x_m and y_m} we see that for fixed \(m\)
    the coefficients \(\px_m (η)\) and \(\py_m (η)\) can be expressed as some fixed
    polynomial in \(η\). For the congruence this means
    \begin{align*}
      \px_{m + 1} (a) &= 2 a \px_m (a) - \px_{m - 1} (a)
                     \equiv 2 b \px_m (b) - \px_m{m - 1} (b) = \px_{m + 1} (b)
                     \mod (c)
    \end{align*}
    and for \(\py_{m + 1}\) completely analogously.
  \end{plist}
\end{proof}

\begin{lem}\label{lem:congruence x_2m+k}
  Let \(K\) be a number field and \(a ∈ \algint\) such that \(δ = δ(a) \not\in
  K\). Then for \(m, k ∈ ℕ\) such that \(m ± k ≥ 0\) the following congruence
  holds in $\algint$.
  \[
    \px_{2 m ± k} \equiv - \px_k \mod (\px_m)
  \]
\end{lem}
\begin{proof}
  By applying the addition formulas of \cref{lem:addition formulas} twice and
  using that \(\px_m\) and \(\py_m\) solve Pell's equation~\eqref{eq:Pell} one obtains
  \begin{align*}
    \px_{2m ± k} &= \px_m \px_{m ± k} + δ^2 \py_m \py_{m ± k}
                \equiv δ^2 \py_m (\py_m \px_k ± \px_m \py_k) \equiv\\
               &\equiv δ^2 \py_m^2 \px_k = (\px_m^2 - 1) \px_k
                \equiv -\px_k \mod (\px_m).
  \end{align*}
\end{proof}

At this point for the first time in this section I state a result that is no
direct generalization of a result in~\cite{Davis1973} and present proofs given
in~\cite{Denef1980}. Note however that the results are nevertheless true for the
case \(K = ℚ\) and \(\algint = ℤ\).

\begin{lem}
  Let \(K\) be a number field and \(a ∈ \algint\) such that \(δ = δ(a) \not\in
  K\). Then for all non-negative integers \(k, m ∈ ℕ\) the following congruence
  holds in \(\algint\).~\cite[cf.][Lem.~6.3.2.(2)]{Shlapentokh2007}
  \[
    \px_{2km} \equiv (-1)^k \mod (\px_m)
  \]
\end{lem}
\begin{proof}
  If \(k = 0\) the congruence becomes and identity and if \(k = 1\) the claim
  follows directly from the lemma above. Assuming the claim to be proven for all
  integers lower than \(k\), we find---by applying \cref{lem:addition formulas}
  twice---that
  \begin{align*}
    \px_{2km} &= \px_{2(k-1)m}\px_{2m} + δ^2 \py_{2 (k-1) m}\py_{2m} \equiv
               (-1)^k + δ^2 \py_{2 (k-1) m}\py_{2m} = \\
              &= (-1)^k + δ^2 \py_{2 (k-1) m} \; 2 \px_m\py_m \equiv
               (-1)^k \mod (\px_m)
  \end{align*}
\end{proof}

\begin{lem}
  Let \(K\) be a number field and \(a ∈ \algint\) such that \(δ = δ(a) \not\in K\).
  Then for all \(η ∈ \algint \setminus \set{0}\) there exists an \(m ∈ ℕ\) such that
  \(η \mid \py_m\) in \(\algint[{L}]\).
  % QUESTION
  \todo{Is \(η \mid y\) in  \(\algint\)?}
\end{lem}
\begin{proof}
  The quotient of the non-trivial, finitely generated \(\algint\)-modules \(\algint[L]\) and \((2 δ η)\)
  % TODO
  \todo{Do I need this lemma?}

  Let \(m\) be the order of the group of units in the finite ring
  \(\algint[L]/(2 δ η)\), where \((2 δ η)\) denotes the principal ideal
  generated by \(2 δ η\) in \(\algint[{L}]\). Then \(ε^{±m} \equiv 1 \mod (2 δ
  η)\). Hence, \(2 δ η \mid ε^m - ε^{-m}\) in \(\algint[{L}]\) and therefore
  \[
    \left. η \;\middle\vert\; \frac{ε^m - ε^{-m}}{2 δ} \right.
  \]
  in \(\algint[{L}]\), where the right hand side equals \(\py_m\) by
  \cref{lem:real part of epsilon}.
\end{proof}

\begin{lem}\label{lem:subgroup of ker N L/K}
  Let \(K\) be a number field and \(a ∈ \algint\) such that \(δ = δ(a) \not\in
  K\). Then the set
  \[
    S = \set{α + δ β \mid (α, β) ∈ \algint^2 \text{ is a solution to~\eqref{eq:Pell} with parameter } d = δ^2}
  \]
  is a subgroup of the kernel of the norm map \(N_{L/K}: U_L → U_K\), where \(U_K\) and \(U_L\) denote the groups of units in \(\algint\) and \(\algint[L]\) respectively.
\end{lem}
\begin{proof}
  First of all, note that, if \(α + δ β ∈ S\), so is \(\overline{α + δ β} = α - δ
  β ∈ S\) because
  \[
    α^2 - d {(-β)}^2 = α^2 - d β^2 = 1.
  \]
  Now let \(α + δ β\) be an arbitrary element of \(S\), then
  \[
    N_{L/K}(α + δ β) = (α + δ β) \left(\overline α + δ β \right) = α^2 - d β^2 = 1.
  \]
  This implies that \(α + δ β ∈ \ker N_{L / K}\) but also that \(α + δ β\) is a
  unit, as \(α - δ β\) is its inverse. The product of two arbitrary elements \(α_1 + δ β_1, α_2 + δ β_2 ∈ S\) is
  \[
    (α_1 + δ β_1)(α_2 + δ β_2) = (α_1 α_2 + δ^2 β_1 β_2) + δ (α_1 β_2 + α_2 β_1).
  \]
  We apply the automorphism \(\overline\cdot\) and multiply to obtain
  \[
    (α_1 + δ β_1)(α_2 + δ β_2)\,\overline{(α_1 + δ β_1)(α_2 + δ β_2)} =
    (α_1 + δ β_1)\overline{(α_1 + δ β_1)}(α_2 + δ β_2)\overline{(α_2 + δ β_2)} = 1.
  \]
  As a consequence, \(S\) is closed under multiplication and the claim is
  proven.
\end{proof}

\begin{lem}\label{lem:rank of N_L/K U_L}
  Let \(L\) and \(K\) be number fields as defined above.
  The image \(N_{L / K}\left(U_L\right) ≤ U_K\) has finite index in \(U_K\).
\end{lem}
\begin{proof}
  I claim that \(N_{L / K}\left(U_L\right)\) contains \(α^2\) for every \(α ∈ U_K\).
  This is because the restriction \(σ_i^K|_{\algint}\) is just the identity on
  \(\algint\) for \(i ∈ \set{1, 2}\) and therefore, \(N_{L / K}(α) = α^2\) for all
  \(α ∈ U_K \subseteq U_L\).

  Let now \(k := \rk K\) and identify \(U_K = ℤ_ℓ \times ℤ^{k}\), where \(ℤ_ℓ =
  μ(K)\) is the finite cyclic group of roots of unity in \(K\)
  (cf.~\cref{thm:Dirichlet}). Consider the following \(k\) elements
  \[
    (\overline 0,1,0,…,0), \; (\overline 0,0,1,0,…,0), \; …, \; (\overline 0, 0, …, 0, 1)
  \]
  contained in \(U_K\). By the claim their ‘squares’ are contained in \(N_{L / K}\left(U_L\right)\) i.e.
  \[
    (\overline 0,2,0,…,0), \; (\overline 0,0,2,0,…,0), \; …, \; (\overline 0, 0, …, 0, 2) ∈ N_{L / K}\left(U_L\right).
  \]
  As a consequence, the direct product
  \[
    G := \set{\overline 0} \times \underbrace{2 ℤ \times … \times 2 Z}_{k\text{-times}}
  \]
  is a subgroup of \(N_{L / K}\left(U_L\right)\) and therefore
  \[
    [U_K : N_{L / K}\left(U_L\right)] ≤ [U : G] ≤ ℓ\, 2^k < ∞.
  \]
\end{proof}

As for the free ranks of \(U_K\), \(U_L\), \(N_{L / K}\left(U_L\right)\) and \(S\) the
lemma above implies that \(\rk N_{L / K}\left(U_L\right) = \rk U_K\) and
therefore, as an immediate consequence of the first isomorphism theorem~\cite[see][II~§1, p.~89]{Lang2002} the following inequality holds
\begin{equation}\label{eq:rank of S}
  \rk S ≤ \rk \ker N_{L / K} = \rk U_L - \rk U_K.
\end{equation}

Before proving the main theorems of this section (\todo{reference them}) I
sketch how \textcite{Davis1973} proceeds in proving that \textsc{H10} is
undecidable over \(ℤ\).

First he proves using the sequences above that the exponential function is
Diophantine~\cite[Thm 3.3]{Davis1973}. Then he is able to extend the language of
Diophantine predicates by \emph{bounded existential} and \emph{bounded universal
quantifiers} i.e.\ by
\begin{align*}
  {(∃y)}_{≤x}ϕ(x, y) &⇔ ∃y\; (y ≤ x ∧ ϕ(x, y)),\\
  {(∀y)}_{≤x}ϕ(x, y) &⇔ ∀y\; (y > x ∨ ϕ(x, y))
\end{align*}
where \(ϕ\) is a positive existential formula~\cite[Thm 5.1]{Davis1973}. The first
one is easily seen to be Diophantine as the order relation on \(ℕ^{*}\) is
Diophantine. Proving the second claim takes the rest of the section. Now using
this result together with the sequence number theorem~\cite[Thm 1.3]{Davis1973}
Davis proofs that a function is Diophantine if and only if it is computable~\cite[Thm 6.1]{Davis1973}.

This already implies the undecidability of \textsc{H10} over \(ℤ\) as Davis
introduced Diophantine pairing functions in~\cite[Thm 1.1]{Davis1973} and
therefore all domains of Diophantine and therefore all computable functions are
Diophantine. But the domains of computable functions are exactly the
semi-decidable subsets of \(ω\) and we have already met a semi-decidable set that
is not decidable, namely the halting set \(\mathcal{K}\). As a consequence, there is a
Diophantine set that is not decidable, or put differently there exists a
polynomial \(p_{\mathcal{K}} ∈ ℤ[X, \seq{Y}]\), such that for every Turing machine \(\mathbb
A\), that can decide some polynomial identities, there exists an \(α ∈ ℤ\) such
that \(\mathbb A\) cannot correctly decide
\[
  ∃\seq{β}: p_{\mathcal{K}}(α, \seq{β}) = 0.
\]

\subsection{Diophantine definition of \(ℤ\) over \(K\)}

For the remainder of this section let \(K\) be a totally real number field or a
number field with exactly one pair of non-real embeddings of degree \(n := [K :
ℚ] ≥ 3\) over the rationals \(ℚ\). In the notation of \cref{thm:Dirichlet} this
means that either \(r = n > 1\), or \(r = n - 2 > 0\) and \(s = 1\).

Furthermore, let us assume that \(σ_1 = \id_K, σ_2, …, σ_n: K → ℂ\) are all
embeddings of \(K\) into the complex pane \(ℂ\). If \(s = 1\) we demand without
loss of generality that $K, σ_2(K) \not\subset ℝ$ and that $σ_2(α) = α$ for all
\(α ∈ K\). In other words, \((σ_1, σ_2)\) is the pair of complex embeddings and
all other morphisms embed \(K\) into the reals \(ℝ\).

\begin{lem}\label{lem:L over K is quadratic}
  Let \(K\) be a number field of degree \(n > 1\) over \(ℚ\). If \(a ∈ \algint\) satisfies
  \begin{equation}\label{eq:approximations of a}
    \begin{cases}
      r = n\\
      a > 2^{2(n + 1)}\\
      0 < σ_i(a) < \frac{1}{2} &\text{for } 1 < i ≤ n
    \end{cases}
    \quad \text{or} \quad
    \begin{cases}
      r = n - 2\\
      |σ_i(a)| > 2^{2(n + 1)} &\text{for } i ∈ \set{1, 2}\\
      0 < σ_i(a) < \frac{1}{2} &\text{for } 2 < i ≤ n
    \end{cases},
  \end{equation}
  then \(δ(a) = \sqrt{a^2 - 1}\) is not contained in \(K\).
\end{lem}
\begin{proof}
  By assumption we have \(0 < σ_n(a) < 1/2\) and therefore
  \(a^2 - 1 < 0\) cannot be a square in \(σ_i(K)\). But then \(δ(a) = \sqrt{a^2 - 1}\) cannot be in \(K\).
\end{proof}

We will prove later using Minkowski's theorem on convex bodies
(\cref{thm:Minkowski}), that an algebraic integer $a ∈ \algint$ satisfying
\eqref{eq:approximations of a} exists and therefore the condition on \(a\) from
the previous section holds.

\begin{rem}
  As the expansion \(L = K[δ] / K\) is quadratic by \cref{lem:L over K is
  quadratic}, every \(σ_i\) can be extended to exactly two embeddings \(σ_{i1}\)
  and \(σ_{i2}\) of \(L\) into the complex plane \(ℂ\) by ‘composing’ with
  \(σ_1^K\) or \(σ_2^K\). This yields
  \begin{equation}\label{eq:def of sigma ij}
    \begin{aligned}
      σ_{i1}(α + δβ) &= σ_i(α) + \sqrt{{σ_i(a)}^2 - 1}\, σ_i(β) \quad \text{and} \\
      σ_{i1}(α + δβ) &= σ_i(α) - \sqrt{{σ_i(a)}^2 - 1}\, σ_i(β)
    \end{aligned}
  \end{equation}
  for all \(α, β ∈ \algint\) and all \(1 ≤ i ≤ n\).
\end{rem}

I will identify the field \(L\) with its embedding \(σ_{11}(L)\) and write \(η\)
instead of \(σ_{11}(η)\) for its elements.

\begin{lem}
  Let \(K\) be a number field of degree \(n > 1\) over \(ℚ\) and let \(a ∈
  \algint\) be such that \eqref{eq:approximations of a} is satisfied. Then
  \begin{thmlist}
    \item if \(r = n\), only \(σ_{11}\) and \(σ_{12}\) embed \(L\) into the
    reals, and
    \item if \(r = n - 2\), the field \(L\) is totally complex.
  \end{thmlist}
\end{lem}
\begin{proof}
  \begin{plist}
    \item If \(K\) is totally real and \(i > 1\), then \(0 < σ_i(a) < 1/2\) and
    therefore the radicands in \eqref{eq:def of sigma ij} are both negative. As
    a consequence, \((σ_{i1}, σ_{i2})\) is a pair of non-real embeddings.

    On the other hand, if \(i = 1\) then \(a > 2^{2(n + 1)} > 1\) and the
    radicands are both positive. We deduce that \(σ_{11}\) and \(σ_{12}\) are
    both real embeddings and \(L\) is a subfield of the reals by our
    identification.
  \end{plist}
\end{proof}

degree \(n > 1\) over the rationals \(ℚ\), and let \(\seq{σ}\) denote the embeddings
of \(K\) into the reals \(ℝ\). Assume \(a ∈ \algint\) satisfies \(δ = δ(a) \not\in
\algint\) as above and additionally
\begin{equation} \label{eq:embeddings of a into reals v1}
  σ_1(a) ≥ 2^{2n}, \quad |σ_i(a)| ≤ \frac 12, \quad \text{for } 2 ≤ i ≤ n.
\end{equation}
% QUESTION
\todo{Does this imply that delta is not in O K?}

Then note firstly that \(a\) cannot be a rational integer as they are preserved
point-wise by each embedding \(σ_i\) for \(1 ≤ i ≤ n\). As the
extension \(L/K\) is quadratic, every \(σ_i\) can be extended to exactly two
embeddings \(σ_{i1}\) and \(σ_{i2}\) of \(L\) into the complex plane \(ℂ\) by
‘composing’ with \(σ_1^K\) or \(σ_2^K\). This yields
\[
  σ_{i1}(α + δβ) = σ_i(α) + \sqrt{{σ_i(a)}^2 - 1} σ_i(β) \quad \text{and} \quad
  σ_{i1}(α + δβ) = σ_i(α) - \sqrt{{σ_i(a)}^2 - 1} σ_i(β)
\]
for all \(α, β ∈ \algint\). As \(|σ_i(a)| ≤ 1/2\) for \(2 ≤ i ≤ n\) the radicands
in the equation above are both negative and as a consequence all \(σ_{i1},
σ_{i2}\) are conjugate pairs of non-real embeddings of \(L\). Only \(σ_{11}\) and
\(σ_{12}\) embed \(L\) into the reals \(ℝ\). I will identify \(L\) with the image of its
embedding \(σ_{11}(L) \subseteq ℝ\) and write \(η\) instead of \(σ_{11}(η)\) for all
\(η ∈ L\).

\begin{lem}
  Suppose \(K\) is a totally real number field and \(a ∈ \algint\) satisfies \(δ =
  δ(a) \not\in \algint\) as well as~\eqref{eq:embeddings of a into reals v1}.
  Then for all positive integers \(m\), and all pairs \((i, j) ∈ \set{2, …, n}
  \times \set{1, 2}\) the following inequalities hold
  \begin{thmlist}
    \item \(a / 2 < δ < a\);
    \item \(\Re (σ_{ij}(δ))= 0\), i.e.\ the image of \(δ\) is imaginary, and \(1 / 2 < | σ_{ij}(δ) | < 1\);
    \item \(a < ε < 2a\) and \(| σ_{ij}(ε) | = 1\);
    \item \(ε^m / (4a) < \py_m < ε^m / a\) and \(|σ_i(\py_m)| < 2\);
    \item \(ε^m / 2 < \px_m < ε^m\) and \(|σ_i(\px_m)| < 1\).
  \end{thmlist}
\end{lem}
\begin{proof}
  As \(a ∈ \algint\) the images \(σ_{11}(a) = a\) and \(σ_1(a)\) coincide. Therefore, \(a ≥ 2^{2n}\) by~\eqref{eq:embeddings of a into reals v1}.
  \begin{plist}
    \item We have
      \[
        \frac{a^2}{4} = a^2 - \frac{3a^2}{4} ≤ a^2 - \frac{3}{4} 2^{4n} < a^2 - 1 < a^2
      \]
      As \(a > 0\) and \(δ = \sqrt{a^2 - 1} ≥ 0\) one can take square roots in the inequality above and obtain the claim.
    \item As was mentioned above, by~\eqref{eq:embeddings of a into reals v1}
      the radicand in \(σ_i(δ) = \sqrt{{σ_i(a)}^2 - 1}\) is negative yielding the
      first claim. To approximate the absolute value of \(σ_i(δ)\), one notes
      \[
        |σ_i(δ)|^2 = {\Im(σ_i(δ))}^2 = 1 - {σ_i(a)}^2 ≥ 1 - \frac{1}{4} ≥ \frac{1}{4}
      \]
      and, since \({σ_i(a)}^2 ≥ 0\),
      \[
        |σ_i(δ)|^2 ≤ 1
      \]
      as well. Again taking square roots yields the claim.
    \item By (i) \(0 < δ < a\) and therefore
      \[
        a < a + δ = ε < 2a.
      \]
      The other claim follows from
      \[
        |σ_{ij}(ε)|^2 = σ_{ij}(ε) \; \overline{σ_{ij}(ε)} =
        \left(σ_i(a) + σ_{i1}(δ)\right) \; \left(σ_i(a) - σ_{i1}(δ)\right) =
        {σ_i(a)}^2 - ({σ_i(a)}^2 - 1) = 1,
      \]
      where the second identity is a consequence of (ii).
    \item This should follow from \cref{lem:real part of epsilon}.
      % TODO
      \todo{write me}
    \item This should follow from \cref{lem:real part of epsilon}.
      % TODO
      \todo{write me}
  \end{plist}
\end{proof}

As a next step, we want to show that we have already found all solutions of
Pell's equation~\eqref{eq:Pell}. For this we need some lemmas.

Recall the group \(S ≤ \ker N_{L / K}\) defined in \cref{lem:subgroup of ker N
L/K}. We have seen in \cref{lem:rank of N_L/K U_L} and the subsequent
inequality~\eqref{eq:rank of S} that the free rank of \(S\) can be bound from
above. Note that \(ε\) is contained in \(S\). Therefore, the following lemma implies
that \(\rk S\) can be bound from below by \(1\).

\begin{lem}\label{lem:rank of S is greater 0}
  Let \(K\) be a totally real number field and \(a ∈ \algint\) satisfies \(δ =
  δ(a) \not\in \algint\) as well as~\eqref{eq:embeddings of a into reals v1}.
  Then \(ε(a) = ε\) is torsion free i.e.\ there is a group isomorphism \(⟨ε⟩ \cong
  ℤ\).
\end{lem}
\begin{proof}
  Assume otherwise that \(ε^m = 1\) for some positive integer \(m\) i.e. \(ε\) is an
  \(m\)-th root of unity. As we have identified \(L\) with a subset of \(ℝ\), this
  leaves \(±1\) as values for \(ε = 1 + δ\). This is impossible as \(δ \not\in
  \algint\).
\end{proof}

As a consequence, of \cref{thm:Dirichlet} we find that
\[
  1 ≤ \rk S ≤ \rk U_L - \rk U_K ≤ (n + 1) - n = 1.
\]
In other words, we can write every \(η\) in \(S\) as a product \(ζ ε_0^k\), where \(ζ ∈ U_L\) is a root of unity and \(ε_0\)
