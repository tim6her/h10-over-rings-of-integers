% !TEX encoding = UTF-8
% !TEX TS-program = xelatex
% !TEX spellcheck = en_GB
% !TEX engine = xelatex
% !TEX root = ../Herbstrith-H10_over_AI.tex
%
% ██   ██  ██  ██████      ████████  ██████  ████████    ██████  ███████
% ██   ██ ███ ██  ████        ██    ██    ██    ██       ██   ██ ██
% ███████  ██ ██ ██ ██        ██    ██    ██    ██ █████ ██████  █████
% ██   ██  ██ ████  ██        ██    ██    ██    ██       ██   ██ ██
% ██   ██  ██  ██████         ██     ██████     ██       ██   ██ ███████ ██

I closely follow the papers of \textcite{Denef1980} and \textcite{Pheidas1988},
whose structure in term heavily depends on the article \citetitle{Davis1973} by
\textcite{Davis1973}. This way, one can prove the undecidability of Hilbert's
tenth problem over rings of algebraic integers in totally real number fields and
number fields with one pair of non-real embeddings and at least one real
embedding in one go. This approach is also present in Sections 6.3 and 7.2 of
the study~\cite{Shlapentokh2007} by Shlapentokh. The same author has proven the
second result independently of Pheidas in her thesis~\cite{Shlapentokh1989}.

\subsection{Finitely many easy lemmas}
% ██      ███████ ███    ███ ███    ███  █████  ███████
% ██      ██      ████  ████ ████  ████ ██   ██ ██
% ██      █████   ██ ████ ██ ██ ████ ██ ███████ ███████
% ██      ██      ██  ██  ██ ██  ██  ██ ██   ██      ██
% ███████ ███████ ██      ██ ██      ██ ██   ██ ███████

We start by defining two sequences, that satisfy Pell's equation stated below.

\begin{equation} \label{eq:Pell}
    x^2 - d y^2 = 1
\end{equation}

Using modified versions of the techniques presented
by~\textcite{Matijasevic1970}, it will be shown that the index \(m\) can be
obtained in a Diophantine way from \(\py_m(a)\) for certain subsequences of the
sequences defined below.

\begin{defin}
  Let \(K\) be an algebraic number field, \(\algint\) its ring of algebraic
  integers and fix \(a ∈ \algint\). One defines \(δ(a) := \sqrt{a^2 - 1}\) and
  \(ε(a) := a + δ(a)\), where we demand that \(-π/2 < \arg δ(a) ≤ π/2\). If
  \(δ(a) \not\in K\) one defines \(\px_m(a), \py_m(a) ∈ \algint\) for \(m ∈ ℕ\)
  by
  \begin{equation}\label{eq:def of x and y}
    \px_m(a) + δ(a) \py_m(a) = {(ε(a))}^m.
  \end{equation}
\end{defin}

This definition includes the case \(K = ℚ\) with \(\algint = ℤ\) of~\cite{Davis1973}. However, I am using the slightly modified notation of~\cite{Denef1980,Pheidas1988}.

Under the assumptions of the definition \(δ(a)\) is a root of the monic
quadratic polynomial
\[
  X^2 - a^2 + 1 ∈ \algint {[X]}.
\]
Therefore, the extension \(K[δ(a)] / K\) is quadratic and \(δ(a) ∈
\algint[{K[δ(a)]}]\). The sequences \(\px_m(a)\) and \(\py_m(a)\) are well
defined for each \(m ∈ ℕ\) as they correspond to the coefficients of
\({(ε(a))}^m\) in \(K[δ(a)]/K\) with respect to the basis \(\lbrace 1,
δ(a)\rbrace\). If the reference is clear, I will omit the dependency on \(a\)
writing \(δ, ε, \px_m\) and \(\py_m\).
% QUESTION Why does every number field contain such an element \(a\)?
\todo{Why does every number field contain such an element \(a\)?}
In the following the number field \(K[δ(a)]\) will be denoted by \(L\).

\begin{rem}
  As \(L/K\) has degree two, there is exactly one pair of field
  automorphisms\footnote{A field extension of degree two is necessarily normal.}
  on \(L\) preserving \(K\) point-wise, namely \(σ_1^{K}(α + δβ) = α + δβ\) and
  \(σ_2^K(α + δβ) = α - δβ\) for \(α, β ∈ \algint\). The latter will be denoted
  by \(\sigmaK{η} = σ_2^K(η)\) to emphasize the analogy of complex
  conjugation.
\end{rem}

Let me now collect some properties of these sequences. The proofs are
generalized versions of the ones given in~\cite{Davis1973}.

\begin{lem}
  Let \(K\) be an algebraic number field and \(a ∈ \algint\) such that \(δ(a) \not\in K\). Then
  \begin{thmlist}
    \item\label{lem:epsilon is unit}
    \(ε\) is a unit in \(\algint[L]\), its
    inverse is given by \(ε^{-1} = a - δ = \sigmaK{ε}\), and
    \item\label{lem:x and y solve Pells equation}
    \(\px_m, \py_m\) satisfy Pell's equation~\eqref{eq:Pell} for all \(m
    ∈ ℕ\), using \(d = {δ(a)}^2\) as parameter.
  \end{thmlist}
\end{lem}
\begin{proof}
  \begin{plist}
    \item We have \(ε \; (a - δ) = (a + δ) (a - δ) = a^2 - δ^2 = 1\) as desired.
    \item One uses induction on \(m\). If \(m = 0\), the pair \(\px_0 = 1\) and \(\py_0 =
    0\) yields a trivial solution to equation~\eqref{eq:Pell}. Let the claim be
    proven for all pairs \(\px_n, \py_n\) with \(n ≤ m\). Then rewriting the definition
    of \(\px_{m + 1}, \py_{m + 1}\) one obtains
    \[
      \px_{m + 1} + δ \py_{m + 1} = ε^{m + 1} = (\px_m + δ \py_m)ε.
    \]
    Applying the automorphism \(\sigmaK{\cdot}\) implies
    \[
      \sigmaK{\px_{m + 1} + δ \py_{m + 1}} = \px_{m + 1} - δ \py_{m + 1} = (\px_m - δ \py_m) ε^{-1}
    \]
    and multiplication of both equations yields
    \[
      \px_{m + 1}^2 - d \py_{m + 1}^2 = \sigmaK{\px_{m + 1} + δ \py_{m + 1}} (\px_{m + 1} - δ \py_{m + 1}) = 1
    \]
    as claimed.
  \end{plist}
\end{proof}

The defining equation
\[
  \px_m + δ \py_m = ε^m = {(\px_1 + δ \py_1)}^m
\]
can be seen as an analogue of the trigonometric identity
\[
  \cos m + i \sin m = e^{im} = {(\cos 1 + i \sin 1)}^m,
\]
where \(\px_m\) plays the role of \(\cos m\), \(\py_m\) the one of \(\sin m\), and \(i\) is replaced by \(δ\). In this view Pell's equation~\eqref{eq:Pell} is the analogue of the Pythagorean identity
\[
  {\cos (m)}^2 + {\sin (m)}^2 = 1.
\]

The next lemma proves the identities corresponding to \(\cos m = \Re e^{im}\),
\(\sin m = \Im e^{im}\), and the addition formulas.

\begin{lem}
  Let \(K\) be an algebraic number field and \(a ∈ \algint\) such that \(δ = δ(a) \not\in K\). Then for all \(m, k ∈ ℕ\) one has
  \begin{thmlist}
    \item\label{lem:real part of epsilon}
    \(\px_m = (ε^m + ε^{-m}) / 2\) and \(\py_m = (ε^m - ε^{-m}) / (2 δ)\), as well as,
    \item\label{lem:addition formulas}
    \(\px_{m ± k} = \px_m \px_k ± δ^2 \py_m \py_k\), and
    \(\py_{m ± k} = \px_k \py_m ± \px_m \py_k\).
  \end{thmlist}
\end{lem}
\begin{proof}
  \begin{plist}
    \item In \cref{lem:epsilon is unit} we have seen that \(ε^{-1} =
    \sigmaK{ε}\) and therefore \(ε^{-m} = {\left(\sigmaK{ε}\right)}^m\). Observe that for arbitrary \(α, β ∈ \algint\) we have
    \[
      α + β y + \sigmaK{α + δ β} = 2α \quad \text{and} \quad
      α + β y - \sigmaK{α - δ β} = 2δ β.
    \]
    Now, setting \(α + δ β = ε^m\) yields the claim.
    \item By the defining equation for \(\px_m\) and \(\py_m\) we have
    \begin{align*}
      \px_{m + k} + δ \py_{m + k} &= ε^{m + k} = (\px_m + δ \py_m) (\px_k + δ \py_k) =\\
                            &= (\px_m \px_k + δ^2 \py_m \py_k) + δ (\px_m \py_k + \px_k \py_m)
    \end{align*}
    and thusly
    \begin{align*}
      \px_{m + k} &= \px_m \px_k + δ^2 \py_m \py_k, \\
      \py_{m + k} &= \px_m \py_k + \px_k \py_m.
    \end{align*}
    The identities for \(\px_{m - k}\) and \(\py_{m - k}\) follow analogously.
  \end{plist}
\end{proof}

Setting \(k = 1\) in the lemma above, one obtains \(\px_{m ± 1} = a \px_m ± δ^2
\py_m\) and \(\py_{m ± 1} = a \py_m ± \px_m\). A further immediate consequence
of this lemma is the subsequent one, which brings divisibility into play.

\begin{lem}
  Let \(K\) be a number field and \(a ∈ \algint\) such that \(δ = δ(a) \not\in K\).
  Then for all \(m, k ∈ ℕ\), \(k ≠ 0\) we have that
  \begin{thmlist}
    \item\label{lem:y m divides y mk}
    \(\py_m\) divides \(\py_{mk}\) in \(\algint\),

    \item \(\py_{mk} \equiv k \px_m^{k - 1} \py_m \mod \left(\py_m^3\right)\) in
    \(\algint\), as well as

    \item\label{lem:x m and y m are relative prime}
    the principal ideals \((\px_m)\) and \(\py_m\) are relative prime in
    \(\algint\) for all \(m ∈ ℕ\)
  \end{thmlist}
\end{lem}
\begin{proof}
  \begin{plist}
    \item I argue by induction on \(k\). The claim is trivial if \(k = 1\) and
    \cref{lem:addition formulas} implies that
    \[
      \py_{m(k + 1)} = \px_m \py_{mk} + x{mk} \py_m.
    \]

    If the claim is proven for all factors lower than \(k + 1\), one finds that
    \(\py_m \mid \py_{mk}\) and \(\py_m \mid \py_m\) trivially. As a consequence, \(\py_m \mid
    \py_{m(k + 1)}\).

    \item Again the defining equation yields
    \begin{align*}
      \px_{mk} + δ \py_{mk} &= ε^{mk} = {(\px_m + δ \py_m)}^k = \\
                        &= \sum_{j = 0}^k \binom k j \px_m^{k - j} \py_m^j δ^j \\
      \intertext{and}
      \py_{mk} &= \sum_{\substack{j=0\\ j \text{ odd}}}^k
                \binom k j \px_m^{k - j} \py_m^j δ^{j -1}.
    \end{align*}
    In the equation above all terms for \(j > 1\) are divisible by \(\py_m^3\)
    and thusly vanish modulo \(\left(\py_m^3\right)\). The only term remaining
    is \(k \px_m^{k - 1} \py_m\) as claimed.

    \item Since \((\px_m, \py_m)\) is a solution to bells equation, we know that
    \[
      1 = \px_m^2 - (a^2 - 1) \py_m^2.
    \]
    is contained in the sum of ideals \((\px_m) + (\py_m)\) and thus the ideals
    are relative prime as claimed.
  \end{plist}
\end{proof}

The next lemma even though being easy to prove provides a valuable tool in
studying the sequences \(\px_m\) and \(\py_m\). It derives a recursive
definition and lets one prove properties of the sequences, by proving them for
\(m ∈ \lbrace 0, 1 \rbrace\) and inferring the properties for \(m + 1\) from
\(m\) and \(m - 1\).

\begin{lem}\label{lem:recursion for x_m and y_m}
  Let \(K\) be a number field and \(a ∈ \algint\) such that \(δ = δ(a) \not\in
  K\). For \(m > 1\) the following recursive conditions hold in \(\algint\).
  \begin{align*}
    \px_{m + 1} &= 2 a \px_m - \px_{m - 1}, & \px_1 = a, \;& \px_0 = 1 \\
    \py_{m + 1} &= 2 a \py_m - \py_{m - 1}, & \py_1 = 1, \;& \py_0 = 0 \\
  \end{align*}
\end{lem}
\begin{proof}
  The initial conditions follow from \(ε = a + δ\) and \(ε^0 = 1\). To prove the
  the difference equations one uses \cref{lem:addition formulas} and obtains
  \begin{align*}
    \px_{m + 1} &= a \px_m + δ^2 \py_m,  &  \py_{m + 1} &= a \py_m + \px_m, \\
    \px_{m - 1} &= a \px_m - δ^2 \py_m,  &  \py_{m - 1} &= a \py_m - \px_m.
  \end{align*}
  Summation yields \(\px_{m + 1} + \px_{m - 1} = 2 a \px_m\) and \(\py_{m + 1} + \py_{m - 1}
  = 2 a \py_m\).
\end{proof}

One applies the previous lemma to prove some congruence conditions.

\begin{lem}
  Let \(K\) be a number field and \(a, b, c ∈ \algint\) such that \(δ(a), δ(b)
  \not\in K\). Then for all \(m ∈ ℕ\) the following congruences hold in
  \(\algint\).
  \begin{plist}
    \item \(\py_m (a) \equiv m \mod (a - 1)\)
    \item If \(a \equiv b \mod (c)\), then \(\px_m (a) \equiv \px_m (b) \mod (c)\) and
    \(\py_m(a) \equiv \py_m(b) \mod (c)\).
  \end{plist}
\end{lem}
\begin{proof}
  Both congruences become equalities if \(m = 0\). As for \(m = 1\),
  the first congruence is again an equality as \(\py_1 (a) = 1\) independently of
  \(a\). The second claim is trivial since \(x_1 (η) = η\) and \(\py_1 (η) = 1\) for \(η
  ∈ \lbrace a, b \rbrace\). At this point one proceeds inductively and assumes
  the claims to be proven for all indices lower than \(m + 1\).

  \begin{plist}
    \item Note that \(a \equiv 1 \mod (a - 1)\) and thusly by
    \cref{lem:recursion for x_m and y_m}
    \[
      \py_{m + 1} = 2 a \py_m - \py_{m - 1} \equiv 2 m - (m - 1) = m + 1 \mod (a - 1)
    \]
    as claimed.

    \item Using \cref{lem:recursion for x_m and y_m} we see that for fixed \(m\)
    the coefficients \(\px_m (η)\) and \(\py_m (η)\) can be expressed as some fixed
    polynomial in \(η\). For the congruence this means
    \begin{align*}
      \px_{m + 1} (a) &= 2 a \px_m (a) - \px_{m - 1} (a)
                     \equiv 2 b \px_m (b) - \px_m{m - 1} (b) = \px_{m + 1} (b)
                     \mod (c)
    \end{align*}
    and for \(\py_{m + 1}\) completely analogously.
  \end{plist}
\end{proof}

\begin{lem}\label{lem:congruence x_2m+k}
  Let \(K\) be a number field and \(a ∈ \algint\) such that \(δ = δ(a) \not\in
  K\). Then for \(m, k ∈ ℕ\) such that \(m ± k ≥ 0\) the following congruence
  holds in \(\algint\).
  \[
    \px_{2 m ± k} \equiv - \px_k \mod (\px_m)
  \]
\end{lem}
\begin{proof}
  By applying the addition formulas of \cref{lem:addition formulas} twice and
  using that \(\px_m\) and \(\py_m\) solve Pell's equation~\eqref{eq:Pell} one obtains
  \begin{align*}
    \px_{2m ± k} &= \px_m \px_{m ± k} + δ^2 \py_m \py_{m ± k}
                \equiv δ^2 \py_m (\py_m \px_k ± \px_m \py_k) \equiv\\
               &\equiv δ^2 \py_m^2 \px_k = (\px_m^2 - 1) \px_k
                \equiv -\px_k \mod (\px_m).
  \end{align*}
\end{proof}

At this point for the first time in this section I state a result that is no
direct generalization of a result in~\cite{Davis1973} and present proofs given
in~\cite{Denef1980} or \cite{Shlapentokh2007}. Note however that the results are
nevertheless true for the case \(K = ℚ\) and \(\algint = ℤ\).

\begin{lem}
  Let \(K\) be a number field and \(a ∈ \algint\) such that \(δ = δ(a) \not\in
  K\). Then for all non-negative integers \(k, m ∈ ℕ\) the following congruence
  holds in \(\algint\).
  \[
    \px_{2km} \equiv (-1)^k \mod (\px_m)
  \]
\end{lem}
\begin{proof}
  If \(k = 0\) the congruence becomes and identity and if \(k = 1\) the claim
  follows directly from the lemma above. Assuming the claim to be proven for all
  integers lower than \(k\), we find---by applying \cref{lem:addition formulas}
  twice---that
  \begin{align*}
    \px_{2km} &= \px_{2(k-1)m}\px_{2m} + δ^2 \py_{2 (k-1) m}\py_{2m} \equiv
               (-1)^k + δ^2 \py_{2 (k-1) m}\py_{2m} = \\
              &= (-1)^k + δ^2 \py_{2 (k-1) m} \; 2 \px_m\py_m \equiv
               (-1)^k \mod (\px_m)
  \end{align*}
\end{proof}

% \begin{lem}
%   Let \(K\) be a number field and \(a ∈ \algint\) such that \(δ = δ(a) \not\in K\).
%   Then for all \(η ∈ \algint \setminus \set{0}\) there exists an \(m ∈ ℕ\) such that
%   \(η \mid \py_m\) in \(\algint[{L}]\).
%   % QUESTION
%   \todo{Is \(η \mid y\) in  \(\algint\)?}
% \end{lem}
% \begin{proof}
%   The quotient of the non-trivial, finitely generated \(\algint\)-modules \(\algint[L]\) and \((2 δ η)\)
%   % TODO
%   \todo{Do I need this lemma?}
%
%   Let \(m\) be the order of the group of units in the finite ring
%   \(\algint[L]/(2 δ η)\), where \((2 δ η)\) denotes the principal ideal
%   generated by \(2 δ η\) in \(\algint[{L}]\). Then \(ε^{±m} \equiv 1 \mod (2 δ
%   η)\). Hence, \(2 δ η \mid ε^m - ε^{-m}\) in \(\algint[{L}]\) and therefore
%   \[
%     \left. η \;\middle\vert\; \frac{ε^m - ε^{-m}}{2 δ} \right.
%   \]
%   in \(\algint[{L}]\), where the right hand side equals \(\py_m\) by
%   \cref{lem:real part of epsilon}.
% \end{proof}

\begin{lem}\label{lem:subgroup of ker N L/K}
  Let \(K\) be a number field and \(a ∈ \algint\) such that \(δ = δ(a) \not\in
  K\). Then the set
  \[
    S = \set{α + δ β :
             (α, β) ∈ \algint^2
             \text{ is a solution to~\eqref{eq:Pell} with parameter } d = δ^2}
  \]
  is a subgroup of the kernel of the norm map \(N_{L/K}: U_L → U_K\), where
  \(U_K\) and \(U_L\) denote the groups of units in \(\algint\) and
  \(\algint[L]\) respectively.
\end{lem}
\begin{proof}
  First of all, note that, if \(α + δ β ∈ S\), so is \(\sigmaK{α + δ β} = α - δ
  β ∈ S\) because
  \[
    α^2 - d {(-β)}^2 = α^2 - d β^2 = 1.
  \]
  Now let \(α + δ β\) be an arbitrary element of \(S\), then
  \[
    N_{L/K}(α + δ β) = (α + δ β) \sigmaK{α + δ β} = α^2 - d β^2 = 1.
  \]
  This implies that \(α + δ β ∈ \ker N_{L / K}\) but also that \(α + δ β\) is a
  unit, as \(α - δ β\) is its inverse. The product of two arbitrary elements \(α_1 + δ β_1, α_2 + δ β_2 ∈ S\) is
  \[
    (α_1 + δ β_1)(α_2 + δ β_2) = (α_1 α_2 + δ^2 β_1 β_2) + δ (α_1 β_2 + α_2 β_1).
  \]
  We apply the automorphism \(\sigmaK{\cdot}\) and multiply to obtain
  \begin{align*}
    (α_1 α_2 + δ^2 β_1 β_2)^2 - δ^2 (α_1 β_2 + α_2 β_1)^2 &=
    (α_1 + δ β_1)(α_2 + δ β_2)\,\sigmaK{(α_1 + δ β_1)(α_2 + δ β_2)} =\\
    &=
    (α_1 + δ β_1)\sigmaK{α_1 + δ β_1}(α_2 + δ β_2)\sigmaK{α_2 + δ β_2} = 1.
  \end{align*}
  As a consequence, \(S\) is closed under multiplication and the claim is
  proven.
\end{proof}

\begin{lem}\label{lem:rank of N_L/K U_L}
  Let \(L\) and \(K\) be number fields as defined above.
  The image \(N_{L / K}\left(U_L\right) ≤ U_K\) has finite index in \(U_K\).
\end{lem}
\begin{proof}
  I claim that \(N_{L / K}\left(U_L\right)\) contains \(α^2\) for every \(α ∈ U_K\).
  This is because the restriction \(σ_i^K|_{\algint}\) is just the identity on
  \(\algint\) for \(i ∈ \set{1, 2}\) and therefore, \(N_{L / K}(α) = α^2\) for all
  \(α ∈ U_K \subseteq U_L\).

  Let now \(k := \rk U_K\) and identify \(U_K = μ(K) \times ℤ^{k}\), where
  \(μ(K)\) is the finite cyclic group of roots of unity in \(K\)
  (cf.~\cref{thm:Dirichlet}). Consider the following \(k\) elements
  \[
    ([0],1,0,…,0), \; ([0],0,1,0,…,0), \; …, \; ([0], 0, …, 0, 1)
  \]
  contained in \(U_K\). By the claim their ‘squares’ are contained in \(N_{L / K}\left(U_L\right)\) i.e.
  \[
    ([0],2,0,…,0), \; ([0],0,2,0,…,0), \; …, \; ([0], 0, …, 0, 2) ∈ N_{L / K}\left(U_L\right).
  \]
  As a consequence, the direct product
  \[
    G := \set{[0]} \times \underbrace{2 ℤ \times … \times 2 Z}_{k\text{-times}}
  \]
  is a subgroup of \(N_{L / K}\left(U_L\right)\) and therefore
  \[
    [U_K : N_{L / K}\left(U_L\right)] ≤ [U : G] = ℓ\, 2^k < ∞.
  \]
\end{proof}

As for the free ranks of \(U_K\), \(U_L\), \(N_{L / K}\left(U_L\right)\) and
\(S\) the lemma above implies that \(\rk N_{L / K}\left(U_L\right) = \rk U_K\)
and therefore, as an immediate consequence of the first isomorphism
theorem~\cite[see][II~§1, p.~89]{Lang2002} the following inequality holds
\begin{equation}\label{eq:rank of S} \rk S ≤ \rk \ker N_{L / K} = \rk U_L - \rk
U_K. \end{equation}

Before proving the main theorem of this section (\todo{reference it}) I
sketch how \textcite{Davis1973} proceeds in proving the \textsc{DPRM}-theorem.

\DPRM*

First he proves using the sequences above that the exponential function is
Diophantine over \(ℕ\) \cite[Thm 3.3]{Davis1973}. Then he is able to extend the
language of Diophantine predicates by \emph{bounded existential} and
\emph{bounded universal quantifiers}, i.e.\ by
\begin{align*}
  {(∃y)}_{≤x}ϕ(x, y) &⇔ ∃y\; (y ≤ x ∧ ϕ(x, y)),\\
  {(∀y)}_{≤x}ϕ(x, y) &⇔ ∀y\; (y > x ∨ ϕ(x, y))
\end{align*}
where \(ϕ\) is a positive existential formula~\cite[Thm 5.1]{Davis1973}. The
first one is easily seen to be Diophantine as the order relation on \(ℕ^{*}\) is
Diophantine. Proving the second claim takes the rest of the section. Now using
this result together with the sequence number theorem~\cite[Thm 1.3]{Davis1973}
Davis proofs that a function is Diophantine over \(ℕ\) if and only if it is
computable~\cite[Thm 6.1]{Davis1973}.

This already implies the \textsc{DPRM}-theorm as Davis has introduced
Diophantine pairing functions in~\cite[Thm 1.1]{Davis1973} and therefore all
ranges of Diophantine---and therefore all computable functions---are Diophantine
over \(ℕ\). But the ranges of computable functions are exactly the
semi-decidable subsets of \(ω\) by \cref{pro:characterizations of ce sets}, thus
proving the claim of the theorem.

\subsection{Diophantine definition of \(ℤ\) over \(K\)}
% ██████  ██  ██████  ██████  ██   ██        ██████  ███████ ███████
% ██   ██ ██ ██    ██ ██   ██ ██   ██        ██   ██ ██      ██
% ██   ██ ██ ██    ██ ██████  ███████        ██   ██ █████   █████
% ██   ██ ██ ██    ██ ██      ██   ██        ██   ██ ██      ██
% ██████  ██  ██████  ██      ██   ██ ██     ██████  ███████ ██   ██

For the remainder of this section let \(K ≠ ℚ\) be a totally real number field
or a number field with exactly one pair of non-real embeddings of degree \(n :=
[K : ℚ] ≥ 3\) over the rationals \(ℚ\). In the notation of \cref{thm:Dirichlet}
this means that either \(r = n > 1\), or \(r = n - 2 > 0\) and \(s = 1\). As
before we set \(L = K[{δ(a)}]\), where \(δ(a) \not\in \algint\) is a root of
\(X^2 - a^2 + 1\) and \(-π/2 < \arg δ(a) ≤ π/2\).

Furthermore, let us assume that \(σ_1 = \id_K, σ_2, …, σ_n: K → ℂ\) are all
embeddings of \(K\) into the complex pane \(ℂ\). If \(s = 1\) we demand without
loss of generality that \(K, σ_2(K) \not\subset ℝ\) and that \(σ_2(α) =
\overline{α}\) for all \(α ∈ K\). In other words, \((σ_1, σ_2)\) is the pair of
complex embeddings and all other morphisms embed \(K\) into the reals \(ℝ\).

\begin{lem}\label{lem:L over K is quadratic}
  Let \(K ≠ ℚ\) be a number field of degree \(n\) over \(ℚ\). If \(a ∈ \algint\)
  satisfies
  \begin{equation}\label{eq:approximations of a}
    \begin{cases}
      r = n > 1\\
      a > 2^{2(n + 1)}\\
      0 < σ_i(a) < \frac{1}{2} &\text{for } 1 < i ≤ n
    \end{cases}
    \quad \text{or} \quad
    \begin{cases}
      r = n - 2 > 0\\
      |σ_i(a)| > 2^{2(n + 1)} &\text{for } i ∈ \set{1, 2}\\
      0 < σ_i(a) < \frac{1}{2} &\text{for } 2 < i ≤ n
    \end{cases},
  \end{equation}
  then \(δ(a) = \sqrt{a^2 - 1}\) is not contained in \(K\).
\end{lem}
\begin{proof}
  By assumption we have \(0 < σ_n(a) < 1/2\) and therefore \(σ_n(a)^2 - 1 < 0\)
  cannot be a square in the real number field \(σ_n(K) \subseteq ℝ\). As \(K\)
  is isomorphic to \(σ_n(K)\), the algebraic integer \(δ(a) = \sqrt{a^2 - 1}\)
  cannot be contained in \(K\).
\end{proof}

We will prove later using Minkowski's theorem on convex bodies
(\cref{thm:Minkowski}), that an algebraic integer \(a ∈ \algint\) satisfying
\eqref{eq:approximations of a} exists and therefore the condition on \(a\) from
the previous section holds.

\begin{rem}
  As the expansion \(L / K\) is quadratic by \cref{lem:L over K is
  quadratic}, every \(σ_i\) can be extended to exactly two embeddings \(σ_{i1}\)
  and \(σ_{i2}\) of \(L\) into the complex plane \(ℂ\) by ‘composing’ with
  \(σ_1^K\) or \(σ_2^K\). This yields
  \begin{equation}\label{eq:def of sigma ij}
    \begin{aligned}
      σ_{i1}(α + δβ) &= σ_i(α) + \sqrt{{σ_i(a)}^2 - 1}\, σ_i(β) \quad \text{and} \\
      σ_{i2}(α + δβ) &= σ_i(α) - \sqrt{{σ_i(a)}^2 - 1}\, σ_i(β)
    \end{aligned}
  \end{equation}
  for all \(α, β ∈ \algint\) and all \(1 ≤ i ≤ n\).
\end{rem}

I will identify the field \(L\) with its embedding \(σ_{11}(L)\) and write \(x\)
instead of \(σ_{11}(x)\) for its elements.

\begin{lem}\label{lem:r and s for tr and opnr}
  Let \(K ≠ ℚ\) be a number field of degree \(n\) over \(ℚ\) and let \(a ∈
  \algint\) be such that \eqref{eq:approximations of a} is satisfied. Then
  \begin{thmlist}
    \item if \(r = n\), only \(σ_{11}\) and \(σ_{12}\) embed \(L\) into the
    reals, and
    \item if \(r = n - 2\), the field \(L\) is totally complex.
  \end{thmlist}
\end{lem}
\begin{proof}
  \begin{plist}
    \item If \(K\) is totally real and \(i > 1\), then \(0 < σ_i(a) < 1/2\) and
    therefore the radicands in \eqref{eq:def of sigma ij} are both negative. As
    a consequence, \((σ_{i1}, σ_{i2})\) is a pair of non-real embeddings.

    On the other hand, if \(i = 1\) then \(a > 2^{2(n + 1)} > 1\) and the
    radicands are both positive. We deduce that \(σ_{11}\) and \(σ_{12}\) are
    both real embeddings and \(L\) is a subfield of the reals by our
    identification.
    \item As \(σ_1\) and \(σ_2\) are already complex embeddings, so are
    \(σ_{11}, σ_{12}, σ_{21}\) and \(σ_{22}\). For the remaining embeddings one
    argues completely analogously to (i).
  \end{plist}
\end{proof}

\begin{lem}\label{lem:properties of sigma epsilon}
  Let \(K ≠ ℚ\) be a number field of degree \(n\) over \(ℚ\) and let \(a ∈
  \algint\) be such that \eqref{eq:approximations of a} is satisfied. If \(s\) is the number of pairs of non-real embeddings of \(K\), then
  \begin{thmlist}
    \item \(σ_{i1}(ε)^{-1} = σ_{i2}(ε)\) for all \(1 ≤ i ≤ n\),
    \item \(σ_{i1}(ε)\) and \(σ_{i2}(ε)\) are complex conjugates for \(s + 1 < i ≤ n\), and
    \item\label{lem:modulus of sigma epsilon}
    \(|σ_{i1}(ε)| = |σ_{i2}(ε)| = 1\) for \(s + 1 < i ≤ n\).
  \end{thmlist}
\end{lem}
\begin{proof}
  In \cref{lem:epsilon is unit} we have seen, that the claim holds true for \(i
  = 1\). We extend this method to obtain the results for the other cases. For
  all \(1 ≤ i ≤ n\) we have
  \begin{align*}
    σ_{i1}(ε) σ_{i2}(ε) &= (σ_i(a) + σ_{i1}(δ)) (σ_i(a) - σ_{i1}(δ)) =\\
      &= σ_i(a)^2 - σ_{i1}(δ)^2 = σ_i(a)^2 - σ_i(a)^2 + 1 = 1.
  \end{align*}

  For all \(s + 1 < i ≤ n\) we have defined \(σ_i: K → ℂ\) to be a real
  embedding. Thus \(σ_i(a)\) is a real number and as \(0 < σ_i(a) < 1/2\), we
  find that \(σ_{i1}(δ)\) is purely imaginary. Hence, we deduce that
  \(σ_{i1}(ε)\) and \(σ_{i2}(ε)\) are complex conjugates. But then the complex
  moduli of these algebraic integers must coincide, leaving no other option than
  \(|σ_{i1}(ε)| = |σ_{i2}(ε)| = 1\).
\end{proof}

Before we can start proving some approximations for the complex moduli of \(ε,
δ\) and \(a\), we need to fix some notations.

\begin{defin}
  Let \(K ≠ ℚ\) be a number field of degree \(n\) over \(ℚ\) and let \(a ∈
  \algint\) be such that \eqref{eq:approximations of a} is satisfied. For \(1 ≤
  i ≤ n\) we set
  \begin{thmlist}
    \item \(a_i := σ_{i}(a)\),
    \item \(ε_i := σ_{i1}(ε)\) if \(|σ_{i1}(ε)| ≥ 1\) and \(ε_i := σ_{i2}(ε)\)
    otherwise, and
    \item \(δ_i := σ_{i1}(δ)\).
  \end{thmlist}
\end{defin}

\begin{rem}
  \begin{exlist}
    \item In the definition above we could have equivalently defined \(ε_i :=
    σ_{i1}(ε)\) for \(s + 1 < i ≤ n\), as by \cref{lem:modulus of sigma epsilon}
    the complex modulus of \(σ_{i1}(ε)\) is \(1\).
    \item Note that by \eqref{eq:def of sigma ij} we have \(σ_{i2}(δ) = -δ_i\)
    and therefore \(|δ_i| = |σ_{i2}(δ)|\) for all \(1 ≤ i ≤ n\).
  \end{exlist}
\end{rem}

We will use the following fact from number theory.

\begin{lem}
  If an algebraic integer \(η\) and all its conjugates have complex
  modulus \(1\), then \(η\) is a root of unity.
\end{lem}
\begin{proof}
  Let \(η = \seq{η}\) be all the conjugates of \(η\). Consider the polynomial
  \[
    f_k(X) := (X - η_1^k)…(X - η_n^k) = X^n + a_{k, n - 1} X^{n - 1} + … + a_{k, 0}
  \]
  for all integers \(k ∈ ℕ\). Since \(f_k ∈ ℤ[X]\) has integral coefficients and
  all \(|η_i| = 1\), we find that \(|a_{k, s}| ≤ {n \choose s}\). This implies
  that there can only be finitely many different \(f_k\). Now choose \(ℓ > k\)
  with \(f_ℓ = f_k\), then \(η^ℓ = η^k\) and \(η\) is a \((ℓ - k)\)-th root of
  unity.
\end{proof}

\begin{lem}
  Suppose \(K ≠ ℚ\) is a number field of degree \(n\) over \(ℚ\) and let \(a ∈
  \algint\) be such that \eqref{eq:approximations of a} is satisfied.
  Then the following inequalities hold.
  \begin{thmlist}
    \item\label{lem:approx for delta i 1}
    \(|a_i|/2 < |δ_i| < |a_i| + 1\) for \(1 ≤ i ≤ s + 1\).
    \item\label{lem:approx for delta i 2}
     \(1/2 < | δ_i | < 1\) for \(s + 1 < i ≤ n\).
    \item\label{lem:modulus of elements in the kernel}
     If \(η ∈ \ker N_{L/K}\) then \(|σ_{ij}(η)| = 1\) for \(s + 1 < i ≤ n\) and
     \(j ∈ \set{1, 2}\). Furthermore, \(|η| = 1\) if and only if  \(η\) is a
     root of unity.
    \item \(\left\vert |a| - \sqrt{|a^2 - 1|} \right\vert < 1\).
    \item\label{lem:approximation of epsilon with a}
     \(|a| < |ε_1| < 2|a| + 1\).
    \item\label{lem:epsilon is not a root of unity}
    \(ε\) is not a root of unity.
  \end{thmlist}
\end{lem}
\begin{proof}
  \begin{plist}
    \item By assumption we have \(|a_i| > 2^{2(n + 1)}\) and therefore
      \begin{align*}
        \frac{|a_i|^2}{4} &= |a_i|^2 - \frac{3a_i^2}{4} ≤ |a_i|^2 - \frac{3}{4} 2^{4(n + 1)} < |a_i|^2 - 1 ≤\\
        &≤ |δ_i| = |a_i^2 - 1| ≤ |a_i|^2 + 1 < (|a_i| + 1)^2
      \end{align*}
    \item Again by our assumption \(|a_i| < 1/2\). Thus, we find
    \[
      \frac{1}{4} < \frac{3}{4} < 1 - a_i^2 = |δ_i|^2 < 1.
    \]
    \item As in \cref{lem:properties of sigma epsilon} one uses that
    \(\overline{σ_{i1}(δ)} = σ_{i2}(δ)\) for all \(s + 1 < i ≤ n\) and finds
    for \(η = α + δ β ∈ \ker N_{L /K}\) that
    \begin{align*}
      σ_{i1}(η) &= σ_i(α) + σ_{i1}(δ) σ_i(β) \text{ and}\\
      σ_{i2}(η) &= σ_i(α) + σ_{i2}(δ) σ_i(β) = σ_i(α) - σ_{i1}(δ) σ_i(β)
    \end{align*}
    are complex conjugates. Now one can deduce,
    \[
      1 = N_{L/K}(η) = σ_{i1}(η) σ_{i2}(η) = |σ_{ij}(η)|^2
    \]
    for both \(j = 1\) and \(2\).

    To prove the second part of the claim, we notice that all roots
    of unity have complex modulus \(1\), so one direction is trivial.

    Let now \(η = α + δ β ∈ \ker N_{L / K}\) and additionally \(|η| = 1\), we
    differentiate two cases. If \(K\) is totally real, then all embeddings of
    the algebraic integer \(η\) have complex modulus \(1\). Therefore, \(η\) is
    a root of unity.

    If on the other hand, there exists a pair of non-real embeddings of \(K\),
    then note firstly, that the complex conjugate \(\overline{δ}\) is a root of the polynomial
    \[
      X^2 - \overline{a}^2 + 1 = X^2 - σ_2(a) + 1.
    \]
    As a consequence, \(σ_{2j}(δ) = (-1)^{1 + j} \overline{δ}\) for \(j ∈ \set{1, 2}\). We deduce that
    \[
      \overline{σ_{11}(η)} = \overline{η} = \overline{α + δ β} = σ_2(α) +
      σ_{21}(δ) σ_{2}(β) = σ_{21}(η)
    \]
    and
    \[
      \overline{σ_{12}(η)} = \overline{α - δ β} = σ_2(α) - σ_{21}(δ) σ_{2}(β)
      = σ_{21}(η) = σ_2(α) + σ_{22}(δ) σ_{2}(β) = σ_{22}(η).
    \]
    This implies that \(|σ_{21}(η)| = |η| = 1\) and
    \(|σ_{22}(η)| = |σ_{12}(η)|\). Finally, note that \(N_{L / ℚ} = N_{K / ℚ}
    \circ N_{L / K}\) and therefore
    \[
      1 = |N_{K / ℚ}(1)| = |N_{K / ℚ} \circ N_{L / K} (η)| = \left\vert \prod_{\substack{1 ≤ i ≤ n\\ 1 ≤ j ≤ 2}} σ_{ij}(η) \right\vert = |σ_{12}(η)| |σ_{22}(η)| = |σ_{12}(η)|^2.
    \]

    \item The inequality
    \[
      0 ≤ \left\vert |a| - \sqrt{|a^2 - 1|} \right\vert < 1
    \]
    is equivalent to \(|a|^2 - 2 |a| + 1 < |a^2 - 1|\). But this equality can easily seen to be satisfied, as
    \[
      |a|^2 + 1 < |a|^2 + 2|a| - 1 = |a^2 - 1| + 2 |a|
    \]
    and the claim is proven.

    \item Consider the inequality
    \[
      |ε_1|^2 = |a + δ_1|^2 = |a^2 + 2 a δ_1 + δ_1^2| ≥ |2a^2 + 2a δ_1| - 1 = 2 |a| |ε_1| - 1
    \]
    which can be rewritten as
    \[
      0 ≤ |ε_1|^2 - 2 |a| |ε_1| + 1 = \left(|ε_1| - |a| - \sqrt{|a|^2 - 1}\right) \left(|ε_1| - |a| + \sqrt{|a|^2 - 1}\right)
    \]
    Thus, either both factors are negative real numbers or both are positive. In the first case
    \[
      0 < |ε_1| < |a| - \sqrt{|a|^2 - 1} \overset{\text{by (iv)}}{<} 1,
    \]
    which is impossible. Hence, both factors are positive and
    \[
      |ε_1| > |a| + \sqrt{|a|^2 - 1}  > |a|,
    \]
    proving the first estimate. The second inequality follows from \(|ε_1| = |a + δ_1| < 2|a| + 1\) by (i).
    % QUESTION Can one get rid of the \(+ 1\)? I doubt the proof of
    % Shlapentokh, but will check, if the stronger estimate is needed. It seems
    % that it is not needed.
    \todo{Can one get rid of the \(+ 1\)? I doubt the proof of Shlapentokh, but will check, if the stronger estimate is needed. It seems that it is not needed.}

    \item Note that by (v), \(|ε_1| > |a| > 2^{2(n + 1)} > 1\) and therefore the
    complex modulus of \(ε\) cannot be equal to \(1\). The claim follows from
    (iii).
  \end{plist}
\end{proof}

As a next step, we want to show that we have already found all solutions of
Pell's equation~\eqref{eq:Pell}. For this we need some lemmas.

Recall the group \(S ≤ \ker N_{L / K}\) defined in \cref{lem:subgroup of ker N
L/K}. We have seen in \cref{lem:rank of N_L/K U_L} and the subsequent
inequality~\eqref{eq:rank of S} that the free rank of \(S\) can be bound from
above by \(\rk U_L - \rk U_K\). I claim that this difference of ranks is equal
to \(1\) in both cases of algebraic number fields we are considering.

If \(K ≠ ℚ\) is totally real, then by Dirichlet's unit theorem
(\cref{thm:Dirichlet}) we find that \(\rk U_K = n\) and by \cref{lem:r and s for
tr and opnr} that \(\rk U_L = 1 + n\). If on the other hand, \(K\) satisfies
\(r = n - 2 > 0\) then \(\rk U_K = n - 1\) and by \cref{lem:r and s for tr and
opnr} we have \(\rk U_L = n\).

Note that \(ε\) is contained in \(S\) and by the previous lemma, \(ε\) is not a
root of unity. Thus the group \(⟨ε⟩ ≤ S\) has free rank at least equal to \(1\). We deduce that
\[
  \rk ⟨ε⟩ = \rk S = 1.
\]
Thus, there exists a unit \(ε_0 ∈ S\) such that for all \(η ∈ S\)  there exists
a root of unity \(ζ\) and an integer \(k\), such that \(η = ζ ε_0^k\). However,
even more is true, as one can set \(ε = ε_0\), but before we can prove
this, we need a lemma.

\begin{lem}
  Let \(K ≠ ℚ\) be a number field and let \(a ∈ \algint\) satisfy
  \eqref{eq:approximations of a}. Furthermore, let \(ε_0\) be a generator of the torsion free part of \(S\). Then \(2δ | (ε_0 - ε_0^{-1})\) and
  \begin{thmlist}
    \item if \(K\) is totally real, then
    \[
      |N_{L/ℚ} (2 δ)| > a^2 \quad \text{ and } \quad
      |N_{L/ℚ} (ε_0 - ε_0^{-1})| < 2^{2n} |ε_0|^2;
    \]

    \item if \([K: ℚ] ≥ 3\) and \(K\) has exactly one pair of non-real embeddings, then
    \[
      |N_{L/ℚ} (2 δ)| > a^4 \quad \text{ and } \quad
      |N_{L/ℚ} (ε_0 - ε_0^{-1})| < 2^{2n} |ε_0|^4.
    \]
  \end{thmlist}
\end{lem}
\begin{proof}
  Let \(ε_0 = α + δβ\) for some \(α, β ∈ \algint\), then \(ε_0^{-1} = α - δ β\)
  (cf.~\cref{lem:subgroup of ker N L/K}) and
  \[
    ε_0 - ε_0^{-1} = 2δ\, β,
  \]
  proving that \(2δ | (ε_0 - ε_0^{-1})\).

  We assert without loss of generality that \(ε = ζε_0^k\), where \(|ε_0| ≥ 1\).
  Then for all \(1 ≤ i ≤ n\) and all \(1 ≤ j ≤ 2\), we have
  \[
    |σ_{ij}(ε)| = |σ_{ij}(ζε_0^k)| = |σ_{ij}(ε_0)|^k,
  \]
  and thus, \(|ε_1| = |ε_0|^k ≥ |ε_0| ≥ 1\). Furthermore, by \cref{lem:modulus of elements in the kernel} the following inequality holds
  \begin{equation}\label{eq:approximation of epsilon 0}
    |σ_{i1}(ε_0) - σ_{i1}(ε_0^{-1})| |σ_{i2}(ε_0) - σ_{i2}(ε_0^{-1})| =
    |σ_{i1}(ε_0) - σ_{i1}(ε_0^{-1})|^2 ≤ 2^2
  \end{equation}
  for all \(s + 1 < i ≤ n\).

  \begin{plist}
    \item If \(K ≠ ℚ\) is totally real, then by \cref{lem:approx for delta i 1}
    and (ii) we find that
    \[
      |N_{L/ℚ}(2δ)| = 2^{2n} \prod_{i = 1}^n |δ_i|^2 > \frac{2^{2n}}{2^{2n - 2}} \frac{|a|^2}{4} = a^2.
    \]

    To see the second inequality, we use \eqref{eq:approximation of epsilon 0} to find
    \begin{align*}
      |N_{L/ℚ}(ε_0 - ε_0^{-1})| &=
        \prod_{\substack{1 ≤ i ≤ n\\1 ≤ j ≤ 2}} |σ_{ij}(ε_0) - σ_{ij}(ε_0^{-1})|
        \prod_{i = 1}^n |σ_{i1}(ε_0) - σ_{i1}(ε_0^{-1})|^2 ≤ \\
        &≤ 2^{2n - 2} |ε_0 - ε_0^{-1}|^2 ≤ 2^{2n - 2} (|ε_0| + 1)^2
        < 2^{2n} |ε_0|^2
    \end{align*}
    as claimed.

    \item Completely analogously using the fact, that
    \begin{align*}
      &|σ_{11}(ε_0) - σ_{11}(ε_0)| |σ_{12}(ε_0) - σ_{12}(ε_0)|
      |σ_{21}(ε_0) - σ_{21}(ε_0)| |σ_{22}(ε_0) - σ_{22}(ε_0)| =\\
      &|σ_{11}(ε_0) - σ_{11}(ε_0)|^2 |\overline{σ_{21}(ε_0) - σ_{21}(ε_0)}|^2 =
      |σ_{11}(ε_0) - σ_{11}(ε_0)|^4.
    \end{align*}
  \end{plist}
\end{proof}

\begin{pro}\label{pro:epsilon essentially generaltes S}
  Let \(K ≠ ℚ\) be a number field and let \(a ∈ \algint\) satisfy
  \eqref{eq:approximations of a}. Then for every \(η ∈ S\) there exists an
  integer \(k\) and a root of unity \(ζ ∈ L\) such that \(η = ζ ε^k\).
\end{pro}
\begin{proof}
  By the discussion above all that is left to prove is that in the equation
  \[
    ε_1 = ζε_0^k
  \]
  with \(|ε_0| ≥ 1\) the integer \(k\) is \(1\). Then
  \[
    ε_0 = ζ^{-1}ε_1 = ζ^{-1}ε^{± 1}
  \]
  and the proposition is proven.

  Assume to the contrary, that \(k ≥ 2\), then \(|ε_1| ≥ |ε_0|^2 ≥ 1\).
  By the lemma above \(2δ | (ε_0 - ε_0^{-1})\) and therefore
  \[
    N_{L / K}(2δ) ≤ N_{L / K}(ε_0 - ε_0^{-1}).
  \]
  The previous lemma implies now that \(1 < |a|^{2m_0} < 2^{2n}|ε_0|^{2m_0}\),
  where \(m_0 ∈ \set{1, 2}\) is chosen accordingly. Applying
  \cref{lem:approximation of epsilon with a} yields
  \[
    |a|^2 < 2^{2n} |ε_0|^{2} ≤ 2^{2n} |ε_1| < 2^{2n + 1} (|a| + 1) < 2^{2n + 2} |a|.
  \]
  But this is a contradiction to the assumption \(|a| > 2^{2n + 2}\) of
  \eqref{eq:approximations of a}.
\end{proof}

Recall the sequences \((\px_m)_{m ∈ ℕ}\) and \((\py_m)_{m ∈ ℕ}\) defined in
\eqref{eq:def of x and y}. In the special cases of the field \(K\) we are
considering one can derive further properties.

\begin{lem}
  Let \(K ≠ ℚ\) be a number field and let \(a ∈ \algint\) satisfy
  \eqref{eq:approximations of a}. Then the following inequality holds
  \[
    m < |σ_i(\px_m)|
  \]
  for all non-negative integers \(m ∈ ℕ\) and all \(1 ≤ i ≤ s + 1\).
\end{lem}
\begin{proof}
  By \cref{lem:approximation of epsilon with a} we know that \(|ε_1| > |a| > 1\)
  and since \(ε_1 = ε^{±1}\), \cref{lem:real part of epsilon} implies
  \[
    |x_m| = \frac{|ε_1^m + ε_1^{-m}|}{2} ≥ \frac{|ε_1|^{m} + 1}{2}
    > \frac{|a|^{m} + 1}{2} > \frac{2^{2(n+1)m} + 1}{2} > m
  \]

  If \(s = 1\), then \(σ_2(\px_m)\) is the complex conjugate of \(σ_1(\px_m) =
  \px_m \)
  their moduli must coincide.
\end{proof}

\begin{lem}\label{lem:approximations of sigma x and sigma y}
  Let \(K ≠ ℚ\) be a number field and let \(a ∈ \algint\) satisfy
  \eqref{eq:approximations of a}. There exists a constant \(C\) depending on
  \(K\) and \(a\) such that for all \(k ∈ ℕ \setminus \set{0}\) there exist \(m,
  h ∈ ℕ\) with \(k | m\) and \(k | h\), and
  \begin{align*}
    |σ_i(\px_m)| &> \frac{1}{2},\\
    |σ_i(\py_h)| &> C
  \end{align*}
  for \(s + 1 < i ≤ n\).
\end{lem}
\begin{proof}
  By \cref{lem:modulus of sigma epsilon} we know that \(|ε_j| = 1\) for \(s + 1
  < j ≤ n\), but in the same lemma we have proven, that \(ε\) is not a root of
  unity. It follows that there exist arguments \(ϑ_{s + 2}, …, ϑ_n ∈ ℝ\) such
  that
  \[
    ε_j = e^{i π ϑ_j}.
  \]
  Let \(A = \set{1, ϑ_{j_1}, … ϑ_{j_s}}\) be a maximal \(ℚ\)-linear independent
  subset of \(\set{1, ϑ_{s + 2}, …, ϑ_{n}}\). Since \(ε\) is not a root of
  unity, none of the \(ϑ_j\) can be rational and \(A\) contains an element
  unequal to \(1\).

  Let \(J_0\) be the set of indices of
  elements in \(A\), then the construction implies that for all \(s + 1 < t ≤
  n\) there exist integers \(b_t, b_{tj} ∈ ℤ\) with the property
  \[
    ε_r^{b_t} = \prod_{j ∈ J_0} ε_j^{b_{tj}}.
  \]
  We set \(b := \prod_{t = s + 2}^n b_t\) and find for all \(s + 1 < t ≤
  n\) integers \(c_{tj} ∈ ℤ\) with
  \[
    ε_t^b = \prod_{j ∈ J_0} ε_j^{c_{tj}}.
  \]

  We exponent this expression by a multiple \(ℓ ∈ ℤ\) of \(k\) and rewrite it
  to obtain
  \begin{align}
    \begin{split}\label{eq:approximation of sigma x and sigma y}
    σ_t(\px_{ℓb}) + σ_{t1}(δ) σ_t(\py_{ℓb}) &= ε_t^{ℓb} = \prod_{j ∈ J_0} ε_j^{ℓ c_{tj}}\\
      &= e^{iπ \sum_{j ∈ J_0} ℓ c_{tj} ϑ_j} =\\
      &= \cos\left(π \sum_{j ∈ J_0} ℓ c_{tj} ϑ_j\right) +
         i \sin\left(π \sum_{j ∈ J_0} ℓ c_{tj} ϑ_j\right).
    \end{split}
  \end{align}

  By continuity of \(| \cos(π ϑ) |\) in \(ϑ\), we can find
  a constant \(λ > 0\) such that \(1 - \cos(π ϑ) | < 1/2\) whenever \(|ϑ| <
  λ\).

  Let \(c_0 = \max_{t,j}(|c_{tj}|)\). By Kronecker's
  theorem~(\ref{thm:Kronecker})
  we can find an integer \(ℓ ∈ ℕ\) divisible by \(k\) such that for all \(j ∈
  J_0\) there exists an \(ℓ_j ∈ ℤ\) with the property that
  \[
    |ℓ ϑ_j - ℓ_j| < \frac{λ}{c_0n}.
  \]
   This implies that
   \[
    \left| \sum_{j ∈ J_0} ℓ c_{tj} ϑ_j - \sum_{j ∈ J_0} ℓ_j c_{tj} \right| < λ
   \]

  From \eqref{eq:approximation of sigma x and sigma y} and the fact that
  \(σ_{t1}(δ)\) is uniformly continuous we deduce that
  \[
    |σ_t(\px_{ℓb})| =
    \left| \cos\left(π \sum_{j ∈ J_0} ℓ c_{tj} ϑ_j\right) \right| =
    \left| \cos\left(π \sum_{j ∈ J_0} ℓ c_{tj} ϑ_j - π \sum_{j ∈ J_0} ℓ_j c_{tj}\right) \right| < \frac{1}{2}.
  \]
  % QUESTION How can I deduce that \(\sum_{j ∈ J_0} ℓ_j c_{tj}\) is divisible by 2?
  \todo{How can I deduce that \(\sum_{j ∈ J_0} ℓ_j c_{tj}\) is divisible by 2?}

  To prove the claimed bound for \(\py_h\) let \(C_t := \sum_{j ∈ J_0} c_{tj}\)
  for all \(s + 1 < t ≤ n\) and fix a constant \(C_0 ∈ ℕ\) such that \(C_0 >
  \max_t (|C_t|)\) and in the prime factorization of \(C_0\) appear at least as
  many twos as in the prime factorization of \(C_t\). In other words,
  \[
    \ord_2 C_0 ≥ \ord_2 C_t, \quad \text{ for all } s + 1 < t ≤ n.
  \]

  As \(|\sin(π ϑ)|\) is uniformly continuous on the compact interval \([-1,
  1]\) we can find for all positive \(λ_1 > 0\) a real number \(0 < λ_2 < 1/4\)
  such that \(|\sin(π ϑ) - \sin(π φ)| < λ_1\) whenever \(|ϑ - φ| < λ_2\). We
  apply Kronecker's theorem again to obtain an integer \(ℓ ∈ ℤ\) divisible by
  \(k\) such that for all \(j ∈ J_0\) there exists an integer \(ℓ_j ∈ ℤ\) with
  the property
  \[
    \left| ℓ ϑ_j - ℓ_j - \frac{1}{2 C_0} \right| < \frac{λ_2}{2 C_0}.
  \]

  Then we find that
  \[
    \left| \sum_{j ∈ J_0} \left( ℓ c_{tj} ϑ_j - ℓ_j c_{tj}\right) - \frac{C_t}{2 C_0}\right| =
    \left| \sum_{j ∈ J_0} \left( ℓ c_{tj} ϑ_j - ℓ_j c_{tj} - \frac{c_{tj}}{2 C_0}\right)\right| < λ_2
  \]
  Set now \(λ_1 = 1/2 | \sin(π C_t / (2 C_0))\) then we can use \eqref{eq:approximation of sigma x and sigma y} to obtain
  \begin{align*}
    |σ_{t1} σ_t (y_{ℓb})| &=
      \left| \sin \left(π \sum_{j ∈ J_0} ℓ c_{tj} ϑ_j \right)\right| =
      \left| \sin \left(π \sum_{j ∈ J_0} ℓ c_{tj} ϑ_j  - ℓ_j c_{tj}\right)\right| >\\
      &> \frac{1}{2} \left| \sin \left( \frac{π C_t}{2 C_0}\right)\right|
  \end{align*}
  for all \(s + 1 < t ≤ n\).
  % QUESTION Why is \(\sum_j ℓ_j c_{tj}\) divisible by \(2\)?
  \todo{Why is \(\sum_j ℓ_j c_{tj}\) divisible by \(2\)?}
  Note that \(\sin(π C_t / (2 C_0)) ≠ 0\) as \(C_t / (2 C_0)\) cannot be an integer by the choice of \(C_0\). Now set
  \[
    C := \min_{s + 1 < t ≤ n} \frac{\left| \sin \left(π \frac{C_t}{2 C_0}\right) \right|}{2 |σ_{t1}(δ)|}
  \]
  then \(C\) satisfies the claim.
\end{proof}

\begin{lem}
  Let \(K\) be a number field of degree \(n > 0\) over \(ℚ\) and let \(a ∈
  \algint\) satisfy \eqref{eq:approximations of a}. Furthermore, let \(s\)
  denote the number of pairs of non-real embeddings. If \(\py_{eh}\) satisfies
  \cref{lem:approximations of sigma x and sigma y} for an arbitrary but fixed
  positive integer \(k ∈ ℕ \setminus \set{0}\) dividing \(h\), where \(e ∈ ℕ\)
  is an integer such that
  \begin{equation}\label{eq:def of e}
    |ε_1^e| > \frac{4 |δ_1|^{s + 1}}{C^{n - s - 1}}
  \end{equation}
  holds for the constant \(C\) of the same lemma, then
  \begin{thmlist}
    \item \(\py_{eh} | \py_{eℓ}\) implies \(h | ℓ\) in \(ℤ\) and
    \item \(\py_{eh}^2 | \py_{eℓ}\) implies \(h \py_{eh} | ℓ\) in \(\algint\).
  \end{thmlist}
\end{lem}
\begin{proof}
  Let \(q ∈ ℕ\) be such that \(0 < r < h\). As before let \(L := K[δ]\). Considering the
  field norm \(N_{L / ℚ}\) of \(\py_{eq}\) we use \cref{lem:real part of
  epsilon} to obtain
  \begin{align*}
    | N_{L / ℚ} (\py_{er}) | &=
        \prod_{i = 1}^n \prod_{j = 1}^2 \left| σ_{ij} \left( \frac{ε^{er} - ε^{-er}}{2δ}\right) \right| ≤\\
      &≤ \prod_{i = 1}^{s + 1} \frac{|ε_i^{er} - ε_i^{-er}| |ε_i^{-er} - ε_i^{er}|}{4 |δ_i|^2} \prod_{i = s + 2}^n \frac{1}{|δ_i|^2}\\
      &≤ \frac{4 |ε_1|^{2er(s + 1)}}{4^{s + 1} |N_{K/ℚ}(a^2 - 1)|} <
      \frac{|ε_1|^{2er (s + 1)}}{4^{s}},
  \end{align*}
  where the approximations follow completely analogously as in the proof of
  \cref{pro:epsilon essentially generaltes S}. As \(\py_{er}\) is in \(\algint\)
  we deduce that
  \[
    |N_{K / ℚ} (\py_{er})| < \frac{|ε_1|^{er (s + 1)}}{2^{s}}.
  \]

  On the other hand, by our assumption on \(|σ_i(\py_{eh})|\) we know that
  \begin{align*}
    |N_{K / ℚ} (\py_{eh})| &= \prod_{i = 1}^n | σ_i(\py_{eh}) | ≥
        C^{n - s - 1} \prod_{i = 1}^{s + 1} |σ_i(\py_{eh}) | =\\
      &= C^{n - s - 1} \prod_{i = 1}^{s + 1} \frac{|ε_i^{eh} - ε_i^{-eh}|}{2 |δ_i|} ≥
        C^{n - s - 1} \left( \frac{|ε_1|^{eh} - 1}{2 |δ_1|} \right)^{s + 1} >\\
      &> C^{n - s - 1} \frac{|ε_1|^{eh (s + 1)}}{2^{s + 2} |δ_1|^{s + 1}},
  \end{align*}
  where the second inequality follows from \(|ε_1^{-1}| ≤ 1\). Using our
  assumption on \(e\), it follows that
  \begin{equation}\label{eq:norm of x eh and x er}
    |N_{K / ℚ} (\py_{eh})| > | N_{L / ℚ}(\py_{er}) |.
  \end{equation}

  Let \(\py_{eh} | \py_{eℓ}\) and  set \(ℓ = t h + r\) for \(t, r ∈ ℕ\) with \(0 ≤ r < h\). Assume to reach a contradiction that \(r > 0\), then
  \[
    \py_{eℓ} = \py_{eth + er} \overset{\text{\cref{lem:addition formulas}}}{=}
    \py_{eth}\px_{er} + \px_{eth} \py_{er}.
  \]
  By \cref{lem:y m divides y mk} we know that \(\py_{eh} | \py_{eth}\) and
  consequently \(\py_{eh} | \px_{eth} \py_{er}\) in \(\algint\). But by
  \cref{lem:x m and y m are relative prime} the principal ideals \((\px_{eth})\)
  and \((\py_{eth})\) are relative prime. Now \cref{lem:y m divides y mk}
  implies that \((\px_{eth})\) and \((\py_{eh})\) are relative prime as well.
  And therefore \(y_{eh} | y_{er}\), contradicting \eqref{eq:norm of x eh and x
  er}. Consequently, \(r = 0\) and \(h | ℓ\) in \(ℤ\).

  To see the second divisibility condition we assume \(\py_{eh}^2 | \py_{eℓ}\)
  in \(\algint\). Then by the first part of the lemma, we know that there exists
  an integer \(t\) such that \(ℓ = th\). Using the binomial formula we obtain
  \begin{align*}
    \px_{eth} + δ\py_{eth} &= (ε^{eh})^t = (\px_{eh} + δ \py_{eh})^t =\\
      &= \sum_{\substack{0 ≤ i ≤ t\\2 \text{ even}}} \begin{pmatrix}t \\ i\end{pmatrix} \px_{eh}^{t - i} δ^i \py_{eh}^i +
      δ \sum_{\substack{0 ≤ i ≤ t\\2 \text{ odd}}} \begin{pmatrix}t \\ i\end{pmatrix} \px_{eh}^{t - i} δ^{i - 1} \py_{eh}^i
  \end{align*}
  and, since \((\px_{eh})\)  and \((\py_{eh})\) are relative prime, we can
  conclude that
  \[
    0 \equiv \py_{eth} \equiv t \px_{eh}^{t - 1} \py_{eh} \equiv t \py_{eh} \mod (\py_{eh}^2).
  \]
  It follows that \(\py_{eh} | t\) in \(\algint\) and therefore \(h \py_{eh} |
  ℓ\).
\end{proof}

%
% \begin{lem}\label{lem:rank of S is greater 0}
%   Let \(K\) be a totally real number field and \(a ∈ \algint\) satisfies \(δ =
%   δ(a) \not\in \algint\) as well as~\eqref{eq:embeddings of a into reals v1}.
%   Then \(ε(a) = ε\) is torsion free i.e.\ there is a group isomorphism \(⟨ε⟩ \cong
%   ℤ\).
% \end{lem}
% \begin{proof}
%   Assume otherwise that \(ε^m = 1\) for some positive integer \(m\) i.e. \(ε\) is an
%   \(m\)-th root of unity. As we have identified \(L\) with a subset of \(ℝ\), this
%   leaves \(±1\) as values for \(ε = 1 + δ\). This is impossible as \(δ \not\in
%   \algint\).
% \end{proof}
%
% As a consequence, of \cref{thm:Dirichlet} we find that
% \[
%   1 ≤ \rk S ≤ \rk U_L - \rk U_K ≤ (n + 1) - n = 1.
% \]
% In other words, we can write every \(η\) in \(S\) as a product \(ζ ε_0^k\), where \(ζ ∈ U_L\) is a root of unity and \(ε_0\)
