% !TEX encoding = UTF-8
% !TEX TS-program = xelatex
% !TEX spellcheck = en_GB
% !TEX engine = xelatex
% !TEX root = ../Herbstrith-H10_over_AI.tex
%
% ██   ██  ██  ██████      ████████  ██████  ████████    ██████  ███████
% ██   ██ ███ ██  ████        ██    ██    ██    ██       ██   ██ ██
% ███████  ██ ██ ██ ██        ██    ██    ██    ██ █████ ██████  █████
% ██   ██  ██ ████  ██        ██    ██    ██    ██       ██   ██ ██
% ██   ██  ██  ██████         ██     ██████     ██       ██   ██ ███████ ██

I will closely follow the papers of \textcite{Denef1980} and
\textcite{Pheidas1988}, whose structure in turn heavily depends on the article
\citetitle{Davis1973} by \textcite{Davis1973}. This way, one can prove the
undecidability of Hilbert's tenth problem over rings of algebraic integers in
totally real number fields and number fields with one pair of non-real
embeddings and at least one real embedding in one go. This approach is also
present in Sections 6.3 and 7.2 of the study~\cite{Shlapentokh2007} by
Shlapentokh. The same author has proven the second result independently of
Pheidas in her thesis~\cite{Shlapentokh1989}.

\subsection{Finitely many easy lemmas}
% ██      ███████ ███    ███ ███    ███  █████  ███████
% ██      ██      ████  ████ ████  ████ ██   ██ ██
% ██      █████   ██ ████ ██ ██ ████ ██ ███████ ███████
% ██      ██      ██  ██  ██ ██  ██  ██ ██   ██      ██
% ███████ ███████ ██      ██ ██      ██ ██   ██ ███████

We start by defining two sequences, that satisfy Pell's equation stated below.

\begin{equation} \label{eq:Pell}
    x^2 - d y^2 = 1
\end{equation}

Using modified versions of the techniques presented
by~\textcite{Matijasevic1970}, it will be shown that the index \(m\) can be
obtained in a Diophantine way from \(\py_m(a)\) for certain subsequences of the
sequences defined below.

\begin{defin}
  Let \(K\) be an algebraic number field, \(\algint\) its ring of algebraic
  integers and fix \(a ∈ \algint\). One defines \(δ(a) := \sqrt{a^2 - 1}\) and
  \(ε(a) := a + δ(a)\), where we demand that \(-π/2 < \arg δ(a) ≤ π/2\). If
  \(δ(a) \not\in K\) one defines \(\px_m(a), \py_m(a) ∈ \algint\) for \(m ∈ ℕ\)
  by
  \begin{equation}\label{eq:def of x and y}
    \px_m(a) + δ(a) \py_m(a) = {(ε(a))}^m.
  \end{equation}
\end{defin}

This definition includes the case \(K = ℚ\) with \(\algint = ℤ\)
of~\cite{Davis1973}. However, I am using the slightly modified notation
of~\cite{Denef1980,Pheidas1988}. Under the assumptions of the definition,
\(δ(a)\) is a root of the monic quadratic polynomial
\[
  X^2 - a^2 + 1 ∈ \algint {[X]}.
\]
Therefore, the extension \(K[δ(a)] / K\) is quadratic and \(δ(a) ∈
\algint[{K[δ(a)]}]\) in an algebraic integer. The sequences \(\px_m(a)\) and
\(\py_m(a)\) are well defined for each \(m ∈ ℕ\) as they correspond to the
coefficients of \({(ε(a))}^m\) in \(K[δ(a)]/K\) with respect to the basis
\(\lbrace 1, δ(a)\rbrace\). If the reference is clear, I will omit the
dependency on \(a\) writing \(δ, ε, \px_m\) and \(\py_m\) instead.
% QUESTION Why does every number field contain such an element \(a\)?
%\todo{Why does every number field contain such an element \(a\)?}
%In the following the number field \(K[δ(a)]\) will be denoted by \(L\).

\begin{rem}
  As \(L/K\) with \(L = K[δ]\) has degree two, there is exactly one pair of
  field automorphisms on \(L\) preserving \(K\) point-wise, namely \(σ_1^{K}(α
  + δβ) = α + δβ\) and \(σ_2^K(α + δβ) = α - δβ\) for \(α, β ∈ \algint\). The
  latter will be denoted by \(\sigmaK{η} = σ_2^K(η)\) to emphasize the analogy
  of complex conjugation.
\end{rem}

\begin{exam}
  Consider the number field \(K := ℚ[√2]\). It is not hard to check that
  \(\set{1, √2}\) is an integral basis for \(\algint\). We may choose \(a = 2\)
  as the integer in the definition of the sequences. Indeed, we find \(δ =
  √{a^2 - 1} = √3\) and if \(√3\) were contained in \(K\) then it would be an
  algebraic integer in \(\algint\). Thus, there exist \(m, n ∈ ℤ\) such that
  \[
    √3 = m + n√2.
  \]
  Now since neither \(√3\) nor \(√{3/2}\) are rational integers, we may assume
  that both \(m\) and \(n\) are non-zero. But then
  \[
    3 = m^2 + 2 m n √2 + 2 n^2
  \]
  which is equivalent to
  \[
    √2 = \frac{3 - m^2 - 2 n^2}{2 m n} ∈ ℚ.
  \]
  Note that \(ε = a + δ = 2 + √3\) then by definition of the sequences we have
  for all \(m ∈ ℕ\) that
  \begin{align*}
    \px_m + δ\py_m &= ε^m = \left(2 + √3\right)^m =
      \sum_{j = 0}^m \binom{m}{j} 2^{m - j} √{3^{j}} \\
    &= \sum_{\substack{j=0\\ j \text{ even}}}^m
          \binom{m}{j} 2^{m - j} 3^{\frac{j}{2}} +
       √3 \sum_{\substack{j=0\\ j \text{ odd}}}^m
          \binom{m}{j} 2^{m - j} 3^{\frac{j - 1}{2}}.
  \end{align*}
  Thus, we have found that
  \begin{align*}
    \px_m &= \sum_{\substack{j=0\\ j \text{ even}}}^m
          \binom{m}{j} 2^{m - j} 3^{\frac{j}{2}}\\
    \intertext{and}
    \py_m &= \sum_{\substack{j=0\\ j \text{ odd}}}^m
       \binom{m}{j} 2^{m - j} 3^{\frac{j - 1}{2}}.
  \end{align*}
\end{exam}

Let me now collect some properties of these sequences. The proofs are
generalized versions of the ones given in Davis' paper~\cite{Davis1973}.

\begin{lem}
  Let \(K\) be an algebraic number field and \(a ∈ \algint\) such that \(δ(a) \not\in K\). Then
  \begin{thmlist}
    \item\label{lem:epsilon is unit}
    \(ε\) is a unit in \(\algint[L]\), its
    inverse is given by \(ε^{-1} = a - δ = \sigmaK{ε}\), and
    \item\label{lem:x and y solve Pells equation}
    \(\px_m, \py_m\) satisfy Pell's equation~\eqref{eq:Pell} for all \(m
    ∈ ℕ\), using \(d = {δ(a)}^2\) as parameter.
  \end{thmlist}
\end{lem}
\begin{proof}
  \begin{plist}
    \item We have \(ε \; (a - δ) = (a + δ) (a - δ) = a^2 - δ^2 = 1\) as desired.

    \item One uses induction on \(m\). If \(m = 0\), the pair \(\px_0 =
    1\) and \(\py_0 = 0\) yields a trivial solution to
    equation~\eqref{eq:Pell}. Let the claim be proven for all pairs
    \(\px_n, \py_n\) with \(n ≤ m\). Then rewriting the definition of
    \(\px_{m + 1}, \py_{m + 1}\) one obtains
    \[
      \px_{m + 1} + δ \py_{m + 1} = ε^{m + 1} = (\px_m + δ \py_m)ε.
    \]
    Applying the automorphism \(\sigmaK{\cdot}\) implies
    \[
      \sigmaK{\px_{m + 1} + δ \py_{m + 1}} = \px_{m + 1} - δ \py_{m + 1} =
      (\px_m - δ \py_m) ε^{-1}
    \]
    and multiplication of both equations yields
    \[
      \px_{m + 1}^2 - d \py_{m + 1}^2 = 
      (\px_{m + 1} + δ \py_{m + 1}) (\px_{m + 1} - δ \py_{m + 1}) = 1,
    \]
    as claimed.
  \end{plist}
\end{proof}

The defining equation
\[
  \px_m + δ \py_m = ε^m = {(\px_1 + δ \py_1)}^m
\]
can be seen as an analogue of the trigonometric identity
\[
  \cos m + i \sin m = e^{im} = {(\cos 1 + i \sin 1)}^m,
\]
where \(\px_m\) plays the role of \(\cos m\), \(\py_m\) the one of \(\sin m\),
and \(i\) is replaced by \(δ\). In this view Pell's equation~\eqref{eq:Pell} is
the analogue of the Pythagorean identity
\[
  {\cos (m)}^2 + {\sin (m)}^2 = 1.
\]

The next lemma proves the identities corresponding to \(\cos m = \Re e^{im}\),
\(\sin m = \Im e^{im}\), and the addition formulas.

\begin{lem}
  Let \(K\) be an algebraic number field and \(a ∈ \algint\) such that \(δ = δ(a) \not\in K\). Then for all \(m, k ∈ ℕ\) one has
  \begin{thmlist}
    \item\label{lem:real part of epsilon}
    \(\px_m = (ε^m + ε^{-m}) / 2\) and \(\py_m = (ε^m - ε^{-m}) / (2 δ)\), as well as,
    \item\label{lem:addition formulas}
    \(\px_{m ± k} = \px_m \px_k ± δ^2 \py_m \py_k\), and
    \(\py_{m ± k} = \px_k \py_m ± \px_m \py_k\).
  \end{thmlist}
\end{lem}
\begin{proof}
  \begin{plist}
    \item In \cref{lem:epsilon is unit} we have seen that \(ε^{-1} =
    \sigmaK{ε}\) and therefore \(ε^{-m} = {\left(\sigmaK{ε}\right)}^m\). Observe
    that for arbitrary \(α, β ∈ \algint\) we have
    \[
      α + β δ + \sigmaK{α + δ β} = 2α \quad \text{and} \quad
      α + β δ - \sigmaK{α + δ β} = 2δ β.
    \]
    Now, setting \(α + δ β = ε^m\) yields the claim.
    
    \item By the defining equation for \(\px_{m + k}\) and \(\py_{m + k}\) we
    have
    \begin{align*}
      \px_{m + k} + δ \py_{m + k} &= ε^{m + k} = 
        (\px_m + δ \py_m) (\px_k + δ \py_k) =\\
      &= (\px_m \px_k + δ^2 \py_m \py_k) + δ (\px_m \py_k + \px_k \py_m)
    \end{align*}
    and thus
    \begin{align*}
      \px_{m + k} &= \px_m \px_k + δ^2 \py_m \py_k, \\
      \py_{m + k} &= \px_m \py_k + \px_k \py_m.
    \end{align*}
    The identities for \(\px_{m - k}\) and \(\py_{m - k}\) follow analogously.
  \end{plist}
\end{proof}

Setting \(k = 1\) in the lemma above, one obtains \(\px_{m ± 1} = a \px_m ± δ^2
\py_m\) and \(\py_{m ± 1} = a \py_m ± \px_m\). A further immediate consequence
of this lemma is the subsequent one, which brings divisibility into play.

\begin{lem}
  Let \(K\) be a number field and \(a ∈ \algint\) such that \(δ = δ(a) \not\in K\).
  Then for all \(m, k ∈ ℕ\), \(k ≠ 0\) we have that
  \begin{thmlist}
    \item\label{lem:y m divides y mk}
    \(\py_m\) divides \(\py_{mk}\) in \(\algint\),

    \item \(\py_{mk} \equiv k \px_m^{k - 1} \py_m \mod \left(\py_m^3\right)\) in
    \(\algint\), as well as

    \item\label{lem:x m and y m are relative prime}
    the principal ideals \((\px_m)\) and \((\py_m)\) are relative prime in
    \(\algint\) for all \(m ∈ ℕ\)
  \end{thmlist}
\end{lem}
\begin{proof}
  \begin{plist}
    \item I argue by induction on \(k\). The claim is trivial if \(k = 1\) and
    \cref{lem:addition formulas} implies that
    \[
      \py_{m(k + 1)} = \px_m \py_{mk} + x{mk} \py_m.
    \]

    If the claim is proven for all factors up to \(k\), one finds that
    \(\py_m \mid \py_{mk}\) and \(\py_m \mid \py_m\) trivially. As a
    consequence, \(\py_m \mid \py_{m(k + 1)}\).

    \item Again the defining equation yields
    \begin{align*}
      \px_{mk} + δ \py_{mk} &= ε^{mk} = {(\px_m + δ \py_m)}^k =
                        \sum_{j = 0}^k \binom k j \px_m^{k - j} \py_m^j δ^j \\
      \intertext{and}
      \py_{mk} &= \sum_{\substack{j=0\\ j \text{ odd}}}^k
                \binom k j \px_m^{k - j} \py_m^j δ^{j -1}.
    \end{align*}
    In the equation above all terms for \(j > 1\) are divisible by \(\py_m^3\)
    and thus vanish modulo \(\left(\py_m^3\right)\). The only term remaining
    is \(k \px_m^{k - 1} \py_m\) as claimed.

    \item Since \((\px_m, \py_m)\) is a solution to bells equation, we know that
    \[
      1 = \px_m^2 - (a^2 - 1) \py_m^2.
    \]
    is contained in the sum of ideals \((\px_m) + (\py_m)\) and thus the ideals
    are relative prime as claimed.
  \end{plist}
\end{proof}

The next lemma even though being easy to prove provides a valuable tool in
studying the sequences \(\px_m\) and \(\py_m\). It derives a recursive
definition and lets one prove properties of the sequences, by proving them for
\(m ∈ \lbrace 0, 1 \rbrace\) and inferring the properties for \(m + 1\) from
\(m\) and \(m - 1\).

\begin{lem}\label{lem:recursion for x_m and y_m}
  Let \(K\) be a number field and \(a ∈ \algint\) such that \(δ = δ(a) \not\in
  K\). For \(m > 1\) the following recursive conditions hold in \(\algint\).
  \begin{align*}
    \px_{m + 1} &= 2 a \px_m - \px_{m - 1}, & \px_1 = a, \;& \px_0 = 1 \\
    \py_{m + 1} &= 2 a \py_m - \py_{m - 1}, & \py_1 = 1, \;& \py_0 = 0 \\
  \end{align*}
\end{lem}
\begin{proof}
  The initial conditions follow from \(ε = a + δ\) and \(ε^0 = 1\). To prove the
  the difference equations one uses \cref{lem:addition formulas} and obtains
  \begin{align*}
    \px_{m + 1} &= a \px_m + δ^2 \py_m,  &  \py_{m + 1} &= a \py_m + \px_m, \\
    \px_{m - 1} &= a \px_m - δ^2 \py_m,  &  \py_{m - 1} &= a \py_m - \px_m.
  \end{align*}
  Summation yields \(\px_{m + 1} + \px_{m - 1} = 2 a \px_m\) and \(\py_{m + 1} + \py_{m - 1}
  = 2 a \py_m\).
\end{proof}

One applies the previous lemma to prove some congruence conditions.

\begin{lem}
  Let \(K\) be a number field and \(a, b, c ∈ \algint\) such that \(δ(a), δ(b)
  \not\in K\). Then for all \(m ∈ ℕ\) the following congruences hold in
  \(\algint\).
  \begin{thmlist}
    \item\label{lem:m congruent y m}
      \(\py_m (a) \equiv m \mod (a - 1)\)

    \item\label{lem:a congruent b mod c}
      If \(a \equiv b \mod (c)\), then
      \(\px_m (a) \equiv \px_m (b) \mod (c)\) and
      \(\py_m(a) \equiv \py_m(b) \mod (c)\).
  \end{thmlist}
\end{lem}
\begin{proof}
  Both congruences become equalities if \(m = 0\). As for \(m = 1\), the first
  congruence is again an equality as \(\py_1 (a) = 1\) independently of \(a\).
  The second claim is trivial since \(x_1 (η) = η\) and \(\py_1 (η) = 1\) for
  \(η ∈ \lbrace a, b \rbrace\). At this point one proceeds inductively and
  assumes the claims to be proven for all indices lower than \(m + 1\).

  \begin{plist}
    \item Note that \(a \equiv 1 \mod (a - 1)\) and thus by
    \cref{lem:recursion for x_m and y_m}
    \[
      \py_{m + 1} = 2 a \py_m - \py_{m - 1} \equiv 2 m - (m - 1) =
      m + 1 \mod (a - 1)
    \]
    as claimed.

    \item Using \cref{lem:recursion for x_m and y_m} again, we see that for
    fixed \(m\) the coefficients \(\px_m (η)\) and \(\py_m (η)\) can be
    expressed as some fixed polynomial in \(η\). For the congruence this means
    \begin{align*}
      \px_{m + 1} (a) &= 2 a \px_m (a) - \px_{m - 1} (a)
                     \equiv 2 b \px_m (b) - \px_{m - 1} (b) = \px_{m + 1} (b)
                     \mod (c)
    \end{align*}
    and for \(\py_{m + 1}\) completely analogously.
  \end{plist}
\end{proof}

\begin{lem}\label{lem:congruence x_2m+k}
  Let \(K\) be a number field and \(a ∈ \algint\) such that \(δ = δ(a) \not\in
  K\). Then for \(m, k ∈ ℕ\) such that \(m ± k ≥ 0\) the following congruence
  holds in \(\algint\).
  \[
    \px_{2 m ± k} \equiv - \px_k \mod (\px_m)
  \]
\end{lem}
\begin{proof}
  By applying the addition formulas of \cref{lem:addition formulas} twice and
  using that \(\px_m\) and \(\py_m\) solve Pell's equation~\eqref{eq:Pell} one obtains
  \begin{align*}
    \px_{2m ± k} &= \px_m \px_{m ± k} + δ^2 \py_m \py_{m ± k}
                \equiv δ^2 \py_m (\py_m \px_k ± \px_m \py_k) \\
               &\equiv δ^2 \py_m^2 \px_k = (\px_m^2 - 1) \px_k
                \equiv -\px_k \mod (\px_m).
  \end{align*}
\end{proof}

At this point for the first time in this section I state a result that is no
direct generalization of a result of~\textcite{Davis1973} and present proofs
given in~\cite{Denef1980} or \cite{Shlapentokh2007}. Note however that the
results are nevertheless true for the case \(K = ℚ\) and \(\algint = ℤ\).

\begin{lem}
  Let \(K\) be a number field and \(a ∈ \algint\) such that \(δ = δ(a) \not\in
  K\). Then for all non-negative integers \(k, m ∈ ℕ\) the following congruence
  holds in \(\algint\).
  \[
    \px_{2km} \equiv (-1)^k \mod (\px_m)
  \]
\end{lem}
\begin{proof}
  If \(k = 0\) the congruence becomes and identity and if \(k = 1\) the claim
  follows directly from the lemma above. Assuming the claim to be proven for all
  integers lower than \(k\), we find---by applying \cref{lem:addition formulas}
  twice---that
  \begin{align*}
    \px_{2km} &= \px_{2(k-1)m}\px_{2m} + δ^2 \py_{2 (k-1) m}\py_{2m} \equiv
               (-1)^k + δ^2 \py_{2 (k-1) m}\py_{2m} \\
              &= (-1)^k + δ^2 \py_{2 (k-1) m} \; 2 \px_m\py_m \equiv
               (-1)^k \mod (\px_m)
  \end{align*}
\end{proof}

\begin{lem}\label{lem:forcing divisibility}
  Let \(K\) be a number field and \(a ∈ \algint\) such that \(δ = δ(a) \not\in
  K\). Then for all \(η ∈ \algint \setminus \set{0}\) there exists an \(m ∈ ℕ\)
  such that \(η \mid \py_m\) in \(\algint\).
\end{lem}
\begin{proof}
  I claim that the factor ring \(\algint[L]/(2 δ η)\), where \((2 δ η)\) denotes
  the principal ideal generated by \(2 δ η\) in \(\algint[{L}] =
  \algint[{K[δ]}]\), is finite.
  
  To show this let \(α ∈ (2 δ η) \setminus \set{0}\) and let \(a_0 ∈ ℤ\) be the
  constant term of its minimal polynomial \(μ_{ℚ, α}(X) := α^m + … + a_1 α +
  a_0 ∈ ℤ[X]\). Since \(α\) is non-zero and \(μ_{ℚ, α}\) is irreducible, \(a_0\)
  is non-zero as well. Furthermore, note that \(a_0 = -α^m - … - a_1 α ∈ (α)\)
  and thus we have the inclusion of \(\algint[L]\)-ideals
  \[
    (0) \subsetneq (a_0) ⊂ (α) ⊂ (2 δ η).
  \]
  If we can show that \(\algint[L]/(a_0)\) is finite, then the
  \(\algint[L]/(a_0)\)-ideal \((2 δ η) / (a_0)\) is finite as well. Observing
  \[
    \algint[L]/(2 δ η) \cong (\algint[L] / (a_0)) \; / \; ((2 δ η) / (a_0))
  \]
  will prove the claim. Let \(\seq[ℓ]{ζ} ∈ \algint[L]\) with \(ℓ := [L : ℚ]\) be
  an integral basis of \(\algint[L]\) over \(ℚ\). Then every \(β ∈ \algint[L]\)
  can be written as \(β = k_1 ζ_1 + … k_ℓ ζ_ℓ\) for some \(\seq[ℓ]{k} ∈ ℤ\). But
  every \(k_i\) must belong to one of at most \(|a_0|\) many congruence classes
  modulo \(a_0 ℤ ⊂ a_0 \algint[L] = (a_0)\). Thus, \(k_i\) must belong to one of
  at most \(|a_0|\) cosets of \(\algint[L]/(a_0)\). Since \(\seq[ℓ]{k}\)
  determine every \(β ∈ \algint[L]\) uniquely, the factor ring
  \(\algint[L]/(a_0)\) can have at most cardinality \(|a_0|^ℓ\).
  
  Let \(m\) be the order of the group of units in the finite ring
  \(\algint[L]/(2 δ η)\). Then \(ε^{±m} \equiv 1 \mod (2 δ η)\). Hence, \(2 δ η
  \mid ε^m - ε^{-m}\) in \(\algint[{L}]\) and therefore
  \[
    \left. η \;\middle\vert\; \frac{ε^m - ε^{-m}}{2 δ} \right.
  \]
  in \(\algint[{L}]\), where the right hand side equals \(\py_m\) by
  \cref{lem:real part of epsilon}. Thus, there exists \(ζ ∈ \algint[L]\) such
  that \(ηζ = \py_m\). Now since \(η\) is non-zero, it is invertible in \(K\).
  Hence, \(ζ = \py_m η^{-1}\) is contained in \(K\). In fact, since \(\algint\)
  is integrally closed, we even find that \(ζ\) is contained in \(\algint\) and
  \(η\) divides \(\py_m\) in \(\algint\) as claimed.
\end{proof}

\begin{lem}\label{lem:subgroup of ker N L/K}
  Let \(K\) be a number field and \(a ∈ \algint\) such that \(δ = δ(a) \not\in
  K\). Then the set
  \[
    G := \set{α + δ β :
             (α, β) ∈ \algint^2
             \text{ is a solution to~\eqref{eq:Pell} with parameter } d = δ^2}
  \]
  is a subgroup of the kernel of the norm map \(N_{L/K}: U_L → U_K\), where
  \(U_K\) and \(U_L\) denote the groups of units in \(\algint\) and
  \(\algint[L]\) respectively.
\end{lem}
\begin{proof}
  First of all, note that, if \(α + δ β ∈ G\), so is \(\sigmaK{α + δ β} = α - δ
  β ∈ G\) because
  \[
    α^2 - d {(-β)}^2 = α^2 - d β^2 = 1.
  \]
  Now let \(α + δ β\) be an arbitrary element of \(G\), then
  \[
    N_{L/K}(α + δ β) = (α + δ β) \sigmaK{α + δ β} = α^2 - d β^2 = 1.
  \]
  This implies that \(α + δ β ∈ \ker N_{L / K}\) but also that \(α + δ β\) is a
  unit, as \(α - δ β\) is its inverse. The product of two arbitrary elements \(α_1 + δ β_1, α_2 + δ β_2 ∈ G\) is
  \[
    (α_1 + δ β_1)(α_2 + δ β_2) = (α_1 α_2 + δ^2 β_1 β_2) + δ (α_1 β_2 + α_2 β_1).
  \]
  We apply the automorphism \(\sigmaK{\cdot}\) and multiply to obtain
  \begin{align*}
    (α_1 α_2 + δ^2 β_1 β_2)^2 - δ^2 (α_1 β_2 + α_2 β_1)^2 &=
    (α_1 + δ β_1)(α_2 + δ β_2)\,\sigmaK{(α_1 + δ β_1)(α_2 + δ β_2)} =\\
    &=
    (α_1 + δ β_1)\sigmaK{α_1 + δ β_1}(α_2 + δ β_2)\sigmaK{α_2 + δ β_2} = 1.
  \end{align*}
  As a consequence, \(G\) is closed under multiplication and the claim is
  proven.
\end{proof}

\begin{lem}\label{lem:rank of N_L/K U_L}
  Let \(L\) and \(K\) be number fields as defined above.
  The image \(N_{L / K}\left(U_L\right) ≤ U_K\) has finite index in \(U_K\).
\end{lem}
\begin{proof}
  I claim that \(N_{L / K}\left(U_L\right)\) contains \(α^2\) for every \(α ∈ U_K\).
  This is because the restriction \(σ_i^K|_{\algint}\) is just the identity on
  \(\algint\) for \(i ∈ \set{1, 2}\) and therefore, \(N_{L / K}(α) = α^2\) for all
  \(α ∈ U_K \subseteq U_L\).

  Let now \(k := \rk U_K\) and identify \(U_K = μ(K) \times ℤ^{k}\), where
  \(μ(K)\) is the finite cyclic group of roots of unity in \(K\)
  (cf.~\cref{thm:Dirichlet}). Consider the following \(k\) elements
  \[
    ([0],1,0,…,0), \; ([0],0,1,0,…,0), \; …, \; ([0], 0, …, 0, 1)
  \]
  contained in \(U_K\). By the claim their ‘squares’ are contained in \(N_{L / K}\left(U_L\right)\) i.e.
  \[
    ([0],2,0,…,0), \; ([0],0,2,0,…,0), \; …, \; ([0], 0, …, 0, 2) ∈ N_{L / K}\left(U_L\right).
  \]
  As a consequence, the direct product
  \[
    G := \set{[0]} \times \underbrace{2 ℤ \times … \times 2 ℤ}_{k\text{-times}}
  \]
  is a subgroup of \(N_{L / K}\left(U_L\right)\) and therefore
  \[
    [U_K : N_{L / K}\left(U_L\right)] ≤ [U_K : G] < ∞.
  \]
\end{proof}

As for the free ranks of \(U_K\), \(U_L\), \(N_{L / K}\left(U_L\right)\) and
\(G\) the lemma above implies that \(\rk N_{L / K}\left(U_L\right) = \rk U_K\)
and therefore, as an immediate consequence of the first isomorphism
theorem~\cite[see][II~§1, p.~89]{Lang2002} the following inequality holds
\begin{equation}\label{eq:rank of G}
    \rk G ≤ \rk \ker N_{L / K} = \rk U_L - \rk U_K.
\end{equation}

Before proving the main result of this section (\cref{cor:ZZ is Diophantine over
O K}) I sketch how \textcite{Davis1973} establishes the \textsc{DPRM}-theorem.

\DPRM*

First he proves using the sequences above that the exponential function is
Diophantine over \(ℕ\) \cite[Thm 3.3]{Davis1973}. Then he is able to extend the
language of Diophantine predicates by \emph{bounded existential} and
\emph{bounded universal quantifiers}, i.e.\ by
\begin{align*}
  {(∃y)}_{≤x}ϕ(x, y) \quad &⇔ \quad ∃y\; (y ≤ x ∧ ϕ(x, y)),\\
  {(∀y)}_{≤x}ϕ(x, y) \quad &⇔ \quad ∀y\; (y > x ∨ ϕ(x, y))
\end{align*}
where \(ϕ\) is a positive existential formula~\cite[Thm 5.1]{Davis1973}. The
first one is easily seen to be Diophantine as the order relation on \(ℕ\) is
Diophantine. Proving the second claim takes the rest of the section. Now using
this result together with the sequence number theorem~\cite[Thm 1.3]{Davis1973}
Davis proofs that a function is Diophantine over \(ℕ\) if and only if it is
computable~\cite[Thm 6.1]{Davis1973}.

This already implies the \textsc{DPRM}-theorm as Davis has introduced
Diophantine pairing functions in~\cite[Thm 1.1]{Davis1973} and therefore all
ranges of Diophantine---and therefore all computable functions---are Diophantine
over \(ℕ\). But the ranges of computable functions are exactly the
semi-decidable subsets of \(ω\) by \cref{pro:characterizations of ce sets}, thus
proving the claim of the theorem.

\subsection{Diophantine definition of \(ℤ\) over \(K\)}
% ██████  ██  ██████  ██████  ██   ██        ██████  ███████ ███████
% ██   ██ ██ ██    ██ ██   ██ ██   ██        ██   ██ ██      ██
% ██   ██ ██ ██    ██ ██████  ███████        ██   ██ █████   █████
% ██   ██ ██ ██    ██ ██      ██   ██        ██   ██ ██      ██
% ██████  ██  ██████  ██      ██   ██ ██     ██████  ███████ ██   ██

For the remainder of this section let \(K ≠ ℚ\) be a totally real number field
or a number field with exactly one pair of non-real embeddings of degree \(n :=
[K : ℚ] ≥ 3\) over the rationals \(ℚ\). For any number field \(K\) we set
\(r_K\) to be the number of real embeddings of \(K\) and \(s_K\) to be the
number of pairs of complex-conjugate embeddings of \(K\). Then the conditions on
the number fields we are considering in this section can be restated as \(r_K =
n > 1\), or \(r_K = n - 2 > 0\) and \(s_K = 1\) respectively. As before we set
\(L = K[{δ(a)}]\), where \(δ(a) \not\in \algint\) is a root of \(X^2 - a^2 + 1\)
and \(-π/2 < \arg δ(a) ≤ π/2\).

Furthermore, let us assume that \(σ_1 = \id_K, σ_2, …, σ_n: K → ℂ\) are all
embeddings of \(K\) into the complex pane \(ℂ\). If \(s_K = 1\) we demand
without loss of generality that \(K, σ_2(K) \not\subset ℝ\) and that \(σ_2(α) =
\overline{σ_1(α)}\) for all \(α ∈ K\). In other words, \((σ_1, σ_2)\) is the
pair of complex embeddings and all other morphisms embed \(K\) into the reals
\(ℝ\).

\begin{lem}\label{lem:L over K is quadratic}
  Let \(K ≠ ℚ\) be a number field of degree \(n\) over \(ℚ\). If \(a ∈ \algint\)
  satisfies
  \begin{equation}\label{eq:approximations of a}
    \begin{cases}
      r_K = n > 1\\
      a > 2^{2(n + 1)}\\
      0 < σ_i(a) < \frac{1}{2} &\text{for } 1 < i ≤ n
    \end{cases}
    \quad \text{or} \quad
    \begin{cases}
      r_K = n - 2 > 0\\
      |σ_i(a)| > 2^{2(n + 1)} &\text{for } i ∈ \set{1, 2}\\
      0 < σ_i(a) < \frac{1}{2} &\text{for } 2 < i ≤ n
    \end{cases},
  \end{equation}
  then \(δ(a) = \sqrt{a^2 - 1}\) is not contained in \(K\).
\end{lem}
\begin{proof}
  By assumption we have \(0 < σ_n(a) < 1/2\) and therefore \(σ_n(a)^2 - 1 < 0\)
  cannot be a square in the real number field \(σ_n(K) \subseteq ℝ\). As \(K\)
  is isomorphic to \(σ_n(K)\), the algebraic integer \(δ(a) = \sqrt{a^2 - 1}\)
  cannot be contained in \(K\).
\end{proof}

\begin{lem}
  Let \(K ≠ ℚ\) be a number field of degree \(n\) over \(ℚ\) and \(s_K ∈ \set{0,
  1}\) the number of pairs of non-real embeddings \(σ: K → ℂ\). Then there
  exists an algebraic integer \(a ∈ \algint\) that satisfies
  \eqref{eq:approximations of a}.
\end{lem}
\begin{proof}
  We will apply the strong approximation theorem~(\ref{thm:strong
  approximation}) to prove the existence of such an algebraic integer \(a\). To
  this end, we consider the set of absolute values \(\mathcal{F}_K :=
  \set{|\cdot|_1, |\cdot|_{s_K + 1}, …, |\cdot|_n}\), where \(|\cdot|_i\)
  denotes the absolute value defined by
  \[
    |x|_i := |σ_i(x)|_ℂ.
  \]
  
  By the strong approximation theorem there exists \(b ∈ K\) such that
  \begin{align}
    \left\vert σ_i(b) - \frac{1}{2^5} \right\vert =
    \left\vert σ_i \left(b - \frac{1}{2^5}\right) \right\vert & <
    \frac{1}{2^6}
    & \text{for} \; s_K + 1 < i ≤ n \; \text{and}
    \label{eq:approx of b 1}\\
    |b|_{\mathfrak{p}} & ≤ 1 & \text{for every prime ideal} \; \mathfrak{p}
    \label{eq:approx of b 2}
  \end{align}
  holds. Note that \eqref{eq:approx of b 2} implies that \(b ∈ \algint\) is an
  algebraic integer by \cref{lem:p adic absolute values}. Form \eqref{eq:approx
  of b 1} we firstly conclude, that \(b\) is non-zero as
  \[
    \left\vert σ_n(0) - \frac{1}{2^5} \right\vert = \frac{1}{2^5} >
    \frac{1}{2^6}.
  \]
  Secondly, we find that for all \(s_K + 1 < i ≤ n\) we have
  \[
    |σ_i(b)| = \left\vert σ_i(b) - \frac{1}{2^5} + \frac{1}{2^5} \right\vert ≤
    \left\vert σ_i(b) - \frac{1}{2^5} \right\vert + \frac{1}{2^5} <
    \frac{1}{2^6} + \frac{1}{2^5} < \frac{1}{2^4}.
  \]
  Now since \(b\) is non-zero, we know that
  \[
    1 ≤ |N_{K/ℚ}(b)| =
    \prod_{i = 1}^{s_K + 1} |σ_i(a)| \prod_{i = s_K + 2}^{n} |σ_i(a)| <
    |σ_1(b)|^{s_K + 1} 2^{-4 (n - s_K + 2)}.
  \]
  We conclude that
  \[
    2^{4 (n + 1)} ≤ 2^{4 (n - s_K + 2)} < |σ_1(b)|^{s_K + 1} < |σ_1(b)|^2.
  \]
  Setting \(a := b\) finishes the proof for the case \(r_K = n - 2\). If all
  embeddings are real, we set \(a := |b|_ℂ\).
\end{proof}


\begin{exam}
  Consider again the case of \(K := ℚ[√2]\). Since both embeddings of \(K\)
  into \(ℂ\) are uniquely determined by \(σ(√2) = ± √2\) it suffices to find two
  integers \(k, ℓ ∈ ℤ\) such that \(|k + ℓ √2| > 2^{2 \cdot 2 + 2} = 64\) and
  \(|k - ℓ √2| < 1/2\). Then we can set \(a := k + ℓ √2\) and the \(a\) fulfils
  \eqref{eq:approximations of a}. Such a pair of integers is given by \(a = 34 +
  24 √2\). In this case
  \[
    δ = \sqrt{4 {\left(12 \sqrt{2} + 17\right)}^{2} - 1}.
  \]
\end{exam}

\begin{rem}
  As the expansion \(L / K\) is quadratic by \cref{lem:L over K is
  quadratic}, every \(σ_i\) can be extended to exactly two embeddings \(σ_{i1}\)
  and \(σ_{i2}\) of \(L\) into the complex plane \(ℂ\) by ‘composing’ with
  \(σ_1^K\) or \(σ_2^K\). This yields
  \begin{equation}\label{eq:def of sigma ij}
    \begin{aligned}
      σ_{i1}(α + δβ) &= σ_i(α) + \sqrt{{σ_i(a)}^2 - 1}\, σ_i(β) \quad \text{and} \\
      σ_{i2}(α + δβ) &= σ_i(α) - \sqrt{{σ_i(a)}^2 - 1}\, σ_i(β)
    \end{aligned}
  \end{equation}
  for all \(α, β ∈ \algint\) and all \(1 ≤ i ≤ n\).
\end{rem}

I will identify the field \(L\) with its embedding \(σ_{11}(L)\) and write \(x\)
instead of \(σ_{11}(x)\) for its elements.

\begin{lem}\label{lem:r and s for tr and opnr}
  Let \(K ≠ ℚ\) be a number field of degree \(n\) over \(ℚ\) and let \(a ∈
  \algint\) be such that \eqref{eq:approximations of a} is satisfied. Then
  \begin{thmlist}
    \item if \(r_K = n\), only \(σ_{11}\) and \(σ_{12}\) embed \(L\) into the
    reals, and
    \item if \(r_K = n - 2\), the field \(L\) is totally complex.
  \end{thmlist}
\end{lem}
\begin{proof}
  \begin{plist}
    \item If \(K\) is totally real and \(i > 1\), then \(0 < σ_i(a) < 1/2\) and
    therefore the radicands in \eqref{eq:def of sigma ij} are both negative. As
    a consequence, \((σ_{i1}, σ_{i2})\) is a pair of non-real embeddings.

    On the other hand, if \(i = 1\) then \(a > 2^{2(n + 1)} > 1\) and the
    radicands are both positive. We deduce that \(σ_{11}\) and \(σ_{12}\) are
    both real embeddings and \(L\) is a subfield of the reals by our
    identification.

    \item As \(σ_1\) and \(σ_2\) are already non-real embeddings and \(σ_i^K\)
    preserve \(σ_j(K)\) point-wise (\(1 ≤ i, j ≤ 2\)), \(σ_{11}, σ_{12},
    σ_{21}\) and \(σ_{22}\) are non-real as well. For the remaining embeddings
    one argues completely analogously to (i).
  \end{plist}
\end{proof}

\begin{lem}\label{lem:properties of sigma epsilon}
  Let \(K ≠ ℚ\) be a number field of degree \(n\) over \(ℚ\) and let \(a ∈
  \algint\) be such that \eqref{eq:approximations of a} is satisfied. If \(s_K\)
  is the number of pairs of non-real embeddings of \(K\), then
  \begin{thmlist}
    \item \(σ_{i1}(ε)^{-1} = σ_{i2}(ε)\) for all \(1 ≤ i ≤ n\),

    \item \(σ_{i1}(ε)\) and \(σ_{i2}(ε)\) are complex conjugates for \(s_K + 1 <
    i ≤ n\), and

    \item\label{lem:modulus of sigma epsilon}
    \(|σ_{i1}(ε)| = |σ_{i2}(ε)| = 1\) for \(s_K + 1 < i ≤ n\).
  \end{thmlist}
\end{lem}
\begin{proof}
  In \cref{lem:epsilon is unit} we have seen, that the claim holds true for \(i
  = 1\). We extend this method to obtain the results for the other cases. For
  all \(1 ≤ i ≤ n\) we have
  \begin{align*}
    σ_{i1}(ε) σ_{i2}(ε) &= (σ_i(a) + σ_{i1}(δ)) (σ_i(a) - σ_{i1}(δ)) =\\
      &= σ_i(a)^2 - σ_{i1}(δ)^2 = σ_i(a)^2 - σ_i(a)^2 + 1 = 1.
  \end{align*}

  For all \(s_K + 1 < i ≤ n\) we have defined \(σ_i: K → ℂ\) to be a real
  embedding. Thus \(σ_i(a)\) is a real number and as \(0 < σ_i(a) < 1/2\), we
  find that \(σ_{i1}(δ)\) is purely imaginary. Hence, we deduce that
  \(σ_{i1}(ε)\) and \(σ_{i2}(ε)\) are complex conjugates. But then the complex
  moduli of these algebraic integers must coincide, leaving no other option than
  \(|σ_{i1}(ε)| = |σ_{i2}(ε)| = 1\).
\end{proof}

Before we can start proving some approximations for the complex moduli of \(ε,
δ\) and \(a\), we need to fix some notations.

\begin{defin}
  Let \(K ≠ ℚ\) be a number field of degree \(n\) over \(ℚ\) and let \(a ∈
  \algint\) be such that \eqref{eq:approximations of a} is satisfied. For \(1 ≤
  i ≤ n\) we set
  \begin{thmlist}
    \item \(a_i := σ_{i}(a)\),
    \item \(ε_i := σ_{i1}(ε)\) if \(|σ_{i1}(ε)| ≥ 1\) and \(ε_i := σ_{i2}(ε)\)
    otherwise, and
    \item \(δ_i := σ_{i1}(δ)\).
  \end{thmlist}
\end{defin}

\begin{rem}
  \begin{exlist}
    \item In the definition above we could have equivalently defined \(ε_i :=
    σ_{i1}(ε)\) for \(s_K + 1 < i ≤ n\), as by \cref{lem:modulus of sigma
    epsilon} the complex modulus of \(σ_{i1}(ε)\) is \(1\).

    \item Note that by \eqref{eq:def of sigma ij} we have \(σ_{i2}(δ) = -δ_i\)
    and therefore \(|δ_i| = |σ_{i2}(δ)|\) for all \(1 ≤ i ≤ n\).
  \end{exlist}
\end{rem}

We will use the following result by Kronecker.

\begin{lem}
  If a non-zero algebraic integer \(η\) and all its conjugates have complex
  modulus not exceeding \(1\), then \(η\) is a root of unity.
\end{lem}
\begin{proof}
  Let \(M := ℚ[η]\) and \(n := [M : ℚ]\). Since \(1, η, η^2, η^3, …\) is a
  sequence contained in \(M\), all minimal polynomials \(μ_{ℚ, η^k}\) have at
  most degree \(n\). As all of the conjugates of \(η\) lie within the closed
  unit disk, so do all the complex conjugates of \(η^k\) (for \(k ∈ ℕ\)). This
  implies that all the coefficients of the minimal polynomials \(μ_{ℚ, η^k}\)
  lie between \(±\max\set{{n \choose j} : 0 ≤ j ≤ n}\). We deduce that the
  sequence \(1, η, η^2, η^3, …\) can at most have finitely many distinct terms
  and \(η\) is a root of unity.
\end{proof}

\begin{lem}
  Suppose \(K ≠ ℚ\) is a number field of degree \(n\) over \(ℚ\) and let \(a ∈
  \algint\) be such that \eqref{eq:approximations of a} is satisfied.
  Then the following inequalities hold.
  \begin{thmlist}
    \item\label{lem:approx for delta i 1}
    \(|a_i|/2 < |δ_i| < |a_i| + 1\) for \(1 ≤ i ≤ s_K + 1\).

    \item\label{lem:approx for delta i 2}
     \(1/2 < | δ_i | < 1\) for \(s_K + 1 < i ≤ n\).

    \item\label{lem:modulus of elements in the kernel}
     If \(η ∈ \ker N_{L/K}\) then \(|σ_{ij}(η)| = 1\) for \(s_K + 1 < i ≤ n\)
     and \(j ∈ \set{1, 2}\). Furthermore, \(|η| = 1\) if and only if  \(η\) is a
     root of unity.

    \item \(|a| - \sqrt{|a^2 - 1|} < 1\).

    \item\label{lem:approximation of epsilon with a}
     \(|a| < |ε_1| < 2|a| + 1\).

    \item\label{lem:epsilon is not a root of unity}
    \(ε\) is not a root of unity.
  \end{thmlist}
\end{lem}
\begin{proof}
  \begin{plist}
    \item By assumption we have \(|a_i| > 2^{2(n + 1)}\) and therefore
      \begin{align*}
        \frac{|a_i|^2}{4} &= |a_i|^2 - \frac{3 |a_i| ^2}{4} ≤
            |a_i|^2 - \frac{3}{4} 2^{4(n + 1)} < |a_i|^2 - 1 \\
          &≤ |δ_i|^2 = |a_i^2 - 1| ≤ |a_i|^2 + 1 < (|a_i| + 1)^2
      \end{align*}

    \item Again by our assumption \(|a_i| < 1/2\). Thus, we find
    \[
      \frac{1}{4} < \frac{3}{4} < 1 - a_i^2 = |δ_i|^2 < 1.
    \]

    \item As in \cref{lem:properties of sigma epsilon} one uses that
    \(\overline{σ_{i1}(δ)} = σ_{i2}(δ)\) for all \(s_K + 1 < i ≤ n\) and finds
    for \(η = α + δ β ∈ \ker N_{L /K}\) that
    \begin{align*}
      σ_{i1}(η) &= σ_i(α) + σ_{i1}(δ) σ_i(β) \text{ and}\\
      σ_{i2}(η) &= σ_i(α) + σ_{i2}(δ) σ_i(β) = σ_i(α) - σ_{i1}(δ) σ_i(β)
    \end{align*}
    are complex conjugates. Now one can deduce,
    \[
      1 = N_{L/K}(η) = σ_{i1}(η) σ_{i2}(η) = |σ_{ij}(η)|^2
    \]
    for both \(j = 1\) and \(2\).

    To prove the second part of the claim, we notice that all roots of unity
    have complex modulus \(1\), so one direction is trivial. Let now \(η = α +
    δ β ∈ \ker N_{L / K}\) and additionally \(|η| = 1\), we differentiate two
    cases. If \(K\) is totally real, then all embeddings of the algebraic
    integer \(η\) have complex modulus \(1\). Therefore, \(η\) is a root of
    unity.

    If on the other hand, \(s_K = 1\), then note firstly, that the complex
    conjugate \(\overline{δ}\) is a root of the polynomial
    \[
      X^2 - {\overline{a}}^2 + 1 = X^2 - σ_2(a)^2 + 1.
    \]
    As a consequence, \(σ_{2j}(δ) = (-1)^{1 + j} \overline{δ}\) for \(j ∈
    \set{1, 2}\). We deduce that
    \[
      \overline{σ_{11}(η)} = \overline{η} = \overline{α + δ β} = σ_2(α) +
      σ_{21}(δ) σ_{2}(β) = σ_{21}(η)
    \]
    and
    \[
      \overline{σ_{12}(η)} = \overline{α - δ β} = σ_2(α) - σ_{21}(δ) σ_{2}(β)
      = σ_{21}(η) = σ_2(α) + σ_{22}(δ) σ_{2}(β) = σ_{22}(η).
    \]
    This implies that \(|σ_{21}(η)| = |η| = 1\) and
    \(|σ_{22}(η)| = |σ_{12}(η)|\). Finally, note that \(N_{L / ℚ} = N_{K / ℚ}
    \circ N_{L / K}\) and therefore
    \[
      1 = |N_{K / ℚ}(1)| = |N_{K / ℚ} \circ N_{L / K} (η)| =
          \left\vert
            \prod_{\substack{1 ≤ i ≤ n\\ 1 ≤ j ≤ 2}} σ_{ij}(η)
          \right\vert = |σ_{12}(η)| |σ_{22}(η)| = |σ_{12}(η)|^2.
    \]

    \item The inequality
    \[
      |a| - \sqrt{|a^2 - 1|} < 1
    \]
    is equivalent to \(|a|^2 - 2 |a| + 1 < |a^2 - 1|\). But this inequality can
    easily seen to be satisfied, as
    \[
      |a|^2 + 1 < |a|^2 + 2|a| - 1 ≤ |a^2 - 1| + 2 |a|
    \]
    and the claim is proven.

    \item Consider the inequality
    \[
      |ε_1|^2 = |a + δ_1|^2 = |a^2 + 2 a δ_1 + δ_1^2| ≥ |2a^2 + 2a δ_1| - 1 =
                2 |a| |ε_1| - 1
    \]
    which can be rewritten as
    \[
      0 ≤ |ε_1|^2 - 2 |a| |ε_1| + 1 =
          \left(|ε_1| - |a| - \sqrt{|a|^2 - 1}\right)
          \left(|ε_1| - |a| + \sqrt{|a|^2 - 1}\right)
    \]
    Thus, either both factors are non-positive real numbers or both are
    non-negative. In the first case
    \[
      0 < |ε_1| ≤ |a| - \sqrt{|a|^2 - 1} \overset{\text{by (iv)}}{<} 1,
    \]
    which is impossible. Hence, both factors are non-negative and
    \[
      |ε_1| > |a| + \sqrt{|a|^2 - 1}  > |a|,
    \]
    proving the first estimate. The second inequality follows from \(|ε_1| = |a
    + δ_1| < 2|a| + 1\) by (i).

    \item Note that by (v), \(|ε_1| > |a| > 2^{2(n + 1)} > 1\) and therefore the
    complex modulus of \(ε\) cannot be equal to \(1\). The claim follows from
    (iii).
  \end{plist}
\end{proof}

As a next step, we want to show that we have essentially found all solutions of
Pell's equation~\eqref{eq:Pell}. For this we need some lemmas.

Recall the group \(G ≤ \ker N_{L / K}\) defined in \cref{lem:subgroup of ker N
L/K}. We have seen in \cref{lem:rank of N_L/K U_L} and the subsequent
inequality~\eqref{eq:rank of G} that the free rank of \(G\) can be bound from
above by \(\rk U_L - \rk U_K\). I claim that this difference of ranks is equal
to \(1\) in both cases of algebraic number fields we are considering.

If \(K ≠ ℚ\) is totally real, then by Dirichlet's unit
theorem~(\ref{thm:Dirichlet}) we find that \(\rk U_K = n - 1\) and by
\cref{lem:r and s for tr and opnr} that \(\rk U_L = n\). If on the other hand,
\(K\) satisfies \(r = n - 2 > 0\) then \(\rk U_K = n - 2\) and by \cref{lem:r
and s for tr and opnr} we have \(\rk U_L = n - 1\).

Note that \(ε\) is contained in \(G\) and by the previous lemma, \(ε\) is not a
root of unity. As a consequence, the group \(⟨ε⟩ ≤ G\) has free rank at least
equal to \(1\). We deduce that
\[
  \rk ⟨ε⟩ = \rk G = 1.
\]
Thus, there exists a unit \(ε_0 ∈ G\) such that for all \(η ∈ G\)  there exists
a root of unity \(ζ ∈ \algint[L]\) and an integer \(k\), such that \(η = ζ
ε_0^k\). However, even more is true, as one can set \(ε = ε_0\), but before we
can prove this, we need a lemma.

\begin{lem}
  Let \(K ≠ ℚ\) be a number field and let \(a ∈ \algint\) satisfy
  \eqref{eq:approximations of a}. Furthermore, let \(ε_0\) be a generator of the
  torsion free part of \(G\). Then \(2δ \mid (ε_0 - ε_0^{-1})\) and
  \begin{thmlist}
    \item if \(K\) is totally real, then
    \[
      |N_{L/ℚ} (2 δ)| > a^2 \quad \text{ and } \quad
      |N_{L/ℚ} (ε_0 - ε_0^{-1})| < 2^{2n} |ε_0|^2;
    \]

    \item if \([K: ℚ] ≥ 3\) and \(K\) has exactly one pair of non-real embeddings, then
    \[
      |N_{L/ℚ} (2 δ)| > a^4 \quad \text{ and } \quad
      |N_{L/ℚ} (ε_0 - ε_0^{-1})| < 2^{2n} |ε_0|^4.
    \]
  \end{thmlist}
\end{lem}
\begin{proof}
  Let \(ε_0 = α + δβ\) for some \(α, β ∈ \algint\), then \(ε_0^{-1} = α - δ β\)
  (cf.~\cref{lem:subgroup of ker N L/K}) and
  \[
    ε_0 - ε_0^{-1} = 2δ\, β,
  \]
  proving that \(2δ \mid (ε_0 - ε_0^{-1})\).

  We assert without loss of generality that \(ε = ζε_0^k\), where \(|ε_0| ≥ 1\).
  Then for all \(1 ≤ i ≤ n\) and all \(1 ≤ j ≤ 2\), we have
  \[
    |σ_{ij}(ε)| = |σ_{ij}(ζε_0^k)| = |σ_{ij}(ε_0)|^k,
  \]
  and thus, \(|ε_1| = |ε_0|^k ≥ |ε_0| ≥ 1\). Furthermore, by \cref{lem:modulus
  of elements in the kernel} the following inequality holds
  \begin{equation}\label{eq:approximation of epsilon 0}
    |σ_{i1}(ε_0) - σ_{i1}(ε_0^{-1})| |σ_{i2}(ε_0) - σ_{i2}(ε_0^{-1})| ≤ 4
  \end{equation}
  for all \(s_K + 1 < i ≤ n\).

  \begin{plist}
    \item If \(K ≠ ℚ\) is totally real, then by \cref{lem:approx for delta i 1}
    and (ii) we find that
    \[
      |N_{L/ℚ}(2δ)| = 2^{2n} \prod_{i = 1}^n |δ_i|^2 >
        \frac{2^{2n}}{2^{2n - 2}} \frac{|a|^2}{4} = a^2.
    \]

    To see the second inequality, we use \eqref{eq:approximation of epsilon 0}
    to find
    \begin{align*}
      |N_{L/ℚ}(ε_0 - ε_0^{-1})| &=
        \prod_{\substack{1 ≤ i ≤ n\\1 ≤ j ≤ 2}}
          |σ_{ij}(ε_0) - σ_{ij}(ε_0^{-1})| \\
        & ≤ 2^{2n - 2} |ε_0 - ε_0^{-1}|^2 ≤ 2^{2n - 2} (|ε_0| + 1)^2
          < 2^{2n} |ε_0|^2
    \end{align*}
    as claimed.

    \item Completely analogously using the fact, that
    \begin{align*}
      |σ_{11}(ε_0) & - σ_{11}(ε_0^{-1})| |σ_{12}(ε_0) - σ_{12}(ε_0^{-1})|
      |σ_{21}(ε_0) - σ_{21}(ε_0^{-1})| |σ_{22}(ε_0) - σ_{22}(ε_0^{-1})| \\
      & = |σ_{11}(ε_0) - σ_{11}(ε_0^{-1})|^2 |
        \overline{σ_{21}(ε_0) - σ_{21}(ε_0^{-1})}|^2 =
      |σ_{11}(ε_0) - σ_{11}(ε_0^{-1})|^4.
    \end{align*}
  \end{plist}
\end{proof}

\begin{pro}\label{pro:epsilon essentially generaltes G}
  Let \(K ≠ ℚ\) be a number field and let \(a ∈ \algint\) satisfy
  \eqref{eq:approximations of a}. Then for every \(η ∈ G\) there exists an
  integer \(k\) and a root of unity \(ζ ∈ L\) such that \(η = ζ ε^k\).
\end{pro}
\begin{proof}
  By the discussion above all that is left to prove is that in the equation
  \[
    ε_1 = ζε_0^k
  \]
  with \(|ε_0| ≥ 1\) the integer \(k\) is \(1\). Then
  \[
    ε_0 = ζ^{-1}ε_1 = ζ^{-1}ε^{± 1}
  \]
  and the proposition is proven.

  Assume to the contrary, that \(k ≥ 2\), then \(|ε_1| ≥ |ε_0|^2 ≥ 1\).
  By the lemma above \(2δ \mid (ε_0 - ε_0^{-1})\) and therefore
  \[
    |N_{L / ℚ}(2δ)| ≤ |N_{L / ℚ}(ε_0 - ε_0^{-1})|.
  \]
  The previous lemma implies now that \(1 < |a|^{2m_0} < 2^{2n}|ε_0|^{2m_0}\),
  where \(m_0 = s_K + 1 ∈ \set{1, 2}\) is chosen accordingly. Applying
  \cref{lem:approximation of epsilon with a} yields
  \[
    |a|^{2 m_0} < 2^{2n} |ε_0|^{2 m_0} ≤ 2^{2n} |ε_1|^{m_0} <
            2^{2n + m_0} {(|a| + 1)}^{m_0} < 2^{2n + 1 + m_0} |a|^{m_0}.
  \]
  If \(m_0\) is one then the inequality above reads \(|a|^2 < 2^{2n + 2}|a|\),
  which is a contradiction to \eqref{eq:approximations of a}. If on the other
  hand \(m_0 = 2\) holds then we obtain \(|a|^4 < 2^{2n + 3} |a|^2\). But this
  can be transformed upon dividing by \(|a|^2\) and taking square roots to
  \(|a| < 2^{n + 3/2}\), which is again a contradiction to
  \eqref{eq:approximations of a}.
\end{proof}

Recall the sequences \((\px_m)_{m ∈ ℕ}\) and \((\py_m)_{m ∈ ℕ}\) defined in
\eqref{eq:def of x and y}. If \(K\) is totally real we can conclude from the
proposition above that all solutions of Pell's equation with the parameter \(d =
a^2 - 1\) are of the form
\[
  (±\px_m(a), ±\py_m(a))
\]
for some integer \(m ∈ ℕ\). This is because \(L = K[√d]\) has two real
embeddings (\(σ_{1,1}\) and \(σ_{1,2}\)) and the only real roots of unity are
\(±1\). If \(K\) has one pair of non-real embeddings and at least one real
embedding. This argument can no longer be used as \(L\) is totally complex in
this case. One can however impose a Diophantine condition on the solutions of
Pell's equation to force them to be of this shape
\cite[cf.][Lem.~3]{Pheidas1988}.

\begin{cor}\label{cor:forcing sequences}
  Let \(K\) be a number field of positive degree \(n\) over \(ℚ\) and let \(a ∈
  \algint\) satisfy \eqref{eq:approximations of a}. Then there exists a
  constant \(ν ∈ ℕ \setminus \set{0}\) with the property, that if \(x'+
  δ y' ∈ G\) and
  \begin{equation}\label{eq:forcing solutions to be sequences}
     {(x' + δ y')}^{ν} = x + δ y
  \end{equation}
  for two algebraic integers \(x, y ∈ \algint\) then there exists an index \(m
  ∈ ℕ\) such that
  \[
    (x, y) = (±\px_m, ±\py_m).
  \]
  In particular, \((x, y)\) is a solution to Pell's equation.
\end{cor}
\begin{proof}
  Set \(ν := |μ(K)|\) which is finite by Dirichlet's unit
  theorem~(\ref{thm:Dirichlet}). By the proposition there exists a root of
  unity \(ζ ∈ L\) and an integer \(m_1 ∈ ℕ\) such that \(ζε^{m_1} = x' + δ y'\).
  We conclude that
  \[
    x + δ y = (x' + δ y')^ν = (ζ ε^{m_1})^ν = ζ^ν ε^{m_1 ν} = ε^{m_1 ν}
  \]
  and therefore that \((x, y) = (±\px_{m_1 ν}, ±\py_{m_1 ν})\). The claim
  follows from \cref{lem:x and y solve Pells equation}.
\end{proof}

Note that \eqref{eq:forcing solutions to be sequences} is not a Diophantine
relation over \(\algint\) as \(δ\) is not contained in \(K\). We can however use
the binomial theorem to rewrite the identity as
\[
  x + δ y = {(x' + δ y')}^{ν} = \sum_{i = 0}^ν \binom{ν}{i} x'^{ν - i} δ^i y'^i
\]
and by equating coefficients of \(δ\) we get the two Diophantine definitions
\begin{align*}
  x &= \sum_{\substack{i = 1\\i \text{ even}}}^ν
          \binom{ν}{i} x'^{ν - i} (a^2 - 1)^{\frac{i}{2}} y'^i\\
  \intertext{and}
  y &= \sum_{\substack{i = 1\\i \text{ odd}}}^ν
          \binom{ν}{i} x'^{ν - i} (a^2 - 1)^{\frac{i - 1}{2}} y'^i\\
\end{align*}
As a next step we derive further properties of the sequences \((\px_m)_{m ∈ ℕ}\)
and \((\py_m)_{m ∈ ℕ}\).

\begin{lem}\label{lem:m smaller x m}
  Let \(K ≠ ℚ\) be a number field and let \(a ∈ \algint\) satisfy
  \eqref{eq:approximations of a}. Then the following inequality holds
  \[
    m < |σ_i(\px_m)|
  \]
  for all non-negative integers \(m ∈ ℕ\) and all \(1 ≤ i ≤ s_K + 1\).
\end{lem}
\begin{proof}
  By \cref{lem:approximation of epsilon with a} we know that \(|ε_1| > |a| > 1\)
  and since \(ε_1 = ε^{±1}\), \cref{lem:real part of epsilon} implies
  \[
    |x_m| = \frac{|ε_1^m + ε_1^{-m}|}{2}
    > \frac{|a|^{m} - 1}{2} > \frac{2^{2(n+1)m} - 1}{2} > m
  \]

  If \(s_K = 1\), then \(σ_2(\px_m)\) is the complex conjugate of \(σ_1(\px_m) =
  \px_m \). As a consequence, their moduli must coincide.
\end{proof}

\begin{lem}\label{lem:approximations of sigma x and sigma y}
  Let \(K ≠ ℚ\) be a number field and let \(a ∈ \algint\) satisfy
  \eqref{eq:approximations of a}. There exists a constant \(C > 0\) depending on
  \(K\) and \(a\) such that for all \(k ∈ ℕ \setminus \set{0}\) there exist \(m,
  h ∈ ℕ\) with \(k \mid m\) and \(k \mid h\), and
  \begin{align*}
    |σ_i(\px_m)| &> \frac{1}{2},\\
    |σ_i(\py_h)| &> C
  \end{align*}
  for \(s_K + 1 < i ≤ n\).
\end{lem}
\begin{proof}
  Fix any positive integer \(k\). By \cref{lem:modulus of sigma epsilon} we know
  that \(|ε_j| = 1\) for \(s_K + 1 < j ≤ n\). It follows
  that there exist arguments \(ϑ_{s_K + 2}, …, ϑ_n ∈ ℝ\) such that
  \[
    ε_j = e^{i π ϑ_j}.
  \]
  Let \(A = \set{ϑ_{j_1}, …, ϑ_{j_s}}\) be a maximal \(ℤ\)-linear independent
  subset of \(\set{ϑ_{s_K + 2}, …, ϑ_{n}}\). Since \(ε\) is not a root of unity
  by \cref{lem:epsilon is not a root of unity}, none of the \(ϑ_j\) can be
  rational. Indeed, if \(ϑ_j = p/q\) then \(ε_j^{2q} = e^{i π 2 p} = 1\) and
  \(ε_j\) is a root of unity. Hence, \(A\) contains at least one element.

  Let \(J_0 := \set{\seq[s]{j}}\) be the set of indices of elements in \(A\),
  then the construction implies that for all \(s_K + 1 < r ≤ n\) there exist
  integers \(b_r, b_{rj} ∈ ℤ\) with \(b_r ≠ 0\) such that
  \[
    b_r ϑ_r = \sum_{j ∈ J_0} b_{rj} ϑ_j.
  \]
  For otherwise, \(1, ϑ_{j_1}, …, ϑ_{j_s}, ϑ_{r}\) would be \(ℚ\)-linear
  independent, contradicting the maximality of \(A\). In other words, we have
  that
  \[
    ε_r^{b_r} = \prod_{j ∈ J_0} ε_j^{b_{rj}}.
  \]
  We set \(b := \prod_{r = s_K + 2}^n b_r ≠ 0\) and find for all \(s_K + 1 < r ≤
  n\) integers \(c_{rj} ∈ ℤ\) with
  \[
    ε_r^b = \prod_{j ∈ J_0} ε_j^{c_{rj}}.
  \]

  We exponent this expression by a multiple \(ℓ ∈ kℤ\) of \(k\), whose value
  will be fixed later, and rewrite it to obtain
  \begin{align}
    \begin{split}\label{eq:approximation of sigma x and sigma y}
        σ_r(\px_{ℓb}) + σ_{r1}(δ) σ_r(\py_{ℓb}) &= ε_r^{ℓb} =
            \prod_{j ∈ J_0} ε_j^{ℓ c_{rj}} =
            e^{iπ \sum_{j ∈ J_0} ℓ c_{rj} ϑ_j} =\\
          &= \cos\left(π \sum_{j ∈ J_0} ℓ c_{rj} ϑ_j\right) +
            i \sin\left(π \sum_{j ∈ J_0} ℓ c_{rj} ϑ_j\right).
    \end{split}
  \end{align}

  By continuity of \(| \cos(π ϑ) |\) in \(ϑ\), we can find
  a constant \(λ > 0\) such that \(1 - |\cos(π ϑ)| < 1/2\) whenever \(|ϑ| <
  λ\). Or put differently, \(|\cos(π ϑ)| > 1/2\).

  Let \(c_0 = \max_{r,j}(|c_{rj}|)\). Setting \(α_j := k ϑ_j\), \(β_j = 0\), and
  \(N = 1\) we obtain by Kronecker's
  theorem~(\ref{thm:Kronecker}) integers \(\tilde{ℓ}, \tilde{ℓ}_j\) with
  \(\tilde{ℓ} > 0\) such that
  \[
    |\tilde{ℓ} k ϑ_j - \tilde{ℓ}_j| < \frac{λ}{2 c_0 n}
  \]
  holds for all \(j ∈ J_0\) simultaneously. But then
  \[
    |2 \tilde{ℓ} k ϑ_j - 2 \tilde{ℓ}_j| < \frac{λ}{c_0n}
  \]
  holds as well and we set \(ℓ := 2 \tilde{ℓ} k\) and \(ℓ_j := 2 \tilde{ℓ}_j\).
  This does not only implies that
  \[
    \left| \sum_{j ∈ J_0} ℓ c_{rj} ϑ_j - \sum_{j ∈ J_0} ℓ_j c_{rj} \right| < λ
  \]
  but also by the choice of the \(ℓ_j\) that \(\sum_{j ∈ J_0} ℓ_j c_{rj} ∈ ℤ\)
  is divisible by \(2\) for all \(s_K + 1 < r ≤ n\). From
  \eqref{eq:approximation of sigma x and sigma y} we conclude that
  \[
    |σ_r(\px_{ℓb})| =
      \left| \cos\left(π \sum_{j ∈ J_0} ℓ c_{rj} ϑ_j\right) \right| =
      \left| \cos\left(π \sum_{j ∈ J_0} ℓ c_{rj} ϑ_j -
                       π \sum_{j ∈ J_0} ℓ_j c_{rj}\right) \right| > \frac{1}{2}.
  \]
  Setting \(m := ℓb\) proves the first claim as \(ℓ\) is divisible by \(k\).

  To prove the claimed bound for \(\py_h\) let \(C_r := \sum_{j ∈ J_0}
  c_{rj}\) for all \(s_K + 1 < r ≤ n\) and fix a constant \(C_0 ∈ ℕ\) such that
  \(C_0 > \max_r (|C_r|)\) and in the prime factorization of \(C_0\) appear at
  least as many twos as in all the prime factorizations of the \(C_r\). In other
  words,
  \[
    \ord_2 C_0 ≥ \ord_2 C_t, \quad \text{ for all } s_K + 1 < t ≤ n.
  \]

  As \(|\sin(π ϑ)|\) is uniformly continuous on the compact interval \([-1, 1]\)
  we can find for all positive \(λ_1 > 0\) a real number \(0 < λ_2 < 1/4\) such
  that \(||\sin(π ϑ)| - |\sin(π φ)|| < λ_1\) whenever \(|ϑ - φ| < λ_2\) and
  \(ϑ, φ ∈ [-1, 1]\) are satisfied. We apply Kronecker's theorem again with the
  parameters \(α_j := k ϑ_j\), \(β_j := 1/(4 C_0)\), and \(N := 1\) to obtain
  integers \(\tilde{ℓ}, \tilde{ℓ}_j\) with \(\tilde{ℓ} > 0\) and the property
  that
  \[
    \left\vert \tilde{ℓ} k ϑ_j - \tilde{ℓ}_j - \frac{1}{4 C_0}\right\vert
     < \frac{λ_2}{2 C_0 n}
  \]
  holds for all \(j ∈ J_0\) simultaneously. We again multiply by \(2\) to obtain
  \[
  \left\vert 2 \tilde{ℓ} k ϑ_j - 2 \tilde{ℓ}_j - \frac{1}{2 C_0}\right\vert
    < \frac{λ_2}{C_0 n}
  \]
  and set \(ℓ := 2 \tilde{ℓ} k\) and \(ℓ_j := 2 \tilde{ℓ}_j\) for all \(s_K + 1
  < r ≤ n\). Hence, we can deduce that
  \[
    \left|
      \sum_{j ∈ J_0} \left( ℓ c_{rj} ϑ_j - ℓ_j c_{rj}\right) - \frac{C_r}{2 C_0}
    \right| =
    \left|
      \sum_{j ∈ J_0}
        \left( ℓ c_{rj} ϑ_j - ℓ_j c_{rj} - \frac{c_{rj}}{2 C_0}\right)
    \right| < λ_2
  \]
  and again that \(\sum_{j ∈ J_0}ℓ_j c_{rj}\) is divisible by \(2\) for all
  \(s_K + 1 < r ≤ n\). Set now
  \[
    λ_1 :=  \frac{
      \left\vert\sin\left(
        \frac{π C_r}{2 C_0}
      \right) \right\vert}
      {2} 
  \]
  then we can use \eqref{eq:approximation of sigma x and sigma y} to obtain
  \begin{align*}
    \left\vert
      |σ_{r1}(δ) σ_r (y_{ℓb})| -
        \left| \sin\left( π \frac{C_r}{2 C_0} \right)\right|
    \right\vert &=
      \left|\left|
          \sin \left(π \sum_{j ∈ J_0} ℓ c_{rj} ϑ_j \right)
        \right| -
        \left\vert \sin\left( π \frac{C_r}{2 C_0} \right)\right\vert
       \right\vert \\
      &= \left|\left| \sin \left(π \sum_{j ∈ J_0} ℓ c_{rj} ϑ_j  -
              π \sum_{j ∈ J_0}ℓ_j c_{rj}\right)
          \right| -
          \left\vert \sin\left( π \frac{C_r}{2 C_0} \right)\right\vert
        \right| \\
      &< λ_1 = \frac{1}{2} \left| \sin \left( π \frac{C_r}{2 C_0}\right)\right|
  \end{align*}
  for all \(s_K + 1 < r ≤ n\). Thus, we can conclude that \(|σ_{r1}(δ) σ_r
  (y_{ℓb})| > λ_1\).

  Note that \(\sin(π C_r / (2 C_0)) ≠ 0\) as \(C_r / (2 C_0)\) cannot be an
  integer by the choice of \(C_0\). Now set
  \[
    C :=
      \min_{r}
        \frac{\left|
          \sin \left(π \frac{C_r}{2 C_0}\right)
        \right|}{2 |σ_{r1}(δ)|}
  \]
  then \(C\) satisfies the claim.
\end{proof}

\begin{lem}\label{lem:defining e}
  Let \(K\) be a number field of degree \(n > 0\) over \(ℚ\) and let \(a ∈
  \algint\) satisfy \eqref{eq:approximations of a}. If \(\py_{eh}\) satisfies
  \cref{lem:approximations of sigma x and sigma y} for an arbitrary but fixed
  positive integer \(k ∈ ℕ \setminus \set{0}\) dividing \(h\), where \(e ∈ ℕ\)
  is an integer such that
  \begin{equation}\label{eq:def of e}
    |ε_1|^e > \frac{2^{s_K + 1} |δ_1|^{s_K + 1}}{C^{n - s_K - 1}}
  \end{equation}
  holds for the constant \(C\) of the same lemma, then
  \begin{thmlist}
    \item \(\py_{eh} \mid \py_{eℓ}\) in \(\algint\) implies
    \(h \mid ℓ\) in \(ℤ\) and

    \item \(\py_{eh}^2 \mid \py_{eℓ}\) in \(\algint\) implies
    \(h \py_{eh} \mid ℓ\) in \(\algint\).
  \end{thmlist}
\end{lem}
\begin{proof}
  Note that such an integer \(e\) must exist as \(|ε_1| > 1\) by
  \cref{lem:approximation of epsilon with a}. Let now \(r ∈ ℕ\) be such that
  \(0 < r < h\). As before let \(L := K[δ]\). Considering the field norm \(N_{L
  / ℚ}\) of \(\py_{eq}\) we use \cref{lem:real part of epsilon} to obtain
  \begin{align*}
    | N_{L / ℚ} (\py_{er}) | &=
        \prod_{i = 1}^n
          \prod_{j = 1}^2
            \left| σ_{ij} \left( \frac{ε^{er} - ε^{-er}}{2δ}\right) \right| \\
      &≤ \prod_{i = 1}^{s_K + 1}
        \frac{|ε_i^{er} - ε_i^{-er}| |ε_i^{-er} - ε_i^{er}|}{4 |δ_i|^2}
        \prod_{i = s_K + 2}^n \frac{1}{|δ_i|^2}\\
      &≤ \frac{4^{s_K + 1} |ε_1|^{2er(s_K + 1)}}
              {4^{s_K + 1} |N_{K/ℚ}(a^2 - 1)|} < |ε_1|^{2er (s_K + 1)},
  \end{align*}
  where the approximations follow completely analogously as in the proof of
  \cref{pro:epsilon essentially generaltes G}. As \(\py_{er}\) is in \(\algint\)
  we deduce that
  \[
    |N_{K / ℚ} (\py_{er})| < |ε_1|^{er (s_K + 1)}.
  \]

  On the other hand, by our assumption on \(|σ_i(\py_{eh})|\) we know that
  \begin{align*}
    |N_{K / ℚ} (\py_{eh})| &= \prod_{i = 1}^n | σ_i(\py_{eh}) | ≥
        C^{n - s_K - 1} \prod_{i = 1}^{s_K + 1} |σ_i(\py_{eh}) | \\
      &= C^{n - s_K - 1} \prod_{i = 1}^{s_K + 1} \frac{|ε_i^{eh} -
           ε_i^{-eh}|}{2 |δ_i|} ≥
        C^{n - s_K - 1}
        \left( \frac{|ε_1|^{eh} - 1}{2 |δ_1|} \right)^{s_K + 1} \\
      &> C^{n - s_K - 1}
        \frac{|ε_1|^{eh (s_K + 1)}}{2^{s_K + 1} |δ_1|^{s_K + 1}},
  \end{align*}
  where the second inequality follows from \(|ε_1^{-1}| ≤ 1\). Using our
  assumption on \(e\), it follows that
  \begin{equation}\label{eq:norm of x eh and x er}
    |N_{K / ℚ} (\py_{eh})| > | N_{K / ℚ}(\py_{er}) |.
  \end{equation}

  Let \(\py_{eh} \mid \py_{eℓ}\) and  set \(ℓ = t h + r\) for \(t, r ∈ ℕ\) with
  \(0 ≤ r < h\). Assume to reach a contradiction that \(r > 0\), then
  \[
    \py_{eℓ} = \py_{eth + er} \overset{\text{\cref{lem:addition formulas}}}{=}
    \py_{eth}\px_{er} + \px_{eth} \py_{er}.
  \]
  By \cref{lem:y m divides y mk} we know that \(\py_{eh} \mid \py_{eth}\) and
  consequently \(\py_{eh} \mid \px_{eth} \py_{er}\) in \(\algint\). But by
  part~(iii) of the same lemma the principal ideals \((\px_{eth})\)
  and \((\py_{eth})\) are relative prime in \(\algint\). Now \cref{lem:y m
  divides y mk} implies that \((\px_{eth})\) and \((\py_{eh})\) are relative
  prime as well. And therefore \(y_{eh} \mid y_{er}\), contradicting
  \eqref{eq:norm of x eh and x er}. Consequently, \(r = 0\) and \(h \mid ℓ\) in
  \(ℤ\).

  To see the second divisibility condition we assume \(\py_{eh}^2 \mid
  \py_{eℓ}\) in \(\algint\). Then by the first part of the lemma, we know that
  there exists an integer \(t\) such that \(ℓ = th\). Using the binomial formula
  we obtain
  \begin{align*}
    \px_{eth} + δ\py_{eth} &= (ε^{eh})^t = (\px_{eh} + δ \py_{eh})^t =\\
      &= \sum_{\substack{0 ≤ i ≤ t\\2 \text{ even}}}
        \binom{t}{i} \px_{eh}^{t - i} δ^i \py_{eh}^i +
      δ \sum_{\substack{0 ≤ i ≤ t\\2 \text{ odd}}}
        \binom{t}{i} \px_{eh}^{t - i} δ^{i - 1} \py_{eh}^i
  \end{align*}
  and, since \((\px_{eh})\)  and \((\py_{eh})\) are relative prime, we can
  conclude that
  \[
    0 \equiv \py_{eth} \equiv t \px_{eh}^{t - 1} \py_{eh} \equiv t \py_{eh}
    \mod (\py_{eh}^2).
  \]
  It follows that \(\py_{eh} \mid t\) and therefore \(h \py_{eh}
  \mid ℓ\) (both in \(\algint\)).
\end{proof}

As a next step we prove a similar result for \(\px_m\).

\begin{lem}\label{lem:congruences for x m}
  Let \(K\) be a number field of degree \(n > 0\) over \(ℚ\) and let \(a ∈
  \algint\) satisfy \eqref{eq:approximations of a}. If \(\px_{m}\) satisfies
  \[
    |σ_i(\px_m)| > ½
  \]
  for \(s_K + 1 < i ≤ n\) then for all integers \(ℓ, j  ∈ ℤ\) and some sign \(ς
  ∈ \set{-1, 1}\) we have that \(\px_ℓ \equiv \px_j ς \mod (\px_m)\) in
  \(\algint\) implies \(ℓ \equiv ± j \mod m\) in \(ℤ\).
\end{lem}
\begin{proof}
  Set \(ℓ = 2 m ℓ_1 + r_1\) as well as \(j = 2 m j_1 + r_2\) with \(-m < r_1,
  r_2 ≤ m\). Without loss of generality we may assume that \(0 ≤ r_1, r_2 ≤ m\)
  holds, since we have
  \[
    \px_ℓ = \px_{2 m ℓ_1 ± r_1} \equiv -\px_{r_1} \mod (\px_m)
    \quad \text{and} \quad
    \px_j = \px_{2 m j_1 ± r_2} \equiv -\px_{r_2} \mod (\px_m)
  \]
  by \cref{lem:congruence x_2m+k}. Thus, we can deduce that \(\px_{r_1} \equiv
  \px_{r_2} ς \mod (\px_m)\). We will prove that \(\px_{r_1} = \px_{r_2} ς\) and
  will deduce \(r_1 = r_2 ς\). Assume otherwise that \(r_1 ≠ r_2\) and
  \(\px_{r_1} ≠ \px_{r_2} ς\) then from the congruence it follows that
  \begin{equation}\label{eq:approx of x r1  x r2 and x m}
    |N_{K/ℚ} (\px_m)| ≤ |N_{K/ℚ} (\px_{r_1} - \px_{r_2} ς)|.
  \end{equation}

  To reach a contradiction we assume without loss of generality that \(r_1 ≤
  r_2\) and apply \cref{lem:real part of epsilon} to express \(\px_m,
  \px_{r_1}\) and \(\px_{r_2}\) as the ‘real part’ of a power of \(ε\). Now
  as in the proof of \cref{lem:defining e}, we have that
  \begin{equation}\label{eq:norm of p x}
    \begin{split}
    |N_{K/ℚ}(\px_m)| &=
      \prod_{i = 1}^{s_K + 1} |σ_i(\px_m)| \prod_{i = s_K + 2}^n |σ_i(\px_m)|≥
        \frac{1}{2^{n - s_K - 1}} |σ_1(\px_m)|^{s_K + 1} \\
      &= \frac{1}{2^{n - s_K - 1}}
            \left| \frac{ε_1^m + ε_1^{-m}}{2} \right|^{s_K + 1} ≥
          \frac{(|ε_1|^m - 1)^{s_K + 1}}{2^n}
    \end{split}
  \end{equation}
  Again the last inequality follows from the triangular inequality and the fact
  that \(|ε_1^{-1}| ≤ 1\). To estimate the norm of \(\px_{r_1} - \px_{r_2} ς\)
  we note that by \cref{lem:properties of sigma epsilon} we know for all \(s_K +
  1 < i ≤ n\) that
  \[
    |σ_i(\px_{r_1}) - σ_i(\px_{r_2} ς)| ≤
    \frac{|ε_i^{r_1} - ε_i^{-r_1}| + |ς| |ε_i^{r_2} - ε_i^{-r_2}|}{2} ≤
    \frac{|ε_i|^{r_1} + |ε_i|^{-r_1} + |ε_i|^{r_2} + |ε_i|^{-r_2}}{2} = 2.
  \]
  Thus, we can conclude that
  \begin{equation}
    \begin{split}\label{eq:norm of x r1 plus x r2}
    |N_{K/ℚ}(\px_{r_1} - \px_{r_2} ς)| &=
      \prod_{i = 1}^n |σ_i(\px_{r_1}) - σ_i(\px_{r_2} ς)| ≤
      2^{n - s_K - 1} |σ_i(\px_{r_1}) - σ_i(\px_{r_2} ς)|^{s_K + 1} \\
    &≤   2^{n - s_K - 1} \left(
      \frac{|ε_1|^{r_1} + |ε_1|^{-r_1} + |ε_1|^{r_2} + |ε_1|^{-r_2}}{2}
    \right)^{s_K + 1} \\
    &≤ 2^{n - s_K - 1}(|ε_1|^{r_2} + 1)^{s_K + 1}.
    \end{split}
  \end{equation}

  We know by \cref{lem:approximation of epsilon with a} that \(|ε_1| > 2\). So
  \(|ε_1|^{r_2} + 1 < 2 |ε_1|^{r_2}\) and which implies that
  \[
    |ε_1|^m - 1 > ½ |ε_1|^m.
  \]
  Now notice that \(s_K ≤ 1\) and thus we can deduce from \eqref{eq:norm of p
  x} and \eqref{eq:norm of x r1 plus x r2} that
  \begin{align}\label{eq:norm of x m and x r1 plus x r2 1}
    |N_{K/ℚ}(\px_m)| &≥ \frac{|ε_1|^{m (s_K + 1)}}{2^{n + s_K + 1}} ≥
    \frac{|ε_1|^{m (s_K + 1)}}{2^{n + 2}}\\
    \intertext{and}\label{eq:norm of x m and x r1 plus x r2 2}
    |N_{K/ℚ}(\px_{r_1} - \px_{r_2} ς)| &≤ 2^{n} |ε_1|^{r_2 (s_K + 1)}.
  \end{align}

  We will now distinguish two cases. Either \(r_2 < m\) then we have that
  \[
    2^{2n + 2} < |ε_1|^{s_K + 1} ≤
    |ε_1|^{(m - r_2)(s_K + 1)},
  \]
  which together with the inequalities in \eqref{eq:norm of x m and x r1 plus x
  r2 1} and \eqref{eq:norm of x m and x r1 plus x r2 2} implies
  \[
    |N_{K/ℚ} (\px_m)| > |N_{K/ℚ} (\px_{r_1} - \px_{r_2} ς)|.
  \]
  Thus, we have reached a contradiction with \eqref{eq:approx of x r1  x r2 and
  x m} and we must assume that \(r_2 = m\) holds. But this implies that
  \(\px_m\) actually divides \(\px_{r_1}\), yielding
  \[
    |N_{K/ℚ} (\px_m)| ≤ |N_{K/ℚ} (\px_{r_1})|.
  \]
  As for the norm of \(\px_{r_1}\) we find that
  \begin{align*}
    |N_{K/ℚ} (\px_{r_1})| &=
         |σ_1(\px_{r_1})|^{s_K + 1} \prod_{i = s_K + 2}^n |σ_i(\px_{r_1})| \\
      &≤ |σ_1(\px_{r_1})|^{s_K + 1} \prod_{i = s_K + 2}^n
                \frac{|ε_i|^{r_1} + |ε_i|^{-r_1}}{2} \\
      &≤ \left(\frac{|ε_1|^{r_1} + |ε_1|^{-r_1}}{2}\right)^{s_K + 1} ≤\\
      &≤ \left(\frac{|ε_1|^{r_1} + 1}{2}\right)^{s_K + 1} ≤
         |ε_1|^{r_1 (s_K + 1)}
  \end{align*}
  holds. Remember that we are assuming that \(r_1 ≠ r_2\) and we have already
  shown \(r_2 = m\). Thus, we can conclude that \(2^{n + 2} < |ε_1|^{m - r_1}\).
  But then the inequality above implies together with \eqref{eq:norm of x m and
  x r1 plus x r2 1} that
  \[
    |N_{K/ℚ} (\px_m)| > |N_{K/ℚ} (\px_{r_1})|,
  \]
  which is again an contradiction.

  As a consequence, our assumption that both \(r_1 ≠ r_2\) and \(\px_{r_1} ≠
  \px_{r_2} ς\) are true cannot hold. So the only thing left to check is that
  \(\px_{r_1} = \px_{r_2} ς\) implies \(r_1 = ± r_2\), which yields the desired
  congruence of rational integers. We know that \(\px_{r_1} = \px_{r_2} ς\)
  implies \(ε^{r_1} + ε^{-r_1} = (ε^{r_2} + ε^{-r_2}) ς\). If \(ς = 1\) we find
  that this is equivalent to
  \[
    0 = ε^{r_1} (ε^{r_1} + ε^{-r_1} - ε^{r_2} - ε^{-r_2}) =
      ε^{2 r_1} + 1 - ε^{r_1 + r_2} - ε^{r_1 - r_2} =
      (ε^{r_1 + r_2} - 1) (ε^{r_1 - r_2} - 1).
  \]
  Hence, \(r_1 = ± r_2\) as claimed. If on the other hand, \(ς = -1\) holds we
  find
  \[
  0 = -ε^{r_2} (ε^{r_1} + ε^{-r_1} - ε^{r_2} - ε^{-r_2}) =
    -ε^{r_2 + r_1} -ε^{r_2 - r_1} + ε^{2 r_2} + 1 =
    (ε^{r_2 + r_1} - 1) (ε^{r_2 - r_1} - 1),
  \]
  yielding \(r_1 = ± r_2\).
\end{proof}

\begin{lem}\label{lem:def of b}
  Let \(K\) be a number field of degree \(n\) over \(ℚ\) and let \(a ∈ \algint\)
  satisfy \eqref{eq:approximations of a}. Then for all positive \(m ∈ ℕ
  \setminus \set{0}\) and all constants \(C_1, C_2 > 0\) there exists an
  algebraic integer \(b\) such that
  \begin{thmlist}
    \item \(b \equiv 1 \mod (\py_m(a))\),

    \item \(b \equiv a \mod (\px_m(a))\), and

    \item \(b\) satisfies \(|σ_i(b)| > C_1\) for all \(1 ≤ i ≤ s_K + 1\) and
    \(|σ_i(b)| < C_2\) for all \(s_K + 1 < i ≤ n\). In particular, we may whish
    that \(b\) satisfies the approximations of the embeddings of \(a\) in
    \eqref{eq:approximations of a}.
  \end{thmlist}
\end{lem}
\begin{proof}
  Set
  \[
    b :=
    \left(
      \px_m(a)^2 + \py_m(a)^2 (a^2 - 1)
    \right)^{2s}
    \left(
      \px_m(a)^4 + a \left(1 - \px_m(a)^2\right)^2
    \right)
  \]
  for some positive integer \(s\), whose value will be determined later. Note
  that since \(\px_m(a)\) and \(\py_m(a)\) solve Pell's equation, we have
  \[
    \px_m(a)^2 - (a^2 - 1)\py_m(a)^2 = 1,
  \]
  which implies that \((a^2 - 1) \py_m(a)^2 \equiv -1 \mod (\px_m(a))\). Thus,
  condition~(ii) is satisfied. As for condition~(i), we note that by the same
  argument \(\px_m(a)^2 \equiv 1 \mod (\py_m(a))\) holds.

  To prove the last claim note that for all positive \(m\), we have
  \[
    |σ_i(\px_m(a))| = \left| \frac{ε_i(a)^m + ε_i(a)^{-m}}{2} \right| ≤
    \frac{|ε_i(a)|^m + |ε_i(a)|^{-m}}{2} = 1
  \]
  for all \(s_K + 1 < i ≤ n\). The only way equality can be reached in the
  inequality above is if \(|ε_i(a)^m + ε_i(a)^{-m}| = |ε_i(a)|^m +
  |ε_i(a)|^{-m}\). But this can only happen, if their arguments coincide, which
  implies that both \(ε_i(a)^{m}\) and \(ε_i(a)^{-m}\) are real numbers with
  complex modulus \(1\). Hence, \(ε_i(a)^{2m} = 1\) and \(ε(a)\) would be a
  \(2m\)-th root of unity, contradicting \cref{lem:epsilon is not a root of
  unity}. We conclude that the approximation of the modulus \(|σ_i(\px_m(a))| <
  1\) holds for all \(s_K + 1 < i ≤ n\). But since
  \[
    |N_{K/ℚ}(\px_m(a))| =
      |σ_1(\px_m(a))|^{s_K + 1} \prod_{i = s_K + 2}^n |σ_i(\px_m(a))|
  \]
  is a positive integer we must also have \(|σ_1(\px_m(a))| > 1\).

  As for the embeddings of \(b\) we have
  \[
    σ_i(b) =
    \left(
      σ_i(\px_m(a))^2 + σ_i(\py_m(a))^2 (a_i^2 - 1)
    \right)^{2s}
    \left(
      σ_i(\px_m(a))^4 + a_i \left(1 - σ_i(\px_m(a))^2\right)^2
    \right)
  \]
  where \(|a_i| > 1\) for all \(1 ≤ i ≤ s_K + 1\) and \(|a_i| < 1\) for all
  \(s_K + 1 < i ≤ n\) by \eqref{eq:approximations of a}. Let us first consider
  the case for \(1 ≤ i ≤ s_K + 1\). Then
  \[
  \left|
    σ_i(\px_m(a))^2 + σ_i(\py_m(a))^2 (a_i^2 - 1)
  \right| = |2 σ_i(\px_m(a))^2 - 1| ≥ 2 |σ_i(\px_m(a))|^2 - 1 > 1
  \]
  holds and thus \(|σ_i(b)|\) is strictly increasing in \(s\). On the other
  hand, if \(s_K + 1 < i ≤ n\) we note that \(σ_i(\px_m)\) is a real number as
  \(σ_i : K → ℝ\) is a real embedding and conclude that both
  \begin{align*}
    σ_i(\px_m(a))^2 - σ_i(\py_m(a))^2 (1 - a_i^2) &< σ_i(\px_m(a))^2 < 1 \\
    \intertext{as well as}
    σ_i(\px_m(a))^2 - σ_i(\py_m(a))^2 (1 - a_i^2) &= 2 σ_i(\px_m(a))^2 - 1 > -1
  \end{align*}
  hold. As a consequence, the modulus \(|σ_i(b)|\) is strictly decreasing in
  \(s\). Hence, we can arrange for \(|σ_i(b)| > C_1\) and \(|σ_j(b)| < C_2\) to
  hold (\(1 ≤ i ≤ s_K + 1 < j ≤ n\)), as claimed.
\end{proof}

Finally, we have all the tools at hand to present a Diophantine representation
of \(ℤ\) over \(\algint\).

\begin{thm}
  Let \(K\) be a number field of degree \(n > 0\) over \(ℚ\) and let \(a ∈
  \algint\) satisfy \eqref{eq:approximations of a}. Let \(\seq{σ}\) be all
  embeddings of \(K\) into \(ℂ\), where we demand that \(σ_i\) is a non-real
  embedding if and only if \(i ≤ 2 s_K\). Furthermore, let \(C\) be the bound
  defined in \cref{lem:approximations of sigma x and sigma y}, \(e\) defined as
  in \eqref{eq:def of e}, and \(ν := |μ(K)|\). Then the set \(S\) defined by the
  following relations is Diophantine over \(\algint\) and satisfies
  \(νℕ ⊂ S ⊂ ℤ\).
  \begin{align}
    ξ ∈ S \quad &⇔ \quad
      ∃ x, y, w, z, u, v, s, t, x', y', w', z', u', v', s', t', b ∈ \algint:
      \nonumber \\
      &\begin{cases}
      \label{eq:sequences}
        x'^2 - (a^2 - 1) y'^2 = 1 \\
        w'^2 - (a^2 - 1) z'^2 = 1 \\
        u'^2 - (a^2 - 1) v'^2 = 1 \\
        s'^2 - (b^2 - 1) t'^2 = 1 \\
      \end{cases}\\
      \label{eq:force sequences}
      &\begin{cases}
        x + δ y = {(x' + δ(a) y')}^ν \\
        u + δ v = {(u' + δ(a) v')}^ν \\
        s + δ y = {(s' + δ(b) t')}^ν
      \end{cases} \\
      \label{eq:force sequences for w and z}
      & w + δ z = {(w' + δ(a) z')}^{νe} \\
      \label{eq:approx of embeddings 1}
      & 0 < σ_i(b) < 2^{-18} \quad \text{for all } s_K + 1 < i ≤ n \\
      \label{eq:approx of embeddings 2}
      & |σ_i(z)| ≥ C, \quad |σ_i(u)| ≥ ½ \quad \text{for all } s_K + 1 < i ≤ n\\
      \label{eq:v is non zero}
      & v ≠ 0 \\
      \label{eq:z 2 divides v}
      & z^2 \mid v \\
      \label{eq:cong 1}
      & b \equiv 1 \mod (z), \quad b \equiv a \mod (u) \\
      \label{eq:cong 2}
      & s \equiv x \mod (u) \\
      \label{eq:cong 3}
      & t \equiv ξ \mod (z) \\
      \label{eq:divisibilty of z}
      & 2^{n + 1} \prod_{i = 0}^{n - 1} (ξ + i)^n (x + i)^n \mid z
  \end{align}
\end{thm}
\begin{proof}
  Note that the set \(S\) defined by the relations above is indeed Diophantine
  since
  \begin{itemize}
    \item \eqref{eq:force sequences} as well as \eqref{eq:force sequences for w
    and z} can be rewritten in a Diophantine form (over \(\algint\)) as was
    demonstrated below the proof of \cref{cor:forcing sequences};

    \item \eqref{eq:approx of embeddings 1} as well as \eqref{eq:approx of
    embeddings 2} can be rewritten in a Diophantine form by
    \cref{lem:approximations of embeddings are Diophantine}; and

    \item \eqref{eq:v is non zero} is Diophantine by \cref{ex:being non zero is
    Diophantine}.
  \end{itemize}
  Finally, the conjunction of all of these Diophantine relations is Diophantine
  by \cref{lem:intersections and unions}.

  First suppose that the relations above have a common solution \(ξ\). We need
  to show that \(ξ\) is a rational integer. To see this I first claim that \(b\)
  satisfies \eqref{eq:approximations of a}. Indeed, the part for the embeddings
  \(σ_i(b)\) with \(s_K + 1 < i ≤ n\) are guaranteed by \eqref{eq:approx of
  embeddings 1}. For the embeddings \(σ_i(b)\) with \(1 ≤ i ≤ s_K + 1\) we note
  that since \(b\) is an algebraic integer, its norm \(N_{K/ℚ}(b)\) must have an
  absolute value of at least one. Thus, we have
  \[
    1 ≤ |N_{K/ℚ}(b)| = |σ_1(b)|^{s_K + 1} \prod_{i = s_K + 2}^n |σ_i(b)| ≤
      |σ_1(b)|^{s_K + 1} 2^{-18 (n - s_K - 1)},
  \]
  which implies
  \[
    |σ_1(b)| ≥ 2^{18 \frac{n - s_K - 1}{s_K + 1}}.
  \]
  Now if \(s_K = 0\), then \(n ≥ 2\) and since \(18 (n - 1) > 2 (n + 1)\) the
  claim holds. If on the other hand \(s_K = 1\), then we have demanded that \(n
  ≥ 3\) holds and again since \(9 (n - 2) > 2(n + 1)\) the claim holds true.

  Now \eqref{eq:sequences}, \eqref{eq:force sequences}, and \eqref{eq:force
  sequences for w and z} imply by \cref{cor:forcing sequences} that there exist
  integers \(k, h, m, j ∈ N\) such that
  \begin{align*}
    x &= ±\px_k(a),    & y &= ±\py_k(a),   \\
    w &= ±\px_{eh}(a), & z &= ±\py_{eh}(a),\\
    u &= ±\px_m(a),    & v &= ±\py_m(a), \\
    s &= ±\px_j(b),\; \text{and}    & t &= ±\py_j(b).
  \end{align*}
  We can thus rewrite conditions~\eqref{eq:approx of embeddings 2} to
  \eqref{eq:cong 3} to obtain
  \begin{align}
    \label{eq:approx of embeddings 2 v2}
    & |σ_i(\py_{eh}(a))| ≥ C, \quad
      |σ_i(\px_m(a))| ≥ ½ \quad \text{for all } s_K + 1 < i ≤ n,\\
    \label{eq:v is non zero v2}
    & \py_m(a) ≠ 0, \\
    \label{eq:z 2 divides v v2}
    & \py_{eh}^2 \mid \py_m(a), \\
    \label{eq:cong 1 v2}
    & b \equiv 1 \mod (\py_{eh}(a)), \quad b \equiv a \mod (\px_m(a)), \\
    \label{eq:cong 2 v2}
    & \px_j(b) \equiv ±\px_k(a) \mod (\px_m(a)), \text{ and} \\
    \label{eq:cong 3 v2}
    & ±\py_j(b) \equiv ξ \mod (\py_{eh}(a)).
  \end{align}

  Now from \cref{lem:m congruent y m} we can conclude that
  \[
    \py_j(b) \equiv j \mod (b - 1).
  \]
  By \eqref{eq:cong 1 v2} this implies that
  \[
    \py_j(b) \equiv j \mod (\py_{eh}(a))
  \]
  holds. Now from condition~\eqref{eq:cong 3 v2} we can deduce that the
  congruence
  \begin{equation}\label{eq:j congruent xsi mod y eh}
    j \equiv ± ξ \mod (\py_{eh}(a))
  \end{equation}
  must be satisfied. Furthermore, from \eqref{eq:cong 1 v2} and \cref{lem:a
  congruent b mod c} we can infer that
  \[
    \px_j (b) \equiv \px_j (a) \mod \px_m (a)
  \]
  holds, implying together with \eqref{eq:cong 2 v2}
  \[
    \px_j(a) \equiv ±\px_k(a) \mod (\px_m(a)).
  \]
  Now we can use \cref{lem:congruences for x m}, whose assumption on \(\px_m\)
  is satisfied by \eqref{eq:approx of embeddings 2 v2}, to deduce that
  \begin{equation}\label{eq:k congruent j mod m}
    k \equiv ± j \mod m.
  \end{equation}
  Again \eqref{eq:approx of embeddings 2 v2} allows us to apply
  \cref{lem:defining e} so that we can infer
  \[
    \py_{eh}(a) \mid m \; \text{ in } \algint
  \]
  from \cref{eq:z 2 divides v v2}. We use this relation and find from
  \eqref{eq:k congruent j mod m} that
  \[
    k \equiv ± j \mod (\py_{eh}(a))
  \]
  must hold. We can now infer from \eqref{eq:j congruent xsi mod y eh} that
  \[
    k \equiv ± ξ \mod (\py_{eh}(a))
  \]
  holds. From \eqref{eq:divisibilty of z} it follows by \cref{lem:8} that
  \[
    |σ_i(ξ)| < ½ |N_{K/ℚ}(\py_{eh}(a))|^{\frac{1}{n}}
  \]
  for all \(1 ≤ i ≤ n\). Analogously, one can deduce from the same condition
  that
  \[
    |σ_i(k)|  = k ≤ |σ_i(\px_k(a))| < ½ |N_{K/ℚ}(\py_{eh}(a))|^{\frac{1}{n}}
  \]
  holds, where the first inequality follows from \cref{lem:m smaller x m}. Thus,
  all the conditions of the strong vertical method~(\cref{thm:strong vertical
  method}) are satisfied for \(k\) and \(ξ\) and we find
  \[
    ξ = ± k ∈ ℤ
  \]
  as claimed.

  To show the other direction let \(ξ = ℓν ∈ νℕ\) be given. We set \(k := ξ, x
  :=\px_k(a), x' := \px_ℓ(a), y := \px_k(a)\) and \(y' := \px_ℓ(a)\), then the
  parts of \eqref{eq:sequences} and \eqref{eq:force sequences} involving \(x,
  x', y\) and \(y'\) can be satisfied. By \cref{lem:forcing divisibility} we can
  find an index \(h' ∈ ℕ\) such that
  \[
    2^{n + 1} \prod_{i = 0}^{n - 1} (ξ + i)^n (x + i)^n \mid \py_{h'}(a).
  \]
  By \cref{lem:y m divides y mk} we can set \(h := νh'\) and \(z :=
  \py_{eh}(a)\) then \eqref{eq:divisibilty of z} is satisfied. Now
  \cref{lem:defining e} implies that \(|σ_i(z)| = |σ_i(\py_{eh}(a))| ≥ C\) for
  \(s_K + 1 < i ≤ n\). Set \(w:= \px_{eh}(a)\) then the parts of
  \eqref{eq:sequences}, \eqref{eq:force sequences for w and z}, and
  \eqref{eq:approx of embeddings 2} involving \(z, z', w\) and \(w'\) can be
  satisfied, by setting \(z' := \py_{h'}(a)\) as well as \(w' :=
  \px_{h'}(a)\). To obtain algebraic integers \(u, u', v\) and \(v'\)
  satisfying the respective parts of \eqref{eq:approx of embeddings 2},
  \eqref{eq:v is non zero}, and \eqref{eq:z 2 divides v} we appeal to
  \cref{lem:forcing divisibility} to find an index \(m' ∈ ℕ\) such that
  \({\py_{eh}(a)}^2\) divides \(\py_{m'}(a)\). Now apply
  \cref{lem:approximations of sigma x and sigma y} to find an index \(m ∈ ℕ\)
  that is divisible by \(νm'\), such that \(|σ_i(\px_m(a)| > 1/2\) for all
  \(s_K + 1 < i ≤ n\). By \cref{lem:y m divides y mk} we have that
  \({\py_{eh}(a)}^2\) divides \(\py_m(a)\). Thus, we can set \(u := \px_m(a)\)
  and \(v := \py_m(a)\) (and \(u', v'\) accordingly). By \cref{lem:def of b} we
  can find an algebraic integer \(b ∈ \algint\) satisfying \eqref{eq:approx of
  embeddings 1}, \eqref{eq:cong 1}, and \eqref{eq:approximations of a}. Finally,
  set \(s := \px_k(b), t := \py_k(b)\), and \(s', t'\) accordingly. From
  \eqref{eq:cong 1} and \cref{lem:a congruent b mod c} it follows that
  \eqref{eq:cong 2} holds. Condition \eqref{eq:cong 3} follws completely
  analogously as in the first part.
\end{proof}

\begin{cor}\label{cor:ZZ is Diophantine over O K}
  Let \(K\) be a totally real number field or a number field with exactly one
  pair of non-real embeddings and at least one real embedding. Then \(ℤ\) is
  Diophantine over \(\algint\).
\end{cor}
\begin{proof}
  By the theorem there exists a Diophantine set \(S ⊂ \algint\) with the
  property \(ν ℕ ⊂ S ⊂ ℤ\), where \(ν := |μ(K)|\). Thus, \(ℤ\) can be defined in
  a Diophantine way as follows
  \begin{align*}
    α ∈ ℤ &⇔ ∃ β_1, β_2, β_3 ∈ \algint :\\
        & α = β_1 β_2 + β_3\\
        & β_1 ∈ S\\
        & β_2 ∈ \set{-1, 1}) ∧ (β_3 ∈ \set{0, 1, …, ν - 1})= 0.
  \end{align*}
\end{proof}

We have just seen that the rational integers are Diophantine over rings of
algebraic integers \(\algint\) if the number field \(K ≠ ℚ\) is totally real, or
\([K : ℚ] ≥ 3\) and there is exactly one pair of complex embeddings. From our
observations in \cref{cor:Diophantine theory is undecidable} Hilbert's tenth
problem over these rings is not decidable. The restriction on the degree in the
second case can be omitted, since \textcite{Denef1975} showed in
\citeyear{Denef1975} that \(ℤ\) is Diophantine over rings of algebraic integers
in quadratic number fields. This result was further strengthened by
\textcite{Denef1978}. They proved that \(ℤ\) is Diophantine over \(\algint\) if
\(K / M\) is a quadratic extension of a totally real number field \(M\).
\textcite{Shapiro1989} used these results to deduce that all cyclotomic fields
posses a Diophantine definition of \(ℤ\) over their rings of algebraic integers.
More generally, they deduced this result for all fields \(K\), for which \(K/ℚ\)
is normal and the Galois group of the extension is abelian.

All these results make the conjecture of \textcite{Denef1978}, that such a
Diophantine definition of \(ℤ\) over \(\algint\) exists for all number fields
\(K\), very plausible. This is especially true since promising techniques using
elliptic curves have been developed, for instance by \textcite{Poonen2002}.

Notice however that a Diophantine definition of \(ℤ\) might not be necessary for
the Diophantine theory \(\mathtt{H10}(\struc{O}_K)\) to be undecidable. Indeed,
there might exist rings of algebraic integers whose Diophantine theory forms a
kind of set that \textcite{Post1944} calls ‘creative’. Then
\(\mathtt{H10}(\struc{O}_K)\) would be undecidable as well, but the halting set
is not many-one reducible to this Diophantine theory. As a consequence of
\cref{thm:CE sets are Diophantine}, such a ring of integers cannot posses a
Diophantine definition of \(ℤ\).
