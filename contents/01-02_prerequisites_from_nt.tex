% !TeX encoding = UTF-8
% !TeX TS-program = xelatex
% !TeX spellcheck = en_GB
% !TeX root = ../Herbstrith-H10_over_AI.tex
% TODO
\todo{Fill out these theorems and definitions}

\subsection{Definitions}

\begin{defin}[class number $c_k$]
    pass
\end{defin}

\begin{defin}[Abelian extension]
    pass
\end{defin}

\subsection{Theorems}

Let $V$ be an $n$-dimensional vector space over the reals $ℝ$. If $\seq[k]{e} ∈
V$ ($k ≤ n$) are linearly independent over $ℝ$, then the free Abelian group
\[
  Λ = ℤ e_1 + … + ℤ e_k
\]
is called a \emph{lattice}. Note that $ℤ + \sqrt{2} ℤ$ is not a lattice in this
sense, because $1$ and $\sqrt{2}$ are linearly dependent over $ℝ$.

Let $μ$ be the measure corresponding to the usual Euclidean volume%
\footnote{I.e. $μ$ is the Lebesgue measure on $ℝ^n$ and therefore the Haar
measure with respect to the locally compact, Abelian group $⟨Λ, +⟩$.}
on $ℝ^n$. Then the \emph{fundamental parallelepiped}
\[
  D = \set{\sum_{i=1}^n α_i e_i \;\middle\vert\; α_i ∈ [0, 1]}
\]
has the volume
\[
  μ(D) = |\det \left( \seq{e} \right)|
\]
for a fixed lattice $Λ = ℤ e_1 + … + ℤ e_n$ in $ℝ^n$. One calls such a lattice whose rank coincides with the dimension of the vector space $ℝ^n$ a \emph{full} lattice.

\begin{thm}[Minkowski's theorem] \label{thm:Minkowski}
  Let $Λ = ℤ e_1 + … + ℤ e_n$ be a full lattice in the $n$-dimensional $ℝ$-vector space $V$ and let $D$ denote its fundamental parallelepiped. If $T \subseteq V$ is compact, convex and symmetric in the origin, i.e. if $α ∈ T$ so is $-α ∈ T$, and
  \[
    μ(T) ≥ 2^n μ(D).
  \]
  Then $T$ contains a non-zero lattice point $α ∈ Λ \setminus \set{0}$.
\end{thm}

For a proof of Minkowski's theorem and further details on lattices see \cite[4.4, p.~28]{Neukirch2006} or \cite[Thm.~4.19]{Milne2017}.

\begin{thm}[Dirichlet's unit theorem]\label{thm:Dirichlet}
    see \cite[Thm.~5.1]{Milne2017}
\end{thm}

\begin{thm}[Local-Global--Principle / Hasse-Minkowski theorem]
    see \cite[§ 66]{Meara2000}
\end{thm}
