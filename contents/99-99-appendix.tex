% !TEX encoding = UTF-8
% !TEX TS-program = xelatex
% !TEX spellcheck = en_GB
% !TEX engine = xelatex
% !TEX root = ../Herbstrith-H10_over_AI.tex
% 
%    ###    ########  ########  ########  ######## ##    ## ########
%   ## ##   ##     ## ##     ## ##     ## ##       ###   ## ##     ##
%  ##   ##  ##     ## ##     ## ##     ## ##       ####  ## ##     ##
% ##     ## ########  ########  ########  ######   ## ## ## ##     ##
% ######### ##        ##        ##        ##       ##  #### ##     ##
% ##     ## ##        ##        ##        ##       ##   ### ##     ##
% ##     ## ##        ##        ##        ######## ##    ## ########

\section{Simulating Turing machines}%
\label{app:turing}%

I have published a simulator of Turing machines implemented in \emph{Haskell}
at \url{https://github.com/tim6her/h10-turing-machines}. To obtain a copy of
the source code and build it using \emph{GHC} and \emph{cabal} run

\begin{lstlisting}[language=bash]
  git clone https://github.com/tim6her/h10-turing-machines.git
  cd h10-turing-machines
  cabal setup && cabal build && cabal install
\end{lstlisting}

To run the example codes for Turing machines enter the folder ‘listings’ and
start \verb+ghci+. The following listing shows how to run the Turing machine
deciding the tally encoding of non-negative integers. It might be necessary to
turn on Unicode printing in your \emph{GHC} installation.

\begin{lstlisting}
  >>> import Automaton.TuringMachine
  >>> :l tally
  >>> let d = toTransition tally "error" -- mark the errornous state
  >>> let turing = TuringMachine "start" '_' "halt" d
  >>> "§1111" >>> turing -- Tally encoding of 4
  Just "§1"
  >>> "§1011" >>> turing -- Not tally encoded
  Just "§0"
\end{lstlisting}

A full documentation of the Turing machine simulator is available on the
\emph{GitHub} repository.

\section{Polynomials}\label{app:polynomials}

The following listings show a \emph{Haskell} implementationf the monoid of
monomials and the algebra of polynomials in countably many indeterminates. Note
that the axioms of monoids and algebras respectively are only heuristically
verified but not formally proven.

\lstinputlisting[frame=tb,
                 caption=A \emph{Haskell} implementation of monomials in countably many indeterminates,
                 label=lst:monomials]{./listings/Monomial.hs}

\lstinputlisting[frame=tb,
                caption=A \emph{Haskell} implementation of polynomials in countable many indeterminates,
                label=lst:polynomials]{./listings/Polynomial.hs}
