\documentclass[10pt,a4paper]{scrbook}

% Input Type and AMS-Packages 
\usepackage{amsmath,amsfonts,amssymb,amsthm}
\usepackage{unicode-math}

% Typography
\usepackage{fontspec}
\setmainfont{EB Garamond}
\newfontfamily\headingfont[RawFeature={+c2sc,+scmp},
                           Letters=SmallCaps]{EB Garamond}
\setsansfont[Scale=MatchLowercase]{Open Sans}
\setmonofont[Scale=MatchUppercase]{Fira Mono}
\setmathtt[Scale=MatchUppercase]{Fira Mono}

\renewcommand{\scshape}{\headingfont}

\usepackage{polyglossia}
\setdefaultlanguage[variant=british]{english}
\setotherlanguage{german}
\usepackage{floatrow}
\usepackage{booktabs}

\usepackage{csquotes}
\usepackage{microtype}
\usepackage{enumitem}
\usepackage{subcaption}

\usepackage{letltxmacro}

\newlist{thmlist}{enumerate}{1}			% thmlist-s may only be used in theorem environments
\setlist[thmlist]{label=(\roman{thmlisti}), ref=\thethm.(\roman{thmlisti}),noitemsep}
\newlist{exlist}{enumerate}{1}
\setlist[exlist]{label=(\arabic{exlisti}), ref=(\arabic{exlisti}),noitemsep,leftmargin=0pt,itemindent=2\parindent}
\newlist{plist}{enumerate}{1}
\setlist[plist]{label=(\roman{plisti}), ref=(\roman{plisti}),noitemsep,leftmargin=0pt,itemindent=2\parindent}
\newlist{clist}{enumerate}{2}
\setlist[clist]{label*=(\alph*), ref=(\alph*),noitemsep,leftmargin=0pt,itemindent=2\parindent}

\usepackage{setspace}
\renewcommand{\arraystretch}{1.5}

\usepackage[symbol]{footmisc}		% Use symbols for footnotes
\usepackage{perpage}				% Package to reset counters at the end of each page.
\MakePerPage{footnote}				% Reset footnote-counter at the end of each page

\usepackage{listingsutf8}	% Typeset code.
\lstset{extendedchars=true,
        basicstyle=\ttfamily,
        commentstyle=\color{gray},
        stringstyle=\color{darkgray},
        xleftmargin=\parindent,
        language=haskell}

% BibLaTex
\usepackage[style=numeric,backend=biber]{biblatex}
\addbibresource{./references.bib}


% Math, Operators and Theorems
\usepackage{thmtools}	% More Flexibility in Theorem Styles + Provides in Combination with ref-Packages the 
						% Possibility to Refer to Theorem Style in Reference.
\usepackage{faktor}		% Displays factor groups

\numberwithin{equation}{section}

\DeclareMathOperator{\N}{\mathbb{N}}
\DeclareMathOperator{\Z}{\mathbb{Z}}
\DeclareMathOperator{\Q}{\mathbb{Q}}
\DeclareMathOperator{\R}{\mathbb{R}}
\DeclareMathOperator{\C}{\mathbb{C}}
\DeclareMathOperator{\F}{\mathbb{F}}

\DeclareMathOperator{\Aut}{Aut}
\DeclareMathOperator{\id}{id}

\DeclareMathOperator{\kernel}{ker}
\DeclareMathOperator{\im}{im}
\DeclareMathOperator{\End}{\mathrm{End}}
\DeclareMathOperator{\Hom}{\mathrm{Hom}}
\DeclareMathOperator{\Mod}{\mathrm{mod}}
\DeclareMathOperator{\D}{\mathrm{D}}
\DeclareMathOperator{\lcm}{\mathrm{lcm}}
\DeclareMathOperator{\ord}{\mathrm{ord}}

\newcommand{\sta}{\texttt{§}}
\newcommand{\emp}{\texttt{\_}}
\newcommand{\zer}{\mathtt 0}
\newcommand{\one}{\mathtt 1} 
\newcommand{\sstart}{s_{\text{start}}}
\newcommand{\shalt}{s_{\text{halt}}}
\newcommand{\scheck}{s_{\text{check}}}
\newcommand{\enc}[1]{\ulcorner #1 \urcorner}


\declaretheoremstyle[
    spaceabove=6pt, spacebelow=6pt,
    headfont=\headingfont,
    notefont=\mdseries, notebraces={(}{)},
    bodyfont=\itshape,
    postheadspace=1em
    ]{mythm}
\declaretheoremstyle[
    spaceabove=6pt, spacebelow=6pt,
    headfont=\headingfont,
    notefont=\mdseries, notebraces={(}{)},
    bodyfont=\normalfont,
    postheadspace=1em
    ]{mydef}

\declaretheorem[
	name=Theorem,
    style=mythm,
  	refname={theorem,theorems},		%Lower Case Versions of Theorem Type
  	Refname={Theorem,Theorems},
  	numberwithin=section]{thm}
\declaretheorem[
	name=Lemma,
    style=mythm,
	refname={lemma,lemmas},
	Refname={Lemma,Lemmas},
	sibling=thm]{lem}
\declaretheorem[
	name=Proposition,
    style=mythm,
	refname={proposition,propositions},
	Refname={Proposition,Propositions},
	sibling=thm]{pro}
\declaretheorem[
	name=Corollary,
    style=mythm,
	refname={corollary,corollarys},
	Refname={Corollary,Corollarys},
	sibling=thm]{cor}

\declaretheorem[
	name=Definition,
	style=mydef,
	numbered=no]{defin}
\declaretheorem[
	name=Example,
	style=mydef,
	numbered=no]{exam}
	
\declaretheorem[
	name=remark,
	style=remark,
	numbered=no]{rem}
\renewcommand*{\proofname}{proof}



%\newtheorem{theo}{Theorem}[section]
%\newtheorem{lemma}[theo]{Lemma}
%\newtheorem{prop}[theo]{Proposition}
%\newtheorem{cor}[theo]{Corollary}
%\theoremstyle{definition}
%\newtheorem*{defin}{Definition}
%\newtheorem*{exam}{Example}
%\theoremstyle{remark}
%\newtheorem*{rem}{Remark}

%TikZ and TikZ-Styles
\usepackage{tikz}

\usepackage{pgfcore}
\usetikzlibrary{arrows}

\tikzstyle{every node}=[circle,draw=black,font=\small,text=black,inner sep=1pt, minimum size=5mm]
\tikzstyle{edge from parent}=[draw=black]
\tikzstyle{st}=[black!20, line join=round, line width=2mm]
\tikzstyle{st1}=[black!40, line join=round, line width=2mm]

\tikzstyle{bgn}=[node distance=6mm, font=normal]
\tikzstyle{hlarrow}=[<->, shorten <=1pt, shorten >=1pt, draw, thick]

\tikzstyle{level 1}=[sibling distance=60mm]
\tikzstyle{level 2}=[sibling distance=30mm]
\tikzstyle{level 3}=[sibling distance=15mm]
\tikzstyle{level 4}=[sibling distance=7.5mm]
\tikzstyle{level 5}=[sibling distance=3.75mm, level distance=5mm]


% New- and Renewcommands

\newcommand{\seq}[2][n]{#2_{1},\ldots,#2_{#1}}
\newcommand{\nicefrac}[2]{#1/#2}




% ----------------------------------------------------------
% Only for drafts!
\usepackage{todonotes}
\usepackage{showlabels}	%Used in Drafts to Print references.
%-----------------------------------------------------------



% References
\usepackage{hyperref}
\usepackage[capitalize]{cleveref}

\hypersetup{
	pdftitle={Unbounded Burnside Problem for Groups},
	pdfauthor={Tim Benedikt Herbstrith}}

%\Crefname{thm}{Theorem}{Theorems}
%\Crefname{lem}{Lemma}{Lemmas}
%\Crefname{pro}{Proposition}{Propositions}
%\Crefname{cor}{Corollary}{Corollaries}

%\crefname{thm}{thm.}{thms.}
%\crefname{lem}{lemma}{lemmas.}
%\crefname{pro}{prop.}{props.}
%\crefname{cor}{cor.}{cors.}

\Crefname{thm}{Thm.}{Thms.}
\Crefname{lem}{Lemma}{Lemmas.}
\Crefname{pro}{Prop.}{Props.}
\Crefname{cor}{Cor.}{Cors.}

\addtotheorempostheadhook[thm]{\crefalias{thmlisti}{thm}}
\addtotheorempostheadhook[lem]{\crefalias{thmlisti}{lem}}
\addtotheorempostheadhook[pro]{\crefalias{thmlisti}{pro}}
\addtotheorempostheadhook[cor]{\crefalias{thmlisti}{cor}}

% Author and Title (Not in Use)
\author{Tim Benedikt Herbstrith}
\title{Unbounded Burnside Problem for Groups}
%------------------------------------------------------------------------------------------------------------------

\begin{document}

% Using Layers in TikZ
\pgfdeclarelayer{background}
\pgfsetlayers{background,main}

% Page Breaks for Displayed Formulas
\allowdisplaybreaks

\frontmatter

% !TeX encoding = UTF-8
% !TeX TS-program = xelatex
% !TeX spellcheck = en_GB
% !TeX root = ../Herbstrith-H10_over_AI.tex

\begin{titlepage}
%\vspace*{-2cm}  % bei Verwendung von vmargin.sty
\begin{flushright}
    \includegraphics{uni-logo}
\end{flushright}
\vspace{0.5cm}

\begin{center}  % Diplomarbeit ODER Magisterarbeit ODER Dissertation
    \Huge{\textsf{\textbf{\MakeUppercase{
        Masterarbeit
    }}}}
    \vspace{1.5cm}

    \large{\textsf{  % Diplomarbeit ODER Magisterarbeit ODER Dissertation
                     % (Dies ist erst die Ueberschrift!)
        Titel der Masterarbeit
    }}
    \vspace{.1cm}

    \LARGE{\textsf{ On Hilbert's Tenth Problem over\\
                    Rings of Algebraic Integers
    }}
    \vfill

    \large{\textsf{  % Verfasserin ODER Verfasser (Ueberschrift)
        Verfasser
    }}

    \Large{\textsf{  Tim Benedikt Herbstrith
    }}
    \vfill

    \large{\textsf{
        angestrebter akademischer Grad  % (Ueberschrift)
    }}

    \Large{\textsf{  % Magistra ODER Magister ODER Doktorin ODER Doktor
                     % ACHTUNG: Kuerzel "Mag.a" oder "Dr.in" nicht zulaessig
        Master of Science (MSc.)
    }}

\vspace{1.5cm}

\noindent\textsf{Wien, im Monat Mai 2018}  % <<<<< ORT, MONAT UND JAHR EINTRAGEN
\vfill

\noindent\begin{tabular}{@{}ll}
\textsf{Studienkennzahl lt.\ Studienblatt:}
&
\textsf{A 033 821}  % <<<<< STUDIENKENNZAHL EINTRAGEN
\\
    % BEI DISSERTATIONEN:
%\textsf{Dissertationsgebiet lt. Studienblatt:}
    % ANSONSTEN:
\textsf{Studienrichtung lt.\ Studienblatt:}
&
\textsf{Mathematik}  % <<<<< DISSGEBIET/STUDIENRICHTUNG EINTRAGEN
\\
% Betreuerin ODER Betreuer:
\textsf{Betreuer: }
&
\textsf{ao.~Univ.-Prof.~Mag.~Dr.~Ch.~Baxa}  % <<<<< NAME EINTRAGEN
\end{tabular}
\end{center}
\end{titlepage}

\newpage%
\thispagestyle{empty}%
\vspace*{\fill}%
\noindent%
\begin{footnotesize}%
If not stated otherwise, the text of \emph{On Hilbert's Tenth Problem over Rings
of Algebraic Integers} by Tim B. Herbstrith is licenced under a \textsc{Creative
Commons Attribution-NonCommercial-ShareAlike 4.0 International License}. All
code snippets are provided `as is' without warranty of any kind, express or
implied, including but not limited to the warranties of merchantability, fitness
for a particular purpose and non-infringement under the terms of the
\textsc{MIT} licence. The source codes and the aforementioned licences are
published at
\begin{center}
 \url{https://github.com/tim6her/h10-over-rings-of-integers}
\end{center}
\end{footnotesize}
\cleardoublepage


\begin{german}
\section*{Vorwort}
\todo{Schreibe das Vorwort}
\end{german}

\vspace{1.5cm}

\section*{Abstract}
\todo{Write the abstract}
\newpage
\thispagestyle{empty}
\tableofcontents


\mainmatter
\onehalfspacing

\chapter{Prerequisites and central notions}

Before stating Hilbert's 10th problem and proving its undecidability in
certain rings of algebraic integers, we remind the reader on some
notions of theoretical computer science and number theory, as well as
fix some notations.

\section{Different perspectives on an old problem}
\input{./contents/01-01_different_perspectives}

\section{Prerequisites from theoretical computer science}
\input{./contents/01-02_preliminaries_from_tcs}

\section{Binary Trees and Their Automorphism Group}
%\input{./Inhalt/BinaryTrees}

\section{Definition and Elementary Properties of the First Grigorchuk Group}
%\input{./Inhalt/FirstGrigorchukGroup}

\chapter{Burnside's Problems}
%\input{./Inhalt/HistoryOfBurnsidesProblems}

\section{A Counterexample to the Unbounded Burnside Problem}
%\input{./Inhalt/BurnsidesProblem}

\section{Growth of the First Grigorchuk Group}
%\input{./Inhalt/Growth}

\clearpage
\appendix
\chapter{Appendix}\label{sec:Appendix}
%\input{./Inhalt/Appendix}

\backmatter
\vspace{\fill}
\printbibliography

\listoftodos
\end{document}
